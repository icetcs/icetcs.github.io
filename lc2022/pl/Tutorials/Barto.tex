%% FIRST RENAME THIS FILE <yoursurname>.tex.
%% BEFORE COMPLETING THIS TEMPLATE, SEE THE "READ ME" SECTION
%% BELOW FOR INSTRUCTIONS.
%% TO PROCESS THIS FILE YOU WILL NEED TO DOWNLOAD asl.cls from
%% http://aslonline.org/abstractresources.html.


\documentclass[bsl,meeting]{asl}

\AbstractsOn

\pagestyle{plain}

\def\urladdr#1{\endgraf\noindent{\it URL Address}: {\tt #1}.}


\newcommand{\NP}{}
%\usepackage{verbatim}

\begin{document}
\thispagestyle{empty}

%% BEGIN INSERTING YOUR ABSTRACT DIRECTLY BELOW;
%% SEE INSTRUCTIONS (1), (2), (3), and (4) FOR PROPER FORMATS

\NP%
\absauth{Libor Barto}%
\meettitle{Algebra and Logic in the Complexity of Constraints}%
\affil{Department of Algebra, Faculty of Mathematics and Physics, Charles University}%
\meetemail{libor.barto@mff.cuni.cz}%
\urladdr{www.karlin.mff.cuni.cz/\string~barto}

%% INSERT TEXT OF ABSTRACT DIRECTLY BELOW

What kind of mathematical structure in computational problems allows for efficient algorithms? This fundamental question now has a satisfactory answer for a rather broad class of computational problems, so called fixed-template finite-domain Constraint Satisfaction Problems (CSPs). This answer, due to Bulatov and Zhuk, stems from the interplay between algebra and logic, similar to the classical connection between permutation groups and first-order definability. 

Th aim of this tutorial is to explain this algebra-logic interplay, show how it is applied in CSPs, and discuss some of the major research directions.


\end{document}
