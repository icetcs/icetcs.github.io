%% FIRST RENAME THIS FILE <yoursurname>.tex.
%% BEFORE COMPLETING THIS TEMPLATE, SEE THE "READ ME" SECTION
%% BELOW FOR INSTRUCTIONS.
%% TO PROCESS THIS FILE YOU WILL NEED TO DOWNLOAD asl.cls from
%% http://aslonline.org/abstractresources.html.


\documentclass[bsl,meeting]{asl}

\AbstractsOn

\pagestyle{plain}

\def\urladdr#1{\endgraf\noindent{\it URL Address}: {\tt #1}.}


\newcommand{\NP}{}
%\usepackage{verbatim}

\begin{document}
\thispagestyle{empty}

%% BEGIN INSERTING YOUR ABSTRACT DIRECTLY BELOW;
%% SEE INSTRUCTIONS (1), (2), (3), and (4) FOR PROPER FORMATS

\NP%
\absauth{Karen Lange}%
\meettitle{Classification via effective lists}%
\affil{Wellesley College}%
\meetemail{karen.lange@wellesley.edu}%
%\affil{affiliation second author}%
%\meetemail{email second author}%
\urladdr{https://www.wellesley.edu/math/faculty/karen\_lange}

%% INSERT TEXT OF ABSTRACT DIRECTLY BELOW
``Classifying” a natural collection of structures is  a common goal in mathematics.  Providing a classification can mean different things, e.g., identifying a set of invariants that settle the isomorphism problem or  creating a list of all structures of a given kind without repetition of isomorphism type. Here we discuss recent work on classifications of the latter kind from the perspective of computable structure theory.   We’ll consider natural classes of computable structures such as vector spaces, equivalence relations, algebraic fields, and trees to better understand the nuances of classification via effective lists and its relationship to other forms of classification in this setting.

\end{document}
