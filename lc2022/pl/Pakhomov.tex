%% FIRST RENAME THIS FILE <yoursurname>.tex.
%% BEFORE COMPLETING THIS TEMPLATE, SEE THE "READ ME" SECTION
%% BELOW FOR INSTRUCTIONS.
%% TO PROCESS THIS FILE YOU WILL NEED TO DOWNLOAD asl.cls from
%% http://aslonline.org/abstractresources.html.


\documentclass[bsl,meeting]{asl}
\AbstractsOn

\pagestyle{plain}

\def\urladdr#1{\endgraf\noindent{\it URL Address}: {\tt #1}.}


\newcommand{\NP}{}
%\usepackage{verbatim}

\begin{document}
\thispagestyle{empty}

%% BEGIN INSERTING YOUR ABSTRACT DIRECTLY BELOW;
%% SEE INSTRUCTIONS (1), (2), (3), and (4) FOR PROPER FORMATS

\NP%
\absauth{Fedor Pakhomov}%
\meettitle{Limits of applicability of Gödel's second incompleteness theorem}%
\affil{Ghent University, Krijgslaan 281, B9000 Ghent, Belgium}%
\meetemail{fedor.pakhomov@ugent.be}%

The celebrated  Gödel's second incompleteness theorem is the result that roughly speaking says that no strong enough consistent theory could prove its own consistency. In this talk I will first give an overview of the current state of research on the limits of applicability of the theorem. And second I will present two recent results: first is due to me \cite{Pak19} and the second is due to Albert Visser and me \cite{PV21}. The first result is an example of a weak natural theory that proves the arithmetization of its own consistency. The second result is a general theorem with the flavor of Second Incompleteness Theorem that is applicable to arbitrary weak first-order theories rather than to extension of some base system. Namely the theorem states that no finitely axiomatizable first-order theory one-dimensionally interprets its own extension by predicative comprehension.\\

\bibliographystyle{plain}
\bibliography{bibliography}
%$[1]$ Fedor Pakhomov. {\it A weak set theory that proves its own consistency}. Preprint, arXiv:1907.00877, 2019, https://arxiv.org/abs/1907.00877\\
%$[2]$ Fedor Pakhomov and Albert Visser. {\it Finitely axiomatized theories lack self-comprehension}. Preprint, arXiv:2109.02548, 2021, https://arxiv.org/abs/2109.02548
\end{document}
