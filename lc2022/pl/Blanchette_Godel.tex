%% FIRST RENAME THIS FILE <yoursurname>.tex.
%% BEFORE COMPLETING THIS TEMPLATE, SEE THE "READ ME" SECTION
%% BELOW FOR INSTRUCTIONS.
%% TO PROCESS THIS FILE YOU WILL NEED TO DOWNLOAD asl.cls from
%% http://aslonline.org/abstractresources.html.


\documentclass[bsl,meeting]{asl}

\AbstractsOn

\pagestyle{plain}

\def\urladdr#1{\endgraf\noindent{\it URL Address}: {\tt #1}.}


\newcommand{\NP}{}
%\usepackage{verbatim}

\begin{document}
\thispagestyle{empty}

%% BEGIN INSERTING YOUR ABSTRACT DIRECTLY BELOW;
%% SEE INSTRUCTIONS (1), (2), (3), and (4) FOR PROPER FORMATS
\NP%
\absauth{Patricia Blanchette}%
\meettitle{Formalism in Logic}%
\affil{University of Notre Dame}%
\meetemail{blanchette.1@nd.edu}%

\urladdr{http://sites.nd.edu/patricia-blanchette/}

%% INSERT TEXT OF ABSTRACT DIRECTLY BELOW
Logic became `formal' at the end of the 19th century primarily in pursuit of deductive rigor within mathematics. But by the early 20th century, a formal treatment of logic had become essential to two new streams in the current of logic: the collection of crucial `semantic' notions surrounding the idea of categoricity, and the project of examining the tools of logic themselves, in the way that`s crucial for the treatment of completeness (in its various guises). This lecture discusses the variety of different tasks that have been assigned the notion of formalization in the recent history of logic, with an emphasis on some of the ways in which the distinct purposes of formalization are not always in harmony with one another.

\end{document}
