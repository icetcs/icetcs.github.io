
\documentclass[bsl,meeting]{asl}

\AbstractsOn

\pagestyle{plain}

\def\urladdr#1{\endgraf\noindent{\it URL Address}: {\tt #1}.}


\newcommand{\NP}{}
%\usepackage{verbatim}


\begin{document}
\thispagestyle{empty}


\NP  
\absauth{Luca San Mauro}
\meettitle{Computable reductions of equivalence relations}
\affil{Department of Mathematics, Sapienza University of Rome}
\meetemail{luca.sanmauro@uniroma1.it}


The study of the complexity of equivalence relations has been a major
thread of research in diverse areas of logic. A reduction of an equivalence relation $E$ on a domain $X$ to an equivalence relation $F$ on
a domain $Y$ is a function $f : X \rightarrow Y$ which induces an injection on the quotient sets, $X_{/E}\rightarrow Y_{/F}$.
 In the literature, there are two main definitions for this reducibility: 
\begin{itemize}
\item In descriptive set theory, \emph{Borel reducibility} is defined by assuming that $X$ and $Y$ are Polish spaces and $f$ is Borel;
\item  In computability theory, \emph{computable reducibility} is defined by
assuming that $X$ and $Y$ coincide with the set of natural numbers
and $f$ is computable.
\end{itemize}

The theory of Borel equivalence relations is
a central field of modern descriptive set theory and it shows deep connections with topology, group theory, combinatorics, model theory, and ergodic
theory -- to name a few. On the other hand, computable reducibility dates back to the 1970s and it found remarkable applications in a diverse collection of fields, including the theory of numberings, proof theory, computable structure theory, combinatorial algebra, and theoretical computer science. 


Despite the clear analogy between the two notions, for a long time the study of Borel
and computable reducibility were conducted independentely. Yet, it is rapidly emerging a theory of computable reductions which blends ideas from both computability theory and descriptive set theory.  This tutorial will overview such a theory. We will present computable, or computably enumerable, analogs of fundamental concepts from the Borel theory (e.g., benchmark equivalence relations, dichotomy results,  orbit equivalence relations, the Friedman-Stanley jump), highlighting differences and similarities between  the Borel and the computable setting.  We will also report on recent progress in the abstract study of computable reducibility, focusing on both
 local structures of equivalence relations of given complexity and the global structure of all equivalence relations on the natural numbers.
\end{document}