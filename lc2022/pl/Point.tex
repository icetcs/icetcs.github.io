\documentclass[bsl,meeting]{asl}

\AbstractsOn

\pagestyle{plain}

\def\urladdr#1{\endgraf\noindent{\it URL Address}: {\tt #1}.}


\begin{document}
\thispagestyle{empty}




\absauth{Fran\c coise Point}
\meettitle{On differential expansions of topological fields}
\affil{Department of Mathematics, Mons University, 7000 Mons, Belgium}
\meetemail{Francoise.Point@umons.ac.be}


\par  A. Tarski, A. Robinson and A. Macintyre have described languages for which real-closed fields, algebraically closed valued fields, p-adically closed fieds admit quantifier elimination (and as a consequence one has a good understanding of definable sets in these structures). In particular, these structures are respectively o-minimal, C-minimal, p-minimal (more generally of dp-rank $1$). More recently one has described satisfactory languages for which the corresponding theories admit elimination of imaginaries. 
In his work on Shelah's conjecture on fields with the non independence property (NIP), W. Johnson has shown that a field of dp-rank $1$, which is not strongly minimal can be endowed with a definable field  topology. 
\par Differential expansions of (topological) fields of characteristic $0$, where there is a priori no interactions between the derivation and the topology, have been first considered by  M. Singer in the case of real-closed fields and he showed that the theory of differential ordered fields has a model companion. 
This was later generalized by M. Tressl in the class of large fields (a class of fields introduced by F. Pop). 
\par In this talk, we consider the following setting.
Given a large field of characteristic $0$ endowed with a definable field  topology and its theory $T$, we denote by $T_{\delta}$ the theory of differential expansions of models of $T$ by a derivation $\delta$  (satisfying the usual axiom: $\delta(x+y)=\delta(x)+\delta(y)\wedge \delta(xy)=\delta(x)y+x\delta(y)$). 
Under some further conditions on definable subsets in models of $T$, we show the following.
 The class of existentially closed models of $T_{\delta}$ is first-order axiomatisable by a theory $T_{\delta}^*$. Properties such as: quantifier elimination, the NIP property, elimination of imaginaries transfer from $T$ to $T_\delta^*$. In order to show the last result, we first prove a cell decomposition theorem for models of $T$, applying a similar strategy as for topological fields of dp-rank $1$ due to P. Simon and E. Walsberg and then we show that there are no new open definable sets in models of $T_{\delta}^*$.
This approach can be applied to certain theories of pairs of models of $T$.
 These results were obtained in collaboration with N. Guzy and P. Cubides Kovacsics.
\par Then, using that the theories $T$ we consider, are geometric theories (the topological dimension is well-behaved), 
we pursue our analysis to describe finite-dimensional definable groups in models of $T_{\delta}^{*}$. We relate them to definable groups in models of $T$, using Weil's approach to recover an algebraic group from generic data.
This last part is ongoing work with A. Pillay and K. Peterzil.
\end{document}