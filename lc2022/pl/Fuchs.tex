%% FIRST RENAME THIS FILE <yoursurname>.tex.
%% BEFORE COMPLETING THIS TEMPLATE, SEE THE "READ ME" SECTION
%% BELOW FOR INSTRUCTIONS.
%% TO PROCESS THIS FILE YOU WILL NEED TO DOWNLOAD asl.cls from
%% http://aslonline.org/abstractresources.html.


\documentclass[bsl,meeting]{asl}

\AbstractsOn

\pagestyle{plain}

\def\urladdr#1{\endgraf\noindent{\it URL Address}: {\tt #1}.}


\newcommand{\NP}{}
%\usepackage{verbatim}

\begin{document}
\thispagestyle{empty}

%% BEGIN INSERTING YOUR ABSTRACT DIRECTLY BELOW;
%% SEE INSTRUCTIONS (1), (2), (3), and (4) FOR PROPER FORMATS

\NP%
\absauth{Gunter Fuchs}%
\meettitle{Blurry HOD - a sketch of a landscape}%
\affil{CUNY College of Staten Island and Graduate Center}%
\meetemail{gunter.fuchs@csi.cuny.edu}%
%\affil{affiliation second author}%
%\meetemail{email second author}%
\urladdr{www.math.csi.cuny.edu/\textasciitilde fuchs}

%% INSERT TEXT OF ABSTRACT DIRECTLY BELOW

Classically, a set is ordinal definable if it is the unique object satisfying a formula with ordinal parameters. Generalizing this concept, given a cardinal $\kappa$, I call a set ${<}\kappa$-blurrily definable if it is one of less than $\kappa$ many objects satisfying a formula with ordinal parameters (called a ${<}\kappa$-blurry definition). By considering the hereditary versions of this notion, one arrives at a hierarchy of inner models, one for each cardinal $\kappa$: the collection of all hereditarily ${<}\kappa$-blurrily ordinal definable sets, which I call ${<}\kappa$-HOD. In a ZFC-model, this hierarchy spans the entire spectrum from HOD to V. 

The special cases $\kappa=\omega$ and $\kappa=\omega_1$ have been previously considered, but no systematic study of the general setting has been done, it seems. One main aspect of the study is the notion of a leap, that is, a cardinal at which a new object becomes hereditarily blurrily definable. The talk splits into two parts: first, the ZFC-provable properties of blurry HOD, which are surprisingly rich, and second, the effects of forcing on the structure of blurry HOD and the achievable leap constellations.
\end{document}
