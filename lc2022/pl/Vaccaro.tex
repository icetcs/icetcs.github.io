%% FIRST RENAME THIS FILE <yoursurname>.tex.
%% BEFORE COMPLETING THIS TEMPLATE, SEE THE "READ ME" SECTION
%% BELOW FOR INSTRUCTIONS.
%% TO PROCESS THIS FILE YOU WILL NEED TO DOWNLOAD asl.cls from
%% http://aslonline.org/abstractresources.html.


\documentclass[bsl,meeting]{asl}

\AbstractsOn

\pagestyle{plain}

\def\urladdr#1{\endgraf\noindent{\it URL Address}: {\tt #1}.}


\newcommand{\NP}{}
%\usepackage{verbatim}

\begin{document}
\thispagestyle{empty}

%% BEGIN INSERTING YOUR ABSTRACT DIRECTLY BELOW;
%% SEE INSTRUCTIONS (1), (2), (3), and (4) FOR PROPER FORMATS

\NP%
\absauth{Andrea Vaccaro}%
\meettitle{Games on classifiable C*-algebras}%
\affil{Universit\'e de Paris}%
\meetemail{vaccaro@imj-prg.fr}%
%\affil{affiliation second author}%
%\meetemail{email second author}%
%\urladdr{}

%% INSERT TEXT OF ABSTRACT DIRECTLY BELOW

One of the major themes of research in the study of C*-algebras, over the last decades, has been Elliott’s program to classify separable nuclear C*-algebras by their tracial and K-theoretic data, customarily represented in the so-called Elliott Invariant. In this talk I will analyze some subclasses of algebras (such as approximately finite C*-algebras) which fall within the scope of Elliott Classification Program from the perspective of infinitary continuous logic. More specifically, I will discuss how the techniques developed to classify nuclear C*-algebras can be combined with metric analogues of Ehrenfeucht–Fra\"iss\'e games, allowing to reduce the study of elementary equivalence between C*-algebras to the analogous relation on the discrete structures (groups and ordered groups) composing the Elliott Invariant. I will moreover show how this reduction can be employed to build classes of classifiable C*-algebras of arbitrarily high Scott rank.
\end{document}
