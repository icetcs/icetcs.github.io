\documentclass[bsl,meeting]{asl}

\AbstractsOn

\pagestyle{plain}
\newtheorem{definition}{Definition}

\def\urladdr#1{\endgraf\noindent{\it URL Address}: {\tt #1}.}


\newcommand{\NP}{}

\begin{document}
\thispagestyle{empty}
\NP  
\absauth{Vittorio Cipriani}
\meettitle{Equivalence relations and learning of algebraic structures}
\affil{Dipartimento di informatica, scienze matematiche e fisiche, Universit\`{a} degli studi di
Udine, Via delle Scienze 206, Udine (UD), Italy}
\meetemail{cipriani.vittorio@spes.uniud.it}
%% INSERT TEXT OF ABSTRACT DIRECTLY BELOW

In this talk we present some results related to algorithmic learning of algebraic structures. In a series of papers \cite{FKS-19, bazhenov2020learning, bazhenov2021turing} the authors developed a framework in which a learner receives larger and larger pieces of an arbitrary
copy of a computable structure and, at each stage, is required to output a conjecture about the isomorphism type of such a structure. The learning is successful if the conjectures eventually stabilize to a correct
guess. Borrowing ideas from descriptive set theory, we aim to calibrate the complexity of nonlearnable families, offering a new hierarchy based on reducibility between equivalence relations. To do so, we define the notion of $E$-learnability. 
\begin{definition}
A family of structures $\mathfrak{K}$ is $E$-learnable if there is function $\Gamma : 2^\omega \rightarrow 2^\omega$ which continuously reduce $\mathsf{LD}(\mathfrak{K})_{/\cong}$ to $E$, where $\mathsf{LD}(\mathfrak{K})$ is the collection of all copies of the structures from $\mathfrak{K}$.
\end{definition}

For example, we show that the paradigm introduced at the beginning coincides with $E_0$-learnability, where $E_0$ is the eventual agreement on reals. We then focus on the learning power of well-known benchmark Borel equivalence relations differentiating between learnability of finite and countably infinite families. 
The work presented in this talk is a joint work with Nikolay Bazhenov and Luca San Mauro, and some of the results discussed here can be found in \cite{bcsm21}.


\begin{thebibliography}{10}
\bibitem[BFSM20]{bazhenov2020learning}
Nikolay Bazhenov, Ekaterina Fokina, and Luca San~Mauro.
\newblock Learning families of algebraic structures from informant.
\newblock {\em Information and Computation}, 275:104590, 2020.
\bibitem[FKSM19]{FKS-19}
Ekaterina Fokina, Timo K{\"o}tzing, and Luca San~Mauro.
\newblock Limit learning equivalence structures.
\newblock In Aur\'elien Garivier and Satyen Kale, editors, {\em Proceedings of
  the 30th International Conference on Algorithmic Learning Theory}, volume~98
  of {\em Proceedings of Machine Learning Research}, pages 383--403, Chicago,
  Illinois, 22--24 Mar 2019. PMLR.
  \bibitem[BSM21]{bazhenov2021turing}
Nikolay Bazhenov and Luca San~Mauro.
\newblock On the {T}uring complexity of learning finite families of algebraic
  structures.
\newblock {\em Journal of Logic and Computation}, 2021.
\newblock Published online. arXiv preprint arXiv:2106.14515.
\bibitem[BCSM21]{bcsm21}
Nikolay Bazhenov, Vittorio Ciprani and Luca San Mauro.
\newblock Learning algebraic structures with the help of Borel equivalence relations.
\newblock arXiv preprint available at arxiv.2110.14512.


\end{thebibliography}


\vspace*{-0.5\baselineskip}


\end{document}