%% FIRST RENAME THIS FILE <yoursurname>.tex. 
%% BEFORE COMPLETING THIS TEMPLATE, SEE THE "READ ME" SECTION 
%% BELOW FOR INSTRUCTIONS. 
%% TO PROCESS THIS FILE YOU WILL NEED TO DOWNLOAD asl.cls from 
%% http://aslonline.org/abstractresources.html. 


\documentclass[bsl,meeting]{asl}

\AbstractsOn

\pagestyle{plain}

\def\urladdr#1{\endgraf\noindent{\it URL Address}: {\tt #1}.}


\newcommand{\NP}{}
%\usepackage{verbatim}

\begin{document}
\thispagestyle{empty}

%% BEGIN INSERTING YOUR ABSTRACT DIRECTLY BELOW; 
%% SEE INSTRUCTIONS (1), (2), (3), and (4) FOR PROPER FORMATS

\NP  
\absauth{Patrycja KupŚ and Andrzej Indrzejczak}
\meettitle{Methods of modelling linear time in hypersequent calculus}
\affil{Adam Mickiewicz University in Poznań}
\meetemail{patrycja.kups@amu.edu.pl}


\affil{University of Łódź}
\meetemail{andrzej.indrzejczak@uni.lodz.pl}

%% INSERT TEXT OF ABSTRACT DIRECTLY BELOW
Linear time temporal logics have been considered in a number of proof systems, ranging from tableaux methods, natural deduction and sequent calculi. However, most of these approaches are focused on the extensions of Prior's tense logic (TL). We are particularly interested in systems that model Linear Time Logic (LTL), which extend TL by $Next~Time$ and $Until$ operators. We also choose to focus on the hypersequent calculus (HC) since it proved to increase the expressive power of the sequent calculus in temporal logics. We offer a review of already existing methods of modelling linear time in a hypersequent approach and outline the possibility of providing an efficient proof system for LTL in HC. 


\begin{thebibliography}{10}
	
\bibitem{book_1}
{\scshape Demri, S., Goranko, V. \& Lange, M.},
{\bfseries\itshape Temporal Logics in Computer Science: Finite-State Systems},
Cambridge Tracts in Theoretical Computer Science,
Cambridge University Press,
2016.

\bibitem{col_1}
{\scshape Gaintzarain, J., Hermo, M., Lucio, P., Navarro, M. and Orejas, F.},
{\itshape A Cut-Free and Invariant-Free Sequent Calculus for PLTL},
{\bfseries\itshape Computer Science Logic. CSL 2007. Lecture Notes in Computer Science, vol 4646}
(Duparc, J., Henzinger, T.A., editors),
Springer,
Berlin, Heidelberg,
2007.

\bibitem{jor_1}
{\scshape Indrzejczak, A.},
{\itshape Linear Time in Hypersequent Calculus},
{\bfseries\itshape The Bulletin of Symbolic Logic},
vol.~20 (1), pp.~121--144.

\bibitem{jor_2}
\bysame
{\itshape Cut Elimination in Theorem For Non-Commutative Hypersequent Calculus},
{\bfseries\itshape The Bulletin of Symbolic Logic},
vol.~46 (1), pp.~135--149.




%% INSERT YOUR BIBLIOGRAPHIC ENTRIES HERE; 
%% SEE (4) BELOW FOR PROPER FORMAT.
%% EACH ENTRY MUST BEGIN WITH \bibitem{citation key}
%%
%% IF THERE ARE NO ENTRIES  
%% DELETE THE LINE ABOVE (\begin{thebibliography}{20}) 
%% AND THE LINE BELOW (\end{thebibliography})

\end{thebibliography}


\vspace*{-0.5\baselineskip}
% this space adjustment is usually necessary after a bibliography

\end{document}


%% READ ME
%% READ ME
%% READ ME

INSTRUCTIONS FOR SUPPLYING INFORMATION IN THE CORRECT FORMAT: 

1. Author names are listed as First Last, First Last, and First Last.

\absauth{FirstName1 LastName1, FirstName2 LastName2, and FirstName3 LastName3}


2. Titles of abstracts have ONLY the first letter capitalized,
except for Proper Nouns.

\meettitle{Title of abstract with initial capital letter only, except for
Proper Nouns} 


3. Affiliations and email addresses for authors of abstracts are
  listed separately.

% First author's affiliation
\affil{Department, University, Street Address, Country}
\meetemail{First author's email}
%%% NOTE: email required for at least one author
\urladdr{OPTIONAL}
%
% Second author's affiliation
\affil{Department, University, Street Address, Country}
\meetemail{Second author's email}
\urladdr{OPTIONAL}
%
% Third author's affiliation
\affil{Department, University, Street Address, Country}
% Second author's email
\meetemail{Third author's email}
\urladdr{OPTIONAL}


4. Bibliographic Entries

%%%% IF references are submitted with abstract,
%%%% please use the following formats

%%% For a Journal article
\bibitem{cite1}
{\scshape Author's Name},
{\itshape Title of article},
{\bfseries\itshape Journal name spelled out, no abbreviations},
vol.~XX (XXXX), no.~X, pp.~XXX--XXX.

%%% For a Journal article by the same authors as above,
%%% i.e., authors in cite1 are the same for cite2
\bibitem{cite2}
\bysame
{\itshape Title of article},
{\bfseries\itshape Journal},
vol.~XX (XXXX), no.~X, pp.~XX--XXX.

%%% For a book
\bibitem{cite3}
{\scshape Author's Name},
{\bfseries\itshape Title of book},
Name of series,
Publisher,
Year.

%%% For an article in proceedings
\bibitem{cite4}
{\scshape Author's Name},
{\itshape Title of article},
{\bfseries\itshape Name of proceedings}
(Address of meeting),
(First Last and First2 Last2, editors),
vol.~X,
Publisher,
Year,
pp.~X--XX.

%%% For an article in a collection
\bibitem{cite5}
{\scshape Author's Name},
{\itshape Title of article},
{\bfseries\itshape Book title}
(First Last and First2 Last2, editors),
Publisher,
Publisher's address,
Year,
pp.~X--XX.

%%% An edited book
\bibitem{cite6}
Author's name, editor. % No special font used here
{\bfseries\itshape Title of book},
Publisher,
Publisher's address,
Year.

