%% FIRST RENAME THIS FILE <yoursurname>.tex. 
%% BEFORE COMPLETING THIS TEMPLATE, SEE THE "READ ME" SECTION 
%% BELOW FOR INSTRUCTIONS. 
%% TO PROCESS THIS FILE YOU WILL NEED TO DOWNLOAD asl.cls from 
%% http://aslonline.org/abstractresources.html. 


\documentclass[bsl,meeting]{asl}

\AbstractsOn

\pagestyle{plain}

\def\urladdr#1{\endgraf\noindent{\it URL Address}: {\tt #1}.}


\newcommand{\NP}{}
%\usepackage{verbatim}

\begin{document}
\thispagestyle{empty}

%% BEGIN INSERTING YOUR ABSTRACT DIRECTLY BELOW; 
%% SEE INSTRUCTIONS (1), (2), (3), and (4) FOR PROPER FORMATS

\NP  
\absauth{Pablo Dopico}
\meettitle{Truth-theoretic determinacy revisited}
\affil{Department of Philosophy, King's College London, Strand, London, United Kingdom}
\meetemail{pablo.dopico@kcl.ac.uk}

%% INSERT TEXT OF ABSTRACT DIRECTLY BELOW

 Despite being highly successful, Saul Kripke's theory of truth (1975), based on the so-called fixed-point semantics, has been criticised on the basis of its incapacity to formulate the semantic status of paradoxical sentences such as the Liar. In other words, Kripke's theory treats that and similar sentences as being neither true nor false, but the object language lacks the resources to speak about the \textit{gappy} character of the Liar. From Burge (1979) to Field (2008), the tradition has suggested that this gappy character can be best understood as the idea that the Liar is not determinate or not determinately true. As a result, the main aim of this paper is to explore what being determinate in relation to a truth predicate could mean. We hence propose three different understandings of such notion in the form of three determinacy predicates and offer philosophical motivations for each of them. After that, we test different theories of truth against the background of those three understanding of truth-theoretic determinacy. In particular, we assess Kripke's theory of truth, as well as Solomon Feferman's well-known axiomatization of it (the so-called \textbf{KF}) (Halbach 2014), and Vann McGee's theory of definite truth (1991). Our results suggest that there could be a trade-off between the semantic expressibility of a theory of truth, understood as its ability to capture the semantic status of the Liar, and the logical strength of the theory.


\begin{thebibliography}{10}

\bibitem{cite1}
{\scshape Saul Kripke},
{\itshape Outline of a theory of truth},
{\bfseries\itshape The Journal of Philosophy},
vol.72, no.19, pp.690-716.

\bibitem{cite1}
{\scshape Tyler Burge},
{\itshape Semantical paradox},
{\bfseries\itshape The Journal of Philosophy},
vol.76, no.4, pp.169-198.

%%% For a book
\bibitem{cite3}
{\scshape Hartry Field},
{\bfseries\itshape Saving truth from paradox},
Oxford University Press,
2008.

\bibitem{cite3}
{\scshape Volker Halbach},
{\bfseries\itshape Axiomatic theories of truth},
Cambridge University Press,
2014.

\bibitem{cite3}
{\scshape Vann McGee},
{\bfseries\itshape Truth, vagueness, and paradox: An essay on the logic of truth},
Hackett Publishing,
1991.

%% INSERT YOUR BIBLIOGRAPHIC ENTRIES HERE; 
%% SEE (4) BELOW FOR PROPER FORMAT.
%% EACH ENTRY MUST BEGIN WITH \bibitem{citation key}
%%
%% IF THERE ARE NO ENTRIES  
%% DELETE THE LINE ABOVE (\begin{thebibliography}{20}) 
%% AND THE LINE BELOW (\end{thebibliography})

\end{thebibliography}


\vspace*{-0.5\baselineskip}
% this space adjustment is usually necessary after a bibliography

\end{document}


%% READ ME
%% READ ME
%% READ ME

INSTRUCTIONS FOR SUPPLYING INFORMATION IN THE CORRECT FORMAT: 

1. Author names are listed as First Last, First Last, and First Last.

\absauth{FirstName1 LastName1, FirstName2 LastName2, and FirstName3 LastName3}


2. Titles of abstracts have ONLY the first letter capitalized,
except for Proper Nouns.

\meettitle{Title of abstract with initial capital letter only, except for
Proper Nouns} 


3. Affiliations and email addresses for authors of abstracts are
  listed separately.

% First author's affiliation
\affil{Department, University, Street Address, Country}
\meetemail{First author's email}
%%% NOTE: email required for at least one author
\urladdr{OPTIONAL}
%
% Second author's affiliation
\affil{Department, University, Street Address, Country}
\meetemail{Second author's email}
\urladdr{OPTIONAL}
%
% Third author's affiliation
\affil{Department, University, Street Address, Country}
% Second author's email
\meetemail{Third author's email}
\urladdr{OPTIONAL}


4. Bibliographic Entries

%%%% IF references are submitted with abstract,
%%%% please use the following formats

%%% For a Journal article
\bibitem{cite1}
{\scshape Author's Name},
{\itshape Title of article},
{\bfseries\itshape Journal name spelled out, no abbreviations},
vol.~XX (XXXX), no.~X, pp.~XXX--XXX.

%%% For a Journal article by the same authors as above,
%%% i.e., authors in cite1 are the same for cite2
\bibitem{cite2}
\bysame
{\itshape Title of article},
{\bfseries\itshape Journal},
vol.~XX (XXXX), no.~X, pp.~XX--XXX.

%%% For a book
\bibitem{cite3}
{\scshape Author's Name},
{\bfseries\itshape Title of book},
Name of series,
Publisher,
Year.

%%% For an article in proceedings
\bibitem{cite4}
{\scshape Author's Name},
{\itshape Title of article},
{\bfseries\itshape Name of proceedings}
(Address of meeting),
(First Last and First2 Last2, editors),
vol.~X,
Publisher,
Year,
pp.~X--XX.

%%% For an article in a collection
\bibitem{cite5}
{\scshape Author's Name},
{\itshape Title of article},
{\bfseries\itshape Book title}
(First Last and First2 Last2, editors),
Publisher,
Publisher's address,
Year,
pp.~X--XX.

%%% An edited book
\bibitem{cite6}
Author's name, editor. % No special font used here
{\bfseries\itshape Title of book},
Publisher,
Publisher's address,
Year.

