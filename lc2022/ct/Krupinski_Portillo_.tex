%% FIRST RENAME THIS FILE <yoursurname>.tex. 
%% BEFORE COMPLETING THIS TEMPLATE, SEE THE "READ ME" SECTION 
%% BELOW FOR INSTRUCTIONS. 
%% TO PROCESS THIS FILE YOU WILL NEED TO DOWNLOAD asl.cls from 
%% http://aslonline.org/abstractresources.html. 


\documentclass[bsl,meeting]{asl}

\AbstractsOn

\pagestyle{plain}

\def\urladdr#1{\endgraf\noindent{\it URL Address}: {\tt #1}.}


\newcommand{\NP}{}
%\usepackage{verbatim}

\begin{document}
\thispagestyle{empty}

\NP  
\absauth{Krzysztof Krupi\'{n}ski, Adri\'{a}n Portillo}
\meettitle{On stable quotients}
\affil{University of Wroclaw, pl. Grunwaldzki 2/4 50-384 Wrocław, Poland}
\meetemail{Krzysztof.Krupinski@math.uni.wroc.pl}
\affil{University of Wroclaw, pl. Grunwaldzki 2/4 50-384 Wrocław, Poland}
\meetemail{Adrian.Portillo-Fernandez@math.uni.wroc.pl}

%% BEGIN INSERTING YOUR ABSTRACT DIRECTLY BELOW; 
	We solve two problems from the paper ``On maximal stable quotients of definable groups in NIP theories" by M. Haskel and A. Pillay, which concern maximal stable quotients of groups type-definable in NIP theories. The first result says that if $G$ is a type-definable group in a distal theory, then $G^{st}=G^{00}$ (where $G^{st}$ is the smallest type-definable subgroup with $G/G^{st}$ stable, and $G^{00}$ is the smallest type-definable subgroup of bounded index). In order to get it, we prove that distality is preserved under passing from $T$ to the hyperimaginary expansion $T^{heq}$. The second result is an example of a group $G$ definable in a non-distal, NIP theory for which $G=G^{00}$ but $G^{st}$ is not an intersection of definable groups. Our example is a saturated extension of $(\mathbb{R},+,[0,1])$. Moreover, we make some observations on the question whether there is such an example which is a group of finite exponent. We also take the opportunity and give several characterizations of stability of hyperdefinable sets, involving continuous logic.
%% SEE INSTRUCTIONS (1), (2), (3), and (4) FOR PROPER FORMATS


%% INSERT TEXT OF ABSTRACT DIRECTLY BELOW


\begin{thebibliography}{10}

%% INSERT YOUR BIBLIOGRAPHIC ENTRIES HERE; 
%% SEE (4) BELOW FOR PROPER FORMAT.
%% EACH ENTRY MUST BEGIN WITH \bibitem{citation key}
%%
%% IF THERE ARE NO ENTRIES  
%% DELETE THE LINE ABOVE (\begin{thebibliography}{20}) 
%% AND THE LINE BELOW (\end{thebibliography})

\bibitem{cite1}
{\scshape Haskel, Mike and Pillay, Anand},
{\itshape On maximal stable quotients of definable groups in {$NIP$} theories},
{\bfseries\itshape The Journal of Symbolic Logic},
vol.~83 (2018), no.~1, pp.~117--122.

\end{thebibliography}


\vspace*{-0.5\baselineskip}
% this space adjustment is usually necessary after a bibliography

\end{document}


%% READ ME
%% READ ME
%% READ ME

INSTRUCTIONS FOR SUPPLYING INFORMATION IN THE CORRECT FORMAT: 

1. Author names are listed as First Last, First Last, and First Last.

\absauth{FirstName1 LastName1, Adrian Portillo, and FirstName3 LastName3}


2. Titles of abstracts have ONLY the first letter capitalized,
except for Proper Nouns.

\meettitle{Title of abstract with initial capital letter only, except for
Proper Nouns} 


3. Affiliations and email addresses for authors of abstracts are
  listed separately.

% First author's affiliation
\affil{Department, University, Street Address, Country}
\meetemail{First author's email}
%%% NOTE: email required for at least one author
\urladdr{OPTIONAL}
%
% Second author's affiliation
\affil{Department, University, Street Address, Country}
\meetemail{Second author's email}
\urladdr{OPTIONAL}
%
% Third author's affiliation
\affil{Department, University, Street Address, Country}
% Second author's email
\meetemail{Third author's email}
\urladdr{OPTIONAL}


4. Bibliographic Entries

%%%% IF references are submitted with abstract,
%%%% please use the following formats

%%% For a Journal article
\bibitem{cite1}
{\scshape Author's Name},
{\itshape Title of article},
{\bfseries\itshape Journal name spelled out, no abbreviations},
vol.~XX (XXXX), no.~X, pp.~XXX--XXX.

%%% For a Journal article by the same authors as above,
%%% i.e., authors in cite1 are the same for cite2
\bibitem{cite2}
\bysame
{\itshape Title of article},
{\bfseries\itshape Journal},
vol.~XX (XXXX), no.~X, pp.~XX--XXX.

%%% For a book
\bibitem{cite3}
{\scshape Author's Name},
{\bfseries\itshape Title of book},
Name of series,
Publisher,
Year.

%%% For an article in proceedings
\bibitem{cite4}
{\scshape Author's Name},
{\itshape Title of article},
{\bfseries\itshape Name of proceedings}
(Address of meeting),
(First Last and First2 Last2, editors),
vol.~X,
Publisher,
Year,
pp.~X--XX.

%%% For an article in a collection
\bibitem{cite5}
{\scshape Author's Name},
{\itshape Title of article},
{\bfseries\itshape Book title}
(First Last and First2 Last2, editors),
Publisher,
Publisher's address,
Year,
pp.~X--XX.

%%% An edited book
\bibitem{cite6}
Author's name, editor. % No special font used here
{\bfseries\itshape Title of book},
Publisher,
Publisher's address,
Year.

