In this paper we study frame definability in finitely-valued modal logics and establish two main results via suitable translations: (1) in finitely-valued modal logics one cannot define more classes of frames than are already definable in classical modal logic (cf.~\cite[Thm.~8]{tho}), and (2)  a large family of finitely-valued modal logics define exactly the same classes of frames as classical modal logic (including modal logics based on finite Heyting and MV-algebras). In this way one may observe, for example, that the celebrated Goldblatt--Thomason theorem applies immediately to these logics. In particular, we obtain the central result from~\cite{te} with a much simpler proof  and answer one of the open questions left in that paper. Moreover, the proposed translations allow us to determine the computational complexity of a big class of finitely-valued modal logics. Finally, we show that the first translation we offer  (from finitely-valued modal logic into two-valued modal logic) yields a  0-1 law over models for the former (cf. \cite{kapron1}) as a corollary of W. Oberschelp's generalization \cite{ober} of  Fagin's 0-1 law. In particular, one can show that, over Kripke models for finitely-valued modal logics based on finite frames, for every modal formula  there is a truth-value that it takes almost surely at all worlds.

%
\begin{thebibliography}{10}
%
%%% INSERT YOUR BIBLIOGRAPHIC ENTRIES HERE; 
%%% SEE (4) BELOW FOR PROPER FORMAT.
%%% EACH ENTRY MUST BEGIN WITH \bibitem{citation key}
%%%
%%% IF THERE ARE NO ENTRIES  
%%% DELETE THE LINE ABOVE (\begin{thebibliography}{20}) 
%%% AND THE LINE BELOW (\end{thebibliography})
%

\bibitem{te} 
B.~Teheux. Modal definability for \L ukasiewicz validity relations. \emph{Studia Logica} 104 (2): 343--363 (2016).

\bibitem{tho} 
S.K.~Thomason. Possible worlds and many truth values. \emph{Studia Logica} 37: 195--204 (1978).

\bibitem{kapron1}
J. Y. Halpern and B. M. Kapron, Zero-one laws for modal logic,
\emph{Annals of Pure and Applied Logic}, 69: 157–193 (1994).


\bibitem{ober}
Walter Oberschelp. Asymptotic 0-1 laws in combinatorics. In D.~Jungnickel (ed.), \emph{Combinatorial theory, Lecture Notes in Mathematics} 969:276--292, Springer, 1982.


\end{thebibliography}
