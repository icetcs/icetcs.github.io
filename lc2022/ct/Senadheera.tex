%% FIRST RENAME THIS FILE <yoursurname>.tex. 
%% BEFORE COMPLETING THIS TEMPLATE, SEE THE "READ ME" SECTION 
%% BELOW FOR INSTRUCTIONS. 
%% TO PROCESS THIS FILE YOU WILL NEED TO DOWNLOAD asl.cls from 
%% http://aslonline.org/abstractresources.html. 


\documentclass[bsl,meeting]{asl}

\AbstractsOn

\usepackage[utf8]{inputenc}
\pagestyle{plain}

\def\urladdr#1{\endgraf\noindent{\it URL Address}: {\tt #1}.}


\newcommand{\NP}{}
%\usepackage{verbatim}

\begin{document}
\thispagestyle{empty}

%% BEGIN INSERTING YOUR ABSTRACT DIRECTLY BELOW; 
%% SEE INSTRUCTIONS (1), (2), (3), and (4) FOR PROPER FORMATS

\NP  
\absauth{D. Gihanee Senadheera}
\meettitle{Effective Concept Classes of PAC and PACi Incomparable Degrees and Jump Structure}
\affil{School of Mathematics and Statistical Sciences, Southern Illinois University, 1245 Lincoln Drive, Mail Code 4408, Carbondale IL, USA}
\meetemail{gihanee.s@siu.edu}

%% INSERT TEXT OF ABSTRACT DIRECTLY BELOW

The Probably Approximately Correct (PAC) learning model is a machine learning model introduced by Leslie Valiant in 1984. Similar to Turing reducibility there is a reducibility to this learning model as well. The PACi means a less restricted version of PAC reducibility. Here \textit{i} refers to the independence of the size and the computation time of the PAC reducibility.  
The ordering of concept classes under PAC reducibility is nonlinear, even when restricted to particular concrete examples. We recursively construct two c.e. effective concept classes of incomparable PACi degrees to show that there exist incomparable PACi degrees. Similarly, we can construct for PAC degrees which is analogous to incomparable Turing degrees. The priority construction method is used to construct the two concept classes, which was used by Friedburg and Muchnik in their proof of incomparable Turing degrees.  It was necessary to deal with the size of an effective concept class thus we propose to compute the size of the effective concept class using Kolmogorov complexity. Furthermore, we explore the jump structure of effective concept classes similar to the Turing jump and progress toward embedding $1$degrees.


\begin{thebibliography}{10}

%% INSERT YOUR BIBLIOGRAPHIC ENTRIES HERE; 
%% SEE (4) BELOW FOR PROPER FORMAT.
%% EACH ENTRY MUST BEGIN WITH \bibitem{citation key}
%%
%% IF THERE ARE NO ENTRIES  
%% DELETE THE LINE ABOVE (\begin{thebibliography}{20}) 
%% AND THE LINE BELOW (\end{thebibliography})


\bibitem{cite1}
{\scshape Wesley Calvert},
{\itshape PAC Learning, VC Dimensions, and The Arithmetic Hierarchy},
{\bfseries\itshape Archive for Mathematical Logic},
vol.~54, no.7-8, pp.~871--883.



\bibitem{cite2}
{\scshape Robert Soare}, 
{\bfseries\itshape Recursively Enumerable Sets and Degrees},
Springer-Verlag,
New York,
1987.

\bibitem{cite3}
{\scshape U.V. Vazirani and M.J. Kearns}, 
{\bfseries\itshape An Introduction to Computational Learning Theory},
The MIT Press
1994.

\bibitem{cite4}
{\scshape Ming Li and Paul Vitányi}, 
{\bfseries\itshape An Introduction to Kolmogorov Complexity and Its Applications},
Springer,
Switzerland,
2019.

\bibitem{cite5}
{\scshape Wesley Calvert}, 
{\bfseries\itshape Mathematical Logic and Probability},
2021 Pre-print.



\end{thebibliography}


\vspace*{-0.5\baselineskip}
% this space adjustment is usually necessary after a bibliography

\end{document}


%% READ ME
%% READ ME
%% READ ME

INSTRUCTIONS FOR SUPPLYING INFORMATION IN THE CORRECT FORMAT: 

1. Author names are listed as First Last, First Last, and First Last.

\absauth{FirstName1 LastName1, FirstName2 LastName2, and FirstName3 LastName3}


2. Titles of abstracts have ONLY the first letter capitalized,
except for Proper Nouns.

\meettitle{Title of abstract with initial capital letter only, except for
Proper Nouns} 


3. Affiliations and email addresses for authors of abstracts are
  listed separately.

% First author's affiliation
\affil{Department, University, Street Address, Country}
\meetemail{First author's email}
%%% NOTE: email required for at least one author
\urladdr{OPTIONAL}
%
% Second author's affiliation
\affil{Department, University, Street Address, Country}
\meetemail{Second author's email}
\urladdr{OPTIONAL}
%
% Third author's affiliation
\affil{Department, University, Street Address, Country}
% Second author's email
\meetemail{Third author's email}
\urladdr{OPTIONAL}


4. Bibliographic Entries

%%%% IF references are submitted with abstract,
%%%% please use the following formats

%%% For a Journal article
\bibitem{cite1}
{\scshape Author's Name},
{\itshape Title of article},
{\bfseries\itshape Journal name spelled out, no abbreviations},
vol.~XX (XXXX), no.~X, pp.~XXX--XXX.

%%% For a Journal article by the same authors as above,
%%% i.e., authors in cite1 are the same for cite2
\bibitem{cite2}
\bysame
{\itshape Title of article},
{\bfseries\itshape Journal},
vol.~XX (XXXX), no.~X, pp.~XX--XXX.

%%% For a book
\bibitem{cite3}
{\scshape Author's Name},
{\bfseries\itshape Title of book},
Name of series,
Publisher,
Year.

%%% For an article in proceedings
\bibitem{cite4}
{\scshape Author's Name},
{\itshape Title of article},
{\bfseries\itshape Name of proceedings}
(Address of meeting),
(First Last and First2 Last2, editors),
vol.~X,
Publisher,
Year,
pp.~X--XX.

%%% For an article in a collection
\bibitem{cite5}
{\scshape Author's Name},
{\itshape Title of article},
{\bfseries\itshape Book title}
(First Last and First2 Last2, editors),
Publisher,
Publisher's address,
Year,
pp.~X--XX.

%%% An edited book
\bibitem{cite6}
Author's name, editor. % No special font used here
{\bfseries\itshape Title of book},
Publisher,
Publisher's address,
Year.

