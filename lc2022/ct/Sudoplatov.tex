%% FIRST RENAME THIS FILE <yoursurname>.tex.
%% BEFORE COMPLETING THIS TEMPLATE, SEE THE "READ ME" SECTION
%% BELOW FOR INSTRUCTIONS.
%% TO PROCESS THIS FILE YOU WILL NEED TO DOWNLOAD asl.cls from
%% http://aslonline.org/abstractresources.html.


\documentclass[bsl,meeting]{asl}

\AbstractsOn

\pagestyle{plain}

\def\urladdr#1{\endgraf\noindent{\it URL Address}: {\tt #1}.}

\newtheorem{theorem}{Theorem}


\newcommand{\NP}{}
%\usepackage{verbatim}

\begin{document}
\thispagestyle{empty}

%% BEGIN INSERTING YOUR ABSTRACT DIRECTLY BELOW;
%% SEE INSTRUCTIONS (1), (2), (3), and (4) FOR PROPER FORMATS

\NP \absauth{Beibut Kulpeshov, Sergey Sudoplatov} \meettitle{On
theories of dense spherical orders}

\affil{Kazakh-British Technical University, Almaty, Kazakhstan}
\meetemail{b.kulpeshov@kbtu.kz}

\affil{Sobolev Institute of Mathematics, Novosibirsk State
Technical University, Novosibirsk State University, Novosibirsk,
Russia} \meetemail{sudoplat@math.nsc.ru}
%\urladdr{http://math.nsc.ru/$\sim$sudoplatov/}



\noindent

%% INSERT TEXT OF ABSTRACT DIRECTLY BELOW

We study properties of theories of $n$-spherical orders $K_n$
\cite{aafot, ananat} which naturally generalize linear orders
$K_2$ and circular orders $K_3$ \cite{KulMac, Kulp3, Kulp4}.

A {\em $n$-spherical} order relation, for $n\geq 2$, is described
by a $n$-ary relation $K_n$ satisfying the following conditions:
 (nso1) $\forall x_1,\ldots,x_n (K_n(x_1,x_2,\ldots,x_n)\to K_n(x_2,\ldots,x_n,x_1));$
 (nso2) $\forall x_1,\ldots,x_n
 \biggm((K_n(x_1,\ldots,x_i,\ldots,x_j,\ldots,x_n)\land$
 $K_n(x_1,\ldots,x_j,\ldots,x_i,\ldots,x_n))$ $\leftrightarrow\bigvee\limits_{1\leq k<l\leq n} x_k\approx x_l\biggm)$ for any $1\leq i<j\leq n$;
 (nso3) $\forall x_1,\ldots,x_n\biggm(K_n(x_1,\ldots,x_n)\to$
 $\forall t\left(\bigvee\limits_{i=1}^nK_n(x_1,\ldots,x_{i-1},t,x_{i+1},\ldots,x_n)\right)\biggm);$
 (nso4) $\forall x_1,\ldots,x_n (K_n(x_1,\ldots,x_i,$ $\ldots,$ $x_j,\ldots,x_n)\lor$
 $\lor K_n(x_1,\ldots,x_j,\ldots,x_i,\ldots,x_n)),\,\, 1\leq i<j\leq
 n.$

Structures $\mathcal{A}=\langle A,K_n\rangle$ with $n$-spherical
orders are called {\em $n$-spherical orders}, too.

A $n$-spherical order $K_n$ is called {\em dense} if it contains
at least two elements and for each $(a_1,a_2,a_3,\ldots,a_n)\in
K_n$ with $a_1\ne a_2$ there is $b\notin\{a_1,a_2,\ldots,a_n\}$
with
$
\models K_n(a_1,b,a_3,\ldots,a_n)\wedge K_n(b,a_2,a_3,\ldots,a_n).
$

\begin{theorem}\label{th_dense_isom}
If $\mathcal{A}$ and $\mathcal{B}$ are countable dense
$n$-spherical orders, $n\geq 2$, without endpoints for $n=2$, then
$\mathcal{A}\simeq\mathcal{B}$.
\end{theorem}

\begin{theorem}\label{th_dec}
For any natural $n\geq 2$ the theory $T_n$ of dense $n$-spherical
order is decidable.
\end{theorem}

The research is supported by Committee of Science in Education and
Science Ministry of the Republic of Kazakhstan, Grant No.
AP08855544 and Russian Scientific Foundation, Project
No.~22-21-00044. The work of the second author was carried out in
the framework of the State Contract of the Sobolev Institute of
Mathematics, Project No.~FWNF-2022-0012.

\begin{thebibliography}{10}

\bibitem{aafot} {\scshape  S.V.~Sudoplatov}, {\itshape Arities and aritizabilities of first-order
theories}, Preprint at https://arxiv.org/abs/2112.09593v1 (2021).

\bibitem{ananat} {\scshape  S.V.~Sudoplatov}, {\itshape Almost $n$-ary and almost $n$-aritizable
theories}, Preprint at https://arxiv.org/abs/2112.10330v1 (2021).

\bibitem{KulMac} {\scshape B.Sh.~Kulpeshov, H.D.~Macpherson}, {\itshape Minimality
conditions on circularly ordered structures}, {\bfseries\itshape
Mathematical Logic Quarterly}, Vol.~51, No.~4 (2005),
pp.~377--399.

\bibitem{Kulp3} {\scshape A.B.~Altaeva, B.Sh.~Kulpeshov},
{\itshape On almost binary weakly circularly minimal
structures},{\bfseries\itshape Bulletin of Karaganda University,
Mathematics}, Vol.~78, No.~2 (2015), pp.~74--82.

\bibitem{Kulp4} {\scshape B.Sh.~Kulpeshov}, {\itshape On almost binarity in weakly
circularly minimal structures}, {\bfseries\itshape Eurasian
Mathematical Journal}, Vol.~, No.~2 (2016), pp.~38--49.




%% INSERT YOUR BIBLIOGRAPHIC ENTRIES HERE;
%% SEE (4) BELOW FOR PROPER FORMAT.
%% EACH ENTRY MUST BEGIN WITH \bibitem{citation key}
%%
%% IF THERE ARE NO ENTRIES
%% DELETE THE LINE ABOVE (\begin{thebibliography}{20})
%% AND THE LINE BELOW (\end{thebibliography})

\end{thebibliography}


\vspace*{-0.5\baselineskip}
% this space adjustment is usually necessary after a bibliography

\end{document}


%% READ ME
%% READ ME
%% READ ME

INSTRUCTIONS FOR SUPPLYING INFORMATION IN THE CORRECT FORMAT:

1. Author names are listed as First Last, First Last, and First
Last.

\absauth{FirstName1 LastName1, FirstName2 LastName2, and
FirstName3 LastName3}


2. Titles of abstracts have ONLY the first letter capitalized,
except for Proper Nouns.

\meettitle{Title of abstract with initial capital letter only,
except for Proper Nouns}


3. Affiliations and email addresses for authors of abstracts are
  listed separately.

% First author's affiliation
\affil{Department, University, Street Address, Country}
\meetemail{First author's email}
%%% NOTE: email required for at least one author
\urladdr{OPTIONAL}
%
% Second author's affiliation
\affil{Department, University, Street Address, Country}
\meetemail{Second author's email} \urladdr{OPTIONAL}
%
% Third author's affiliation
\affil{Department, University, Street Address, Country}
% Second author's email
\meetemail{Third author's email} \urladdr{OPTIONAL}


4. Bibliographic Entries

%%%% IF references are submitted with abstract,
%%%% please use the following formats

%%% For a Journal article
\bibitem{cite1}
{\scshape Author's Name}, {\itshape Title of article},
{\bfseries\itshape Journal name spelled out, no abbreviations},
vol.~XX (XXXX), no.~X, pp.~XXX--XXX.

%%% For a Journal article by the same authors as above,
%%% i.e., authors in cite1 are the same for cite2
\bibitem{cite2}
\bysame {\itshape Title of article}, {\bfseries\itshape Journal},
vol.~XX (XXXX), no.~X, pp.~XX--XXX.

%%% For a book


%%% For an article in proceedings
\bibitem{cite4}
{\scshape Author's Name}, {\itshape Title of article},
{\bfseries\itshape Name of proceedings} (Address of meeting),
(First Last and First2 Last2, editors), vol.~X, Publisher, Year,
pp.~X--XX.

%%% For an article in a collection
\bibitem{cite5}
{\scshape Author's Name}, {\itshape Title of article},
{\bfseries\itshape Book title} (First Last and First2 Last2,
editors), Publisher, Publisher's address, Year, pp.~X--XX.

%%% An edited book
\bibitem{cite6}
Author's name, editor. % No special font used here
{\bfseries\itshape Title of book}, Publisher, Publisher's address,
Year.
