%% FIRST RENAME THIS FILE <yoursurname>.tex. 
%% BEFORE COMPLETING THIS TEMPLATE, SEE THE "READ ME" SECTION 
%% BELOW FOR INSTRUCTIONS. 
%% TO PROCESS THIS FILE YOU WILL NEED TO DOWNLOAD asl.cls from 
%% http://aslonline.org/abstractresources.html. 


\documentclass[bsl,meeting]{asl}

\newcommand{\la}{\langle}
\newcommand{\ra}{\rangle}

\DeclareMathOperator{\bprf}{Prf}
\DeclareMathOperator{\ea}{EA}
\DeclareMathOperator{\On}{On}
\DeclareMathOperator{\con}{Con}
\DeclareMathOperator{\true}{True}
\DeclareMathOperator{\ewd}{EWD}
\DeclareMathOperator{\next}{next}
\DeclareMathOperator{\rfn}{RFN}
\DeclareMathOperator{\bgl}{GL}
\DeclareMathOperator{\glp}{GLP}
\DeclareMathOperator{\RC}{RC}
\DeclareMathOperator{\aca}{ACA}
\DeclareMathOperator{\ti}{TI}
\DeclareMathOperator{\is}{I\Sigma}
\DeclareMathOperator{\ip}{I\Pi}
\DeclareMathOperator{\pa}{PA}
\DeclareMathOperator{\monid}{\top}
\newcommand{\w}{\mathbb{W}}
\newcommand{\WL}{\mathbb{W}^{\Lambda}}

\newcommand{\utb}{\ensuremath{\textup{UTB}}\xspace}
\newcommand{\ct}{\ensuremath{\textup{CT}}\xspace}
\AbstractsOn

\pagestyle{plain}

\def\urladdr#1{\endgraf\noindent{\it URL Address}: {\tt #1}.}


\newcommand{\NP}{}
%\usepackage{verbatim}

\begin{document}
\thispagestyle{empty}

%% BEGIN INSERTING YOUR ABSTRACT DIRECTLY BELOW; 
%% SEE INSTRUCTIONS (1), (2), (3), and (4) FOR PROPER FORMATS

\NP  
\absauth{David Fern\'andez-Duque, Joost J. Joosten, and 
Konstantinos Papafilippou}
\meettitle{Hyperarithmetical Worm Battles}
\affil{Department of Mathematics WE16, Ghent University, Ghent, Belgium}
\meetemail{David.FernandezDuque@UGent.be}
\affil{Department of Mathematics WE16, Ghent University, Ghent, Belgium}
\meetemail{Konstantinos.Papafilippou@UGent.be}
\affil{Department of Philosophy, University of Barcelona, Catalonia, Spain}
\meetemail{jjoosten@ub.edu}


%% INSERT TEXT OF ABSTRACT DIRECTLY BELOW
Japaridze's provability logic $\glp$ has one modality $[n]$ for each natural number and has been used by Beklemishev for a proof theoretic analysis of Peano aritmetic ($\pa$) and related theories.
In his analysis, he interprets $\glp$ in arithmetic by interpreting each modality $\la n \ra $ as $\Sigma_n$-$\rfn(T):= \{ \Box_{T} \phi \to \phi : \phi \text{ is a } \Sigma_n \text{ formula} \} $, for some given theory $T$ and with $\Box_T \phi$ standing for the formula: ``$\phi$ is provable in $T$".
He examines what he calls worms, which is the set $\sf W$ of formulas in $\glp$ defined as:
\begin{itemize}
    \item $\monid \in {\sf W}$;
    \item if $A \in {\sf W}$ and $n$ is a natural number, then $\la n \ra A \in {\sf W}$.
\end{itemize}

Among other benefits, this analysis yields \cite{Beklemishev_Survey} the so-called {\em Every Worm Dies} ($\ewd$) principle, a natural combinatorial statement that is similar in spirit to the hercules hydra battle and a bit more closely connected to the assertion of the totality of Hardy functions $H_\alpha$ on ordinals $\alpha < \epsilon_0$. He had then proven that $\ewd$ is equivalent over $\sf \ea := I \Delta_0 + exp$ to the $\Sigma_1$-$\rfn(\pa)$ and hence it is also independent of $\pa$.
Recently, Beklemishev and Pakhomov \cite{Beklemishev_Pakhomov_Reflection_algebras} have studied notions of provability corresponding to transfinite modalities in $\glp$ and they have looked into their connection to some theories of second order arithmetic.
We show \cite{K.Papafil_D.Fernandez-Duque_JoostJ.Joosten_ Hyperarithmetical Worm Battles} that indeed the natural transfinite extension of $\glp$ is sound for this interpretation, and yields similarly an equivalence to the $\Sigma_1$-reflection of the second order theory $\aca$ of arithmetical comprehension with full induction.
We also provide restricted versions of $\ewd$ related to the fragments $\is_n$ of Peano arithmetic.


\begin{thebibliography}{10}

\bibitem{Beklemishev_Survey}
{\scshape L D Beklemishev},
{\itshape Reflection principles and provability algebras in formal arithmetic},
{\bfseries\itshape Russian Mathematical Surveys},
vol.~60 (2005), no.~2, pp.~197--268.

\bibitem{Beklemishev_Pakhomov_Reflection_algebras}
{\scshape Lev D. Beklemishev and Fedor N. Pakhomov},
{\itshape Reflection algebras and conservation results for theories of iterated truth},
{\bfseries\itshape Annals of Pure and Applied Logic},
vol.~173 (2022), no.~5.

\bibitem{K.Papafil_D.Fernandez-Duque_JoostJ.Joosten_ Hyperarithmetical Worm Battles}
{\scshape Fern{\'a}ndez-Duque, David
and Papafilippou, Konstantinos
and Joosten, Joost J.},
{\itshape Hyperarithmetical Worm Battles},
{\bfseries\itshape Logical Foundations of Computer Science}
(Cham),
(Artemov Sergei and Nerode Anil, editors),
vol.~13137,
Publisher Springer International Publishing,
Year 2022,
pp.~52--69.
%% INSERT YOUR BIBLIOGRAPHIC ENTRIES HERE; 
%% SEE (4) BELOW FOR PROPER FORMAT.
%% EACH ENTRY MUST BEGIN WITH \bibitem{citation key}
%%
%% IF THERE ARE NO ENTRIES  
%% DELETE THE LINE ABOVE (\begin{thebibliography}{20}) 
%% AND THE LINE BELOW (\end{thebibliography})

\end{thebibliography}

\vspace*{-0.5\baselineskip}
% this space adjustment is usually necessary after a bibliography

\end{document}


%% READ ME
%% READ ME
%% READ ME

INSTRUCTIONS FOR SUPPLYING INFORMATION IN THE CORRECT FORMAT: 

1. Author names are listed as First Last, First Last, and First Last.

\absauth{FirstName1 LastName1, FirstName2 LastName2, and FirstName3 LastName3}


2. Titles of abstracts have ONLY the first letter capitalized,
except for Proper Nouns.

\meettitle{Title of abstract with initial capital letter only, except for
Proper Nouns} 


3. Affiliations and email addresses for authors of abstracts are
  listed separately.

% First author's affiliation
\affil{Department, University, Street Address, Country}
\meetemail{First author's email}
%%% NOTE: email required for at least one author
\urladdr{OPTIONAL}
%
% Second author's affiliation
\affil{Department, University, Street Address, Country}
\meetemail{Second author's email}
\urladdr{OPTIONAL}
%
% Third author's affiliation
\affil{Department, University, Street Address, Country}
% Second author's email
\meetemail{Third author's email}
\urladdr{OPTIONAL}


4. Bibliographic Entries

%%%% IF references are submitted with abstract,
%%%% please use the following formats

%%% For a Journal article
\bibitem{cite1}
{\scshape Author's Name},
{\itshape Title of article},
{\bfseries\itshape Journal name spelled out, no abbreviations},
vol.~XX (XXXX), no.~X, pp.~XXX--XXX.

%%% For a Journal article by the same authors as above,
%%% i.e., authors in cite1 are the same for cite2
\bibitem{cite2}
\bysame
{\itshape Title of article},
{\bfseries\itshape Journal},
vol.~XX (XXXX), no.~X, pp.~XX--XXX.

%%% For a book
\bibitem{cite3}
{\scshape Author's Name},
{\bfseries\itshape Title of book},
Name of series,
Publisher,
Year.

%%% For an article in proceedings
\bibitem{cite4}
{\scshape Author's Name},
{\itshape Title of article},
{\bfseries\itshape Name of proceedings}
(Address of meeting),
(First Last and First2 Last2, editors),
vol.~X,
Publisher,
Year,
pp.~X--XX.

%%% For an article in a collection
\bibitem{cite5}
{\scshape Author's Name},
{\itshape Title of article},
{\bfseries\itshape Book title}
(First Last and First2 Last2, editors),
Publisher,
Publisher's address,
Year,
pp.~X--XX.

%%% An edited book
\bibitem{cite6}
Author's name, editor. % No special font used here
{\bfseries\itshape Title of book},
Publisher,
Publisher's address,
Year.


