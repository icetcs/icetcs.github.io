%% FIRST RENAME THIS FILE <yoursurname>.tex. 
%% BEFORE COMPLETING THIS TEMPLATE, SEE THE "READ ME" SECTION 
%% BELOW FOR INSTRUCTIONS. 
%% TO PROCESS THIS FILE YOU WILL NEED TO DOWNLOAD asl.cls from 
%% http://aslonline.org/abstractresources.html. 


\documentclass[bsl,meeting]{asl}

\AbstractsOn

\pagestyle{plain}

\def\urladdr#1{\endgraf\noindent{\it URL Address}: {\tt #1}.}


\newcommand{\NP}{}
%\usepackage{verbatim}

\begin{document}
\thispagestyle{empty}

\NP  
\absauth{David Fern\'andez-Duque, Oriola Gjetaj, Andreas Weiermann}
\meettitle{Intermediate Goodstein principles}
\affil{Institute for Analysis, Logic and Discrete Mathematics, Ghent University, Krijgslaan 281 S8, 9000 Ghent, Belgium.}
\meetemail{david.FernandezDuque@UGent.be}
\meetemail{oriola.gjetaj@ugent.be}
\meetemail{andreas.weiermann@ugent.be}


%% INSERT TEXT OF ABSTRACT DIRECTLY BELOW
The Goodstein principle is a natural number-theoretic theorem which is unprovable in Peano arithmetic.The original process proceeds by writing natural numbers in nested exponential k-base normal form, then successively raising the base to k + 1 and subtracting one from the end result. Such sequences always reach zero, but this fact is unprovable in Peano arithmetic. \\
In this talk, we will consider canonical representations of natural numbers using Ackermann function and the function of Grzegorczyk hierarchy. These representations give a natural Goodstein process for which we obtain independence from different theories of reverse mathematics. \\
This is a joint ongoing work with A. Weiermann and D. Fern\'andez-Duque on exploring normal form notations for the Goodstein principle.

\begin{thebibliography}{10}
\bibitem{cite1}
{\scshape  D.Fern\'andez-Duque, O.Gjetaj, A. Weiermann},
{\itshape Intermediate Goodstein principles},
{\bfseries\itshape Mathematics for Computation(M4C)}
(2023),
Accepted for publication.

\bibitem{cite2}
{\scshape D. Fern{\'{a}}ndez{-}Duque and A. Weiermann},
{\itshape Ackermannian Goodstein Sequences of Intermediate Growth},
{\bfseries\itshape Beyond the Horizon of Computability - 16th Conference on Computability
	in Europe, CiE 2020, Fisciano, Italy, June 29 - July 3, 2020, Proceedings}
(Marcella Anselmo and
Gianluca Della Vedova and
Florin Manea and
Arno Pauly, editors),
vol.~12098,
Springer,
(2020),
pp.~163--174.

\bibitem{cite3}
{\scshape T. Arai, D. Fern\'andez-Duque, S. Wainer and A. Weiermann},
	{\itshape Predicatively Unprovable Termination of the {A}ckermannian {G}oodstein Principle},
	{\bfseries\itshape Proceedings of the American Mathematical Society},
	vol.~148,
	(2019),
	pp.~3567--3582

\bibitem{cite4}
{\scshape R.L. Goodstein},
{\itshape On the restricted ordinal theorem},
{\bfseries\itshape J. Symbolic Logic},
9, 
(1944), 
pp.~33--41. 


\bibitem{cite5}
{\scshape L.~Kirby and J.~Paris.},
{\itshape On the restricted ordinal theorem},
{\bfseries\itshape Bulletin of The London Mathematical Society},
14 ,(1982), no. 4, pp.~285--293. 


\bibitem{cite6}
{\scshape A.Weiermann},
{\itshape Ackermannian {G}oodstein principles for first order {P}eano
	arithmetic},
{\bfseries\itshape Sets and Computations, Lecture Notes Series, Institute for
	Mathematical Sciences, National University of Singapore },
vol.~33, WorldScientific Publications, Hackensack, NJ, (2018), pp.~157--181.


%% INSERT YOUR BIBLIOGRAPHIC ENTRIES HERE; 
%% SEE (4) BELOW FOR PROPER FORMAT.
%% EACH ENTRY MUST BEGIN WITH \bibitem{citation key}
%%
%% IF THERE ARE NO ENTRIES  
%% DELETE THE LINE ABOVE (\begin{thebibliography}{20}) 
%% AND THE LINE BELOW (\end{thebibliography})

\end{thebibliography}


\vspace*{-0.5\baselineskip}
% this space adjustment is usually necessary after a bibliography

\end{document}


%% READ ME
%% READ ME
%% READ ME

INSTRUCTIONS FOR SUPPLYING INFORMATION IN THE CORRECT FORMAT: 

1. Author names are listed as First Last, First Last, and First Last.

\absauth{FirstName1 LastName1, FirstName2 LastName2, and FirstName3 LastName3}


2. Titles of abstracts have ONLY the first letter capitalized,
except for Proper Nouns.

\meettitle{Title of abstract with initial capital letter only, except for
Proper Nouns} 


3. Affiliations and email addresses for authors of abstracts are
  listed separately.

% First author's affiliation
\affil{Department, University, Street Address, Country}
\meetemail{First author's email}
%%% NOTE: email required for at least one author
\urladdr{OPTIONAL}
%
% Second author's affiliation
\affil{Department, University, Street Address, Country}
\meetemail{Second author's email}
\urladdr{OPTIONAL}
%
% Third author's affiliation
\affil{Department, University, Street Address, Country}
% Second author's email
\meetemail{Third author's email}
\urladdr{OPTIONAL}


4. Bibliographic Entries

%%%% IF references are submitted with abstract,
%%%% please use the following formats

%%% For a Journal article
\bibitem{cite1}
{\scshape Author's Name},
{\itshape Title of article},
{\bfseries\itshape Journal name spelled out, no abbreviations},
vol.~XX (XXXX), no.~X, pp.~XXX--XXX.

%%% For a Journal article by the same authors as above,
%%% i.e., authors in cite1 are the same for cite2
\bibitem{cite2}
\bysame
{\itshape Title of article},
{\bfseries\itshape Journal},
vol.~XX (XXXX), no.~X, pp.~XX--XXX.

%%% For a book
\bibitem{cite3}
{\scshape Author's Name},
{\bfseries\itshape Title of book},
Name of series,
Publisher,
Year.

%%% For an article in proceedings
\bibitem{cite4}
{\scshape Author's Name},
{\itshape Title of article},
{\bfseries\itshape Name of proceedings}
(Address of meeting),
(First Last and First2 Last2, editors),
vol.~X,
Publisher,
Year,
pp.~X--XX.

%%% For an article in a collection
\bibitem{cite5}
{\scshape Author's Name},
{\itshape Title of article},
{\bfseries\itshape Book title}
(First Last and First2 Last2, editors),
Publisher,
Publisher's address,
Year,
pp.~X--XX.

%%% An edited book
\bibitem{cite6}
Author's name, editor. % No special font used here
{\bfseries\itshape Title of book},
Publisher,
Publisher's address,
Year.

