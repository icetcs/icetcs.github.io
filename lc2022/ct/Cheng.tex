%% FIRST RENAME THIS FILE <yoursurname>.tex.
%% BEFORE COMPLETING THIS TEMPLATE, SEE THE "READ ME" SECTION
%% BELOW FOR INSTRUCTIONS.
%% TO PROCESS THIS FILE YOU WILL NEED TO DOWNLOAD asl.cls from
%% http://aslonline.org/abstractresources.html.


\documentclass[bsl,meeting]{asl}

\AbstractsOn

\pagestyle{plain}

\def\urladdr#1{\endgraf\noindent{\it URL Address}: {\tt #1}.}
\newcommand{\PA}{\mathbf{PA}}
\newcommand{\Godel}{G\"{o}del}

\newcommand{\NP}{}
%\usepackage{verbatim}

\begin{document}
\thispagestyle{empty}

%% BEGIN INSERTING YOUR ABSTRACT DIRECTLY BELOW;
%% SEE INSTRUCTIONS (1), (2), (3), and (4) FOR PROPER FORMATS

\NP
\absauth{Yong Cheng}
\meettitle{The relevance of the incompleteness theorems with Hilbert's concrete
proof theory}
\affil{School of Philosophy, Wuhan University, China}
\meetemail{world-cyr@hotmail.com}

It is widely believed that G\"{o}del's first and second incompleteness theorem ({\sf G1} and
{\sf G2}) undermined Hilbert's program. We examine the relevance of {\sf G1} and {\sf G2} with
Hilbert's concrete proof theory. We argue that even if {\sf G1} and {\sf G2} refute (in a narrow
sense) some original goals of Hilbert's program, G\"{o}del's solutions are not concrete and real in the sense of Hilbert's concrete proof theory. 

We could view {\sf G1} as an existence problem. In the sense of Hilbert's concrete proof
theory, the independent sentence of {\sf PA}  G\"{o}del constructed is an ideal element which is not real and concrete. Even if we can say (in the narrow sense) that {\sf G1} shows there is no strong enough consistent axiomatized formal system in which all true statements are provable, {\sf G1} does not answer the following question in the spirit of Hilbert's concrete proof theory: whether all concrete true arithmetic sentences are provable in {\sf PA}. The research program after G\"{o}del on concrete incompleteness looks for concrete and real solutions of the existence problem of {\sf G1}. We could view this research practice as a realization of Hilbert's concrete proof theory.

It is a popular view that {\sf G2} destroys Hilbert's consistency program. Nonetheless,
there are dissidents (see \cite{Detlefsen}, \cite{Artemov}). Neither G\"{o}del nor Hilbert think Hilbert's consistency program were destroyed by {\sf G2}. Hilbert thinks {\sf G2} only shows one must exploit the finitary standpoint in a sharper way for the consistency proofs (see \cite{Hilbert}). G\"{o}del writes in \cite{Godel} that it is conceivable that there exist finitary proofs that can not be expressed in the
formalism of the basis system. We argue that the current research on the consistency
problem confirms G\"{o}del's view that {\sf G2} does not destroy but leaves Hilbert's program very much alive and even more interesting than it initially was. There is no purely
mathematical solution of the consistency problem since each solution of it is related
to a philosophical question: what is a solution of the consistency problem? A key
issue of this problem is: how to formulate the consistency statement and what is the
``correct" formulation if any? Different formulations of the consistency statement may
lead to different answers of the consistency problem. We examine different methods
to formulate the consistency statement and compare them in the spirit of Hilbert's
concrete proof theory: whether one formulation of the consistency statement is more
concrete than another one. This research is a beginning step toward the interesting
open question: what is a ``natural" consistency statement?

In summary, in a strong sense we argue that G\"{o}del's original {\sf G1} and {\sf G2} have no
relevance with Hilbert's concrete proof theory; but some lines of research after G\"{o}del on {\sf G1} and {\sf G2} can be interpreted as a realization of Hilbert's concrete proof theory in the sense of finding concrete solutions of the existence problem in {\sf G1} and formulating the consistency statement in a more natural and concrete way.



\begin{thebibliography}{100}

\bibitem{Artemov}
Sergei Artemov. The Provability of Consistency. preprint, see arXiv:
1902.07404v5, 2019.


\bibitem{Detlefsen} M. Detlefsen. What does G\"{o}del's second theorem say? Philosophia Mathematica,
9:37-71, 2001.

\bibitem{Godel} K. G\"{o}del. On formally undecidable propositions of Principia Mathematica and
related systems. In Collected Works: Oxford University Press: New York. Editor-inchief:
Solomon Feferman. Volume I: Publications 1929-1936, 1986.

\bibitem{Hilbert} D. Hilbert and P. Bernays. Grundlagen der Mathematik. Vol. I. Springer, 1934.
\end{thebibliography}

\end{document}


%% READ ME
%% READ ME
%% READ ME

INSTRUCTIONS FOR SUPPLYING INFORMATION IN THE CORRECT FORMAT:

1. Author names are listed as First Last, First Last, and First Last.

\absauth{FirstName1 LastName1, FirstName2 LastName2, and FirstName3 LastName3}


2. Titles of abstracts have ONLY the first letter capitalized,
except for Proper Nouns.

\meettitle{Title of abstract with initial capital letter only, except for
Proper Nouns}


3. Affiliations and email addresses for authors of abstracts are
  listed separately.

% First author's affiliation
\affil{Department, University, Street Address, Country}
\meetemail{First author's email}
%%% NOTE: email required for at least one author
\urladdr{OPTIONAL}
%
% Second author's affiliation
\affil{Department, University, Street Address, Country}
\meetemail{Second author's email}
\urladdr{OPTIONAL}
%
% Third author's affiliation
\affil{Department, University, Street Address, Country}
% Second author's email
\meetemail{Third author's email}
\urladdr{OPTIONAL}


4. Bibliographic Entries

%%%% IF references are submitted with abstract,
%%%% please use the following formats

%%% For a Journal article
\bibitem{cite1}
{\scshape Author's Name},
{\itshape Title of article},
{\bfseries\itshape Journal name spelled out, no abbreviations},
vol.~XX (XXXX), no.~X, pp.~XXX--XXX.

%%% For a Journal article by the same authors as above,
%%% i.e., authors in cite1 are the same for cite2
\bibitem{cite2}
\bysame
{\itshape Title of article},
{\bfseries\itshape Journal},
vol.~XX (XXXX), no.~X, pp.~XX--XXX.

%%% For a book
\bibitem{cite3}
{\scshape Author's Name},
{\bfseries\itshape Title of book},
Name of series,
Publisher,
Year.

%%% For an article in proceedings
\bibitem{cite4}
{\scshape Author's Name},
{\itshape Title of article},
{\bfseries\itshape Name of proceedings}
(Address of meeting),
(First Last and First2 Last2, editors),
vol.~X,
Publisher,
Year,
pp.~X--XX.

%%% For an article in a collection
\bibitem{cite5}
{\scshape Author's Name},
{\itshape Title of article},
{\bfseries\itshape Book title}
(First Last and First2 Last2, editors),
Publisher,
Publisher's address,
Year,
pp.~X--XX.

%%% An edited book
\bibitem{cite6}
Author's name, editor. % No special font used here
{\bfseries\itshape Title of book},
Publisher,
Publisher's address,
Year.

