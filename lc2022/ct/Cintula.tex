%% FIRST RENAME THIS FILE <yoursurname>.tex.
%% BEFORE COMPLETING THIS TEMPLATE, SEE THE "READ ME" SECTION
%% BELOW FOR INSTRUCTIONS.
%% TO PROCESS THIS FILE YOU WILL NEED TO DOWNLOAD asl.cls from
%% http://aslonline.org/abstractresources.html.

\let\negmedspace\undefined
\let\negthickspace\undefined

\documentclass[bsl,meeting]{asl}


\AbstractsOn

\pagestyle{plain}

\def\urladdr#1{\endgraf\noindent{\it URL Address}: {\tt #1}.}


\newcommand{\NP}{}
%\usepackage{verbatim}

\begin{document}
\thispagestyle{empty}
\thispagestyle{empty}




\NP
\absauth{Petr Cintula}
\meettitle{Abstract Lindenbaum lemma for non-finitary consequence relations}
\affil{Institute of Computer Science of the Czech Academy of Sciences,\\ Pod Vod\' arenskou V\v e\v z\' i 271, Prague, Czech Republic}
\meetemail{cintula@cs.cas.cz}


%% INSERT TEXT OF ABSTRACT DIRECTLY BELOW

The Lindenbaum lemma is an easy yet crucial result in algebraic logic abstractly formulated as: for any \emph{finitary} consequence relation, the meet-irreducible theories form a basis of the closure system of its theories. While the finitarity restriction is crucial for its usual proof, it is not necessary: there are works (e.g.\ \cite{Goldblatt93,Segerberg94,Sundholm77}) proving it (or its variant for \emph{finitely} meet-irreducible theories) for certain \emph{infinitary structural} consequence relations. The paper~\cite{Bilkova-Cintula-Lavicka:InfinitaryWOLLIC} provides a general result (covering most of the known cases) for \emph{structural} consequence relations with a \emph{countable} Hilbert-style axiomatization and a \emph{strong disjunction}~\cite{CN:TheBook}. Identifying the essential non-structural properties of strong disjunctions we can prove:

\smallskip

\noindent {\bf Lemma} Let $\vdash$ be a consequence relation with a countable axiomatization such that the closure system $\mathcal T_\vdash$ of its theories is a frame (i.e.\ satisfies the corresponding infinite distributive law) and the intersection of two finitely generated theories is finitely generated. Then the finitely meet-irreducible theories form a basis of $\mathcal T_\vdash$.


%\noindent {\bf Lemma} Consider any consequence relation with a countable axiomatization such that the closure system T of its theories is a frame (i.e.\ satisfies the corresponding infinite distributive law) and the intersection of two finitely generated theories is finitely generated. Then the finitely meet-irreducible theories form a basis of T.

%
\begin{thebibliography}{10}

\bibitem{Bilkova-Cintula-Lavicka:InfinitaryWOLLIC}
{\scshape Marta B{\'{i}}lkov{\'{a}}, Petr Cintula, Tom\' a\v s L{\'{a}}vi{\v{c}}ka},
{\itshape Lindenbaum and pair extension lemma in infinitary logics},
{\bfseries\itshape WoLLIC 2018},
(Moss, de~Queiroz, Martinez, editors),
Springer,
2018,
pp.~134--144.

\bibitem{CN:TheBook}
{\scshape Petr Cintula, Carles Noguera},
{\bfseries\itshape  Logic and Implication: An Introduction to the General Algebraic Study of Non-classical Logics},
Springer,
2021.

\bibitem{Goldblatt93}
{\scshape Robert Goldblatt},
{\bfseries\itshape Mathematics of Modality},
CSLI Publications Stanford University,
1993.
 
\bibitem{Segerberg94}
{\scshape Krister Segerberg},
{\itshape A model existence theorem in infinitary propositional modal logic},
{\bfseries\itshape Journal of Philosophical Logic},
vol.~23 (1994), pp.~337--367.
 

\bibitem{Sundholm77}
{\scshape G\" oran Sundholm},
{\itshape A completeness proof for an infinitary tense-logic},
{\bfseries\itshape Theoria},
vol.~43 (1977), pp.~47--51.




%%% INSERT YOUR BIBLIOGRAPHIC ENTRIES HERE;
%%% SEE (4) BELOW FOR PROPER FORMAT.
%%% EACH ENTRY MUST BEGIN WITH \bibitem{citation key}
%%%
%%% IF THERE ARE NO ENTRIES
%%% DELETE THE LINE ABOVE (\begin{thebibliography}{20})
%%% AND THE LINE BELOW (\end{thebibliography})
%
\end{thebibliography}


\vspace*{-0.5\baselineskip}
% this space adjustment is usually necessary after a bibliography

\end{document}

