%% FIRST RENAME THIS FILE <yoursurname>.tex. 
%% BEFORE COMPLETING THIS TEMPLATE, SEE THE "READ ME" SECTION 
%% BELOW FOR INSTRUCTIONS. 
%% TO PROCESS THIS FILE YOU WILL NEED TO DOWNLOAD asl.cls from 
%% http://aslonline.org/abstractresources.html. 


\documentclass[bsl,meeting]{asl}

\AbstractsOn

\pagestyle{plain}

\def\urladdr#1{\endgraf\noindent{\it URL Address}: {\tt #1}.}


\newcommand{\NP}{}
%\usepackage{verbatim}

\begin{document}
\thispagestyle{empty}

%% BEGIN INSERTING YOUR ABSTRACT DIRECTLY BELOW; 
%% SEE INSTRUCTIONS (1), (2), (3), and (4) FOR PROPER FORMATS

\NP  
\absauth{Djamel Eddine Amir, and Mathieu Hoyrup}
\meettitle{Computability of finite simpilicial complexes and homology}
\affil{Université de Lorraine, CNRS, Inria, LORIA, F-54000 Nancy,
	France}
\meetemail{djamel-eddine.amir@loria.fr}
\meetemail{mathieu.hoyrup@inria.fr}
%% INSERT TEXT OF ABSTRACT DIRECTLY BELOW
%\begin{abstract}

	The topological properties of a space have a strong impact on its
	computability properties. The notion of computable type is an interesting
	example of this fact. A space has computable type if every effectively
	compact copy of it is computable. Many spaces have this property,
	such as spheres and closed manifolds \cite{cite1,cite2}.
	A similar notion is defined for pairs with computable type. 
	
	We proved recently a characterization of simplicial pairs with computable
	type (see \cite{cite3}). In particular, we proved that a simplicial
	cone pair has computable type iff it has the surjection property.
	Namely, a simplicial pair~$(Cone(X),X)$ has computable type iff
	every continuous function~$f:Cone(X)\rightarrow Cone(X)$ which is
	the identity in~$X$ is a surjection.
	
	It raises a purely topological question: when does a pair has the
	surjection property? We prove connections between the surjection property
	and homology: for instance, the cone of a graph has the surjection property
	iff every point of its base is in a cycle. We try to generalize this
	to some other simplicial pairs and we explain an open question whose
	positive answer gives a full characterization of simpilicial cone
	pairs which have the surjection property using the relative homology
	of their base pairs.
%\end{abstract}


\begin{thebibliography}{10}
\bibitem{cite1}
{\scshape Joseph S. Miller},
{\itshape Effectiveness for Embedded Spheres and Balls},
{\bfseries\itshape Electronic Notes in Theoretical Computer Science},
vol.~66 (2002), pp.~127--138.


\bibitem{cite2}
{\scshape Zvonko Iljazovi{\'c} and Igor Su{\v s}i{\' c}},
{\itshape Semicomputable manifolds in computable topological spaces},
{\bfseries\itshape Journal of Complexity},
vol.~45 (2018), pp.~83--114.


\bibitem{cite3}
{\scshape Djamel Eddine Amir and Mathieu Hoyrup},
{\itshape Computability of finite simplicial complexes},
 (2022), arxiv:2202.04945.

%% INSERT YOUR BIBLIOGRAPHIC ENTRIES HERE; 
%% SEE (4) BELOW FOR PROPER FORMAT.
%% EACH ENTRY MUST BEGIN WITH \bibitem{citation key}
%%
%% IF THERE ARE NO ENTRIES  
%% DELETE THE LINE ABOVE (\begin{thebibliography}{20}) 
%% AND THE LINE BELOW (\end{thebibliography})

\end{thebibliography}


\vspace*{-0.5\baselineskip}
% this space adjustment is usually necessary after a bibliography

\end{document}


%% READ ME
%% READ ME
%% READ ME

INSTRUCTIONS FOR SUPPLYING INFORMATION IN THE CORRECT FORMAT: 

1. Author names are listed as First Last, First Last, and First Last.

\absauth{FirstName1 LastName1, FirstName2 LastName2, and FirstName3 LastName3}


2. Titles of abstracts have ONLY the first letter capitalized,
except for Proper Nouns.

\meettitle{Title of abstract with initial capital letter only, except for
Proper Nouns} 


3. Affiliations and email addresses for authors of abstracts are
  listed separately.

% First author's affiliation
\affil{Department, University, Street Address, Country}
\meetemail{First author's email}
%%% NOTE: email required for at least one author
\urladdr{OPTIONAL}
%
% Second author's affiliation
\affil{Department, University, Street Address, Country}
\meetemail{Second author's email}
\urladdr{OPTIONAL}
%
% Third author's affiliation
\affil{Department, University, Street Address, Country}
% Second author's email
\meetemail{Third author's email}
\urladdr{OPTIONAL}


4. Bibliographic Entries

%%%% IF references are submitted with abstract,
%%%% please use the following formats

%%% For a Journal article
\bibitem{cite1}
{\scshape Author's Name},
{\itshape Title of article},
{\bfseries\itshape Journal name spelled out, no abbreviations},
vol.~XX (XXXX), no.~X, pp.~XXX--XXX.

%%% For a Journal article by the same authors as above,
%%% i.e., authors in cite1 are the same for cite2
\bibitem{cite2}
\bysame
{\itshape Title of article},
{\bfseries\itshape Journal},
vol.~XX (XXXX), no.~X, pp.~XX--XXX.

%%% For a book
\bibitem{cite3}
{\scshape Author's Name},
{\bfseries\itshape Title of book},
Name of series,
Publisher,
Year.

%%% For an article in proceedings
\bibitem{cite4}
{\scshape Author's Name},
{\itshape Title of article},
{\bfseries\itshape Name of proceedings}
(Address of meeting),
(First Last and First2 Last2, editors),
vol.~X,
Publisher,
Year,
pp.~X--XX.

%%% For an article in a collection
\bibitem{cite5}
{\scshape Author's Name},
{\itshape Title of article},
{\bfseries\itshape Book title}
(First Last and First2 Last2, editors),
Publisher,
Publisher's address,
Year,
pp.~X--XX.

%%% An edited book
\bibitem{cite6}
Author's name, editor. % No special font used here
{\bfseries\itshape Title of book},
Publisher,
Publisher's address,
Year.

