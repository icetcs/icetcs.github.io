%% FIRST RENAME THIS FILE <yoursurname>.tex. 
%% BEFORE COMPLETING THIS TEMPLATE, SEE THE "READ ME" SECTION 
%% BELOW FOR INSTRUCTIONS. 
%% TO PROCESS THIS FILE YOU WILL NEED TO DOWNLOAD asl.cls from 
%% http://aslonline.org/abstractresources.html. 


\documentclass[bsl,meeting]{asl}

\AbstractsOn

\pagestyle{plain}

\def\urladdr#1{\endgraf\noindent{\it URL Address}: {\tt #1}.}


\newcommand{\NP}{}
%\usepackage{verbatim}

\begin{document}
\thispagestyle{empty}

%% BEGIN INSERTING YOUR ABSTRACT DIRECTLY BELOW; 
%% SEE INSTRUCTIONS (1), (2), (3), and (4) FOR PROPER FORMATS

\NP  
\absauth{Pavel Arazim}
\meettitle{The limits of expressing logic}
\affil{Department of logic, Institute of Philosophy of the Czech Academy of Sciences, Jilska 1, Praha, Czech Republic}
\meetemail{arazim@flu.cas.cz}

%% INSERT TEXT OF ABSTRACT DIRECTLY BELOW
In his Tractatus, Wittgenstein dedicates some of the most fascinating, yet also most enigmatic passages to the sphere of the mystical. One of the characteristics of this sphere is supposed to be its ineffability. Any attempts to describe it force us to maim the expressive powers of the language we use. Surpsingly enough, Wittgenstein treats logic in a very similar way in Tractatus. Logic, then, can only be shown, not expressed. Or, to be more precise, logic can only show itself. Besides being ineffable, the mystical is also supposed to be fundamental, in fact much more important than what lies outside it. Therefore, logic also deserves this honourable status, according to Wittgenstein. Nevertheless, logicians today purport to be making explicit all kinds of logical laws which hold in variegated areas, which causes the unprecedented plurality of logics. On the other hand, it is not clear what the import of all this intellectual work is. Is there a lesson to be learned from Wittgenstein for the contemporary philosophy of logic? In order to access this possible lesson, we have to pay attention not only to early Wittgenstein but also to his later development where the notion of game and language game became prominent. I will show that taking seriously Wittgenstein´s motivation - which originates in his discussions with Moritz Schlick and his conception of games - to treat our linguistic activities as games, which are partly playful and unserious, shows us the limits of formal logical systems. They are language games themselves but do not understand themselves properly which causes them to be unsatisfying and turns the plurality of logics into a curse rather than into a blessing, getting us close to the positions of logical nihilists rather than to those of logical pluralists.    
\begin{thebibliography}{10}



\bibitem[Russell(2018)] {russell} Russell, G. (2018). \emph{Logical nihilism: Could there be no logic?}. Philosophical Issues, 28, 1
\bibitem[Schlick(1928)] {schlick} Schlick, M. (1928). \emph{Vom Sinne des Lebens}. Appears in Die Wiener Zeit: Aufsätze, Beiträge und Rezensionen 1926-1936 (J. Friedl & H. Rutte, Eds.). Wien: Springer, 2008. 

\bibitem[Wittgenstein(1921)] {wittgensteint} Wittgenstein, L. (1921). \emph{Logisch-Philosophische Abhandlung}. Annalen der Naturphilosophische, XIV (3/4)
\bibitem[Wittgenstein(1953)] {wittgenstein} Wittgenstein, L. (1953). \emph{Philosophische Untersuchungen}. Oxford: Blackwell

%% INSERT YOUR BIBLIOGRAPHIC ENTRIES HERE; 
%% SEE (4) BELOW FOR PROPER FORMAT.
%% EACH ENTRY MUST BEGIN WITH \bibitem{citation key}
%%
%% IF THERE ARE NO ENTRIES  
%% DELETE THE LINE ABOVE (\begin{thebibliography}{20}) 
%% AND THE LINE BELOW (\end{thebibliography})

\end{thebibliography}


\vspace*{-0.5\baselineskip}
% this space adjustment is usually necessary after a bibliography

\end{document}


%% READ ME
%% READ ME
%% READ ME

INSTRUCTIONS FOR SUPPLYING INFORMATION IN THE CORRECT FORMAT: 

1. Author names are listed as First Last, First Last, and First Last.

\absauth{Pavel Arazim}


2. Titles of abstracts have ONLY the first letter capitalized,
except for Proper Nouns.

\meettitle{Title of abstract with initial capital letter only, except for
Proper Nouns} 


3. Affiliations and email addresses for authors of abstracts are
  listed separately.

% First author's affiliation
\affil{Department, University, Street Address, Country}
\meetemail{First author's email}
%%% NOTE: email required for at least one author
\urladdr{OPTIONAL}
%
% Second author's affiliation
\affil{Department, University, Street Address, Country}
\meetemail{Second author's email}
\urladdr{OPTIONAL}
%
% Third author's affiliation
\affil{Department, University, Street Address, Country}
% Second author's email
\meetemail{Third author's email}
\urladdr{OPTIONAL}


4. Bibliographic Entries

%%%% IF references are submitted with abstract,
%%%% please use the following formats

%%% For a Journal article
\bibitem{cite1}
{\scshape Author's Name},
{\itshape Title of article},
{\bfseries\itshape Journal name spelled out, no abbreviations},
vol.~XX (XXXX), no.~X, pp.~XXX--XXX.

%%% For a Journal article by the same authors as above,
%%% i.e., authors in cite1 are the same for cite2
\bibitem{cite2}
\bysame
{\itshape Title of article},
{\bfseries\itshape Journal},
vol.~XX (XXXX), no.~X, pp.~XX--XXX.

%%% For a book
\bibitem{cite3}
{\scshape Author's Name},
{\bfseries\itshape Title of book},
Name of series,
Publisher,
Year.

%%% For an article in proceedings
\bibitem{cite4}
{\scshape Author's Name},
{\itshape Title of article},
{\bfseries\itshape Name of proceedings}
(Address of meeting),
(First Last and First2 Last2, editors),
vol.~X,
Publisher,
Year,
pp.~X--XX.

%%% For an article in a collection
\bibitem{cite5}
{\scshape Author's Name},
{\itshape Title of article},
{\bfseries\itshape Book title}
(First Last and First2 Last2, editors),
Publisher,
Publisher's address,
Year,
pp.~X--XX.

%%% An edited book
\bibitem{cite6}
Author's name, editor. % No special font used here
{\bfseries\itshape Title of book},
Publisher,
Publisher's address,
Year.

