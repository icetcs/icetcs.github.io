%% FIRST RENAME THIS FILE <yoursurname>.tex.
%% BEFORE COMPLETING THIS TEMPLATE, SEE THE "READ ME" SECTION
%% BELOW FOR INSTRUCTIONS.
%% TO PROCESS THIS FILE YOU WILL NEED TO DOWNLOAD asl.cls from
%% http://aslonline.org/abstractresources.html.


\documentclass[bsl,meeting]{asl}
\usepackage{hyperref}
\usepackage{enumerate}
\AbstractsOn

\pagestyle{plain}

\def\urladdr#1{\endgraf\noindent{\it URL Address}: {\tt #1}.}


\newcommand{\NP}{}
\newcommand{\M}{\ensuremath{\mathcal{M}}} 
\newcommand{\lomegaone}{\ensuremath{\mathcal{L}_{\omega_1,\omega}}}
\newcommand{\LS}{LS}
\newcommand{\cL}{\mathcal{L}}
\def\K{\mbox{\boldmath $K$}}
\def\subm{\prec_{\mbox{\scriptsize \boldmath $K$}}}


\newcommand{\Card}{\mathrm{Card}}
\newcommand{\Loo}{\mathcal{L}_{\omega_1,\omega}}
\newcommand{\N}{\ensuremath{\mathcal{N}}}
\renewcommand{\aa}{\aleph_{\alpha}} %aleph alpha 
\newcommand{\aapo}{\aleph_{\alpha+1}}%aleph alpha plus one
\newcommand{\Ord}{\mathrm{Ord}}
\newcommand{\ZFC}{{\rm{ZFC}}}
\newcommand{\BPFA}{{\rm{BPFA}}}
\newcommand{\CH}{{\rm{CH}}}
 \newcommand{\GCH}{{\rm{GCH}}}


%\usepackage{verbatim}

\begin{document}
\thispagestyle{empty}

%% BEGIN INSERTING YOUR ABSTRACT DIRECTLY BELOW;
%% SEE INSTRUCTIONS (1), (2), (3), and (4) FOR PROPER FORMATS

\NP
\absauth{Ioannis Souldatos}
\meettitle{Characterizing Cardinals by $\mathcal{L}_{\omega_1,\omega}$-sentences in an Absolute Way}
\affil{Department of Mathematics,
Aristotle University of Thessaloniki, Thessaloniki 54124, Greece}
\meetemail{souldatos@math.auth.gr}

%% INSERT TEXT OF ABSTRACT DIRECTLY BELOW

 In \cite{HjorthsKnightsModel},  Hjorth proved  that for every countable ordinal  $\alpha$, there exists a complete $\Loo$-sentence $\phi_\alpha$ that has models of all cardinalities less than or equal to $\aa$, but no models of cardinality  $\aapo$.  
 Unfortunately, his solution yields not one  $\Loo$-sentence $\phi_\alpha$, but a  set of $\Loo$-sentences, one of which is guaranteed to work. 

The following is new: It is independent of the axioms of $\ZFC$ which of the Hjorth sentences works. More specifically, we isolate a diagonalization principle for functions from $\omega_1$ to $\omega_1$ which is a consequence of the \emph{Bounded Proper Forcing Axiom} ($\BPFA$) and then we use this principle to  prove that Hjorth's solution to characterizing $\aleph_2$ in models of $\BPFA$ is different than in models of $\CH$. 

This raises the question whether Hjorth's result can be proved in an \emph{absolute way} and what exactly this means, which we will discuss at the end of the talk. 


This is joint work with Philipp~L\"ucke. 
 
\textsc{References}
\begin{thebibliography}{10}
\bibitem{HjorthsKnightsModel}
\newblock Greg {Hjorth}.
\newblock {Knight's model, its automorphism group, and characterizing the
  uncountable cardinals.}
\newblock {\em {J. Math. Log.}}, 2(1):113--144, 2002.

\bibitem{LuckeSouldatos}
\newblock Philipp~L\"ucke, Ioannis~Souldatos,
\newblock A lower bound for the hanf number for joint embedding.
\newblock \url{https://arxiv.org/abs/2109.07310}
%% INSERT YOUR BIBLIOGRAPHIC ENTRIES HERE;
%% SEE (4) BELOW FOR PROPER FORMAT.
%% EACH ENTRY MUST BEGIN WITH \bibitem{citation key}
%%
%% IF THERE ARE NO ENTRIES
%% DELETE THE LINE ABOVE (\begin{thebibliography}{20})
%% AND THE LINE BELOW (\end{thebibliography})

\end{thebibliography}



\vspace*{-0.5\baselineskip}
% this space adjustment is usually necessary after a bibliography

\end{document}


%% READ ME
%% READ ME
%% READ ME

INSTRUCTIONS FOR SUPPLYING INFORMATION IN THE CORRECT FORMAT:

1. Author names are listed as First Last, First Last, and First Last.

\absauth{FirstName1 LastName1, FirstName2 LastName2, and FirstName3 LastName3}


2. Titles of abstracts have ONLY the first letter capitalized,
except for Proper Nouns.

\meettitle{Title of abstract with initial capital letter only, except for
Proper Nouns}


3. Affiliations and email addresses for authors of abstracts are
  listed separately.

% First author's affiliation
\affil{Department, University, Street Address, Country}
\meetemail{First author's email}
%%% NOTE: email required for at least one author
\urladdr{OPTIONAL}
%
% Second author's affiliation
\affil{Department, University, Street Address, Country}
\meetemail{Second author's email}
\urladdr{OPTIONAL}
%
% Third author's affiliation
\affil{Department, University, Street Address, Country}
% Second author's email
\meetemail{Third author's email}
\urladdr{OPTIONAL}


4. Bibliographic Entries

%%%% IF references are submitted with abstract,
%%%% please use the following formats

%%% For a Journal article
\bibitem{cite1}
{\scshape Author's Name},
{\itshape Title of article},
{\bfseries\itshape Journal name spelled out, no abbreviations},
vol.~XX (XXXX), no.~X, pp.~XXX--XXX.

%%% For a Journal article by the same authors as above,
%%% i.e., authors in cite1 are the same for cite2
\bibitem{cite2}
\bysame
{\itshape Title of article},
{\bfseries\itshape Journal},
vol.~XX (XXXX), no.~X, pp.~XX--XXX.

%%% For a book
\bibitem{cite3}
{\scshape Author's Name},
{\bfseries\itshape Title of book},
Name of series,
Publisher,
Year.

%%% For an article in proceedings
\bibitem{cite4}
{\scshape Author's Name},
{\itshape Title of article},
{\bfseries\itshape Name of proceedings}
(Address of meeting),
(First Last and First2 Last2, editors),
vol.~X,
Publisher,
Year,
pp.~X--XX.

%%% For an article in a collection
\bibitem{cite5}
{\scshape Author's Name},
{\itshape Title of article},
{\bfseries\itshape Book title}
(First Last and First2 Last2, editors),
Publisher,
Publisher's address,
Year,
pp.~X--XX.

%%% An edited book
\bibitem{cite6}
Author's name, editor. % No special font used here
{\bfseries\itshape Title of book},
Publisher,
Publisher's address,
Year.
