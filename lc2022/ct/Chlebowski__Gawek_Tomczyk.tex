%% FIRST RENAME THIS FILE <yoursurname>.tex. 
%% BEFORE COMPLETING THIS TEMPLATE, SEE THE "READ ME" SECTION 
%% BELOW FOR INSTRUCTIONS. 
%% TO PROCESS THIS FILE YOU WILL NEED TO DOWNLOAD asl.cls from 
%% http://aslonline.org/abstractresources.html. 


\documentclass[bsl,meeting]{asl}

\AbstractsOn

\pagestyle{plain}

\def\urladdr#1{\endgraf\noindent{\it URL Address}: {\tt #1}.}


\newcommand{\NP}{}
%\usepackage{verbatim}

\begin{document}
\thispagestyle{empty}

%% BEGIN INSERTING YOUR ABSTRACT DIRECTLY BELOW; 
%% SEE INSTRUCTIONS (1), (2), (3), and (4) FOR PROPER FORMATS

\NP  
\absauth{Szymon Chlebowski$^1$, Marta Gawek$^2$, Agata Tomczyk$^1$}
\meettitle{Propositional identity. From semantics to proof theory}
\affil{$^1$ Adam Mickiewicz University in Poznań\\ 
$^2$ University of Lorraine, CNRS, LORIA}
\meetemail{szymon.chlebowski@amu.edu.pl}

%% INSERT TEXT OF ABSTRACT DIRECTLY BELOW

The aim of the talk is to discuss differences between formal and philosophical interpretations of propositional identity across classical and intuitionistic logic. In classical logic propositional identity is closely related to the abolition of the Fregean Axiom \cite{suszko1975}, according to which sentences are names of truth values. In reference to the ideas from Wittgenstein's  \emph{Tractatus} we may, contrary to Frege, assume that propositions denote situations. In such case, due to Quine \emph{dictum}, we need to introduce criteria of identity of situations, as it was done for example by Suszko \cite{suszko1968}. This picture changes when we move to the constructive environment. Here, propositions can be thought of as types of their own proofs and propositional identity can be interpreted as expressing the notion of identity of proofs \cite{isci}.



\begin{thebibliography}{10}

\bibitem{isci}
{\scshape Szymon Chlebowski \& Dorota Leszczy{\'n}ska-Jasion},
{\itshape An Investigation into Intuitionistic Logic with Identity},
{\bfseries\itshape Bulletin of the Section of Logic},
vol.~48 (2019), no.~4, pp.~259--283.

\bibitem{suszko1975}
{\scshape Roman Suszko},
{\itshape Abolition of the Fregean Axiom},
{\bfseries\itshape Lecture Notes in Mathematics},
vol.~453 (1975), pp.~169--239.

\bibitem{suszko1968}
{\scshape Roman Suszko},
{\itshape Ontology in the Tractatus of L. Wittgenstein},
{\bfseries\itshape Notre Dame Journal of Formal Logic},
vol.~9 (1968), no.~1 pp.~7--33.

%% INSERT YOUR BIBLIOGRAPHIC ENTRIES HERE; 
%% SEE (4) BELOW FOR PROPER FORMAT.
%% EACH ENTRY MUST BEGIN WITH \bibitem{citation key}
%%
%% IF THERE ARE NO ENTRIES  
%% DELETE THE LINE ABOVE (\begin{thebibliography}{20}) 
%% AND THE LINE BELOW (\end{thebibliography})

\end{thebibliography}


\vspace*{-0.5\baselineskip}
% this space adjustment is usually necessary after a bibliography

\end{document}


%% READ ME
%% READ ME
%% READ ME

INSTRUCTIONS FOR SUPPLYING INFORMATION IN THE CORRECT FORMAT: 

1. Author names are listed as First Last, First Last, and First Last.

\absauth{FirstName1 LastName1, FirstName2 LastName2, and FirstName3 LastName3}


2. Titles of abstracts have ONLY the first letter capitalized,
except for Proper Nouns.

\meettitle{Title of abstract with initial capital letter only, except for
Proper Nouns} 


3. Affiliations and email addresses for authors of abstracts are
  listed separately.

% First author's affiliation
\affil{Department, University, Street Address, Country}
\meetemail{First author's email}
%%% NOTE: email required for at least one author
\urladdr{OPTIONAL}
%
% Second author's affiliation
\affil{Department, University, Street Address, Country}
\meetemail{Second author's email}
\urladdr{OPTIONAL}
%
% Third author's affiliation
\affil{Department, University, Street Address, Country}
% Second author's email
\meetemail{Third author's email}
\urladdr{OPTIONAL}


4. Bibliographic Entries

%%%% IF references are submitted with abstract,
%%%% please use the following formats

%%% For a Journal article
\bibitem{cite1}
{\scshape Author's Name},
{\itshape Title of article},
{\bfseries\itshape Journal name spelled out, no abbreviations},
vol.~XX (XXXX), no.~X, pp.~XXX--XXX.

%%% For a Journal article by the same authors as above,
%%% i.e., authors in cite1 are the same for cite2
\bibitem{cite2}
\bysame
{\itshape Title of article},
{\bfseries\itshape Journal},
vol.~XX (XXXX), no.~X, pp.~XX--XXX.

%%% For a book
\bibitem{cite3}
{\scshape Author's Name},
{\bfseries\itshape Title of book},
Name of series,
Publisher,
Year.

%%% For an article in proceedings
\bibitem{cite4}
{\scshape Author's Name},
{\itshape Title of article},
{\bfseries\itshape Name of proceedings}
(Address of meeting),
(First Last and First2 Last2, editors),
vol.~X,
Publisher,
Year,
pp.~X--XX.

%%% For an article in a collection
\bibitem{cite5}
{\scshape Author's Name},
{\itshape Title of article},
{\bfseries\itshape Book title}
(First Last and First2 Last2, editors),
Publisher,
Publisher's address,
Year,
pp.~X--XX.

%%% An edited book
\bibitem{cite6}
Author's name, editor. % No special font used here
{\bfseries\itshape Title of book},
Publisher,
Publisher's address,
Year.

