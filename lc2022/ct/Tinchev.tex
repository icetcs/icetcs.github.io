%% FIRST RENAME THIS FILE <yoursurname>.tex. 
%% BEFORE COMPLETING THIS TEMPLATE, SEE THE "READ ME" SECTION 
%% BELOW FOR INSTRUCTIONS. 
%% TO PROCESS THIS FILE YOU WILL NEED TO DOWNLOAD asl.cls from 
%% http://aslonline.org/abstractresources.html. 


\documentclass[bsl,meeting]{asl}

\AbstractsOn

\pagestyle{plain}

\def\urladdr#1{\endgraf\noindent{\it URL Address}: {\tt #1}.}


\newcommand{\NP}{}
%\usepackage{verbatim}

\begin{document}
\thispagestyle{empty}

%% BEGIN INSERTING YOUR ABSTRACT DIRECTLY BELOW; 
%% SEE INSTRUCTIONS (1), (2), (3), and (4) FOR PROPER FORMATS

\NP  
\absauth{Tinko Tinchev}
\meettitle{Decidability of modal definability problem on  the class of quasilinear frames}
\affil{Faculty of Mathematics and Informatics, Sofia University St. Kliment Ohridski, Blvd. James Bourchier 5, Sofia 1164, Bulgaria}
\meetemail{tinko@fmi.uni-sofia.bg}

%% INSERT TEXT OF ABSTRACT DIRECTLY BELOW

Let $\mathcal{K}$ be the class of all quasilinear Kripke frames, i.e. the accesibility relation is reflexive, transitive and total (linear). Denote by $\mathcal{K}^{\rm fin}$ and by  $\mathcal{K}^{\leq\omega}$ the classes of frames from  $\mathcal{K}^{\rm fin}$ having finetely many, resp. at most countably many, clusters. 
 Our goal is to study the modal definability problem on these three classes.  Remind that a sentence $A$ from the first-order language with equality and one binary predicate symbol is modally definable with respect to some class of frames if there is a modal formula $\varphi$ from the classical propositional modal language  such that $A$ and $\varphi$ are valid in the same frames from the class. Modal definability problem ask whether there exists an algorithm that recognizes modally definable sentences. 

 In this talk we make a heavy use of decidability of Rabins's $S2S$ theory to prove the following.
\begin{theorem}
The modal definability problem is decidable with respect to the classes $\mathcal{K}^{\rm fin}$ and $\mathcal{K}^{\leq\omega}$. 
\end{theorem}

\begin{theorem}
For any  sentence $A$, $A$ is modally definable with respect to $\mathcal{K}$ if and only if $A$ is modally definable with respect to $\mathcal{K}^{\leq\omega}$. 
Therefore, the modal definability problem with respect to 
$\mathcal{K}$  is decidable.
\end{theorem}
 
\begin{theorem}
\begin{enumerate}
 1. There is an algorithm which for any sentence $A$ gives a modal definition of $A$ on $\mathcal{K}^{\rm fin}$, if such modal formula exists.

2. There is an algorithm which for any sentence $A$ gives a modal definition of $A$ on $\mathcal{K}$, if such modal formula exists.
\end{enumerate}
\end{theorem}

%\begin{thebibliography}{10}

%% INSERT YOUR BIBLIOGRAPHIC ENTRIES HERE; 
%% SEE (4) BELOW FOR PROPER FORMAT.
%% EACH ENTRY MUST BEGIN WITH \bibitem{citation key}
%%
%% IF THERE ARE NO ENTRIES  
%% DELETE THE LINE ABOVE (\begin{thebibliography}{20}) 
%% AND THE LINE BELOW (\end{thebibliography})

%\end{thebibliography}


%\vspace*{-0.5\baselineskip}
% this space adjustment is usually necessary after a bibliography

\end{document}


%% READ ME
%% READ ME
%% READ ME

INSTRUCTIONS FOR SUPPLYING INFORMATION IN THE CORRECT FORMAT: 

1. Author names are listed as First Last, First Last, and First Last.

\absauth{FirstName1 LastName1, FirstName2 LastName2, and FirstName3 LastName3}


2. Titles of abstracts have ONLY the first letter capitalized,
except for Proper Nouns.

\meettitle{Title of abstract with initial capital letter only, except for
Proper Nouns} 


3. Affiliations and email addresses for authors of abstracts are
  listed separately.

% First author's affiliation
\affil{Department, University, Street Address, Country}
\meetemail{First author's email}
%%% NOTE: email required for at least one author
\urladdr{OPTIONAL}
%
% Second author's affiliation
\affil{Department, University, Street Address, Country}
\meetemail{Second author's email}
\urladdr{OPTIONAL}
%
% Third author's affiliation
\affil{Department, University, Street Address, Country}
% Second author's email
\meetemail{Third author's email}
\urladdr{OPTIONAL}


4. Bibliographic Entries

%%%% IF references are submitted with abstract,
%%%% please use the following formats

%%% For a Journal article
\bibitem{cite1}
{\scshape Author's Name},
{\itshape Title of article},
{\bfseries\itshape Journal name spelled out, no abbreviations},
vol.~XX (XXXX), no.~X, pp.~XXX--XXX.

%%% For a Journal article by the same authors as above,
%%% i.e., authors in cite1 are the same for cite2
\bibitem{cite2}
\bysame
{\itshape Title of article},
{\bfseries\itshape Journal},
vol.~XX (XXXX), no.~X, pp.~XX--XXX.

%%% For a book
\bibitem{cite3}
{\scshape Author's Name},
{\bfseries\itshape Title of book},
Name of series,
Publisher,
Year.

%%% For an article in proceedings
\bibitem{cite4}
{\scshape Author's Name},
{\itshape Title of article},
{\bfseries\itshape Name of proceedings}
(Address of meeting),
(First Last and First2 Last2, editors),
vol.~X,
Publisher,
Year,
pp.~X--XX.

%%% For an article in a collection
\bibitem{cite5}
{\scshape Author's Name},
{\itshape Title of article},
{\bfseries\itshape Book title}
(First Last and First2 Last2, editors),
Publisher,
Publisher's address,
Year,
pp.~X--XX.

%%% An edited book
\bibitem{cite6}
Author's name, editor. % No special font used here
{\bfseries\itshape Title of book},
Publisher,
Publisher's address,
Year.

