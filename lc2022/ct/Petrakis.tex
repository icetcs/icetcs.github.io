%% FIRST RENAME THIS FILE <yoursurname>.tex. 
%% BEFORE COMPLETING THIS TEMPLATE, SEE THE "READ ME" SECTION 
%% BELOW FOR INSTRUCTIONS. 
%% TO PROCESS THIS FILE YOU WILL NEED TO DOWNLOAD asl.cls from 
%% http://aslonline.org/abstractresources.html. 


\documentclass[bsl,meeting]{asl}

\AbstractsOn

\pagestyle{plain}

\def\urladdr#1{\endgraf\noindent{\it URL Address}: {\tt #1}.}





% 
% 
% 
% \usepackage{amsfonts,qtree,stmaryrd,textcomp}
% \usepackage{pifont,amssymb,amsmath,amsthm,tabularx,graphicx}
% \usepackage{tree-dvips,pictexwd,dcpic}
% \usepackage{bbm}
% \usepackage{bm}
% 
% 
% \usepackage{stmaryrd}
% 
% %\usepackage{undertilde}
% %\usepackage{bussproofs}
% \usepackage[T1]{fontenc}
% \usepackage[latin1]{inputenc}
% %\usepackage[ngerman]{babel}
% \usepackage[text={17cm,26cm},centering]{geometry}
% 
% 
% 
% \usepackage{etex}
% %\usepackage{a4wide}
% \usepackage{fancyhdr}
% \usepackage{graphicx}
% %\usepackage{german}
% %\usepackage[bookmarks,pdfpagelabels]{hyperref}
% %\usepackage[pdfpagelabels]{hyperref}
% %\usepackage{hyperref}
% 
% %\usepackage[latin1]{inputenc}
% %\usepackage{amsthm}
% %\usepackage{appendix}
% %\usepackage[toc,pages]{appendix} % makes a page only with "Appendices".
% \usepackage{enumitem}
% \usepackage{amsmath}
% \usepackage[bbgreekl]{mathbbol}% order sensitive; it is before amssymb
% \usepackage{amssymb}
% \usepackage{amsthm}
% \usepackage[all]{xy}
% 
% 
% 

\newcommand{\B}{\boldsymbol}
   %Bold math symbol, use as \B{a}
\newcommand{\C}[1]{\mathcal{#1}}
   %Euler Script - only caps, use as \C{A}
\newcommand{\D}[1]{\mathbb{#1}}
   %Doubles - only caps, use as \D{A}
\newcommand{\F}[1]{\mathfrak{#1}}
   %Fraktur, use as \F{a}
\newcommand{\M}[1]{\mathscr{#1}}


% 
% 
 \newcommand{\Chu}{\textnormal{\textbf{Chu}}}
% \newcommand{\CMC}{\mathrm{CMC}}
% \newcommand{\CSMC}{\mathrm{CSMC}}
% 
% \newcommand{\tro}{\rightarrowtriangle}
% 
% 
% \newcommand{\mto}{\hookrightarrow}
% 
% %\newcommand{\MOR}{\textnormal{\textbf{Mor}}}
% 
% \newcommand{\op}{\mathrm{op}}
% 
% \newcommand{\sym}{\mathrm{sym}}
% 
% 
% 
% 
% \newcommand{\Fun}{\mathrm{Fun}}
% 
% \newcommand{\cnt}{\mathrm{cnt}}
% 
% 
% 
% \newcommand{\Inf}{\textnormal{\textbf{Inf}}}
% 
% 
% \newcommand{\crTop}{\textnormal{\textbf{crTop}}}
% 
% 
% \newcommand{\fin}{\textnormal{\texttt{fin}}}
% 
% \newcommand{\Aff}{\textnormal{\textbf{Aff}}}
% 
% 
% 
% 
% 
% \newcommand{\Hom}{\mathrm{Hom}}
% 
% 
% 
% 
% \newcommand{\Sub}{\textnormal{\textbf{Sub}}}
% 
% 
% 
% \newcommand{\Groth}{\textnormal{\textbf{Groth}}}
% 
% \newcommand{\ccCat}{\textnormal{\textbf{ccCat}}}
% 
% 
% 
% \newcommand{\Ob}{\mathrm{Ob}}
% 
% 
% \newcommand{\Rel}{\textnormal{\textbf{Rel}}}   
% 
% 
% 
% \newcommand{\smallCat}{\textnormal{\textbf{smallCat}}}
% \newcommand{\finSet}{\textnormal{\textbf{finSet}}}
% \newcommand{\simplSet}{\textnormal{\textbf{simplSet}}}
% 
% \newcommand{\CCC}{\mathrm{CCC}}
% 
% 
% \newcommand{\End}{\mathrm{End}}



\newcommand{\CHU}{\textnormal{\textbf{CHU}}}


\newcommand{\Set}{\mathrm{\mathbf{Set}}}




\newcommand{\Disj}{\B \rrbracket \B \llbracket}

\newcommand{\BISH}{\mathrm{BISH}}





\newcommand{\NP}{}
%\usepackage{verbatim}

\begin{document}
\thispagestyle{empty}









%% BEGIN INSERTING YOUR ABSTRACT DIRECTLY BELOW; 
%% SEE INSTRUCTIONS (1), (2), (3), and (4) FOR PROPER FORMATS

\NP  
\absauth{Iosif Petrakis}
\meettitle{Positive negation in constructive mathematics}
\affil{Mathematisches Institut, Ludwig-Maximilians-Universit\"at M\"unchen, Theresienstrasse 39, D-80333 Munich, Germany}
\meetemail{petrakis@math.lmu.de}

%% INSERT TEXT OF ABSTRACT DIRECTLY BELOW


In standard constructive logic negation is treated as in classical logic in a negativistic and weak way. 
This is in contrast to the use of a positive and strong ``or'' and ``exists''. 
In constructive mathematics~\cite{BB85}
however, we often find a positive and strong approach to negatively defined concepts, 
like that of inequality.
This fact motivates a clear distinction between a positive and strong negation and the standard weak negation.
Bringing together older ideas of Griss and
Nelson and recent work of Shulman~\cite{Sh21} and ours~\cite{Pe21}, we investigate the role of a positive and 
strong negation in Bishop-style constructive mathematics $\BISH$. We define the positive negation of a formula in
$\BISH$, we determine the formulas of $\BISH$ that are used to define the equality of a Bishop set, and we define the 
canonical inequality of a Bishop set through positive negation of its given equality formula. Consequently, many 
seemingly ad hoc definitions of concepts of $\BISH$, such as the complement of a subset, the empty subset,
complemented subsets, and the $F$-complement of a closed set, are canonical definitions through 
positive negation.   




% 
% 
% If $\C C$ is a closed symmetric monoidal category (CSMC) and $\gamma$ is an object of $\C C$, 
% the Chu category $\Chu(\C C, \gamma)$ over 
% $\C C$ and $\gamma$ was defined by Chu 
% in~\cite{Ba79}, as a $*$-autonomous category generated from $\C C$. 
% In~\cite{Bi67} Bishop introduced the thin category 
% $\C P^{\neq}(X)$ 
% of complemented subsets of a set $X$, in order to overcome 
% the problems generated by the use of negation in constructive measure theory. In~\cite{Sh18} Shulman mentions
% that Bishop's complemented subsets correspond roughly to the Chu construction. Here we explain
% this correspondence by showing that there is a Chu representation (a full embedding) of $\C P^{\neq}(X)$ into 
% $\Chu(\Set, X \times X)$. A Chu representation of the category of Bishop spaces into $\Chu(\Set, \D R)$ is shown, as 
% the constructive analogue to the standard Chu representation of the category of topological spaces into $\Chu(\Set, 2)$.
% In order to represent the category of predicates (with objects pairs $(X, A)$, where $A$ is a subset of $X$), 
% and the category of complemented predicates (with objects pairs $(X, A)$, where $A$ is a complemented subset of $X$),
% we generalise the Chu construction by defining the Chu category over a CSMC $\C C$ and an endofunctor 
% $\Gamma \colon \C C \to \C C$.

\begin{thebibliography}{10}

%% INSERT YOUR BIBLIOGRAPHIC ENTRIES HERE; 
%% SEE (4) BELOW FOR PROPER FORMAT.
%% EACH ENTRY MUST BEGIN WITH \bibitem{citation key}
%%
%% IF THERE ARE NO ENTRIES  
%% DELETE THE LINE ABOVE (\begin{thebibliography}{20}) 
%% AND THE LINE BELOW (\end{thebibliography})

% \bibitem{Ba79}
% {\scshape M.~Barr},
% {\bfseries\itshape  $^*$-Autonomous Categories},
% LNM 752, Springer-Verlag, 1979.
% 


\bibitem{BB85}
{\scshape E.~Bishop, D.~Bridges},
{\bfseries\itshape Constructive Analysis},
Springer-Verlag, 1985.

\bibitem{Pe21}
{\scshape I.~Petrakis},
{\itshape Families of Sets in Bishop Set Theory},
{\bfseries\itshape arXiv:2109.04183v1 }
(2021).

\bibitem{Sh21}
{\scshape M.~Shulman},
{\itshape Affine Logic for Constructive Mathematics},
{\bfseries\itshape arXiv:1805.07518v2}
(2021).

\end{thebibliography}


\vspace*{-0.5\baselineskip}
% this space adjustment is usually necessary after a bibliography

\end{document}


%% READ ME
%% READ ME
%% READ ME

INSTRUCTIONS FOR SUPPLYING INFORMATION IN THE CORRECT FORMAT: 

1. Author names are listed as First Last, First Last, and First Last.

\absauth{FirstName1 LastName1, FirstName2 LastName2, and FirstName3 LastName3}


2. Titles of abstracts have ONLY the first letter capitalized,
except for Proper Nouns.

\meettitle{Title of abstract with initial capital letter only, except for
Proper Nouns} 


3. Affiliations and email addresses for authors of abstracts are
  listed separately.

% First author's affiliation
\affil{Department, University, Street Address, Country}
\meetemail{First author's email}
%%% NOTE: email required for at least one author
\urladdr{OPTIONAL}
%
% Second author's affiliation
\affil{Department, University, Street Address, Country}
\meetemail{Second author's email}
\urladdr{OPTIONAL}
%
% Third author's affiliation
\affil{Department, University, Street Address, Country}
% Second author's email
\meetemail{Third author's email}
\urladdr{OPTIONAL}


4. Bibliographic Entries

%%%% IF references are submitted with abstract,
%%%% please use the following formats

%%% For a Journal article
\bibitem{cite1}
{\scshape Author's Name},
{\itshape Title of article},
{\bfseries\itshape Journal name spelled out, no abbreviations},
vol.~XX (XXXX), no.~X, pp.~XXX--XXX.

%%% For a Journal article by the same authors as above,
%%% i.e., authors in cite1 are the same for cite2
\bibitem{cite2}
\bysame
{\itshape Title of article},
{\bfseries\itshape Journal},
vol.~XX (XXXX), no.~X, pp.~XX--XXX.

%%% For a book
\bibitem{cite3}
{\scshape Author's Name},
{\bfseries\itshape Title of book},
Name of series,
Publisher,
Year.

%%% For an article in proceedings
\bibitem{cite4}
{\scshape Author's Name},
{\itshape Title of article},
{\bfseries\itshape Name of proceedings}
(Address of meeting),
(First Last and First2 Last2, editors),
vol.~X,
Publisher,
Year,
pp.~X--XX.

%%% For an article in a collection
\bibitem{cite5}
{\scshape Author's Name},
{\itshape Title of article},
{\bfseries\itshape Book title}
(First Last and First2 Last2, editors),
Publisher,
Publisher's address,
Year,
pp.~X--XX.

%%% An edited book
\bibitem{cite6}
Author's name, editor. % No special font used here
{\bfseries\itshape Title of book},
Publisher,
Publisher's address,
Year.

