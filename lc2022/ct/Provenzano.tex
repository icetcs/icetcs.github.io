%% FIRST RENAME THIS FILE <yoursurname>.tex. 
%% BEFORE COMPLETING THIS TEMPLATE, SEE THE "READ ME" SECTION 
%% BELOW FOR INSTRUCTIONS. 
%% TO PROCESS THIS FILE YOU WILL NEED TO DOWNLOAD asl.cls from 
%% http://aslonline.org/abstractresources.html. 


\documentclass[bsl,meeting]{asl}

\AbstractsOn

\pagestyle{plain}

\def\urladdr#1{\endgraf\noindent{\it URL Address}: {\tt #1}.}


\newcommand{\NP}{}
%\usepackage{verbatim}

\begin{document}
\thispagestyle{empty}

%% BEGIN INSERTING YOUR ABSTRACT DIRECTLY BELOW; 
%% SEE INSTRUCTIONS (1), (2), (3), and (4) FOR PROPER FORMATS

\NP  
\absauth{Philipp Provenzano}
\meettitle{The reverse mathematical strength of hyperations}
\affil{Department of Mathematics, ETH Zürich, Rämistrasse 101, 8092 Zürich, Switzerland}
\meetemail{PProvenzano@web.de}

%% INSERT TEXT OF ABSTRACT DIRECTLY BELOW

Hyperations have been introduced in \cite{FJ13} as a way to transfinitely iterate normal, i.e., strictly increasing continuous, functions on ordinals, refining the notion of Veblen functions. The goal of this talk is to outline a construction of hyperations for certain uniform functions on linear orders in second order arithmetic and discuss the logical strength of preservation of well-foundedness by this construction in the light of reverse mathematics. \\
For the first goal, we will employ the notion of dilators introduced by Girard and show how a sufficiently uniform categorical treatment of finite iterations can be extended to transfinite exponents. Such a construction has already appeared for the standard Veblen hierarchy in \cite{GV84}.\\
The proof-theoretic discussion builds on a framework developed in \cite{PW21}, relating transfinitely iterated syntactic reflection to semantic $\omega$-model reflection. The ordinal analysis of $\operatorname{ATR}_0$ developed there is relativized to an arbitrary normal dilator $T$, yielding an equivalence between the principles ``$\textit{the hyperation of }T\textit{ preserves well-foundedness}$'' and $\Pi^1_2\text{-}\omega\operatorname{RFN}(\Pi^1_1\text{-}\operatorname{BI}_0+~T\textit{ is a dilator})$ over the weak base theory $\operatorname{RCA}_0$.\\

The master thesis on which this talk is based has been supervised by Andreas Weiermann and Fedor Pakhomov from the Logic group at Ghent University. 

\begin{thebibliography}{10}

%% INSERT YOUR BIBLIOGRAPHIC ENTRIES HERE; 
%% SEE (4) BELOW FOR PROPER FORMAT.
%% EACH ENTRY MUST BEGIN WITH \bibitem{citation key}
%%
%% IF THERE ARE NO ENTRIES  
%% DELETE THE LINE ABOVE (\begin{thebibliography}{20}) 
%% AND THE LINE BELOW (\end{thebibliography})

\bibitem{FJ13}
{\scshape David Fern\'{a}ndez{-}Duque and Joost J. Joosten},
{\itshape Hyperations, Veblen Progressions and Transfinite Iteration of Ordinal Functions},
{\bfseries\itshape Annals of Pure and Applied Logic},
vol.~164 (2013), no.~7-8, pp.~785--801.

\bibitem{GV84}
{\scshape Jean-Yves Girard and Jacqueline Vauzeilles},
{\itshape Functors and ordinal notations. I: A functorial construction of the veblen hierarchy},
{\bfseries\itshape Journal of Symbolic Logic},
vol.~49 (1984), no.~3, pp.~713--729.

\bibitem{PW21}
{\scshape Fedor Pakhomov and James Walsh},
{\itshape Reducing $\omega$-model reflection to iterated syntactic reflection},
{\bfseries\itshape Journal of Mathematical Logic},
2021, doi:~10.1142/S0219061322500015

\end{thebibliography}


\vspace*{-0.5\baselineskip}
% this space adjustment is usually necessary after a bibliography

\end{document}


%% READ ME
%% READ ME
%% READ ME

INSTRUCTIONS FOR SUPPLYING INFORMATION IN THE CORRECT FORMAT: 

1. Author names are listed as First Last, First Last, and First Last.

\absauth{FirstName1 LastName1, FirstName2 LastName2, and FirstName3 LastName3}


2. Titles of abstracts have ONLY the first letter capitalized,
except for Proper Nouns.

\meettitle{Title of abstract with initial capital letter only, except for
Proper Nouns} 


3. Affiliations and email addresses for authors of abstracts are
  listed separately.

% First author's affiliation
\affil{Department, University, Street Address, Country}
\meetemail{First author's email}
%%% NOTE: email required for at least one author
\urladdr{OPTIONAL}
%
% Second author's affiliation
\affil{Department, University, Street Address, Country}
\meetemail{Second author's email}
\urladdr{OPTIONAL}
%
% Third author's affiliation
\affil{Department, University, Street Address, Country}
% Second author's email
\meetemail{Third author's email}
\urladdr{OPTIONAL}


4. Bibliographic Entries

%%%% IF references are submitted with abstract,
%%%% please use the following formats

%%% For a Journal article
\bibitem{cite1}
{\scshape Author's Name},
{\itshape Title of article},
{\bfseries\itshape Journal name spelled out, no abbreviations},
vol.~XX (XXXX), no.~X, pp.~XXX--XXX.

%%% For a Journal article by the same authors as above,
%%% i.e., authors in cite1 are the same for cite2
\bibitem{cite2}
\bysame
{\itshape Title of article},
{\bfseries\itshape Journal},
vol.~XX (XXXX), no.~X, pp.~XX--XXX.

%%% For a book
\bibitem{cite3}
{\scshape Author's Name},
{\bfseries\itshape Title of book},
Name of series,
Publisher,
Year.

%%% For an article in proceedings
\bibitem{cite4}
{\scshape Author's Name},
{\itshape Title of article},
{\bfseries\itshape Name of proceedings}
(Address of meeting),
(First Last and First2 Last2, editors),
vol.~X,
Publisher,
Year,
pp.~X--XX.

%%% For an article in a collection
\bibitem{cite5}
{\scshape Author's Name},
{\itshape Title of article},
{\bfseries\itshape Book title}
(First Last and First2 Last2, editors),
Publisher,
Publisher's address,
Year,
pp.~X--XX.

%%% An edited book
\bibitem{cite6}
Author's name, editor. % No special font used here
{\bfseries\itshape Title of book},
Publisher,
Publisher's address,
Year.

