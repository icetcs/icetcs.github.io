%% FIRST RENAME THIS FILE <yoursurname>.tex. 
%% BEFORE COMPLETING THIS TEMPLATE, SEE THE "READ ME" SECTION 
%% BELOW FOR INSTRUCTIONS. 
%% TO PROCESS THIS FILE YOU WILL NEED TO DOWNLOAD asl.cls from 
%% http://aslonline.org/abstractresources.html. 

\newcommand{\NFU}{\textsf{NFU}}
\newcommand{\NF}{\textsf{NF}}
\newcommand{\Inf}{\textsf{Inf}}
\newcommand{\AC}{\textsf{AC}}
\newcommand{\Tarski}{\textsf{Tarski}}

\documentclass[bsl,meeting]{asl}

\AbstractsOn

\pagestyle{plain}

\def\urladdr#1{\endgraf\noindent{\it URL Address}: {\tt #1}.}


\newcommand{\NP}{}
%\usepackage{verbatim}

\begin{document}
\thispagestyle{empty}

%% BEGIN INSERTING YOUR ABSTRACT DIRECTLY BELOW; 
%% SEE INSTRUCTIONS (1), (2), (3), and (4) FOR PROPER FORMATS

\NP  
\absauth{Tin Adlešić, and Vedran Čačić}
\meettitle{Tarski's theorem about choice and the alternative axiomatic extension of \NFU}
\affil{Faculty of Teacher Education, University of Zagreb, Savska cesta 77, Zagreb, Croatia}
\meetemail{tin.adlesic@ufzg.hr}
\affil{Faculty of science -- Department of Mathematics, University of Zagreb, Bijenička cesta 30, Zagreb, Croatia}
\meetemail{veky@math.hr}

%% INSERT TEXT OF ABSTRACT DIRECTLY BELOW
The main advantage of \NFU\ over plain \NF\ is that it does not disprove the axiom of choice. Both the axiom of choice and the axiom of infinity are independent, but relatively consistent with \NFU. Therefore, $\NFU+\Inf+\AC$ is a theory that is rich enough to encompass all the existent mathematics---but with some technical difficulties. Namely, it is hard to work with Kuratowski's ordered pairs because they are not type-leveled, meaning that the ordered pair does not have the same type as its projections. The fortunate circumstance is that everything can be developed irrespective of how we define ordered pairs, but making them type-leveled yields a significant simplification of exposition. A prevalent solution to that problem in contemporary literature is to postulate a new axiom, so-called \emph{axiom of ordered pairs}. From our point of view, the introduction of that axiom lacks the proper motivation and justification for inclusion, and it also creates new problems; it can be expressed only by introducing new primitive notions. Such evasions are a common occurrence in contemporary \NFU.
    
    In theory $\NFU+\Inf+\AC$ we can define type-leveled ordered pairs using Tarski's theorem about choice which is equivalent to the $\AC$. The main drawback for using $\NFU+\Inf+\AC$ is that we first need to develop all the necessary theory with Kuratowski's ordered pairs, prove the equivalence of the $\AC$ to Tarski's theorem, and only then define type-leveled ordered pairs. This is apparently unavoidable. However, we propose an approach which does that hard work only once: to start with $\NFU+\Inf+\Tarski$, then define type-leveled ordered pairs, and then easily prove the equivalence of Tarski's theorem to the $\AC$. In order to justify that shift of axioms, we must show that in $\NFU+\Inf+\AC$ we can prove the equivalence of the $\AC$ to the Tarski's theorem, but then it becomes a self-sufficient result one can just cite afterwards. It is also worth saying that \Tarski{} seems much more justified as an axiom than the ``axiom of ordered pairs''. In order to complete our presentation, we also need to show that the same thing can be done in $\NFU+\Inf+\Tarski$, but the equivalence proof can be mirrored by the former, and that proof will be in fact much simpler. In effect, those two theories are equiconsistent.

\begin{thebibliography}{10}

%% INSERT YOUR BIBLIOGRAPHIC ENTRIES HERE; 
%% SEE (4) BELOW FOR PROPER FORMAT.
%% EACH ENTRY MUST BEGIN WITH \bibitem{citation key}
%%
%% IF THERE ARE NO ENTRIES  
%% DELETE THE LINE ABOVE (\begin{thebibliography}{20}) 
%% AND THE LINE BELOW (\end{thebibliography})
\bibitem{Wagemakers}
{\scshape G. Wagemakers},
{\itshape New Foundations - A survey of Quine's set theory},
{\bfseries\itshape Instituut voor Tall, Logica en Informatie Publication Series},
X-89-02.

\bibitem{Rosser}
{\scshape J. B. Rosser},
{\itshape Logic for mathematician},
{\bfseries\itshape Dover Publications},
2008.

\bibitem{Morris}
{\scshape S. Morris},
{\itshape Quine, New Foundation, and the Philosophy of Set Theory},
{\bfseries\itshape Cambridge University Press},
2018.

\end{thebibliography}


\vspace*{-0.5\baselineskip}
% this space adjustment is usually necessary after a bibliography

\end{document}


%% READ ME
%% READ ME
%% READ ME

INSTRUCTIONS FOR SUPPLYING INFORMATION IN THE CORRECT FORMAT: 

1. Author names are listed as First Last, First Last, and First Last.

\absauth{FirstName1 LastName1, FirstName2 LastName2, and FirstName3 LastName3}


2. Titles of abstracts have ONLY the first letter capitalized,
except for Proper Nouns.

\meettitle{Title of abstract with initial capital letter only, except for
Proper Nouns} 


3. Affiliations and email addresses for authors of abstracts are
  listed separately.

% First author's affiliation
\affil{Department, University, Street Address, Country}
\meetemail{First author's email}
%%% NOTE: email required for at least one author
\urladdr{OPTIONAL}
%
% Second author's affiliation
\affil{Department, University, Street Address, Country}
\meetemail{Second author's email}
\urladdr{OPTIONAL}
%
% Third author's affiliation
\affil{Department, University, Street Address, Country}
% Second author's email
\meetemail{Third author's email}
\urladdr{OPTIONAL}


4. Bibliographic Entries

%%%% IF references are submitted with abstract,
%%%% please use the following formats

%%% For a Journal article
\bibitem{cite1}
{\scshape Author's Name},
{\itshape Title of article},
{\bfseries\itshape Journal name spelled out, no abbreviations},
vol.~XX (XXXX), no.~X, pp.~XXX--XXX.

%%% For a Journal article by the same authors as above,
%%% i.e., authors in cite1 are the same for cite2
\bibitem{cite2}
\bysame
{\itshape Title of article},
{\bfseries\itshape Journal},
vol.~XX (XXXX), no.~X, pp.~XX--XXX.

%%% For a book
\bibitem{cite3}
{\scshape Author's Name},
{\bfseries\itshape Title of book},
Name of series,
Publisher,
Year.

%%% For an article in proceedings
\bibitem{cite4}
{\scshape Author's Name},
{\itshape Title of article},
{\bfseries\itshape Name of proceedings}
(Address of meeting),
(First Last and First2 Last2, editors),
vol.~X,
Publisher,
Year,
pp.~X--XX.

%%% For an article in a collection
\bibitem{cite5}
{\scshape Author's Name},
{\itshape Title of article},
{\bfseries\itshape Book title}
(First Last and First2 Last2, editors),
Publisher,
Publisher's address,
Year,
pp.~X--XX.

%%% An edited book
\bibitem{cite6}
Author's name, editor. % No special font used here
{\bfseries\itshape Title of book},
Publisher,
Publisher's address,
Year.

