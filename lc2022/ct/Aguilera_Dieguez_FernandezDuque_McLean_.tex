%% FIRST RENAME THIS FILE <yoursurname>.tex. 
%% BEFORE COMPLETING THIS TEMPLATE, SEE THE "READ ME" SECTION 
%% BELOW FOR INSTRUCTIONS. 
%% TO PROCESS THIS FILE YOU WILL NEED TO DOWNLOAD asl.cls from 
%% http://aslonline.org/abstractresources.html. 


\documentclass[bsl,meeting]{asl}

\AbstractsOn

\pagestyle{plain}

\def\urladdr#1{\endgraf\noindent{\it URL Address}: {\tt #1}.}


\newcommand{\NP}{}
%\usepackage{verbatim}

\begin{document}
\thispagestyle{empty}

%% BEGIN INSERTING YOUR ABSTRACT DIRECTLY BELOW; 
%% SEE INSTRUCTIONS (1), (2), (3), and (4) FOR PROPER FORMATS

\NP  
\absauth{Juan Pablo Aguilera, Mart\'in Di\'eguez, David Fern\'andez-Duque, and Brett McLean}
\meettitle{G\"odel temporal logic}
% First author's affiliation
\affil{Institute of Discrete Mathematics and Geometry, Vienna University of Technology, Wiedner Hauptstrasse 8--10, 1040 Vienna, Austria}
\meetemail{aguilera@logic.at}
%%% NOTE: email required for at least one author
%\urladdr{OPTIONAL}
%
% Second author's affiliation
\affil{Department of Informatics, University of Angers, 2 Boulevard de Lavoisier, 49045 Angers CEDEX 01, France}
\meetemail{Martin.DieguezLodeiro@univ-angers.fr}
%\urladdr{OPTIONAL}
%
% Third author's affiliation
\affil{Department of Mathematics WE16, Ghent University,  Krijgslaan 281-S8, 9000 Ghent, Belgium}
% Third author's email
\meetemail{David.FernandezDuque@ugent.be}
%\urladdr{OPTIONAL}
%
% Fourth author's affiliation
\affil{Department of Mathematics WE16, Ghent University, Krijgslaan 281-S8, 9000 Ghent, Belgium}
% Fourth author's email
\meetemail{brett.mclean@ugent.be}
%\urladdr{OPTIONAL}

%% INSERT TEXT OF ABSTRACT DIRECTLY BELOW

We investigate a non-classical version of linear temporal
logic (with next $\circ$, eventually $\Diamond$, and henceforth $\Box$ modalities) whose propositional fragment is G\"odel--Dummett logic
(which is well known both as a superintuitionistic logic and
a t-norm fuzzy logic). The importance of both linear temporal logic and of fuzzy logics in computer science is well established. 

We define the logic using two natural
semantics---a real-valued semantics and a semantics where truth values are captured by a linear Kripke frame---and can show that these indeed define one and the same
logic. Although this G\"odel temporal logic does not have any
form of the finite model property for these two semantics, we are able to prove decidability of the validity problem. The proof makes use of \emph{quasimodels} \cite{cite1}, which are a variation on Kripke models where time can be nondeterministic. We can show that every falsifiable formula is falsifiable on a finite
quasimodel, which yields decidability. We then
strengthen this result to PSPACE-complete. Further, we provide a deductive calculus for G\"odel temporal logic with a finite number of axioms and deduction rules, and can show this calculus to be sound and
complete for the above-mentioned semantics.

\begin{thebibliography}{10}

%% INSERT YOUR BIBLIOGRAPHIC ENTRIES HERE; 
%% SEE (4) BELOW FOR PROPER FORMAT.
%% EACH ENTRY MUST BEGIN WITH \bibitem{citation key}
%%
%% IF THERE ARE NO ENTRIES  
%% DELETE THE LINE ABOVE (\begin{thebibliography}{20}) 
%% AND THE LINE BELOW (\end{thebibliography})

%%% For a Journal article
\bibitem{cite1}
{\scshape David Fern\'andez-Duque},
{\itshape Non-deterministic semantics for dynamic topological logic},
{\bfseries\itshape Annals of Pure and Applied Logic},
vol.~157 (2009), no.~2--3, pp.~110--121.

\end{thebibliography}


\vspace*{-0.5\baselineskip}
% this space adjustment is usually necessary after a bibliography

\end{document}


%% READ ME
%% READ ME
%% READ ME

INSTRUCTIONS FOR SUPPLYING INFORMATION IN THE CORRECT FORMAT: 

1. Author names are listed as First Last, First Last, and First Last.

\absauth{FirstName1 LastName1, FirstName2 LastName2, and FirstName3 LastName3}


2. Titles of abstracts have ONLY the first letter capitalized,
except for Proper Nouns.

\meettitle{Title of abstract with initial capital letter only, except for
Proper Nouns} 


3. Affiliations and email addresses for authors of abstracts are
  listed separately.

% First author's affiliation
\affil{Department, University, Street Address, Country}
\meetemail{First author's email}
%%% NOTE: email required for at least one author
\urladdr{OPTIONAL}
%
% Second author's affiliation
\affil{Department, University, Street Address, Country}
\meetemail{Second author's email}
\urladdr{OPTIONAL}
%
% Third author's affiliation
\affil{Department, University, Street Address, Country}
% Second author's email
\meetemail{Third author's email}
\urladdr{OPTIONAL}


4. Bibliographic Entries

%%%% IF references are submitted with abstract,
%%%% please use the following formats

%%% For a Journal article
\bibitem{cite1}
{\scshape Author's Name},
{\itshape Title of article},
{\bfseries\itshape Journal name spelled out, no abbreviations},
vol.~XX (XXXX), no.~X, pp.~XXX--XXX.

%%% For a Journal article by the same authors as above,
%%% i.e., authors in cite1 are the same for cite2
\bibitem{cite2}
\bysame
{\itshape Title of article},
{\bfseries\itshape Journal},
vol.~XX (XXXX), no.~X, pp.~XX--XXX.

%%% For a book
\bibitem{cite3}
{\scshape Author's Name},
{\bfseries\itshape Title of book},
Name of series,
Publisher,
Year.

%%% For an article in proceedings
\bibitem{cite4}
{\scshape Author's Name},
{\itshape Title of article},
{\bfseries\itshape Name of proceedings}
(Address of meeting),
(First Last and First2 Last2, editors),
vol.~X,
Publisher,
Year,
pp.~X--XX.

%%% For an article in a collection
\bibitem{cite5}
{\scshape Author's Name},
{\itshape Title of article},
{\bfseries\itshape Book title}
(First Last and First2 Last2, editors),
Publisher,
Publisher's address,
Year,
pp.~X--XX.

%%% An edited book
\bibitem{cite6}
Author's name, editor. % No special font used here
{\bfseries\itshape Title of book},
Publisher,
Publisher's address,
Year.

