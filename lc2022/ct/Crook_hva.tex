\documentclass[bsl,meeting]{asl}
\begin{document}
We approach the idea of verified machine learning from the perspective of computable analysis. By using the language of computable analysis, particularly that of represented spaces, we can formalize various verification questions of relevance for the machine learning community. These formulations involve function spaces, as well as the spaces of open subsets, overt subsets and compact subsets. We can then show that as long as the appropriate notions of subsets are chosen, and as long as we use ternary logic including an undetermined truth value, these verification questions do become computable.
\begin{thebibliography}{10}

\bibitem{cite1}
{\scshape Tonicha Crook, Jay Morgan, Markus Roggenbach and Arno Pauly},
{\itshape  Computability Perspective on (Verified) Machine Learning},
{\bfseries\itshape arXiv 2102.06585},
\end{thebibliography}
\end{document}
