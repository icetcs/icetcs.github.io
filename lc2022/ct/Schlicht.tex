%%FIRST RENAME THIS FILE <yoursurname>.tex. 
%%BEFORE COMPLETING THIS TEMPLATE, SEE THE "READ ME" SECTION BELOW FOR INSTRUCTIONS.
%% TO PROCESS THIS FILE YOU WILL NEED TO DOWNLOAD asl.cls from  http://aslonline.org/abstractresources.html. 

\documentclass[bsl,meeting]{asl}
\AbstractsOn
\pagestyle{plain}
\def\urladdr#1{\endgraf\noindent{\it URL Address}: {\tt #1}.}
\newcommand{\NP}{}
\usepackage{verbatim}

\begin{document}
\thispagestyle{empty}

%% BEGIN INSERTING YOUR ABSTRACT DIRECTLY BELOW; 
%% SEE INSTRUCTIONS (1), (2), (3), and (4) FOR PROPER FORMATS

\NP  
\absauth{Philipp Schlicht}
\meettitle{Countable ranks at the first and second projective levels}
\affil{School of Mathematics, University of Bristol, Fry Building, Woodland Road, Bristol, BS8 1UG, UK}
\meetemail{philipp.schlicht@bristol.ac.uk}

%% INSERT TEXT OF ABSTRACT DIRECTLY BELOW

Transfinite derivations and computations induce rank functions on sets of reals.  
The complexity of these ranks typically lies at the first or second level of the projective hierarchy or in between them. 
We study arbitrary ranks of countable length at these levels. 
Using robust ordinals, a variant of stable ordinals from proof theory, we calculate the suprema of lengths of countable $\Pi^1_1$ ranks, $\Sigma^1_2$ ranks, $\Pi^1_1$ prewellorders and $\Sigma^1_2$ wellfounded relations, among others. 
They all equal the first $\Sigma_2$-robust ordinal $\tau$, or equivalently, Kechris' ordinal $\gamma^1_2$. 
Furthermore, we obtain results towards a characterisation of those $\Sigma^1_2$ sets that admit countable ranks. 

This is a joint project with Merlin Carl and Philip Welch. 



%\begin{thebibliography}{10}

%% INSERT YOUR BIBLIOGRAPHIC ENTRIES HERE; 
%% SEE (4) BELOW FOR PROPER FORMAT.
%% EACH ENTRY MUST BEGIN WITH \bibitem{citation key}
%%
%% IF THERE ARE NO ENTRIES  
%% DELETE THE LINE ABOVE (\begin{thebibliography}{20}) 
%% AND THE LINE BELOW (\end{thebibliography})

%\end{thebibliography}


\vspace*{-0.5\baselineskip}
% this space adjustment is usually necessary after a bibliography

\end{document}


%% READ ME
%% READ ME
%% READ ME

INSTRUCTIONS FOR SUPPLYING INFORMATION IN THE CORRECT FORMAT: 

1. Author names are listed as First Last, First Last, and First Last.

\absauth{FirstName1 LastName1, FirstName2 LastName2, and FirstName3 LastName3}


2. Titles of abstracts have ONLY the first letter capitalized,
except for Proper Nouns.

\meettitle{Title of abstract with initial capital letter only, except for Proper Nouns} 


3. Affiliations and email addresses for authors of abstracts are
  listed separately.

% First author's affiliation
\affil{Department, University, Street Address, Country}
\meetemail{First author's email}
%%% NOTE: email required for at least one author
\urladdr{OPTIONAL}


% Second author's affiliation
\affil{Department, University, Street Address, Country}
\meetemail{Second author's email}
\urladdr{OPTIONAL}


% Third author's affiliation
\affil{Department, University, Street Address, Country}
% Second author's email
\meetemail{Third author's email}
\urladdr{OPTIONAL}


4. Bibliographic Entries

%%%% IF references are submitted with abstract,
%%%% please use the following formats

%%% For a Journal article
\bibitem{cite1}
{\scshape Author's Name},
{\itshape Title of article},
{\bfseries\itshape Journal name spelled out, no abbreviations},
vol.~XX (XXXX), no.~X, pp.~XXX--XXX.

%%% For a Journal article by the same authors as above,
%%% i.e., authors in cite1 are the same for cite2
\bibitem{cite2}
\bysame
{\itshape Title of article},
{\bfseries\itshape Journal},
vol.~XX (XXXX), no.~X, pp.~XX--XXX.

%%% For a book
\bibitem{cite3}
{\scshape Author's Name},
{\bfseries\itshape Title of book},
Name of series,
Publisher,
Year.

%%% For an article in proceedings
\bibitem{cite4}
{\scshape Author's Name},
{\itshape Title of article},
{\bfseries\itshape Name of proceedings}
(Address of meeting),
(First Last and First2 Last2, editors),
vol.~X,
Publisher,
Year,
pp.~X--XX.

%%% For an article in a collection
\bibitem{cite5}
{\scshape Author's Name},
{\itshape Title of article},
{\bfseries\itshape Book title}
(First Last and First2 Last2, editors),
Publisher,
Publisher's address,
Year,
pp.~X--XX.

%%% An edited book
\bibitem{cite6}
Author's name, editor. % No special font used here
{\bfseries\itshape Title of book},
Publisher,
Publisher's address,
Year.


--------------------------------------------------------

Rules for Abstracts

The following rules apply to abstracts of contributed talks at ASL meetings (including those submitted “by title”); abstracts that do not conform to these requirements will be returned immediately to the authors. Abstracts received after the announced deadline will not be considered.

For ASL meetings in North America (Annual, JMM, APA), at least one author of each contributed abstract must be an ASL member; this does not apply to the ASL European Summer Meeting, Asian Logic Conference, or SLALM. Abstracts are only published as part of the meeting report in The Bulletin of Symbolic Logic if at least one author is a member of the ASL at the time the report is sent for publication. This applies to all ASL meetings.

Individuals are invited to submit contributed papers for presentation at an ASL meeting that have logic research content that lies within the scope of the interests of the ASL. Individuals’ names may appear as authors or co-authors on up to 3 abstracts per meeting; however, each set of authors may only submit one abstract per meeting.

Abstracts of contributed talks are limited to 300 words (about three quarters of a page in the standard, 11pt LaTeX article style), including the title, other heading material, and references.

Authors submitting both invited and contributed abstracts must do so via email, preferably as a LaTex or TeX file (see template below); however, ASCII and Doc will also be accepted. Authors may also submit via post as a hard copy. Please note that if you are submitting an abstract using TeX or Word, it should be fully processed and it must be accompanied by a PDF file in which all mathematical symbols are visible and correct for the typeset version.

When submitting abstracts as Tex files, please use the ASL abstract template, (ASLabstracttemplate.tex, asl.cls) Please rename the template using your last name before submitting it as an attachment. Complete instructions on how to use the template are included in ASLabstracttemplate.tex.

To ensure accurate and timely typesetting:

    Each abstract submission must include the complete address of at least one author, including an email address.
    Abstracts may not include drawings or footnotes.
    The references must be complete: author, title, full name of journal, year, volume, page numbers, and for books, complete publisher data.

Typesetting is done immediately after the submission deadline, so abstracts should be in final form when they are submitted. A typeset booklet containing the abstracts will be distributed to participants at the meeting. Galley proofs are given to authors at the time of the meeting or before, and must be corrected, approved, and returned immediately after receipt; no changes other than corrections of typesetting errors may be made.

