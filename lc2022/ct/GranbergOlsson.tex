%% FIRST RENAME THIS FILE <yoursurname>.tex. 
%% BEFORE COMPLETING THIS TEMPLATE, SEE THE "READ ME" SECTION 
%% BELOW FOR INSTRUCTIONS. 
%% TO PROCESS THIS FILE YOU WILL NEED TO DOWNLOAD asl.cls from 
%% http://aslonline.org/abstractresources.html. 


\documentclass[bsl,meeting]{asl}

\AbstractsOn

\pagestyle{plain}

\def\urladdr#1{\endgraf\noindent{\it URL Address}: {\tt #1}.}


\newcommand{\NP}{}
%\usepackage{verbatim}

\begin{document}
\thispagestyle{empty}

%% BEGIN INSERTING YOUR ABSTRACT DIRECTLY BELOW; 
%% SEE INSTRUCTIONS (1), (2), (3), and (4) FOR PROPER FORMATS

\NP  
\absauth{Mattias Granberg Olsson and Graham Leigh}
\meettitle{Almost negative truth and fixpoints in intuitionistic logic}
\affil{Department of Philosophy, Linguistics and Theory of Science, University of Gothenburg, PO Box 200 SE405 30 G{\"o}teborg, Sweden}
\meetemail{mattias.granberg.olsson@gu.se}
\affil{Department of Philosophy, Linguistics and Theory of Science, University of Gothenburg, PO Box 200 SE405 30 G{\"o}teborg, Sweden}
\meetemail{graham.leigh@gu.se}

%% INSERT TEXT OF ABSTRACT DIRECTLY BELOW
We present work in progress on the relationship between the theory of transfinitely iterated strictly positive fixpoints and axiomatic theories of compositional and disquotational truth for almost negative formulae in intuitionistic logic. The starting point is the result of Cantini \cite{Cantini:1989} and Feferman \cite{Feferman:1991} (extended to the transfinite by Fujimoto \cite{Fujimoto:2011}) that the (classical) theory of positive fixpoints $\widehat{\mathrm{ID}}_1$, the Kripke-Feferman compositional theory of partial truth $\mathrm{KF}$, and the uniformly disquotational theory for truth-positive formulae $\mathrm{PUTB}$ are mutually interpretable. We obtain similar results for the theories of transfinite iterations of strictly positive fixpoints (as in \cite{Ruede_Strahm:2002}) for almost negative operators, and disquotation for almost negative strictly truth-positive sentences, in intuitionistic logic (which we call $\widehat{\mathrm{ID}}_{\alpha}^{\mathrm{i}}(\Lambda)$ and $\mathrm{P \Lambda UTB}_{\alpha}^{\mathrm{i}}$ respectively):
\begin{itemize}
\item First, $\widehat{\mathrm{ID}}_{\alpha}^{\mathrm{i}}(\Lambda)$ is interpretable in $\mathrm{P \Lambda UTB}_{\alpha}^{\mathrm{i}}$ by mimicking the classical proof.
\item Second, $\mathrm{P \Lambda UTB}_{\alpha}^{\mathrm{i}}$ is interpretable (via a compositional theory) in $\widehat{\mathrm{ID}}_{\omega \cdot \alpha}^{\mathrm{i}}$ for limit $\alpha$. This is achieved by using the extra `spacing' between the levels, given by the multiplication by $\omega$, to keep track of the nestings of implications in formulae.
\end{itemize}

\begin{thebibliography}{10}

%% INSERT YOUR BIBLIOGRAPHIC ENTRIES HERE; 
%% SEE (4) BELOW FOR PROPER FORMAT.
%% EACH ENTRY MUST BEGIN WITH \bibitem{citation key}
%%
%% IF THERE ARE NO ENTRIES  
%% DELETE THE LINE ABOVE (\begin{thebibliography}{20}) 
%% AND THE LINE BELOW (\end{thebibliography})

\bibitem{Cantini:1989}
{\scshape Andrea Cantini},
{\itshape Notes on formal theories of truth},
{\bfseries\itshape Zeitschrift f{\"u}r Mathematiscche Logik und Grundlagen der Mathematik},
vol.~35 (1989), no.~2, pp.~97--130.

\bibitem{Feferman:1991}
{\scshape Solomon Feferman},
{\itshape Reflecting on incompleteness},
{\bfseries\itshape The Journal of Symbolic Logic},
vol.~56 (1991), no.~1, pp.~1--49.

\bibitem{Fujimoto:2011}
{\scshape Kentaro Fujimoto},
{\itshape Autonomous progression and transfinite iteration of self-applicable truth},
{\bfseries\itshape The Journal of Symbolic Logic},
vol.~76 (2011), no.~3, pp.~914--945.

\bibitem{Ruede_Strahm:2002}
{\scshape Christian R{\"u}ede and Thomas Strahm},
{\itshape Intuitionistic fixed point theories for strictly positive operators},
{\bfseries\itshape Mathematical Logic Quaterly},
vol.~48 (2002), no.~2, pp.~195--202.
\end{thebibliography}


\vspace*{-0.5\baselineskip}
% this space adjustment is usually necessary after a bibliography

\end{document}


%% READ ME
%% READ ME
%% READ ME

INSTRUCTIONS FOR SUPPLYING INFORMATION IN THE CORRECT FORMAT: 

1. Author names are listed as First Last, First Last, and First Last.

\absauth{FirstName1 LastName1, FirstName2 LastName2, and FirstName3 LastName3}


2. Titles of abstracts have ONLY the first letter capitalized,
except for Proper Nouns.

\meettitle{Title of abstract with initial capital letter only, except for
Proper Nouns} 


3. Affiliations and email addresses for authors of abstracts are
  listed separately.

% First author's affiliation
\affil{Department, University, Street Address, Country}
\meetemail{First author's email}
%%% NOTE: email required for at least one author
\urladdr{OPTIONAL}
%
% Second author's affiliation
\affil{Department, University, Street Address, Country}
\meetemail{Second author's email}
\urladdr{OPTIONAL}
%
% Third author's affiliation
\affil{Department, University, Street Address, Country}
% Second author's email
\meetemail{Third author's email}
\urladdr{OPTIONAL}


4. Bibliographic Entries

%%%% IF references are submitted with abstract,
%%%% please use the following formats

%%% For a Journal article
\bibitem{cite1}
{\scshape Author's Name},
{\itshape Title of article},
{\bfseries\itshape Journal name spelled out, no abbreviations},
vol.~XX (XXXX), no.~X, pp.~XXX--XXX.

%%% For a Journal article by the same authors as above,
%%% i.e., authors in cite1 are the same for cite2
\bibitem{cite2}
\bysame
{\itshape Title of article},
{\bfseries\itshape Journal},
vol.~XX (XXXX), no.~X, pp.~XX--XXX.

%%% For a book
\bibitem{cite3}
{\scshape Author's Name},
{\bfseries\itshape Title of book},
Name of series,
Publisher,
Year.

%%% For an article in proceedings
\bibitem{cite4}
{\scshape Author's Name},
{\itshape Title of article},
{\bfseries\itshape Name of proceedings}
(Address of meeting),
(First Last and First2 Last2, editors),
vol.~X,
Publisher,
Year,
pp.~X--XX.

%%% For an article in a collection
\bibitem{cite5}
{\scshape Author's Name},
{\itshape Title of article},
{\bfseries\itshape Book title}
(First Last and First2 Last2, editors),
Publisher,
Publisher's address,
Year,
pp.~X--XX.

%%% An edited book
\bibitem{cite6}
Author's name, editor. % No special font used here
{\bfseries\itshape Title of book},
Publisher,
Publisher's address,
Year.

