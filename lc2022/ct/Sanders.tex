

\documentclass[bsl,meeting,11pt]{asl}

\usepackage{hyperref, amsrefs}

\AbstractsOn
/Users/sasander/Dropbox/math temp/Talks/Reykjavik_LC2022/Abstract/Sanders_LC22_Two_computational_clusters.tex
\pagestyle{plain}

\def\urladdr#1{\endgraf\noindent{\it URL Address}: {\tt #1}.}


\newcommand{\NP}{}
%\usepackage{verbatim}

\begin{document}
\thispagestyle{empty}



\NP  
\absauth{Sam Sanders}
\meettitle{Two computational clusters in ordinary mathematics}
\affil{Institute for Philosophy II, RUB Bochum, Germany}
\meetemail{sasander@me.com}
\urladdr{http://sasander.wixsite.com/academic}



\medskip

%% INSERT TEXT OF ABSTRACT DIRECTLY BELOW

I provide an overview of recent joint work with Dag Normann on the computability theory of ordinary mathematics (\cites{dagsamXII, dagsamXIII}), as follows.

\smallskip

Given a finite set, perhaps the most basic questions are \emph{how many} elements it has, and \emph{which ones}?  
We study this question in Kleene's \emph{higher-order computability theory}, based on his computation schemes S1-S9 (\cite{longmann}).  
In particular, a central object of study is the higher-order functional $\Omega$ which on input a finite set of real numbers, list the elements as a finite sequence. 

\smallskip

Perhaps surprisingly, the `finiteness' functional $\Omega$ give rise to a \emph{huge and robust} class of computationally equivalent operations, called the $\Omega$-cluster.
For instance, many basic operations on functions of \emph{bounded variation} ($BV$) are part of the $\Omega$-cluster, including those stemming from the well-known \emph{Jordan decomposition theorem}.  
In addition, we identify a second cluster of computationally equivalent objects, called the $\Omega_{1}$-cluster, based on the functional $\Omega_{1}$, the restriction of $\Omega$ to singletons.  
We also show that both clusters include basic operations on \emph{regulated} and \emph{Sobolev space} functions, respectively a well-known super- and sub-class of the class of $BV$-functions. 

\smallskip

Our objects of study are fundamentally \emph{partial} in nature, and we formulate an elegant and equivalent $\lambda$-calculus formulation of S1-S9 to accommodate partial objects.
The advantages of this approach are three-fold: proofs are more transparent in our $\lambda$-calculus approach, all (previously hand-waved) technical details can be settled easily, and we can show that $\Omega_{1}$ and $\Omega$ are not computationally equivalent to any \emph{total} functional.

\smallskip


\begin{thebibliography}{10}
\bib{longmann}{book}{
  author={Longley, John},
  author={Normann, Dag},
  title={Higher-order Computability},
  year={2015},
  publisher={Springer},
  series={Theory and Applications of Computability},
}

\bib{dagsamXII}{article}{
   author={Normann, Dag},
   author={Sanders, Sam},
   title={Betwixt Turing and Kleene},
   journal={LNCS 13137, proceedings of LFCS22},
   pages={pp.\ 18},
   date={2022},
}

\bib{dagsamXIII}{article}{
   author={Normann, Dag},
   author={Sanders, Sam},
   title={On the computational properties of basic mathematical notions},
   journal={Submitted, arxiv: \url{https://arxiv.org/abs/2203.05250}},
   pages={pp.\ 43},
   date={2022},
}



\end{thebibliography}

%\begin{bibdiv}
%\begin{biblist}
%\bibselect{allkeida}
%\end{biblist}
%\end{bibdiv}



\vspace*{-0.5\baselineskip}
% this space adjustment is usually necessary after a bibliography

\end{document}
