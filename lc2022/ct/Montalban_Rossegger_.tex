%% FIRST RENAME THIS FILE <yoursurname>.tex. 
%% BEFORE COMPLETING THIS TEMPLATE, SEE THE "READ ME" SECTION 
%% BELOW FOR INSTRUCTIONS. 
%% TO PROCESS THIS FILE YOU WILL NEED TO DOWNLOAD asl.cls from 
%% http://aslonline.org/abstractresources.html. 


\documentclass[bsl,meeting]{asl}

\AbstractsOn

\pagestyle{plain}

\def\urladdr#1{\endgraf\noindent{\it URL Address}: {\tt #1}.}


\newcommand{\NP}{}
%\usepackage{verbatim}

\begin{document}
\thispagestyle{empty}

%% BEGIN INSERTING YOUR ABSTRACT DIRECTLY BELOW; 
%% SEE INSTRUCTIONS (1), (2), (3), and (4) FOR PROPER FORMATS

\NP  
\absauth{Antonio Montalb\'an, Dino Rossegger}
\meettitle{The structural complexity of models of arithmetic}
\affil{Department of Mathematics, University of California, Berkeley}
\meetemail{antonio@math.berkeley.edu}

\affil{Department of Mathematics, University of California, Berkeley and
Institut of Discrete Mathematics and Geometry, Technische Universit\"at Wien,
Austria}
\meetemail{dino@math.berkeley.edu}

%% INSERT TEXT OF ABSTRACT DIRECTLY BELOW
The Scott rank of a countable structure is the least ordinal $\alpha$ such that all
automorphism orbits of the structure are definable by infinitary
$\Sigma_{\alpha}$ formulas. Montalb\'an showed that the Scott rank of
a structure is a robust measure of its structural and computational complexity
by showing that
various different measures are equivalent. For example, a structure has
Scott rank $\alpha$ if and only if it has a $\Pi_{\alpha+1}$ Scott sentence if
and only if it is uniformly $\pmb \Delta_\alpha^0$ categorical.
In this talk we present results on the Scott rank of models of Peano arithmetic. We show that 
non-standard models of PA have Scott rank at least $\omega$ and that the 
models of PA that have Scott rank $\omega$ are precisely the prime models. We also give
reductions via bi-interpretability of the class of linear orders
to completions $T$ of $PA$. This allows us to exhibit models of $T$ of Scott rank $\alpha$ for every $\omega\leq \alpha\leq \omega_1$.
\end{document}

