%% FIRST RENAME THIS FILE <yoursurname>.tex.
%% BEFORE COMPLETING THIS TEMPLATE, SEE THE "READ ME" SECTION
%% BELOW FOR INSTRUCTIONS.
%% TO PROCESS THIS FILE YOU WILL NEED TO DOWNLOAD asl.cls from
%% http://aslonline.org/abstractresources.html.


\documentclass[bsl,meeting]{asl}

\usepackage{amsfonts}
\usepackage{amssymb}
\usepackage[mathscr]{euscript}
\usepackage{verbatim}
%\usepackage{amsthm}

\AbstractsOn

\pagestyle{plain}

\def\urladdr#1{\endgraf\noindent{\it URL Address}: {\tt #1}.}


\newcommand{\NP}{}
%\usepackage{verbatim}

\newcommand{\mf}[1]{\mathfrak{#1}}
\newcommand{\mc}[1]{\mathcal{#1}}
\newcommand{\mbf}[1]{\mathbf{#1}}
\newcommand{\ms}[1]{\mathscr{#1}}
\newcommand{\msf}[1]{\mathsf{#1}}
\newcommand{\couple}[2]{\langle{#1},{#2}\rangle}
\newcommand{\FA}{\mf{Fm}}
\newcommand{\inverse}[1]{{#1}^{-1}}
\newcommand{\restr}{{\upharpoonright}}

\def\e{{\frac{1}{2}} }
\def\iff{\Leftrightarrow}

\DeclareMathOperator{\Fm}{Fm}
\DeclareMathOperator{\Con}{Con}
\DeclareMathOperator{\img}{img}
\DeclareMathOperator{\Cn}{Cn}

%\newtheorem{claim}{Claim}

\begin{document}
\thispagestyle{empty}

%% BEGIN INSERTING YOUR ABSTRACT DIRECTLY BELOW;
%% SEE INSTRUCTIONS (1), (2), (3), and (4) FOR PROPER FORMATS

\NP
\absauth{Alexej Pynko, Ira Sirko}
\meettitle{Extensions of paraconsistent three-valued
chain
%lattice
logics}
\affil{Cybernetics Institute,
Glushkov p. 40, Kiev, 03680, Ukraine}
\meetemail{pynko@i.ua}

%% INSERT TEXT OF ABSTRACT DIRECTLY BELOW


Given any propositional language $L$
(viz., a set of primary connectives,
treated as operation symbols, when dealing
with $L$-algebras), an {\em $L$-rule\/}
is any expression of the form $\ms{R}=([\Gamma\vdash]\varphi)$,
where $[\Gamma\subseteq]\Fm_L\ni\varphi$,
whereas $\Fm_L$ is the set of $L$-formulas
with variables in %a countable set,
%of the absolutely-free $L$-algebra %$\FA_L$
%freely-generated by
%the set
$V=\{x_i\}_{i\in\omega}$ --- viz.
the carrier of the absolutely-free $L$-algebra $\FA_L$
freely-generated by $V$,
%of propositional variables
natural numbers, including $0$,
being treated as sets of lesser ones,
the set of all them being denoted by $\omega$,
while $\neg|\land/\lor\supset$ is a
$(1|2)$-ary prefix$|$infix connective of $L$
(possibly, a secondary one --- viz., an $L$-formula
with variables in $\{x_j\}_{j\in(1|2)}$).
Then, an {\em $L$-logic\/} $C$ (viz.,
a {\em structural\/} closure operator over $\Fm_L$
--- i.e., with $\img C$ closed under inverse
endomorphisms of $\FA_L$)
is said to ``{\em satisfy\/} $\ms{R}$''$|$``be {\em [$\neg$-para]consistent\/}'',
if $(\varphi|x_1)\in|\not\in C(\varnothing[\cup(\Gamma|\{x_0,\neg x_0\})])$,
an {\em extension of\/} $C$ (viz., an $L$-logic $C'$ with $(\img C')\subseteq(\img
C)$)
%point-wise)
being said to be {\em proper/``relatively axiomatized
by\/ $\ms{R}$''}, if it is ``distinct from $C$''/``the least
extension of $C$ satisfying $\ms{R}$''.
Likewise, any {\em $L$-matrix\/}
(viz., a pair $\mc{A}=\couple{\mf{A}}{D}$,
constituted by its {\em underlying\/} $L$-algebra $\mf{A}$ with
carrier $A$, consisting of its {\em values}, and the set $D\subseteq A$  of
its {\em distinguished\/} values) {\em defines\/} its logic
$\Cn_\mc{A}$ such that
$\{\inverse{h}[D]\mid h\in\hom(\FA_L,\mf{A})\}$
is a closure basis of %the closure system
$\img\Cn_\mc{A}$, %over $\Fm_L$,
as well as said to be {\em
$\neg$-classical\/$|\supset$-implicative},
if ``it has exactly $2[-1]$ [distinguished] values,
while $\neg^\mf{A}$ permutes its unique distinguished
and non-distinguished values''$|\forall a,b\in A:
((a\in D)\Rightarrow(b\in
D))\iff((a\supset^\mf{A}b)\in D)$.
Then, $C$ is said to be {\em $\neg$-classical\/$|\supset$-implicative},
if ``it %(has an extension that)
is defined by
a $\neg$-classical $L$-matrix,
in which case it is consistent but not $\neg$-paraconsistent''$|\forall\Delta\subseteq\Fm_L,\forall\phi,\psi\in\Fm_L:
(\psi\in C(\Delta\cup\{\phi\}))\iff((\phi\supset\psi)\in
C(\Delta))$,
$L$-logics with $\neg$-classical extensions being referred to as
{\em $\neg$-subclassical}.

\begin{theorem}
\label{main-thm}
Let\/ $\mf{A}$ be an $L$-algebra with carrier
$A\triangleq(2\cup\{\e\})$,
$a\triangleq\neg^\mf{A}\e$,
$b\in\{a,1\}$,
%$b'\triangleq1$,
$D\subseteq(A\setminus1)$
%$D\triangleq(A\setminus1)$
and
%$C$ the logic of
$\mc{A}\triangleq\couple{\mf{A}}{D}$.
Suppose\/ $1\in D$,
%$\neg^\mf{A}\e\in D$,
while\/ [the least subalgebra of]\/ %the algebra\/
$\couple{A}{\land^\mf{A},\lor^\mf{A},0,b[+(1-b),\neg^\mf{A}]}$
%of type\/ $\langle2,2,0,0[,1]\rangle$
%[generated by\/ $2$]
is a [complemented] bounded lattice, %with zero\/ $0$ and unit in\/
%$\{1,\neg^\mf{A}\e\}$,
whereas\/ $\Cn_\mc{A}$ is both\/
$\neg$-paraconsistent (i.e., $\{\e,a\}\subseteq D$)
and\/ \{not\/\}
non-$\neg$-subclassical (i.e.,
the subalgebra of\/ $\mf{A}$ generated by\/ $2$
does\/ \{not\/\} contain\/ $\e$)
%$|(\mbf{S}\mf{A})\cap\{2\}|=(0[+1])$)
$\langle$as well as\/
$\supset$-implicative
(i.e., $\mc{A}$ is so)\/$\rangle$.
Then, $\Cn_\mc{A}$ has no consistent proper extension\/
\{other than\/
$\Cn_{(\mc{A}\restr2)/(\mc{A}\times(\mc{A}\restr2))}$
relatively axiomatized by\/
$(((\{x_0\lceil\lor x_1\rceil,\neg x_0\lor x_1\}|\{x_0\lor x_1,\neg x_0\})/\{x_0,\neg x_0\})\vdash([\neg\neg x_0\lor] x_1)/([\lfloor x_1\lor\rfloor\neg\neg]\linebreak x_{1[-1]}))\langle\|
\neg x_0\supset|||\vdash(x_0\supset([\neg\neg] x_{1[-1]}))/\rangle$
[unless $a=\e$]
, %(\{x_0,\neg x_0\}\vdash
%x_1)$,
in which case
the former is a\/ $\neg$-classical proper extension of the latter,
and so the latter is not\/ $\neg$-classical,
while\/ $\Cn_\mc{A}(\varnothing)=
\Cn_{\mc{A}\times(\mc{A}\restr2)}(\varnothing)$,
whereas\/ $\Cn_\mc{A}(\varnothing)\neq
\Cn_{\mc{A}\restr2}(\varnothing)$ iff
$\mc{A}/\!/\Cn_\mc{A}$ is implicative
iff\/
$\{\couple{i}{i}\mid i\in2\}\cup(\{\e\}\times(\frac{1}{a}))$
does not form a subalgebra of\/ $\mf{A}^2$
iff  either\/ $b=\e$
or\/ $\mf{A}$ has a (dual)
discriminator\/\}.
\end{theorem}

This covers arbitrary {\em three-valued\/} expansions of
``the {\em logic of paradox} $LP$''/``Ha\l{}kow\-s\-ka-Zajac' logic
$HZ$'' (with $a=\e$ and $b=(1/\e)$
/``as well as secondary binary connectives $\neg x_0(\lor|\land)\neg x_1$
for primary ones $\land|\lor$'') and %as well as those of
the $\neg$-paraconsistent counterpart of %[s of both]
the implication-less fragment of G\"{o}del's
three-valued logic %$G_3$ [and $G_3$ itself]
resulted from leaving non-distinguished $0$ alone and
taking dual pseudo-complement
%[as well as {\em any\/} binary operation on $A$]
for pseudo-complement, %[and relative one],
in which case $(a|b)=1$, %, and so $b=1$ too,
thus subsuming results originally proved by {\sc Pynko} {\em ad hoc}.

\begin{comment}
 %that is {\em structural\/} in the
%sense that %$\sigma[C(X)]\subseteq C(\sigma[X])$,
%for all $X\subseteq\Fm_L$ and all $\sigm
%$\img C$ is closed under inverse {\em [propositional]\/
%$L$-substitutions\/}
%(viz., endomorphisms of $\FA_L$),
is said to be {\em self-extensional},
provided %its {\em inter-derivability relation\/}
%${\equiv}_C\triangleq
$\{\couple{\phi}{\psi}\in\Fm_L^2\mid C(\phi)=C(\psi)\}\in\Con(\FA_L)$
[in which case the $L'$-fragment of $C$ is so].
 %as well as
%referred to as the$|$ {\em ``logic of''\/$|$``defined by''\/}
%a class $\msf{M}$ of {\em [logical] $L$-matrices\/}
%(viz., pairs constituted by $L$-algebras
%said to be their {\em underlying\/} ones,
%and subsets of the carriers of these),
%elements of which are called their {\em
%whenever $\{\inverse{h}[D]\mid\couple{\mf{A}}{D}\in\msf{M},
%h\in\hom(\FA_L,\mf{A})\}$ is a closure basis
%of $\img C$.
Next, $C$ is said to be {\em \{maximally\/\}/
$\neg$-paraconsistent/\/$\land$-conjunctive},
where $\neg/\land$ is a (secondary)
unary/binary connective of $L$, provided, for some/all
$\phi,\psi\in\Fm_L$, it holds that
$(\psi\not\in
C(\{\phi,\neg\phi\}))/(C(\{\phi,\psi\})=C(\phi\land\psi))$
\{and $C$ has no proper $\neg$-paraconsistent extension\}/,
%respectively,
an $L$-matrix being referred to as {\em
$\neg$-paraconsistent/\/$\land$-conjunctive},
whenever its logic is so.
Likewise, $C$ is said to be {\em /\/\{strongly\}
$\lor$-disjunctive/\/$\supset$-implicative},
where $\lor/\supset$ is a (secondary)
binary connective of $L$,
provided, for all
$\phi,\psi\in\Fm_L$ and all $X\subseteq\Fm_L$, it holds that
/\{both $(((\phi\supset\psi)\supset\phi)\supset\phi)\in
C(\varnothing)$ and\}
$(C(X\cup\{\phi\lor\psi\})=(C(X\cup\{\phi\})\cap C(X\cup\{\psi\})))/
((\psi\in C(X\cup\{\phi\}))\iff((\phi\supset\psi)\in C(X)))$.
%/(in which case it is $\lor_\supset$-disjunctive,
%where
%$(\phi\lor_\supset\psi)\triangleq((\phi\supset\psi)\supset\psi)$).
Then, an $L$-matrix
$\mc{A}=\couple{\mf{A}}{D}$ is said to be {\em
$\lor$-disjunctive/\/$\supset$-implicative},
provided,
for all $a,b\in A$, it holds that
$(((a\not\in/\in D)\Rightarrow(b\in
D)\iff((a(\lor/\supset)^\mf{A}b)\in D)$,
in which case %``it is $\lor_\supset$-disjunctive, while''/
its logic is /strongly so. %$\lor$-disjunctive/$\supset$-implicative.
%and so is the logic of any class of such $L$-matrices.
Further, %an $L$-matrix
$\mc{A}$ %=\couple{\mf{A}}{D}$
is said to be {\em [hereditarily-]simple/reduced},
provided, for each $\theta\in\Con(\mf{A})$
and every $\couple{a}{b}\in\theta$,
it holds that $(\theta[D]\subseteq D)\Rightarrow(a=b)$
[and $\mc{A}$ has no non-simple submatrix].
%Likewise, it is said to be {\em denumerably-generated}, whenever
%$\mf{A}$ is generated by a denumerable subset of $A$.
Finally, /``(a sublogic of) the logic of'' $\mc{A}$ is said to be {\em
$\neg$-%[super-]classical},
[super-/infra-](/sub)classical},
provided $A=(2[\cup\{\e\}])$,
where $(1\lceil+1\rceil)\triangleq\{0\lceil,1\rceil\}$,
while $(\neg^\mf{A}[\restr2])=\Delta^1_2$,
where $\Delta^{0\lfloor+1\rfloor}_2\triangleq\{\couple{i}{\lfloor1-\rfloor i}\mid i\in2\}$,
whereas $0\not\in D\ni1$,
in which case $\mc{A}$ is $\lor$-disjunctive/$\supset$-implicative,
whenever its logic is /[\{strongly\}] so /[and
\{non-\}$\neg$-paraconsistent].
%sublogics of $\neg$-classical $L$-logics being said to be
%{\em $\neg$-subclassical

\begin{theorem} %[Key Universal Characterization of Self-extensionality]
\label{main-thm}
Let\/ $\msf{M}$ be a [fi\-ni\-te] class of [finite %denumerably-ge\-ne\-ra\-ted
he\-re\-di\-ta\-ri\-ly-simple] $L$-matrices and\/ $C$ the logic of\/
$\msf{M}$.
[Suppose\/ $C$ is either strongly
%$\supset$-
implicative
%''\/$|$``
or both %\/ $\land$-
conjunctive
and either %\/ $\lor$-
disjunctive or infra-classical.]
%or\/ $\msf{M}$ consists of a single
%(conjunctive)\/ $\neg$-(super-)classical member.]
Then, $C$ is self-extensional if[f]\/
%for each\/
$\forall\mf{B}\in\pi_0[\msf{M}],\forall a\in B, \forall
b\in(B\setminus\{a\},
\exists\couple{\mf{A}}{D}\in\msf{M},\exists h\in\hom(\mf{B},\mf{A}): %such
(h(a)\in D)\iff(h(b)\not\in D)$.
\end{theorem}

This provides a quite effective algebraic criterion of the
self-extensionality of logics of the optional kind,
almost immediately yielding:
%As an almost immediate consequence of this fundamental
%universal result as well as Lemmas 3.2, 3.3 and
%Remark 3.6 of \cite{My-DB4}, we first have:
%the following new insight into


\begin{corollary}[Four-valued FDE expansions]
\label{SE-FDE}
Let $\mc{A}=\couple{\mf{A}}{D}$ be an $L$-matrix and\/
$C$ the logic of $\mc{A}$.
Suppose $L'\triangleq\{\land,\lor,\neg\}\subseteq L$,
while\/ $\mf{A}\restr L'$ is the non-Boolean De Morgan
diamond over\/ $2^2$ with zero/unit\/ $\couple{0/1}{0/1}$,
whereas $D=\{\couple{1}{j}\mid j\in2\}$
[as well as\/ $C$ is strongly implicative]. %(viz., $C$ is strongly so)].
%(2^2\cap\inverse{\pi}_0[\{1\}])$.
Then, $C$ is self-extensional [iff it is maximally\/
$\neg$-paraconsistent] iff
(it is\/ $\neg$-subclassical
%both\/ $\Delta^0_2\in\mbf{S}\mf{A}$
and)
$\{\couple{\couple{k}{l}}{\couple{l}{k}}\mid k,l\in2\}\in
\hom(\mf{A},\mf{A})$.
\end{corollary}

%This exhausts all {\em four-valued\/} expansions of
%Belnap's ``useful'' four-valued logic.
%positively/negatively covering
%``$L'$-conservative
%fragments/expansions of the purely classically-negative/``bi-lattice$|$specially-implicative'' one.

\begin{corollary}
\label{SE-3}
Let $\mc{A}=\couple{\mf{A}}{D}$ be a\/ $\neg$-super-classical $L$-matrix
and\/ $C$ the logic of $\mc{A}$.
Suppose\/ $C$ is
[non-]\/$\neg$-paraconsistent
%(viz., $\mc{A}$ is so)
as well as
``[strongly]\/
$\supset$-implicative''\/$|\!$``%$\land$-
con\-jun\-c\-ti\-ve\/ \{and
%$\lor$-
dis\-jun\-c\-ti\-ve\/\}
%$\lceil$viz., $\mc{A}$ is\/
%$\supset$-implicative''\/$|$``both\/ $\land$-conjunctive
%and\/ $\lor$-disjunctive''\/$\rceil$
[while\/ $\e\in/\not\in D$]''.
Then, $C$ is self-extensional iff [either it is two-valued
(viz., $\neg$-classical\/
$\lceil$i.e., $\mc{A}$ is not\/
$\lfloor$hereditarily-$\rfloor$simple\/$\rceil$) or]\/
$\,|\!$``for some $m\in(1[+(0/(1\{-1\}))])$''
$h\triangleq(\Delta^{1|m}_2\cup\{\couple{\e}{\e|(0[+(0/(1-(m\cdot\e)))])}\})\in\hom(\mf{A},\mf{A})$, in which
case\/ $\neg^\mf{A}\e=|\neq\e$, while\/
$\couple{\mf{A}}{\inverse{h}[D]}$ is a
%$\neg$-paraconsistent
model of\/ $C$
 %[satisfying the $L$-rule
``whereas, for all $a,b\in A$,
$(a\supset^\mf{A}b)=(\e\|b)$, whenever $a=\|\neq b$''\/$|$,
and so
``\/$\e\in D$
$\langle$in particular, $C$ is non-maximally\/
$\neg$-paraconsistent\/$\rangle$ as well as''\/$|\,$
$C$ is not\/$|\,$ $\neg$-subclassical.
%$2\not\in|\in\mbf{S}\mf{A}$.
\end{corollary}

This both yields a term-wise definitionally
minimal instance of a self-ex\-ten\-si\-o\-n\-al implicative
paraconsistent infra-classical logic
and provides a new and generic insight into
the [non-]self-extensionality of all
known infra-classical non-classical logics such as [Kleene's one,
$LP$, $HZ$ and $P^1$
(as well as their expansions  like \L{u}kasiewicz-style \{not necessarily, finitely-valued\}
logics)
but] G\"{o}del's %non-strongly
implicative non-paraconsistent
conjunctive disjunctive one
$\langle$so showing the necessity of the optional strength
stipulation$\rangle$
and its fragments, as well as into their
%well-known
strong non-implicativity. %of such
%self-extensional
%logics. %w.r.t. no secondary binary connective.



\begin{comment}
Finally, let $L\triangleq\{\neg,\land,\lor,\to\}$
with binary connectives other than $\neg$.
Then, a most representative negative instance of Theorem
\ref{main-thm} with the only distinguished/non-dis\-tin\-gu\-i\-sh\-ed value $1/0$ is
\L{}ukasiewicz'/Sugihara's three-valued logic, %\cite{Luk}/\cite{Sug},
being defined by a $\supset$-implicative,
where
$(\phi\supset\psi)\triangleq((\phi\to(\neg\phi/\psi))\lor\psi)$,
$(3,\neg)$-canonical $L$-matrix
such that $2$ forms a subalgebra of its underlying algebra.
On the other hand, the self-extensional G\"{o}del's three-valued
logic \cite{God} is $\to$-implicative and defined by a
$\neg$-canonical $L$-matrix with the only distinguished value
$1$ that both shows that the optional stipulation ``(strongly)''
is essential in Theorem \ref{main-thm} and provides a new insight
into the well-known  strong
implicativity of the logic w.r.t. no secondary binary connective.

\begin{thebibliography}{10}
\bibitem{God}
{\scshape K. G{\"o}del},
{\itshape Zum intuitionistischen {Aussagen\-kal\-k{\"u}l}},
{\bfseries\itshape An\-zei\-ger der Akademie der Wissenschaften im Wien},
vol.~69 (1932),
pp.~65--66.
\bibitem{Luk}
{\scshape J. {\L}ukasiewicz},
{\itshape O logice tr{\'o}jwarto{\'s}ciowej},
{\bfseries\itshape Ruch Filozoficzny},
vol.~5 (1920),
pp.~170--171.
\end{thebibliography}
\end{comment}


\vspace*{-0.5\baselineskip}
% this space adjustment is usually necessary after a bibliography

\end{document}


%% READ ME
%% READ ME
%% READ ME

INSTRUCTIONS FOR SUPPLYING INFORMATION IN THE CORRECT FORMAT:

1. Author names are listed as First Last, First Last, and First Last.

\absauth{FirstName1 LastName1, FirstName2 LastName2, and FirstName3 LastName3}


2. Titles of abstracts have ONLY the first letter capitalized,
except for Proper Nouns.

\meettitle{Title of abstract with initial capital letter only, except for
Proper Nouns}


3. Affiliations and email addresses for authors of abstracts are
  listed separately.

% First author's affiliation
\affil{Department, University, Street Address, Country}
\meetemail{First author's email}
%%% NOTE: email required for at least one author
\urladdr{OPTIONAL}
%
% Second author's affiliation
\affil{Department, University, Street Address, Country}
\meetemail{Second author's email}
\urladdr{OPTIONAL}
%
% Third author's affiliation
\affil{Department, University, Street Address, Country}
% Second author's email
\meetemail{Third author's email}
\urladdr{OPTIONAL}


4. Bibliographic Entries

%%%% IF references are submitted with abstract,
%%%% please use the following formats

%%% For a Journal article
\bibitem{cite1}
{\scshape Author's Name},
{\itshape Title of article},
{\bfseries\itshape Journal name spelled out, no abbreviations},
vol.~XX (XXXX), no.~X, pp.~XXX--XXX.

%%% For a Journal article by the same authors as above,
%%% i.e., authors in cite1 are the same for cite2
\bibitem{cite2}
\bysame
{\itshape Title of article},
{\bfseries\itshape Journal},
vol.~XX (XXXX), no.~X, pp.~XX--XXX.

%%% For a book
\bibitem{cite3}
{\scshape Author's Name},
{\bfseries\itshape Title of book},
Name of series,
Publisher,
Year.

%%% For an article in proceedings
\bibitem{cite4}
{\scshape Author's Name},
{\itshape Title of article},
{\bfseries\itshape Name of proceedings}
(Address of meeting),
(First Last and First2 Last2, editors),
vol.~X,
Publisher,
Year,
pp.~X--XX.

%%% For an article in a collection
\bibitem{cite5}
{\scshape Author's Name},
{\itshape Title of article},
{\bfseries\itshape Book title}
(First Last and First2 Last2, editors),
Publisher,
Publisher's address,
Year,
pp.~X--XX.

%%% An edited book
\bibitem{cite6}
Author's name, editor. % No special font used here
{\bfseries\itshape Title of book},
Publisher,
Publisher's
