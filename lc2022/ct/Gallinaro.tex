%% FIRST RENAME THIS FILE <yoursurname>.tex. 
%% BEFORE COMPLETING THIS TEMPLATE, SEE THE "READ ME" SECTION 
%% BELOW FOR INSTRUCTIONS. 
%% TO PROCESS THIS FILE YOU WILL NEED TO DOWNLOAD asl.cls from 
%% http://aslonline.org/abstractresources.html. 


\documentclass[bsl,meeting]{asl}

\AbstractsOn

\pagestyle{plain}

\def\urladdr#1{\endgraf\noindent{\it URL Address}: {\tt #1}.}


\newcommand{\NP}{}
%\usepackage{verbatim}

\begin{document}
\thispagestyle{empty}

%% BEGIN INSERTING YOUR ABSTRACT DIRECTLY BELOW; 
%% SEE INSTRUCTIONS (1), (2), (3), and (4) FOR PROPER FORMATS

\NP  
\absauth{Francesco Gallinaro}
\meettitle{Exponential sums equations and tropical geometry}
\affil{School of Mathematics, University of East Anglia, NR4 7TJ, United Kingdom}
\meetemail{mmfpg@leeds.ac.uk}

%% INSERT TEXT OF ABSTRACT DIRECTLY BELOW

In the late 1990s, Boris Zilber made a conjecture on the model theory of the exponential function, the Quasiminimality Conjecture (see \cite{Zil00}, \cite{Zil05}). This predicts that all subsets of the complex numbers that are definable using the language of rings and the exponential function are either countable or cocountable. He then proved that the conjecture would follow if the complex exponential field were a model of a certain theory in an infinitary logic.
Building on Zilber's work, Bays and Kirby have proved in \cite{BK18} that the Quasiminimality Conjecture would follow from just one of Zilber's axioms, the Exponential-Algebraic Closedness Conjecture, which predicts sufficient conditions for systems of equations in polynomials and exponentials to have complex solutions.
In this talk, I will give an introduction to this topic before presenting some recent work which solves the conjecture for a class of algebraic varieties which corresponds to systems of exponential sums. This turns out to be closely related to tropical geometry, a ``combinatorial shadow" of algebraic geometry which reduces some questions about algebraic varieties to questions about polyhedral objects.

\vskip 5mm

\begin{thebibliography}{10}

	\bibitem{BK18}
	{\scshape Martin Bays and Jonathan Kirby},
	{\itshape Pseudo-exponential maps, variants, and quasiminimality},
	{\bfseries\itshape Algebra \& Number Theory},
	vol.12 (2018), no.3, pp.493-549.


	\bibitem{Zil00}
	{\scshape Boris Zilber},
	{\itshape Analytic and pseudo-analytic structure},
	{\bfseries\itshape Lecture Notes in Logic, 19. Logic Colloquium 2000, Paris},
	(Rene Cori, Alexander Razborov, Stevo Todor\v{c}evi\'c, and Carol Wood, editors),
	Cambridge University Press,
	2005,
	pp.392-408.

	\bibitem{Zil05}
	{\scshape Boris Zilber},
	{\itshape Pseudo-exponentiation on algebraically closed fields},
	{\bfseries\itshape Annals of Pure and Applied Logic},
	vol.132 (2005), no.1, pp.67-95.

%% INSERT YOUR BIBLIOGRAPHIC ENTRIES HERE; 
%% SEE (4) BELOW FOR PROPER FORMAT.
%% EACH ENTRY MUST BEGIN WITH \bibitem{citation key}
%%
%% IF THERE ARE NO ENTRIES  
%% DELETE THE LINE ABOVE (\begin{thebibliography}{20}) 
%% AND THE LINE BELOW (\end{thebibliography})

\end{thebibliography}


\vspace*{-0.5\baselineskip}
% this space adjustment is usually necessary after a bibliography

\end{document}


%% READ ME
%% READ ME
%% READ ME

INSTRUCTIONS FOR SUPPLYING INFORMATION IN THE CORRECT FORMAT: 

1. Author names are listed as First Last, First Last, and First Last.

\absauth{FirstName1 LastName1, FirstName2 LastName2, and FirstName3 LastName3}


2. Titles of abstracts have ONLY the first letter capitalized,
except for Proper Nouns.

\meettitle{Title of abstract with initial capital letter only, except for
Proper Nouns} 


3. Affiliations and email addresses for authors of abstracts are
  listed separately.

% First author's affiliation
\affil{Department, University, Street Address, Country}
\meetemail{First author's email}
%%% NOTE: email required for at least one author
\urladdr{OPTIONAL}
%
% Second author's affiliation
\affil{Department, University, Street Address, Country}
\meetemail{Second author's email}
\urladdr{OPTIONAL}
%
% Third author's affiliation
\affil{Department, University, Street Address, Country}
% Second author's email
\meetemail{Third author's email}
\urladdr{OPTIONAL}


4. Bibliographic Entries

%%%% IF references are submitted with abstract,
%%%% please use the following formats

%%% For a Journal article
\bibitem{cite1}
{\scshape Author's Name},
{\itshape Title of article},
{\bfseries\itshape Journal name spelled out, no abbreviations},
vol.~XX (XXXX), no.~X, pp.~XXX--XXX.

%%% For a Journal article by the same authors as above,
%%% i.e., authors in cite1 are the same for cite2
\bibitem{cite2}
\bysame
{\itshape Title of article},
{\bfseries\itshape Journal},
vol.~XX (XXXX), no.~X, pp.~XX--XXX.

%%% For a book
\bibitem{cite3}
{\scshape Author's Name},
{\bfseries\itshape Title of book},
Name of series,
Publisher,
Year.

%%% For an article in proceedings
\bibitem{cite4}
{\scshape Author's Name},
{\itshape Title of article},
{\bfseries\itshape Name of proceedings}
(Address of meeting),
(First Last and First2 Last2, editors),
vol.~X,
Publisher,
Year,
pp.~X--XX.

%%% For an article in a collection
\bibitem{cite5}
{\scshape Author's Name},
{\itshape Title of article},
{\bfseries\itshape Book title}
(First Last and First2 Last2, editors),
Publisher,
Publisher's address,
Year,
pp.~X--XX.

%%% An edited book
\bibitem{cite6}
Author's name, editor. % No special font used here
{\bfseries\itshape Title of book},
Publisher,
Publisher's address,
Year.

