%% FIRST RENAME THIS FILE <yoursurname>.tex.
%% BEFORE COMPLETING THIS TEMPLATE, SEE THE "READ ME" SECTION
%% BELOW FOR INSTRUCTIONS.
%% TO PROCESS THIS FILE YOU WILL NEED TO DOWNLOAD asl.cls from
%% http://aslonline.org/abstractresources.html.


\documentclass[bsl,meeting]{asl}
\usepackage[greek,english]{babel}
\AbstractsOn

\pagestyle{plain}

\def\urladdr#1{\endgraf\noindent{\it URL Address}: {\tt #1}.}


\newcommand{\NP}{}
\newcommand{\ot}{\otimes}
\newcommand{\I}{\mathsf{I}}
\newcommand{\MILL}{\mathtt{MILL}}
\newcommand{\NMILL}{\mathtt{NMILL}}
%\usepackage{verbatim}
\usepackage{teubner}

\begin{document}
\thispagestyle{empty}

%% BEGIN INSERTING YOUR ABSTRACT DIRECTLY BELOW;
%% SEE INSTRUCTIONS (1), (2), (3), and (4) FOR PROPER FORMATS

%\NP
\absauth{Cheng-Syuan Wan}
\meettitle{Proof Theory of Skew Non-Commutative $\MILL$}
\affil{Department of Software Science, Tallinn University of Technology}
\meetemail{cswan@cs.ioc.ee}
%%% NOTE: email required for at least one author
%\urladdr{OPTIONAL}
%

%% INSERT TEXT OF ABSTRACT DIRECTLY BELOW
Monoidal closed categories are models of non-commutative multiplicative intuitionistic linear logic ($\NMILL$).
Skew monoidal closed categories are weak variants of monoidal closed categories \cite{cite2}.
In the skew cases, three natural isomorphisms $\lambda : \I \ot A \cong A$, $\rho : A \cong A \ot \I$, and $\alpha : (A \ot B) \ot C \cong A \ot (B \ot C)$ are merely natural transformations with a specific orientation.
In previous works by Uustalu et al. \cite{cite3} \cite{cite4}, proof theoretical analysis on skew monoidal categories and skew closed categories are investigated.
In particular, the sequent calculus systems modelled by skew monoidal and skew closed categories are respectively constructed.
Moreover, proof theoretical semantics of each system is provided according to Jean-Marc Andreoli's focusing technique \cite{cite1}.

Following the results above, a question arises: is it possible to construct a sequent calculus system naturally modelled by skew monoidal closed categories?
We answer the question positively by constructing a cut-free system $\NMILL^{s}$, a skew version of $\NMILL$.
Furthermore, we study the proof theoretical semantics of $\NMILL^{s}$.
The inspiration also originates from focusing, but we peculiarly employ tag annotations to keep tracking new formulae occurring in antecedent and reducing non-deterministic choices in bottom-up proof search.
Focusing solves the coherence problem of skew monoidal closed categories by providing a decision procedure to determine equality of maps in the free skew monoidal closed category.
% The decision procedure takes two morphisms in the free skew monoidal closed category then maps them into two derivations in the tagged sequent calculus.
% If two derivations are syntactically identical, then we know two original morphisms are equivalent, otherwise they are not.

This is joint work with Tarmo Uustalu (Reykjavik University) and Niccol{\`o} Veltri (Tallinn University of Technology).

\begin{thebibliography}{10}
\bibitem{cite1}
{\scshape Jean-Marc Andreoli},
{\itshape Logic Programming with Focusing Proofs in Linear Logic.},
{\bfseries\itshape Journal of Logic and Computation},
vol.2(3), pp.~297--347.
\bibitem{cite2}
{\scshape Ross Street},
{\itshape Skew-Closed Categories.},
{\bfseries\itshape Journal of Pure and Applied Algebra},
vol.217(6), pp.~973--988.
\bibitem{cite3}
{\scshape Tarmo Uustalu, Niccol{\`o} Veltri, and Noam Zeilberger},
{\itshape Deductive Systems and Coherence for Skew
Prounital Closed Categories.},
{\bfseries\itshape Eletronic Proceedings in Theoretical Computer Science},
vol.332, pp.~35--53.
\bibitem{cite4}
{\scshape Tarmo Uustalu, Niccol{\`o} Veltri, and Noam Zeilberger},
{\itshape The Sequent Calculus of Skew Monoidal Categories.},
{\bfseries\itshape Joachim Lambek: The Interplay of Mathematics,
Logic, and Linguistics}, pp.~377--406.
\end{thebibliography}


\vspace*{-0.5\baselineskip}
%% this space adjustment is usually necessary after a bibliography

\end{document}
