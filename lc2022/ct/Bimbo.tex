\documentclass[bsl,meeting]{asl}
\AbstractsOn
\pagestyle{plain}
\def\urladdr#1{\endgraf\noindent{\it URL Address}: {\tt #1}.}
\begin{document}
\thispagestyle{empty}

\absauth{Katalin Bimb\'o}
\meettitle{Relational semantics for some classical relevance logics}
\affil{Department of Philosophy, University of Alberta, 2--40 Assiniboia
Hall, Edmonton, AB T6G\thinspace2E7, Canada}
\meetemail{bimbo@ualberta.ca}
\urladdr{www.ualberta.ca/\char126bimbo}

The framework called \textit{generalized Galois logics} (or gaggle theory, 
for short) was introduced in [2] to encompass Kripke's semantics for modal 
and intuitionistic logics, J\'onsson \& Tarski's representation of BAO's
and the Meyer--Routley semantics for relevance logics among others.  In 
some cases, gaggle theory gives exactly the semantics defined earlier for 
a logic; in other cases, the semantics differ (cf.\ [3], [1]).  Relational
semantics for classical relevance logics such as $\mathbf{CR}$ and 
$\mathbf{CB}$ are usually defined as a modification of the Meyer--Routley 
semantics for $\mathbf{R}_+$ and $\mathbf{B}_+$, respectively (cf.\ [4]).  
In this talk, I compare the existing semantics for $\mathbf{CB}$ and 
$\mathbf{CR}$ to the semantics that results as an application of gaggle 
theory. 

\begin{thebibliography}{10}

\bibitem{BiDu2008}
{\scshape Bimb{\'o}, Katalin and J.~Michael Dunn},
{\bfseries\itshape Generalized Galois Logics: Relational Semantics of 
Nonclassical Logical Calculi}, CSLI Lecture Notes vol.~188, CSLI
Publications, Stanford, CA, 2008.

\bibitem{Du91}
{\scshape Dunn, J.~Michael},
{\itshape Gaggle theory: An abstraction of Galois connections and residuation,
with applications to negation, implication, and various logical operators},
{\bfseries\itshape Logics in AI: European Workshop JELIA~'90},
(J.~van Eijck, editor), Lecture Notes in Computer Science vol.~478, 
Springer, Berlin, 1991, pp.~31--51.

\bibitem{Du95a}
{\scshape Dunn, J.~Michael},
{\itshape Gaggle theory applied to intuitionistic, modal and relevance
logics}, 
{\bfseries\itshape Logik und Mathematik. Frege-Kolloquium Jena 1993},
(I.~Max and W.~Stelzner, editors),
W.~de Gruyter, Berlin, 1995, pp.~335--368.

\bibitem{Me2003}
{\scshape Meyer, Robert K.},
{\itshape Ternary relations and relevant semantics},
{\bfseries\itshape Annals of Pure and Applied Logic},
vol.~127 (2003), pp.~195--217.

\end{thebibliography}

\vspace*{-0.5\baselineskip}
% this space adjustment is usually necessary after a bibliography

\end{document}
