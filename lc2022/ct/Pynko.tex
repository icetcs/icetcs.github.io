%% FIRST RENAME THIS FILE <yoursurname>.tex.
%% BEFORE COMPLETING THIS TEMPLATE, SEE THE "READ ME" SECTION
%% BELOW FOR INSTRUCTIONS.
%% TO PROCESS THIS FILE YOU WILL NEED TO DOWNLOAD asl.cls from
%% http://aslonline.org/abstractresources.html.


\documentclass[bsl,meeting]{asl}

\usepackage{amsfonts}
\usepackage{amssymb}
%\usepackage[mathscr]{euscript}
\usepackage{verbatim}
%\usepackage{amsthm}

\AbstractsOn

\pagestyle{plain}

\def\urladdr#1{\endgraf\noindent{\it URL Address}: {\tt #1}.}


\newcommand{\NP}{}
%\usepackage{verbatim}

\newcommand{\mr}[1]{\mathrm{#1}}
\newcommand{\mf}[1]{\mathfrak{#1}}
\newcommand{\mc}[1]{\mathcal{#1}}
\newcommand{\mbf}[1]{\mathbf{#1}}
%\newcommand{\ms}[1]{\mathscr{#1}}
\newcommand{\msf}[1]{\mathsf{#1}}
\newcommand{\couple}[2]{\langle{#1},{#2}\rangle}
\newcommand{\FA}{\mf{Fm}}
\newcommand{\inverse}[1]{{#1}^{-1}}
\newcommand{\restr}{{\upharpoonright}}

\def\e{{\frac{1}{2}} }
\def\iff{\Leftrightarrow}

\DeclareMathOperator{\Fm}{Fm}
\DeclareMathOperator{\Con}{Con}
\DeclareMathOperator{\img}{img}
\DeclareMathOperator{\Cn}{Cn}


\begin{document}
\thispagestyle{empty}

%% BEGIN INSERTING YOUR ABSTRACT DIRECTLY BELOW;
%% SEE INSTRUCTIONS (1), (2), (3), and (4) FOR PROPER FORMATS

\NP
\absauth{Alexej Pynko}
\meettitle{Minimally $n$-valued maximally paraconsistent
%minimal %unary
expansions of $LP$}
%the logic of paradox}
\affil{Cybernetics Institute,
Glushkov p. 40, Kiev, 03680, Ukraine}
\meetemail{pynko@i.ua}

%% INSERT TEXT OF ABSTRACT DIRECTLY BELOW


Given any propositional language $L$
(viz., a set of propositional connectives,
treated as operation symbols, when dealing
with $L$-algebras),
a propositional $L$-logic $C$ %where $L$ is a propositional
%language, %/signature,
%constituted by {\em [propositional] connectives},
(viz., a structural %--- i.e, commuting
closure operator over the carrier $\Fm_L$
%constituted by {\em (propositional)\/ $L$-formulas\/}
%with variables in a countable set,
of the absolutely-free $L$-algebra $\FA_L$
freely-generated by the set $V\triangleq\{x_i\}_{i\in\omega}$
of propositional variables
$\langle$as usual, natural numbers, including $0$,
are treated as sets of lesser ones,
the set of all them being denoted by $\omega\rangle$)
%that is structural --- i.e., $\img C$
%commutes with inverse
%propositional $L$-substitutions
%$\langle$viz., endomorphisms of $\FA_L\rangle$)
is said to be {\em [\/\{uniformly/axiomatically\/\} minimally/maximally]\/
``\/$\lfloor$singularly\/$\rfloor$
$\lceil$no-more-than-$\rceil n$-valued''/$\neg$-paraconsisent},
whe\-re ``$n\in(\omega\setminus(1[+1])$''/``$\neg\in L$ is unary'',
provided ``$C$ is {\em defined\/} by a
$\lfloor$one-element$\rfloor$ class $\msf{M}$ of $\lceil$no-more-than-$\rceil n$-valued
{\em $L$-matrices\/}
(viz., pairs of %{\em underlying\/}
$L$-algebras and their
subsets)
--- i.e., $\{\inverse{h}[D]\mid\couple{\mf{A}}{D}\in\msf{M},
h\in\hom(\FA_L,\mf{A})\}$
is a closure basis of $\img C$ ---
[but is not \{singularly\} no-more-than-$(n-1)$-valued]''/``$x_1\not\in
C(\{x_0,\neg x_0\})$ [and $C$ has no $\neg$-paraconsistent
{\em extension\/} $C'$ (viz., an $L$-logic with $(\img C')\subseteq(\img C)$)
such that $C'\{(\varnothing)\}\neq C\{(\varnothing)\}$]'',
an $L$-matrix being said to be {\em $\neg$-paraconsistent},
whenever its (viz., defined by it) logic
is so.
Then, a {\em model of\/} $C$ is any $L$-matrix defining an
extension of $C$.


Let $n\in(\omega\setminus3)$,
$L_{+[-]}\triangleq\{\land,\lor[,\neg]\}$
the propositional language with binary connectives
[other than the unary one $\neg$],
$N_{n[-]}\triangleq\{i\in((n-1)\setminus1)\mid(2\cdot
i)\in(n[-1])\ni(4[-3])\}$,
$L_n\triangleq(L_{+-}\cup\{\partial_i\mid i\in N_{n-}\}\cup
\{\nabla_j\mid j\in N_n\})$
the propositional language with unary connectives
other than those in $L_+$,
$\mf{A}_n$ the $L_n$-algebra with $L_{+-}$-reduct
being the Kleene chain lattice under the natural ordering
on the carrier $n$ of $\mf{A}_n$ as well as operations
$\partial_i^{\mf{A}_n}\triangleq(((i+1)\times\{0\})\cup((n\setminus(i+1))\times\{n-1\}))$,
where $i\in N_{-n}$, and $\nabla_j^{\mf{A}_n}\triangleq
((((n-1)\setminus1)\times\{j\})\cup\{\couple{0}{0},\couple{n-1}{n-1}\})$,
where $j\in N_n$,
while $\mc{A}_n\triangleq\couple{\mf{A}_n}{D_n}$ the $L_n$-matrix
with $D_n\triangleq(n\setminus1)$, whereas
$C_n$ the logic of $\mc{A}_n$.
in which case this is $\neg$-paraconsistent,
while $(L|C)_3=(L_{+-}|LP)$ $|$(viz., the {\em logic of
paradox\/}),
whereas $((((n-1)\setminus1)\times\{1\})\cup\{\couple{0}{0},\couple{n-1}{n-1}\})
\in\hom(\mc{A}_n\restr L_3,\mc{A}_3)$ is both strict and
surjective, and so $C_n$ is an $n$-valued expansion of $LP$
(in particular, $LP$ is [non-minimally] $n$-valued [unless
$n=3$], the $L_+$-fragment of $C_n$ being that of $LP$
\{i.e., that of $PC$\}).

\begin{lemma} %[Key Lemma]
\label{key-lem}
For any\/ $\neg$-paraconsistent model\/ $\couple{\mf{A}}{D}$ of\/
$C_n$,
%has a submatrix of the form\/
there are some
subalgebra\/ $\mf{B}$ of\/ $\mf{A}$ with carrier\/ $B$
and some surjective $h\in\hom(\mf{B},\mf{A}_n)$
with\/ $(B\cap D)=\inverse{h}[D_n]$.
%from $$ onto $\mc{A}_n$.
\end{lemma}

As any $L$-logic is defined by the class of all its models,
Lemma \ref{key-lem} immediately yields:

\begin{theorem}
\label{main-thm}
$C_n$ is both minimally $n$-valued and maximally\/
$\neg$-paraconsistent.
\end{theorem}

On the other hand, elimination of any connective in $L_n\setminus
L_{+-}$ results in a fragment of $C_n$ that is either
not (even uniformly)
minimally $n$-valued or not
(even axiomatically) maximally $\neg$-paraconsistent.
More precisely, we have:

\begin{theorem}
\label{non-thm}
%/``Suppose $n>4$.''
The\/ $L'$-fragment\/ $C'$ of\/ $C_n$
with\/
$L'\subseteq|=(L_n\setminus\{(\partial/\nabla)_i\})$,
$i\in N_{(n-)|n}$,
%$\couple{\mf{A}_n\restr L'}{D_n}$
%Then, the logic of $\mc{A}'$
is not
uniformly\/$|$axiomatically minimally\/$|$maximally
$n$-valued\/$|\neg$-paraconsistent.
\end{theorem}

\begin{theorem}
\label{ext-thm}
Let\/ $C_n^\mr{NP/PC}$ be the $L_n$-logic defined by
``the direct product of $\mc{A}_n$ and''/
$\couple{\mf{A}_n\restr\{0,n-1\}}{\{n-1\}}$.
Then, these are the only proper\/ $\lceil$viz., distinct from\/ $C_n\rceil$
consistent\/ $\lfloor$viz., not defined by\/ $\varnothing\rfloor$ extensions of\/
$C_n$, while the former/latter is /``a proper extension of the former
as well as''
the least
non-$\neg$-paraconsistent/ extension /\/$C'$ of\/ $C_n$
/``such that $(x_1[|\neg x_0\supset(x_0\supset x_1)])\in C'(\{x_0\{\lor x_1\},\neg x_0\lor
x_1\}[|\varnothing])$ [whenever\/ $4\in n$,
where\/ $(x_0\supset x_1)\triangleq(\partial_1\nabla_1\neg x_0\lor
x_1)$]'', whereas\/
$C_n^\mr{NP}(\varnothing)=C_n(\varnothing)(=C_n^\mr{PC}(\varnothing)$ iff\/
$4\not\in n$).
\end{theorem}

\vspace*{-0.5\baselineskip}
% this space adjustment is usually necessary after a bibliography

\end{document}


%% READ ME
%% READ ME
%% READ ME

INSTRUCTIONS FOR SUPPLYING INFORMATION IN THE CORRECT FORMAT:

1. Author names are listed as First Last, First Last, and First Last.

\absauth{FirstName1 LastName1, FirstName2 LastName2, and FirstName3 LastName3}


2. Titles of abstracts have ONLY the first letter capitalized,
except for Proper Nouns.

\meettitle{Title of abstract with initial capital letter only, except for
Proper Nouns}


3. Affiliations and email addresses for authors of abstracts are
  listed separately.

% First author's affiliation
\affil{Department, University, Street Address, Country}
\meetemail{First author's email}
%%% NOTE: email required for at least one author
\urladdr{OPTIONAL}
%
% Second author's affiliation
\affil{Department, University, Street Address, Country}
\meetemail{Second author's email}
\urladdr{OPTIONAL}
%
% Third author's affiliation
\affil{Department, University, Street Address, Country}
% Second author's email
\meetemail{Third author's email}
\urladdr{OPTIONAL}


4. Bibliographic Entries

%%%% IF references are submitted with abstract,
%%%% please use the following formats

%%% For a Journal article
\bibitem{cite1}
{\scshape Author's Name},
{\itshape Title of article},
{\bfseries\itshape Journal name spelled out, no abbreviations},
vol.~XX (XXXX), no.~X, pp.~XXX--XXX.

%%% For a Journal article by the same authors as above,
%%% i.e., authors in cite1 are the same for cite2
\bibitem{cite2}
\bysame
{\itshape Title of article},
{\bfseries\itshape Journal},
vol.~XX (XXXX), no.~X, pp.~XX--XXX.

%%% For a book
\bibitem{cite3}
{\scshape Author's Name},
{\bfseries\itshape Title of book},
Name of series,
Publisher,
Year.

%%% For an article in proceedings
\bibitem{cite4}
{\scshape Author's Name},
{\itshape Title of article},
{\bfseries\itshape Name of proceedings}
(Address of meeting),
(First Last and First2 Last2, editors),
vol.~X,
Publisher,
Year,
pp.~X--XX.

%%% For an article in a collection
\bibitem{cite5}
{\scshape Author's Name},
{\itshape Title of article},
{\bfseries\itshape Book title}
(First Last and First2 Last2, editors),
Publisher,
Publisher's address,
Year,
pp.~X--XX.

%%% An edited book
\bibitem{cite6}
Author's name, editor. % No special font used here
{\bfseries\itshape Title of book},
Publisher,
Publisher's address,
Year.
