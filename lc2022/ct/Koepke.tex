%% FIRST RENAME THIS FILE <yoursurname>.tex. 
%% BEFORE COMPLETING THIS TEMPLATE, SEE THE "READ ME" SECTION 
%% BELOW FOR INSTRUCTIONS. 
%% TO PROCESS THIS FILE YOU WILL NEED TO DOWNLOAD asl.cls from 
%% http://aslonline.org/abstractresources.html. 


\documentclass[bsl,meeting]{asl}

\AbstractsOn

\pagestyle{plain}

\def\urladdr#1{\endgraf\noindent{\it URL Address}: {\tt #1}.}


\newcommand{\NP}{}
%\usepackage{verbatim}

\begin{document}
\thispagestyle{empty}

%% BEGIN INSERTING YOUR ABSTRACT DIRECTLY BELOW; 
%% SEE INSTRUCTIONS (1), (2), (3), and (4) FOR PROPER FORMATS

\NP  
\absauth{Peter Koepke}
\meettitle{Formalizing the Appendix of Kelley's General Topology in the
Naproche Proof Checker} 
\affil{Mathematical Institute, University of Bonn, Endenicher Allee 60, 53115 Bonn, Germany}
\meetemail{koepke@math.uni-bonn.de}

%% INSERT TEXT OF ABSTRACT DIRECTLY BELOW

The $\mathbb{N}$aproche natural language proof checker allows proof checked
mathematical formalizations that read like ordinary mathematical texts
\cite{Naproche1, Naproche}. The easiest way to use the $\mathbb{N}$aproche{} system 
is to install the
well-known Isabelle prover \cite{Isabelle2021-1} and to edit a file with a 
\verb+.ftl.tex+ 
suffix in the Isabelle/jEdit editor. Some example files are included with the
Isabelle distribution.

In my talk I shall describe a $\mathbb{N}$aproche formalization of the appendix of John L. Kelley's \textit{General Topology} \cite{Kelley}. The appendix is a short introduction
to Kelley-Morse set theory which is taken as the foundation for the book and 
which is developed up to some initial results on ordinals and cardinals.
Kelley's text lends itself to formalizations due to its formalistic 
and complete style of writing. Indeed many statements of the appendix can be
transferred almost verbatim to $\mathbb{N}$aproche. Considering, e.g., Kelley's
\begin{quotation}
38 THEOREM {\it If $x$ is a set, then $2^x$ is a set, and for each $y$, $y \subset x$
iff $y \ \varepsilon \  2^x$.}
\end{quotation}
about the power class $2^x$ of $x$, the obvious \LaTeX{} source is a legitimate 
$\mathbb{N}$aproche input, save for the 
commas:
\begin{verbatim}
\begin{theorem}
If $x$ is a set then $2^x$ is a set and for each $y$ $y \subset x$ 
iff $y \in 2^x$.
\end{theorem}
\end{verbatim}

Based on the Kelley formalization I shall conclude with general remarks on the
natural language of mathematics.


\begin{thebibliography}{10}

\bibitem{Kelley}
{\scshape John L. Kelley},
{\bfseries\itshape General Topology},
van Nostrand, 1955.

\bibitem{Naproche1}
{\scshape Adrian De Lon, Peter Koepke, Anton Lorenzen, Adrian Marti,
Marcel Sch{\"u}tz, Makarius Wenzel}
{\itshape The Isabelle/Naproche Natural Language Proof Assistant},
{\bfseries\itshape Automated Deduction -- CADE 28},
Springer LNAI, volume 12699, 614 -- 624,
2021.

\bibitem{Naproche}
Naproche source code on Github: 
\verb+https://github.com/naproche/naproche+


\bibitem{Isabelle2021-1}
{\scshape Isabelle contributors},
{\bfseries\itshape The Isabelle2021-1 release},\\
\verb+https://isabelle.in.tum.de/+,
December 2021.





%% INSERT YOUR BIBLIOGRAPHIC ENTRIES HERE; 
%% SEE (4) BELOW FOR PROPER FORMAT.
%% EACH ENTRY MUST BEGIN WITH \bibitem{citation key}
%%
%% IF THERE ARE NO ENTRIES  
%% DELETE THE LINE ABOVE (\begin{thebibliography}{20}) 
%% AND THE LINE BELOW (\end{thebibliography})

\end{thebibliography}


\vspace*{-0.5\baselineskip}
% this space adjustment is usually necessary after a bibliography

\end{document}


%% READ ME
%% READ ME
%% READ ME

INSTRUCTIONS FOR SUPPLYING INFORMATION IN THE CORRECT FORMAT: 

1. Author names are listed as First Last, First Last, and First Last.

\absauth{FirstName1 LastName1, FirstName2 LastName2, and FirstName3 LastName3}


2. Titles of abstracts have ONLY the first letter capitalized,
except for Proper Nouns.

\meettitle{Title of abstract with initial capital letter only, except for
Proper Nouns} 


3. Affiliations and email addresses for authors of abstracts are
  listed separately.

% First author's affiliation
\affil{Department, University, Street Address, Country}
\meetemail{First author's email}
%%% NOTE: email required for at least one author
\urladdr{OPTIONAL}
%
% Second author's affiliation
\affil{Department, University, Street Address, Country}
\meetemail{Second author's email}
\urladdr{OPTIONAL}
%
% Third author's affiliation
\affil{Department, University, Street Address, Country}
% Second author's email
\meetemail{Third author's email}
\urladdr{OPTIONAL}


4. Bibliographic Entries

%%%% IF references are submitted with abstract,
%%%% please use the following formats

%%% For a Journal article
\bibitem{cite1}
{\scshape Author's Name},
{\itshape Title of article},
{\bfseries\itshape Journal name spelled out, no abbreviations},
vol.~XX (XXXX), no.~X, pp.~XXX--XXX.

%%% For a Journal article by the same authors as above,
%%% i.e., authors in cite1 are the same for cite2
\bibitem{cite2}
\bysame
{\itshape Title of article},
{\bfseries\itshape Journal},
vol.~XX (XXXX), no.~X, pp.~XX--XXX.

%%% For a book
\bibitem{cite3}
{\scshape Author's Name},
{\bfseries\itshape Title of book},
Name of series,
Publisher,
Year.

%%% For an article in proceedings
\bibitem{cite4}
{\scshape Author's Name},
{\itshape Title of article},
{\bfseries\itshape Name of proceedings}
(Address of meeting),
(First Last and First2 Last2, editors),
vol.~X,
Publisher,
Year,
pp.~X--XX.

%%% For an article in a collection
\bibitem{cite5}
{\scshape Author's Name},
{\itshape Title of article},
{\bfseries\itshape Book title}
(First Last and First2 Last2, editors),
Publisher,
Publisher's address,
Year,
pp.~X--XX.

%%% An edited book
\bibitem{cite6}
Author's name, editor. % No special font used here
{\bfseries\itshape Title of book},
Publisher,
Publisher's address,
Year.

