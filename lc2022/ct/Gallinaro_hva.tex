\documentclass[bsl,meeting]{asl}
\begin{document}
In the late 1990s, Boris Zilber made a conjecture on the model theory of the exponential function, the Quasiminimality Conjecture (see \cite{Zil00}, \cite{Zil05}). This predicts that all subsets of the complex numbers that are definable using the language of rings and the exponential function are either countable or cocountable. He then proved that the conjecture would follow if the complex exponential field were a model of a certain theory in an infinitary logic.
Building on Zilber's work, Bays and Kirby have proved in \cite{BK18} that the Quasiminimality Conjecture would follow from just one of Zilber's axioms, the Exponential-Algebraic Closedness Conjecture, which predicts sufficient conditions for systems of equations in polynomials and exponentials to have complex solutions.
In this talk, I will give an introduction to this topic before presenting some recent work which solves the conjecture for a class of algebraic varieties which corresponds to systems of exponential sums. This turns out to be closely related to tropical geometry, a ``combinatorial shadow" of algebraic geometry which reduces some questions about algebraic varieties to questions about polyhedral objects.
\begin{thebibliography}{10}
	\bibitem{BK18}
	{\scshape Martin Bays and Jonathan Kirby},
	{\itshape Pseudo-exponential maps, variants, and quasiminimality},
	{\bfseries\itshape Algebra \& Number Theory},
	vol.12 (2018), no.3, pp.493-549.


	\bibitem{Zil00}
	{\scshape Boris Zilber},
	{\itshape Analytic and pseudo-analytic structure},
	{\bfseries\itshape Lecture Notes in Logic, 19. Logic Colloquium 2000, Paris},
	(Rene Cori, Alexander Razborov, Stevo Todor\v{c}evi\'c, and Carol Wood, editors),
	Cambridge University Press,
	2005,
	pp.392-408.

	\bibitem{Zil05}
	{\scshape Boris Zilber},
	{\itshape Pseudo-exponentiation on algebraically closed fields},
	{\bfseries\itshape Annals of Pure and Applied Logic},
	vol.132 (2005), no.1, pp.67-95.
\end{thebibliography}
\end{document}
