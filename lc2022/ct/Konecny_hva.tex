\documentclass[bsl,meeting]{asl}
\def\Ind#1#2{#1\setbox0=\hbox{$#1x$}\kern\wd0\hbox to 0pt{\hss$#1\mid$\hss}
\lower.9\ht0\hbox to 0pt{\hss$#1\smile$\hss}\kern\wd0}
\def\ind{\mathop{\mathpalette\Ind{}}}
\begin{document}
Study of big Ramsey degrees is an infinitary extension of the study of Ramsey classes. While being stated in a purely combinatorial manner, it is closely connected to model theory (the objects of study are homogeneous structures), topological dynamics (its results are used to construct \emph{universal completion flows} of automorphism groups) and set theory (the tools used are infinitary tree Ramsey theorems such as the Milliken theorem or the Carlson--Simpson theorem, as well as Harrington's application of the method of forcing).

It turns out that \emph{trees of types} are fundamental for understanding big Ramsey degrees. Given an enumeration of an ultrahomogeneous structure, the tree of $n$-types is formed by realised $n$-types over finite initial segments of the enumeration. For structures in binary languages, these trees have bounded branching, but this fails with relations of arity at least three, which makes the problems significantly more complicated.

Recently, we were able to show that an unconstrained homogeneous relational structure has finite big Ramsey degrees if and only if it is $\omega$-categorical (if and only if its tree of $n$-types is finitely branching for every $n$). In particular, this is the first time we were able to handle structures in infinite languages.

This is joint work with Samuel Braunfeld, David Chodounsk\'y, No\'e de Rancourt, Jan Hubi\v{c}ka and Jamal Kawach.
\end{document}