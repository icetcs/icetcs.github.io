%% FIRST RENAME THIS FILE <yoursurname>.tex.
%% BEFORE COMPLETING THIS TEMPLATE, SEE THE "READ ME" SECTION
%% BELOW FOR INSTRUCTIONS.
%% TO PROCESS THIS FILE YOU WILL NEED TO DOWNLOAD asl.cls from
%% http://aslonline.org/abstractresources.html.


\documentclass[bsl,meeting]{asl}
\usepackage{color}
\AbstractsOn

\pagestyle{plain}

\def\urladdr#1{\endgraf\noindent{\it URL Address}: {\tt #1}.}


\newcommand{\NP}{}
%\usepackage{verbatim}

\begin{document}
\thispagestyle{empty}

%% BEGIN INSERTING YOUR ABSTRACT DIRECTLY BELOW;
%% SEE INSTRUCTIONS (1), (2), (3), and (4) FOR PROPER FORMATS

\NP
\absauth{Aibat Yeshkeyev, Olga Ulbrikht, Maira Kassymetova, and Nazgul Shamatayeva}
\meettitle{Properties of lattices of existential formulas of Jonsson beautiful pairs}
\affil{Faculty of Mathematics and Information Technologies, Karaganda Buketov University, University str., 28, Building 2, Kazakhstan}
\meetemail{aibat.kz@gmail.com}
\meetemail{ulbrikht@mail.ru}
\meetemail{mairushaasd@mail.ru}
\meetemail{naz.kz85@mail.ru}

%% INSERT TEXT OF ABSTRACT DIRECTLY BELOW

Let $T$ be a convex $\exists$-complete perfect Jonsson theory of a countable language $L$, $C$ is its semantic model, $T^*=Th(C)$ is the center of theory $T$, $M=\bigcap M_i$, where $M_i\in E_T$, $M_i\subseteq C$ and $E_T$ is the class of existentially closed models of the theory $T$.

\begin{definition}
Let $N,M\in E_T$ and $M\preceq_{\exists_1} N$. We will call a pair $(N,M)$ a $J$-beautiful pair if it satisfies the following conditions:

1) $M$ is $|T|^+$-$\exists_1$-saturated;

2) for each tuple $\bar{b}$ extracted from $N$, each $\exists$-type over $M\cup\{\bar{b}\}$ is realized in $N$.
\end{definition}

Let class $K=\{(M_i,M)\mid M_i\preceq_{\exists_1}C \text{ and } (M_i,M) \text{ is $J$-beautiful pair}\}$. 
Consider the Jonsson spectrum of the class $K$:
\begin{center}
$JSp(K)=\{\Delta\mid\Delta \text{ is Jonsson theory and } \Delta=Th_{\forall\exists}(M_i,M) \text{, where } (M_i,M)\in K\}$.
\end{center}
It is easy to see that the cosemanticness relation on the set of Jonsson theories is an equivalence relation. Then we can consider the $JSp(K)/_{\bowtie}$, which is the factor set of the Jonsson spectrum of the class $K$ with respect to $\bowtie$.

Let $[\Delta]\in JSp(K)/_{\bowtie}$ and $E_n([\Delta])$ be the distributive lattice of equivalence classes of $\varphi^{[\Delta]}=\{\psi\in E_n(L)\mid [\Delta]^*\models \varphi\leftrightarrow\psi,\ \varphi\in E_n(L)\}$.

\begin{definition} Let $T$ be an arbitrary Jonsson theory, then a $\sharp$-companion of a theory $T$ is
a theory $T^\sharp$ of the same signature if it satisfies the following conditions: 

(i) $(T^\sharp)_\forall= T_\forall$;

(ii) if $T_\forall=T^\prime_\forall$, then $T^\sharp=(T^\prime)^\sharp$;

(iii) $T_{\forall\exists}\subseteq T^\sharp$.
\end{definition}
The natural interpretations of the companion $T^\sharp$ are $T^*$, $T^f$, $T^M$, $T^e$, $T^0$, where $T^*$ is the center of Jonsson theory $T$, $T^f$ is the forcing companion of Jonsson theory $T$, $T^M$ is the model
companion of the theory $T$, $T^e=Th(E_T)$, $T^0=Th_{\forall\exists}C$.

\begin{theorem}
Let $[\Delta]\in JSp(K)/_{\bowtie}$ be a perfect class, then the following conditions are equivalent:

(i) $[\Delta]^\sharp$ is complete theory;

(ii) $[\Delta]^\sharp$ is model complete theory.
\end{theorem}
\begin{theorem}
Let $[\Delta]\in JSp(K)/_{\bowtie}$ be a complete for $\exists$-sentences class. Then the following
conditions are equivalent:

(i) $[\Delta]$ is perfect;

(ii) $[\Delta]^*$is model-complete theory;

(iii) $E_n([\Delta])$ is a Boolean algebra.
\end{theorem}

\begin{theorem}
Let $[\Delta]\in JSp(K)/_{\bowtie}$ be a perfect $\forall\exists$-complete convex class, $[\Delta]^\sharp$ be its $\sharp$-companion. Then theory $[\Delta]^\sharp$ is $\omega$-categorical iff the class $[\Delta]$ is $\omega$-categorical.
\end{theorem}

All necessary concepts that are not defined in this thesis can be extracted from \cite{1}.

This research is funded by the Science Committee of the Ministry of Education and Science of the Republic of Kazakhstan (Grant No. AP09260237).

\begin{thebibliography}{10}

%% INSERT YOUR BIBLIOGRAPHIC ENTRIES HERE;
%% SEE (4) BELOW FOR PROPER FORMAT.
%% EACH ENTRY MUST BEGIN WITH \bibitem{citation key}
%%
%% IF THERE ARE NO ENTRIES
%% DELETE THE LINE ABOVE (\begin{thebibliography}{20})
%% AND THE LINE BELOW (\end{thebibliography})

%\bibitem{1}
%{\scshape Barwise J. (Ed.)},
%{\bfseries\itshape Handbook of mathematical logic},
%Part I. Model theory,
%Moscow, Nauka
%1982.


\bibitem{1}
{\scshape Yeshkeyev A.R., Kassymetova M.T.},
{\bfseries\itshape Jonsson theories and their classes of models},
Monograph,
Karaganda,
KSU,
2016.
\end{thebibliography}

\vspace*{-0.5\baselineskip}
% this space adjustment is usually necessary after a bibliography

\end{document}


%% READ ME
%% READ ME
%% READ ME

INSTRUCTIONS FOR SUPPLYING INFORMATION IN THE CORRECT FORMAT:

1. Author names are listed as First Last, First Last, and First Last.

\absauth{FirstName1 LastName1, FirstName2 LastName2, and FirstName3 LastName3}


2. Titles of abstracts have ONLY the first letter capitalized,
except for Proper Nouns.

\meettitle{Title of abstract with initial capital letter only, except for
Proper Nouns}


3. Affiliations and email addresses for authors of abstracts are
  listed separately.

% First author's affiliation
\affil{Department, University, Street Address, Country}
\meetemail{First author's email}
%%% NOTE: email required for at least one author
\urladdr{OPTIONAL}
%
% Second author's affiliation
\affil{Department, University, Street Address, Country}
\meetemail{Second author's email}
\urladdr{OPTIONAL}
%
% Third author's affiliation
\affil{Department, University, Street Address, Country}
% Second author's email
\meetemail{Third author's email}
\urladdr{OPTIONAL}


4. Bibliographic Entries

%%%% IF references are submitted with abstract,
%%%% please use the following formats

%%% For a Journal article
\bibitem{cite1}
{\scshape Author's Name},
{\itshape Title of article},
{\bfseries\itshape Journal name spelled out, no abbreviations},
vol.~XX (XXXX), no.~X, pp.~XXX--XXX.

%%% For a Journal article by the same authors as above,
%%% i.e., authors in cite1 are the same for cite2
\bibitem{cite2}
\bysame
{\itshape Title of article},
{\bfseries\itshape Journal},
vol.~XX (XXXX), no.~X, pp.~XX--XXX.

%%% For a book
\bibitem{cite3}
{\scshape Author's Name},
{\bfseries\itshape Title of book},
Name of series,
Publisher,
Year.

%%% For an article in proceedings
\bibitem{cite4}
{\scshape Author's Name},
{\itshape Title of article},
{\bfseries\itshape Name of proceedings}
(Address of meeting),
(First Last and First2 Last2, editors),
vol.~X,
Publisher,
Year,
pp.~X--XX.

%%% For an article in a collection
\bibitem{cite5}
{\scshape Author's Name},
{\itshape Title of article},
{\bfseries\itshape Book title}
(First Last and First2 Last2, editors),
Publisher,
Publisher's address,
Year,
pp.~X--XX.

%%% An edited book
\bibitem{cite6}
Author's name, editor. % No special font used here
{\bfseries\itshape Title of book},
Publisher,
Publisher's address,
Year.

