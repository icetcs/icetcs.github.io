%% FIRST RENAME THIS FILE <yoursurname>.tex. 
%% BEFORE COMPLETING THIS TEMPLATE, SEE THE "READ ME" SECTION 
%% BELOW FOR INSTRUCTIONS. 
%% TO PROCESS THIS FILE YOU WILL NEED TO DOWNLOAD asl.cls from 
%% http://aslonline.org/abstractresources.html. 


\documentclass[bsl,meeting]{asl}

\AbstractsOn

\pagestyle{plain}

\def\urladdr#1{\endgraf\noindent{\it URL Address}: {\tt #1}.}


\newcommand{\NP}{}
%\usepackage{verbatim}

\begin{document}
\thispagestyle{empty}

%% BEGIN INSERTING YOUR ABSTRACT DIRECTLY BELOW; 
%% SEE INSTRUCTIONS (1), (2), (3), and (4) FOR PROPER FORMATS

\NP  
\absauth{Maria Beatrice Buonaguidi}
\meettitle{Symmetry, locality and hyperintensionality}
\affil{Philosophy Department, King's College London, Strand WC2R 2LS, UK}
\meetemail{maria.buonaguidi@kcl.ac.uk}

%% INSERT TEXT OF ABSTRACT DIRECTLY BELOW
The notion of hyperintensionality and has become of prime importance in contemporary research, to the point that Nolan \cite[p. 149]{Nolan2014-NOLHM} predicted a “hyperintensional revolution" for the 21st century. 
In a recent paper, Odintsov and Wansing \cite{Odintsov2021-ODIRSA} criticize the claim to hyperintensionality of Leitgeb's logic HYPE \cite{Leitgeb2019-LEIHAS}, and suggest a definition of hyperintensionality based on a logic's consequence relation \cite[p. 51]{Odintsov2021-ODIRSA}. According to this criterion, HYPE is not hyperintensional, but merely intensional. However, the picture is not as clear-cut as it seems. Indeed, we can distinguish three notions of hyperintensionality, corresponding to different informal definitions in the hyperintensional metaphysics literature: Odintsov and Wansing only formulate one of these criteria. However, HYPE can be assessed also with respect to the other two. I show it to be hyperintensional with respect to the last, and weakest, criterion. We cannot therefore say that HYPE is hyperintensional proper, but we can say that it is strongly intensional, and not just intensional. In fact, it is precisely the fact that HYPE is strongly intensional and not merely intensional allows to successfully model hyperintensional operators in HYPE-models. This shows that, although HYPE's consequence relation is well-behaved enough to make it stronger than most non-classical logics, HYPE's semantics is especially powerful. 

I note another result showing this feature. Leitgeb argued that HYPE has the disjunction property \cite[p. 346]{Leitgeb2019-LEIHAS}. However, as noted by Odintsov and Wansing \cite[p. 43]{Odintsov2021-ODIRSA}, HYPE can be shown to be equivalent to the logic $N^*_i$, which has been shown not to have the disjunction property \cite[p. 400]{Drobyshevich2015}. I show that HYPE does not have the disjunction property at the level of its consequence relation, but that the disjunction property holds in all HYPE-models.

\begin{thebibliography}{10}
\bibitem{Drobyshevich2015}
{\scshape Sergey A. Drobyshevich},
{\itshape Double negation operator in logic $N^*$},
{\bfseries\itshape Journal of mathematical sciences},
vol.~205 (2015), no.~3, pp.~389--402.

\bibitem{Leitgeb2019-LEIHAS}
{\scshape Hannes Leitgeb},
{\itshape HYPE: a system of hyperintensional logic},
{\bfseries\itshape Journal of philosophical logic},
vol.~48 (2019), no.~2, pp.~305--405. 

\bibitem{Nolan2014-NOLHM}
{\scshape Daniel Nolan},
{\itshape Hyperintensional metaphysics},
{\bfseries\itshape Philosophical Studies},
vol.~171 (2014), no.~1, pp.~149--160.

\bibitem{Odintsov2021-ODIRSA}
{\scshape Sergey Odintsov and Heinrich Wansing},
{\itshape Routley star and hyperintensionality},
{\bfseries\itshape Journal of philosophical logic},
vol.~50 (2021), no.~1, pp.~39--56.

%% INSERT YOUR BIBLIOGRAPHIC ENTRIES HERE; 
%% SEE (4) BELOW FOR PROPER FORMAT.
%% EACH ENTRY MUST BEGIN WITH \bibitem{citation key}
%%
%% IF THERE ARE NO ENTRIES  
%% DELETE THE LINE ABOVE (\begin{thebibliography}{20}) 
%% AND THE LINE BELOW (\end{thebibliography})

\end{thebibliography}


\vspace*{-0.5\baselineskip}
% this space adjustment is usually necessary after a bibliography

\end{document}


%% READ ME
%% READ ME
%% READ ME

INSTRUCTIONS FOR SUPPLYING INFORMATION IN THE CORRECT FORMAT: 

1. Author names are listed as First Last, First Last, and First Last.

\absauth{FirstName1 LastName1, FirstName2 LastName2, and FirstName3 LastName3}


2. Titles of abstracts have ONLY the first letter capitalized,
except for Proper Nouns.

\meettitle{Title of abstract with initial capital letter only, except for
Proper Nouns} 


3. Affiliations and email addresses for authors of abstracts are
  listed separately.

% First author's affiliation
\affil{Department, University, Street Address, Country}
\meetemail{First author's email}
%%% NOTE: email required for at least one author
\urladdr{OPTIONAL}
%
% Second author's affiliation
\affil{Department, University, Street Address, Country}
\meetemail{Second author's email}
\urladdr{OPTIONAL}
%
% Third author's affiliation
\affil{Department, University, Street Address, Country}
% Second author's email
\meetemail{Third author's email}
\urladdr{OPTIONAL}


4. Bibliographic Entries

%%%% IF references are submitted with abstract,
%%%% please use the following formats

%%% For a Journal article
\bibitem{cite1}
{\scshape Author's Name},
{\itshape Title of article},
{\bfseries\itshape Journal name spelled out, no abbreviations},
vol.~XX (XXXX), no.~X, pp.~XXX--XXX.

%%% For a Journal article by the same authors as above,
%%% i.e., authors in cite1 are the same for cite2
\bibitem{cite2}
\bysame
{\itshape Title of article},
{\bfseries\itshape Journal},
vol.~XX (XXXX), no.~X, pp.~XX--XXX.

%%% For a book
\bibitem{cite3}
{\scshape Author's Name},
{\bfseries\itshape Title of book},
Name of series,
Publisher,
Year.

%%% For an article in proceedings
\bibitem{cite4}
{\scshape Author's Name},
{\itshape Title of article},
{\bfseries\itshape Name of proceedings}
(Address of meeting),
(First Last and First2 Last2, editors),
vol.~X,
Publisher,
Year,
pp.~X--XX.

%%% For an article in a collection
\bibitem{cite5}
{\scshape Author's Name},
{\itshape Title of article},
{\bfseries\itshape Book title}
(First Last and First2 Last2, editors),
Publisher,
Publisher's address,
Year,
pp.~X--XX.

%%% An edited book
\bibitem{cite6}
Author's name, editor. % No special font used here
{\bfseries\itshape Title of book},
Publisher,
Publisher's address,
Year.

