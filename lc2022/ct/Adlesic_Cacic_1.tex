%% FIRST RENAME THIS FILE <yoursurname>.tex. 
%% BEFORE COMPLETING THIS TEMPLATE, SEE THE "READ ME" SECTION 
%% BELOW FOR INSTRUCTIONS. 
%% TO PROCESS THIS FILE YOU WILL NEED TO DOWNLOAD asl.cls from 
%% http://aslonline.org/abstractresources.html. 


\documentclass[bsl,meeting]{asl}

\AbstractsOn

\pagestyle{plain}

\def\urladdr#1{\endgraf\noindent{\it URL Address}: {\tt #1}.}


\newcommand{\NP}{}
%\usepackage{verbatim}

\begin{document}
\thispagestyle{empty}

%% BEGIN INSERTING YOUR ABSTRACT DIRECTLY BELOW; 
%% SEE INSTRUCTIONS (1), (2), (3), and (4) FOR PROPER FORMATS

\NP  
\absauth{Tin Adle\v si\'c, and Vedran \v Ca\v ci\'c}
\meettitle{Formalizing assignment of types to terms in \textsf{NFU}}
\affil{Faculty of Teacher Education, University of Zagreb, Savska cesta 77, 10000 Zagreb, Croatia}
\meetemail{tin.adlesic@ufzg.hr}
\affil{Department of Mathematics, Faculty of Science, University of Zagreb, Bijeni\v cka cesta 30, 10000 Zagreb, Croatia}
\meetemail{veky@math.hr}

%% INSERT TEXT OF ABSTRACT DIRECTLY BELOW

Quine's \emph{New Foundations for mathematical logic} from 1937 (upgraded later in 1951) was originally meant as a theory for rigorously formalizing \emph{classes}: collections of objects satisfying certain predicates. In order to avoid the usual paradoxes, Quine stipulated that objects be \emph{stratified}---divided according to types, so that a class is of a higher type than its elements. In search of consistency proof, \textbullet\,proper classes were expelled (to metatheory) leaving only the theory of \emph{sets}, the theory was reworded so \textbullet\,stratification became \emph{syntactic} (the assignment of natural numbers to variables in the formula expressing the defining predicate), and also \textbullet\,\emph{urelements} (``atoms'' without elements, while not being equal to the empty class) were added. The resulting theory, \textsf{NFU}, was shown to be consistent by Jensen in 1969.

In a way, working in \textsf{NFU} seems a lot like working in any ``usual'' set theory---until it comes to the point where it's necessary to justify the existence of a certain set. And there are a lot of such situations: for example, proving a claim by mathematical induction amounts to showing that a certain set is inductive---but first we must ensure it is a set. Checking that a formula is stratified is a straightforward, if boring, task in the basic $\{\in,=\}$-language of set theory, but it becomes much more challenging when we add abstraction terms, functional and new relational symbols. To help us work, we have programmed a framework in Coq which can be used to establish whether the formula is stratified, find the \emph{least} typization, and use that fact in establishing new notions.

\vspace*{-0.5\baselineskip}
% this space adjustment is usually necessary after a bibliography

\end{document}


%% READ ME
%% READ ME
%% READ ME

INSTRUCTIONS FOR SUPPLYING INFORMATION IN THE CORRECT FORMAT: 

1. Author names are listed as First Last, First Last, and First Last.

\absauth{FirstName1 LastName1, FirstName2 LastName2, and FirstName3 LastName3}


2. Titles of abstracts have ONLY the first letter capitalized,
except for Proper Nouns.

\meettitle{Title of abstract with initial capital letter only, except for
Proper Nouns} 


3. Affiliations and email addresses for authors of abstracts are
  listed separately.

% First author's affiliation
\affil{Department, University, Street Address, Country}
\meetemail{First author's email}
%%% NOTE: email required for at least one author
\urladdr{OPTIONAL}
%
% Second author's affiliation
\affil{Department, University, Street Address, Country}
\meetemail{Second author's email}
\urladdr{OPTIONAL}
%
% Third author's affiliation
\affil{Department, University, Street Address, Country}
% Second author's email
\meetemail{Third author's email}
\urladdr{OPTIONAL}


4. Bibliographic Entries

%%%% IF references are submitted with abstract,
%%%% please use the following formats

%%% For a Journal article
\bibitem{cite1}
{\scshape Author's Name},
{\itshape Title of article},
{\bfseries\itshape Journal name spelled out, no abbreviations},
vol.~XX (XXXX), no.~X, pp.~XXX--XXX.

%%% For a Journal article by the same authors as above,
%%% i.e., authors in cite1 are the same for cite2
\bibitem{cite2}
\bysame
{\itshape Title of article},
{\bfseries\itshape Journal},
vol.~XX (XXXX), no.~X, pp.~XX--XXX.

%%% For a book
\bibitem{cite3}
{\scshape Author's Name},
{\bfseries\itshape Title of book},
Name of series,
Publisher,
Year.

%%% For an article in proceedings
\bibitem{cite4}
{\scshape Author's Name},
{\itshape Title of article},
{\bfseries\itshape Name of proceedings}
(Address of meeting),
(First Last and First2 Last2, editors),
vol.~X,
Publisher,
Year,
pp.~X--XX.

%%% For an article in a collection
\bibitem{cite5}
{\scshape Author's Name},
{\itshape Title of article},
{\bfseries\itshape Book title}
(First Last and First2 Last2, editors),
Publisher,
Publisher's address,
Year,
pp.~X--XX.

%%% An edited book
\bibitem{cite6}
Author's name, editor. % No special font used here
{\bfseries\itshape Title of book},
Publisher,
Publisher's address,
Year.

