%% FIRST RENAME THIS FILE <yoursurname>.tex. 
%% BEFORE COMPLETING THIS TEMPLATE, SEE THE "READ ME" SECTION 
%% BELOW FOR INSTRUCTIONS. 
%% TO PROCESS THIS FILE YOU WILL NEED TO DOWNLOAD asl.cls from 
%% http://aslonline.org/abstractresources.html. 


\documentclass[bsl,meeting]{asl}

\AbstractsOn

\pagestyle{plain}

\def\urladdr#1{\endgraf\noindent{\it URL Address}: {\tt #1}.}


\newcommand{\NP}{}
%\usepackage{verbatim}

\begin{document}
\thispagestyle{empty}

%% BEGIN INSERTING YOUR ABSTRACT DIRECTLY BELOW; 
%% SEE INSTRUCTIONS (1), (2), (3), and (4) FOR PROPER FORMATS

\NP  
% \title{}

\absauth{Bruno Bentzen}
\meettitle{Can propositions be intentions, intuitionistically?} %Logic Colloquium 2022}
\affil{School of Philosophy, Zhejiang University, 866 Yuhangtang Rd, China}
\meetemail{bbentzen@zju.edu.cn}

%% INSERT TEXT OF ABSTRACT DIRECTLY BELOW

Dissatisfaction with the philosophical thought of L.~E.~J. Brouwer has led to a growing interest over the last few decades in the
support of his intuitionism from a phenomenological approach, building on ideas from Husserl. The main supporters of this interpretation are Richard Tieszen and Mark van Atten. It is rooted in Heyting's idea that a proposition is an intention which is fulfilled with a proof-object of it. 

%Its 
%point of departure of is the replacement of Brouwer's notion of intuition with
%Husserl's analysis of intuition in terms of fulfillment of intentions. 
%The roots of the idea can be traced mainly to Brouwer's student Arend Heyting's use
%of a Husserlian account of intuition, following a suggestion made by Husserl's student Oskar Becker, in the meaning
%explanations proposed for the intuitionistic logical constants in the early 1930s. 
%The explanation proposed by Heyting is that a mathematical proposition is an
%intention which is fulfilled with a proof-object of that proposition. 


In this talk I argue against this propositions-as-intentions interpretation. I must stress at the outset that the interpretation is already a target of harsh criticisms regarding the incompatibility of Brouwer's and Husserl's positions, mainly from Guillermo Rosado Haddock or Claire Hill. But their objection consists in denying the interpretation its major premise. 
%%
This is not the direction I wish to take in this talk. Instead, I object that even if we grant that the incompatibility can be properly dealt with, as van Atten believes it can, one fundamental issues remain: it is far from clear what the object of an intention corresponding to a proposition should be. I argue that Heyting's own suggestion is inadequate and the most plausible candidates for intentional objects are sets of canonical proof-objects of the propositions. But this thesis immediately leads us to a difficult fulfillment dilemma: for Husserl, an intention is fulfilled when the intended object is genuinely presented to us in just the way it is intended; but here only one element of the set, not the set itself, can fulfill the intention. I conclude that the propositions-as-intentions leads to undesirable consequences. 

%\begin{thebibliography}{10}

%% INSERT YOUR BIBLIOGRAPHIC ENTRIES HERE; 
%% SEE (4) BELOW FOR PROPER FORMAT.
%% EACH ENTRY MUST BEGIN WITH \bibitem{citation key}
%%
%% IF THERE ARE NO ENTRIES  
%% DELETE THE LINE ABOVE (\begin{thebibliography}{20}) 
%% AND THE LINE BELOW (\end{thebibliography})

%\end{thebibliography}


\vspace*{-0.5\baselineskip}
% this space adjustment is usually necessary after a bibliography

\end{document}


%% READ ME
%% READ ME
%% READ ME


