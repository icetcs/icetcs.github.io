%
\documentclass[runningheads]{llncs}
%
\usepackage{graphicx}

%
% If you use the hyperref package, please uncomment the following line
% to display URLs in blue roman font according to Springer's eBook style:
% \renewcommand\UrlFont{\color{blue}\rmfamily}


%own packages & commands
\usepackage[textsize=tiny]{todonotes}
\usepackage{amsfonts}
\usepackage{amsmath}
\usepackage{amssymb}



\begin{document}
% 

\title{Between the Model-Theoretic and the Axiomatic Method of Characterizing Mathematical Truth}

\author{Micha\l\, Tomasz Godziszewski\inst{1,2}\orcidID{0000-0002-3907-2242} }
\institute{University of Warsaw \and  University of \L\'od\'z \\
\email{mtgodziszewski@gmail.com}}



\maketitle

\begin{abstract}
The so-called model-theoretic method of characterizing the notion of truth consists in defining a general
notion of a model of a given formal language L, providing a definition of a binary relation between models
of L and the sentences of L, and finally singling out a concrete model as the standard or the intended one
and declaring that truth simpliciter (of sentences of L) should be understood as truth in this model. Can
we really treat this model-theoretic definition of truth as the definition of (mathematical) truth (say, at
least with respect to the language of arithmetic)?
There are at least two serious problems with this method, one of which is that even if we might assume that our metatheory
can provide us with a determinate concept of the standard model, then the question is: does it
follow that then the concept of truth is definite, complete, determinate or absolute?
In what follows, we provide an analysis of these two particular questions regarding the use of the
notion of standard model in the model-theoretic characterization of the notion of mathematical truth
simpliciter, leading to results that can be interpreted as delivering the following message: not only there
are conceptual problems regarding the way standard models are used in the characterization, but there 
are philosophically justified mathematical reasons for which the appeal to standard models in truth-
theoretic constructions is at least problematic, if not impossible, and therefore, if one's goal is to provide a formal theory of mathematical truth simpliciter, the axiomatic framework is the right method of doing so.

%We conclude with a section describing an application of the axiomatic method of characterizing truth
%to set theory taken as the base theory. By using a result of Gitman and Hamkins we suggest that
%our result characterizing the class of models of ZFC expandable to models of the theory of the so-called
%Compositional Truth allows for an essentially truth-theoretic argument in favor of pluralism in philosophy
%of set theory.
\end{abstract}




\end{document}