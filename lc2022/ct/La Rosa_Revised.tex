%% FIRST RENAME THIS FILE <yoursurname>.tex. 
%% BEFORE COMPLETING THIS TEMPLATE, SEE THE "READ ME" SECTION 
%% BELOW FOR INSTRUCTIONS. 
%% TO PROCESS THIS FILE YOU WILL NEED TO DOWNLOAD asl.cls from 
%% http://aslonline.org/abstractresources.html. 


\documentclass[bsl,meeting]{asl}

\AbstractsOn

\pagestyle{plain}

\def\urladdr#1{\endgraf\noindent{\it URL Address}: {\tt #1}.}


\newcommand{\NP}{}
%\usepackage{verbatim}

\begin{document}
\thispagestyle{empty}

%% BEGIN INSERTING YOUR ABSTRACT DIRECTLY BELOW; 
%% SEE INSTRUCTIONS (1), (2), (3), and (4) FOR PROPER FORMATS

\NP  
\absauth{Elio {La Rosa}}
\meettitle{Normalization of epsilon calculus}
\affil{MCMP, LMU Munich}
\meetemail{lrslei@gmail.com}

%% INSERT TEXT OF ABSTRACT DIRECTLY BELOW


Rule-based reformulations of Hilbert and Bernays' epsilon calculus have been attempted in an effort to provide Gentzen-style normalization procedures. The problem, however, is an open one \cite{Zac17}. In this contribution, a normalization procedure is developed for a system of epsilon calculus based on a structural extension of natural deduction allowing for multiple conclusions. The base propositional system has explicit structural rules and introduction and elimination rules that are both local and `symmetric', in the sense that they do not depend on assumptions and are allowed to branch the derivation upwards and downwards, respectively. The somewhat complicated structure of the derivations in the system, similar to that of \cite{Ung92}, is mitigated by the fact that branching and rule applications can be reinterpreted in a system of `open deduction' \cite{GGP10}. The rules introducing epsilon terms are a simple rule reformulation of the ``critical formulas'' found in the axiomatic version of epsilon calculus.  A normalization procedure is easy to formulate for the propositional part, which allows for a first elimination of detours in the derivation. For what concerns detours caused by applications of rules introducing epsilon terms which do not appear as premises or conclusions in the derivation, reductions based on replacement of epsilon terms indexed by rule instances are defined. The procedure yields the first epsilon theorem and provides a form of the subformula property for the calculus.

\begin{thebibliography}{10}

%% INSERT YOUR BIBLIOGRAPHIC ENTRIES HERE; 
%% SEE (4) BELOW FOR PROPER FORMAT.
%% EACH ENTRY MUST BEGIN WITH \bibitem{citation key}
%%
%% IF THERE ARE NO ENTRIES  
%% DELETE THE LINE ABOVE (\begin{thebibliography}{20}) 
%% AND THE LINE BELOW (\end{thebibliography})

%%% For an article in proceedings
\bibitem{GGP10}
{\scshape Alessio Guglielmi, Tom Gundersen, Michel Parigot},
{\itshape A proof calculus which reduces syntactic bureaucracy},
{\bfseries\itshape 21st International Conference on Rewriting Techniques and Applications}
(Dagstuhl, Germany),
(Christopher Lynch, editor),
vol.~6,
Schloss Dagstuhl--Leibniz-Zentrum fuer Informatik,
2010,
pp.~135--150.

\bibitem{Ung92}
{\scshape Anthony M. Ungar},
{\bfseries\itshape Normalization, cut-elimination, and the theory of proofs},
CLSI Lecture Notes,
Center for the Study of Language and Information,
1992.

\bibitem{Zac17}
{\scshape Richard Zach},
{\itshape Semantics and proof theory of the epsilon calculus},
{\bfseries\itshape 7th Indian Conference on Logic and its Applications}
(Kanpur, India),
(Sujata Ghosh and Sanjiva Prasad, editors),
vol.~10119,
Springer,
2017,
pp.~27--47.
\end{thebibliography}


\vspace*{-0.5\baselineskip}
% this space adjustment is usually necessary after a bibliography

\end{document}

