% Abstract for LC 2022
\documentclass[bsl,meeting]{asl}
\AbstractsOn
\pagestyle{plain}
\def\urladdr#1{\endgraf\noindent{\it URL Address}: {\tt #1}.}
\newcommand{\NP}{}
%\usepackage{verbatim}
\begin{document}
\thispagestyle{empty}
\NP  
\absauth{Joachim Mueller-Theys}
\meettitle{The inhomogeneity of concepts}
\affil{Heidelberg, Germany}
\meetemail{mueller-theys@gmx.de}
\par 
We may think of 
$ P , Q , \ldots \subseteq M $
as properties or concepts. 
$ P $ 
is {\it total} 
:iff 
$ P = M $, 
$ P $ 
is {\it vacuous} 
:iff 
$ P = \emptyset $. 
$ P $ 
is {\it singular} 
:iff 
$ | P \, | = 1 $.
We 
naturally call 
$ P $ 
{\it genuine} 
iff 
$ P $ 
is neither 
vacuous 
nor
singular.  
For 
$ a , b , \ldots \in M $, 
$ P ( a ) $ :iff $ a \in P $. 
We have defined
$ P ${\em -similarity} 
$ a \sim _P b $
by
$ P ( a ) \  \& \  P ( b ) $
and 
$ P ${\em -equivalence} 
$ a \equiv_P b $
by 
$ P ( a ) \Leftrightarrow P ( b ) $.
The basic connection is 
$ \sim _P \  \subseteq \  \equiv _P $. 
Following Leibniz, 
$ a \equiv ^\ast b $ 
$ : \Leftrightarrow $
$ \forall P \subseteq M $
$ a \equiv _P b $
and 
$ a \equiv ^\ast b $
$ \Leftrightarrow $
$ a = b $.  
\par
Phrases of the form 
``$ P $ {\it is} $ P $'',
like 
``human is human'',
may be precisefied by 
$ P ( a ) \,  \& \,  P ( b ) \Rightarrow a \equiv _P b $. 
That's the case 
$ Q : = P $ 
of 
``all $ P $ are $ Q $-equal'', 
viz. 
$ {\it AllEq} _Q (P) $
$ : \Leftrightarrow $
$ \forall a , b \in P \  a \equiv _Q b $.
We had already found and proven 
the Theorem: 
$ {\rm AllEq} _Q (P) $ 
$ \Leftrightarrow $ 
$ {\rm Hom} _Q ( P ) $, 
whereby
homogeneity
$ {\it Hom} _Q ( P ) $ 
is defined by
$ P \subseteq Q 
\vee 
P \subseteq - Q $.
Particularly,
$ {\rm AllEq} _P ( M ) $ 
$ \Leftrightarrow $ 
$ {\rm Hom} _P ( M ) $
$ \Leftrightarrow $ 
$ M \subseteq P \vee M \subseteq - P $
$ \Leftrightarrow $ 
$ P = M \vee P = \emptyset $
$ \Leftrightarrow $ 
($ P $ non-vacuous 
$ \Rightarrow $ 
$ P $ total). 
A sophisticated interpretation of 
``$ P $ {\it is} $ P $''
might now be that 
all $ P $ are equal with respect to all homogeneous $ Q $. 
On the contrary, 
$ {\it Inhom} _Q ( P ) $
$ :\Leftrightarrow $
$ \neg {\rm Hom} _Q ( P ) $
coincides 
with 
{\it heterogeneity}
$ {\it Het} _Q ( P ) $
$ :\Leftrightarrow $
$ P \cap Q \neq \emptyset $ 
$ \& $ 
$ P \cap - Q \neq \emptyset $. 
\par
Sayings of the form 
``$ P $ {\it is not} $ P $'', 
like 
``human is not human'', 
``wine is not wine'', 
``element is not element'', 
may be formalised 
through the existence of {\it differentiating} 
$ Q $,
viz.  
$ {\it Uneq} _Q ( P ) $
$ : \Leftrightarrow $
$ \neg {\rm AllEq} _Q ( P) $,
viz. 
$ P ( a ) \  \& \  P ( b ) $, 
but 
$ a \not\equiv _Q b $
for some 
$ a $, 
$ b $\,: 
Mulder is not criminal, whereas Counter is;
Chasselas is white, while Primitivo is not; 
plutonium is transuranic, while selenium is not.  
\par
Differentiating properties always exist. 
By the Theorem, 
$ {\rm Uneq} _Q ( P ) $
iff 
$ {\rm Inhom} _Q ( P ) $.
Accordingly, 
we call 
$ P $
{\it inhomogeneous} 
iff 
$ {\rm Inhom} _Q ( P ) $
for some 
$ Q $. 
\\
{\sc Theorem}. 
{\it 
All genuine properties are inhomogeneous}.
\\
Proof. 
Let 
$ P $ 
be genuine,
whence 
$ | P | \geq 2 $, 
whereby 
$ P ( a ) $, 
$ P ( b ) $
for some
$ a \neq b $.
By the latter
and  
Leibniz's theorem, 
$ a \not\equiv _Q b $
for some 
$ Q $. 
Thus
$ {\rm Uneq} _Q (P) $, 
whence 
$ {\rm Inhom} _Q ( P ) $. 
\par 
Note. 
For previous achievements and acknowledgments, 
see 
``Similarity and equality'' 
(abstract \& talk, 
2021 ASL Annual Meeting (cf. Long Program), 
The Bulletin of Symbolic Logic 27 (2021), p. 329),
``Mathematical theorems on equality and unequality''
(abstract, 
Logic Colloquium 2021, by title). 
\end{document}