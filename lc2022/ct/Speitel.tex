%% FIRST RENAME THIS FILE <yoursurname>.tex. 
%% BEFORE COMPLETING THIS TEMPLATE, SEE THE "READ ME" SECTION 
%% BELOW FOR INSTRUCTIONS. 
%% TO PROCESS THIS FILE YOU WILL NEED TO DOWNLOAD asl.cls from 
%% http://aslonline.org/abstractresources.html. 


\documentclass[bsl,meeting]{asl}

\AbstractsOn

\pagestyle{plain}

\def\urladdr#1{\endgraf\noindent{\it URL Address}: {\tt #1}.}


\newcommand{\NP}{}
%\usepackage{verbatim}

\begin{document}
\thispagestyle{empty}


\NP  
\absauth{Sebastian G.W. Speitel}
\meettitle{Arithmetic via Carnap-categoricity}
\affil{Institute of Philosophy, University of Bonn}
\meetemail{sgwspeitel@uni-bonn.de}

The existence of non-standard models of first-order Peano-arithmetic (PA) has long been taken to undermine the claim of the mathematical realist that determinate reference to the natural number structure is possible in a non-mysterious, naturalistically acceptable way. The use of logics stronger that FOL to achieve a categorical theory of arithmetic and resolve this referential indeterminacy has been criticised as merely pushing the issue `one level up' into the meta-theory of these logics. This, the model-theoretic sceptic claims, is due to the fact that the resources needed to formulate these logics are just as much in need of justification as reference to the natural number structure itself. 

In \cite{BonnaySpeitel2021} we outlined and defended a novel criterion of logicality based on the idea that logical notions must be \emph{formal} (invariant under isomorphisms) as well as \emph{categorical} (uniquely determinable by inference). A notion satisfying this criterion was termed \emph{Carnap-categorical}. In this talk, I want to show that our criterion offers an attractive and well-motivated answer to the sceptical challenge advanced against the mathematical realist. The reply is based on the Carnap-categoricity of the generalised quantifier ``there are infinitely many'' ($\mathcal{Q}_{0}$). It is this property of $\mathcal{Q}_{0}$ which allows us to successfully mitigate the objection that the indeterminacy affecting reference to the natural number structure simply re-arises at the level of the meta-theory of the logic used to provide a categorical axiomatization of that structure.

I compare this approach with other attempts to justify the move to stronger logics found in the literature and argue that the proposal based on Carnap-categoricity is more robust and thus preferable. I conclude by reflecting on the scope of the response to the sceptical challenge and the remaining sources of indeterminacy.

\begin{thebibliography}{10}

\bibitem{BonnaySpeitel2021}
{\scshape D. Bonnay, S.G.W. Speitel},
{\itshape The Ways of Logicality: Invariance and Categoricity},
{\bfseries\itshape The Semantic Conception of Logic. Essays on Consequence, Invariance, and Meaning}
(G. Sagi and J. Woods, editors),
Cambridge University Press,
2021,
pp.~55--79.

\end{thebibliography}


\vspace*{-0.5\baselineskip}

\end{document}


