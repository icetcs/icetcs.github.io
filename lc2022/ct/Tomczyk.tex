%% FIRST RENAME THIS FILE <yoursurname>.tex. 
%% BEFORE COMPLETING THIS TEMPLATE, SEE THE "READ ME" SECTION 
%% BELOW FOR INSTRUCTIONS. 
%% TO PROCESS THIS FILE YOU WILL NEED TO DOWNLOAD asl.cls from 
%% http://aslonline.org/abstractresources.html. 


\documentclass[bsl,meeting]{asl}

\AbstractsOn

\pagestyle{plain}

\def\urladdr#1{\endgraf\noindent{\it URL Address}: {\tt #1}.}


\newcommand{\NP}{}
%\usepackage{verbatim}

\begin{document}
\thispagestyle{empty}

%% BEGIN INSERTING YOUR ABSTRACT DIRECTLY BELOW; 
%% SEE INSTRUCTIONS (1), (2), (3), and (4) FOR PROPER FORMATS

\NP  
\absauth{Agata Tomczyk}
\meettitle{Sequent Calculus for non-Fregean Boolean theory \textsf{WB}}
\affil{Adam Mickiewicz University, Department of Logic and Cognitive Science}
\meetemail{agata.tomczyk@amu.edu.pl}

The aim of the talk is to present Sequent Calculus for \textsf{WB}---a Boolean extension of the weakest non-Fregean logic \textsf{SCI} (\textit{Sentential Calculus with Identity}) proposed by Roman Suszko \cite{suszko1975}. In \textsf{WB} we consider identity connective `$\equiv$' based on one introduced in \textsf{SCI}. However, \textsf{WB} consists of more tautological identities than \textsf{SCI}, where the only tautological identity was of the form $\phi \equiv \phi$.  In case of \textsf{WB}, $\phi \equiv \chi$ is a tautology if and only if  $\phi \leftrightarrow \chi$ is a tautology of Classical Propositional Calculus. To formalize this notion we introduce proof system $\mathsf{G3}_{\mathsf{WB}}$ (based on ${\ell\mathbf{G}3}_\mathsf{SCI}$ found in \cite{chlebowski2018}) in which each sequent is labelled with marker allowing (or disabling) the use of certain identity-dedicated rules. We will discuss correctness  and invertibility of the proposed rule set and identify issues regarding the cut elimination procedure. We will also discuss ideas concerning sequent calculus for \textsf{WT}, a topological Boolean algebra of situations \cite{suszko1975}.

\begin{thebibliography}{10}

\bibitem{chlebowski2018}
{\scshape Szymon Chlebowski},
{\itshape Sequent Calculi for SCI},
{\bfseries\itshape Studia Logica},
vol.~106 (2018), no.~3, pp.~541--563.

\bibitem{suszko1975}
{\scshape Roman Suszko},
{\itshape Abolition of the Fregean Axiom},
{\bfseries\itshape Lecture Notes in Mathematics},
vol.~453 (1975), pp.~169--239.

\end{thebibliography}


\vspace*{-0.5\baselineskip}
% this space adjustment is usually necessary after a bibliography

\end{document}


%% READ ME
%% READ ME
%% READ ME

INSTRUCTIONS FOR SUPPLYING INFORMATION IN THE CORRECT FORMAT: 

1. Author names are listed as First Last, First Last, and First Last.

\absauth{FirstName1 LastName1, FirstName2 LastName2, and FirstName3 LastName3}


2. Titles of abstracts have ONLY the first letter capitalized,
except for Proper Nouns.

\meettitle{Title of abstract with initial capital letter only, except for
Proper Nouns} 


3. Affiliations and email addresses for authors of abstracts are
  listed separately.

% First author's affiliation
\affil{Department, University, Street Address, Country}
\meetemail{First author's email}
%%% NOTE: email required for at least one author
\urladdr{OPTIONAL}
%
% Second author's affiliation
\affil{Department, University, Street Address, Country}
\meetemail{Second author's email}
\urladdr{OPTIONAL}
%
% Third author's affiliation
\affil{Department, University, Street Address, Country}
% Second author's email
\meetemail{Third author's email}
\urladdr{OPTIONAL}


4. Bibliographic Entries

%%%% IF references are submitted with abstract,
%%%% please use the following formats

%%% For a Journal article
\bibitem{suszko1975}
{\scshape Roman Suszko},
{\itshape Abolition of the Fregean Axiom},
{\bfseries\itshape Lecture Notes in Mathematics},
vol.~453 (1975), pp.~169--239.

%%% For a Journal article by the same authors as above,
%%% i.e., authors in cite1 are the same for cite2
\bibitem{chlebowski2018}
{\scshape Szymon Chlebowski},
{\itshape Sequent Calculi for SCI},
{\bfseries\itshape Studia Logica},
vol.~106 (2018), no.~3, pp.~541--563.

%%% For a book
\bibitem{cite3}
{\scshape Author's Name},
{\bfseries\itshape Title of book},
Name of series,
Publisher,
Year.

%%% For an article in proceedings
\bibitem{cite4}
{\scshape Author's Name},
{\itshape Title of article},
{\bfseries\itshape Name of proceedings}
(Address of meeting),
(First Last and First2 Last2, editors),
vol.~X,
Publisher,
Year,
pp.~X--XX.

%%% For an article in a collection
\bibitem{cite5}
{\scshape Author's Name},
{\itshape Title of article},
{\bfseries\itshape Book title}
(First Last and First2 Last2, editors),
Publisher,
Publisher's address,
Year,
pp.~X--XX.

%%% An edited book
\bibitem{cite6}
Author's name, editor. % No special font used here
{\bfseries\itshape Title of book},
Publisher,
Publisher's address,
Year.

