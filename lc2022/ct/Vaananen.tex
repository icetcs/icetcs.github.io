%% FIRST RENAME THIS FILE <yoursurname>.tex. 
%% BEFORE COMPLETING THIS TEMPLATE, SEE THE "READ ME" SECTION 
%% BELOW FOR INSTRUCTIONS. 
%% TO PROCESS THIS FILE YOU WILL NEED TO DOWNLOAD asl.cls from 
%% http://aslonline.org/abstractresources.html. 


\documentclass[bsl,meeting]{asl}

\AbstractsOn

\pagestyle{plain}

\def\urladdr#1{\endgraf\noindent{\it URL Address}: {\tt #1}.}


\newcommand{\NP}{}
%\usepackage{verbatim}

\begin{document}
\thispagestyle{empty}

%% BEGIN INSERTING YOUR ABSTRACT DIRECTLY BELOW; 
%% SEE INSTRUCTIONS (1), (2), (3), and (4) FOR PROPER FORMATS

\NP  
\absauth{Lauri Hella, Kerkko Luosto, and Jouko V\"a\"an\"anen}
\meettitle{Dimension in team semantics}
\affil{Tampere University}
\meetemail{lauri.hella@tuni.fi}
\affil{Tampere University}
\meetemail{kerkko.luosto@tuni.fi}
\affil{University of Helsinki}
\meetemail{jouko.vaananen@helsinki.fi}

%% INSERT TEXT OF ABSTRACT DIRECTLY BELOW
We introduce three measures of complexity for families of sets. 
Each of the three measures, that we call dimensions, is defined in terms of the minimal number of 
convex subfamilies that are needed for covering the given family: 
for upper dimension, the subfamilies are required to contain a unique 
maximal set, for dual upper dimension a unique minimal set, and for cylindrical 
dimension both a unique maximal and a unique minimal set. In addition to considering dimensions of 
particular families of sets we study the behaviour of dimensions under operators that map families
of sets to new families of sets. We identify natural sufficient criteria for such operators to preserve the
growth class of the dimensions. 

We apply the theory of our dimensions for proving new hierarchy results for logics with 
team semantics. First, we show that the standard logical operators preserve the growth classes
of the families arising from the semantics of formulas in such logics. Second, we show that the upper
dimension of $k+1$-ary dependence, inclusion, independence, anonymity, and exclusion atoms is in a strictly
higher growth class than that of any $k$-ary atoms, whence the $k+1$-ary atoms are not definable 
in terms of any atoms of smaller arity.

Related and earlier work: \cite{ciardelli09}\cite{HLSV}\cite{HS}\cite{LVil}.

%\bibliographystyle{plain}
%\bibliography{hlv22}
%\end{document}

\begin{thebibliography}{10}

%% INSERT YOUR BIBLIOGRAPHIC ENTRIES HERE; 
%% SEE (4) BELOW FOR PROPER FORMAT.
%% EACH ENTRY MUST BEGIN WITH \bibitem{citation key}
%%
%% IF THERE ARE NO ENTRIES  
%% DELETE THE LINE ABOVE (\begin{thebibliography}{20}) 
%% AND THE LINE BELOW (\end{thebibliography})

\bibitem{ciardelli09}
Ivano Ciardelli.
\newblock Inquisitive semantics and intermediate logics.
\newblock Master's thesis, University of Amsterdam, 2009.

\bibitem{HLSV}
Lauri Hella, Kerkko Luosto, Katsuhiko Sano, and Jonni Virtema.
\newblock The expressive power of modal dependence logic.
\newblock In {\em Advances in modal logic. {V}ol. 10}, pages 294--312. Coll.
  Publ., London, 2014.

\bibitem{HS}
Lauri Hella and Johanna Stumpf.
\newblock The expressive power of modal logic with inclusion atoms.
\newblock In {\em Proceedings {S}ixth {I}nternational {S}ymposium on {G}ames,
  {A}utomata, {L}ogics and {F}ormal {V}erification}, volume 193 of {\em
  Electron. Proc. Theor. Comput. Sci. (EPTCS)}, pages 129--143. EPTCS, [place
  of publication not identified], 2015.

\bibitem{LVil}
Martin L\"{u}ck and Miikka Vilander.
\newblock On the succinctness of atoms of dependency.
\newblock {\em Log. Methods Comput. Sci.}, 15(3):Paper No. 17, 28, 2019.

\end{thebibliography}



\vspace*{-0.5\baselineskip}
% this space adjustment is usually necessary after a bibliography

\end{document}


%% READ ME
%% READ ME
%% READ ME

INSTRUCTIONS FOR SUPPLYING INFORMATION IN THE CORRECT FORMAT: 

1. Author names are listed as First Last, First Last, and First Last.

\absauth{FirstName1 LastName1, FirstName2 LastName2, and FirstName3 LastName3}


2. Titles of abstracts have ONLY the first letter capitalized,
except for Proper Nouns.

\meettitle{Title of abstract with initial capital letter only, except for
Proper Nouns} 


3. Affiliations and email addresses for authors of abstracts are
  listed separately.

% First author's affiliation
\affil{Department, University, Street Address, Country}
\meetemail{First author's email}
%%% NOTE: email required for at least one author
\urladdr{OPTIONAL}
%
% Second author's affiliation
\affil{Department, University, Street Address, Country}
\meetemail{Second author's email}
\urladdr{OPTIONAL}
%
% Third author's affiliation
\affil{Department, University, Street Address, Country}
% Second author's email
\meetemail{Third author's email}
\urladdr{OPTIONAL}


4. Bibliographic Entries

%%%% IF references are submitted with abstract,
%%%% please use the following formats

%%% For a Journal article
\bibitem{cite1}
{\scshape Author's Name},
{\itshape Title of article},
{\bfseries\itshape Journal name spelled out, no abbreviations},
vol.~XX (XXXX), no.~X, pp.~XXX--XXX.

%%% For a Journal article by the same authors as above,
%%% i.e., authors in cite1 are the same for cite2
\bibitem{cite2}
\bysame
{\itshape Title of article},
{\bfseries\itshape Journal},
vol.~XX (XXXX), no.~X, pp.~XX--XXX.

%%% For a book
\bibitem{cite3}
{\scshape Author's Name},
{\bfseries\itshape Title of book},
Name of series,
Publisher,
Year.

%%% For an article in proceedings
\bibitem{cite4}
{\scshape Author's Name},
{\itshape Title of article},
{\bfseries\itshape Name of proceedings}
(Address of meeting),
(First Last and First2 Last2, editors),
vol.~X,
Publisher,
Year,
pp.~X--XX.

%%% For an article in a collection
\bibitem{cite5}
{\scshape Author's Name},
{\itshape Title of article},
{\bfseries\itshape Book title}
(First Last and First2 Last2, editors),
Publisher,
Publisher's address,
Year,
pp.~X--XX.

%%% An edited book
\bibitem{cite6}
Author's name, editor. % No special font used here
{\bfseries\itshape Title of book},
Publisher,
Publisher's address,
Year.

