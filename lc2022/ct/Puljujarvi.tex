%% FIRST RENAME THIS FILE <yoursurname>.tex. 
%% BEFORE COMPLETING THIS TEMPLATE, SEE THE "READ ME" SECTION 
%% BELOW FOR INSTRUCTIONS. 
%% TO PROCESS THIS FILE YOU WILL NEED TO DOWNLOAD asl.cls from 
%% http://aslonline.org/abstractresources.html. 


\documentclass[bsl,meeting]{asl}

\usepackage[utf8]{inputenc}
\AbstractsOn

\pagestyle{plain}

\def\urladdr#1{\endgraf\noindent{\it URL Address}: {\tt #1}.}


\newcommand{\NP}{}
%\usepackage{verbatim}

\begin{document}
\thispagestyle{empty}

%% BEGIN INSERTING YOUR ABSTRACT DIRECTLY BELOW; 
%% SEE INSTRUCTIONS (1), (2), (3), and (4) FOR PROPER FORMATS

\NP  
\absauth{Åsa Hirvonen and Joni Puljujärvi}
\meettitle{Games and Scott sentences for positive distances between metric structures}
\affil{Department of Mathematics and Statistics, University of Helsinki, Finland}
\meetemail{joni.puljujarvi@helsinki.fi}

%% INSERT TEXT OF ABSTRACT DIRECTLY BELOW


We develop an abstract framework of infinitary logic, inspired by Henson's positive bounded logic with approximations~\cite{cite1}, and various related Ehrenfeucht--Fraïssé games that we then use to study distances defined on classes of (unbounded) metric structures. We show that the second player has a winning strategy in an EF game of length $\omega$ and precision $\varepsilon>0$ between two separable structures exactly when the distance between the structures is $<\varepsilon$. Using tools from Scott analysis, we then obtain Scott sentences that can express the distance being $<\varepsilon$.

We study two forms of distances: pseudometrics stemming from mapping spaces onto each other with some form of approximate isomorphism, and metrics stemming from measuring the distances between two spaces isometrically embedded into a third space. Our main example of the former notion is the linear isomorphism between Banach spaces, with a fixed bi-Lipschitz constant, and of the latter notion the Kadets distance between Banach spaces.

Unlike in classical Scott analysis or Scott analysis for bounded metric structures with continuous infinitary logic~\cite{cite2}, the traditional methods seem to only give us Scott sentences in $\mathcal{L}_{\omega_2\omega}$ rather than in $\mathcal{L}_{\omega_1\omega}$. However, if we are only concerned with distance $0$, we obtain a variant of the result of~\cite{cite2}, i.e. Scott sentences in $\mathcal{L}_{\omega_1\omega}$.


\begin{thebibliography}{10}

%% INSERT YOUR BIBLIOGRAPHIC ENTRIES HERE; 
%% SEE (4) BELOW FOR PROPER FORMAT.
%% EACH ENTRY MUST BEGIN WITH \bibitem{citation key}
%%
%% IF THERE ARE NO ENTRIES  
%% DELETE THE LINE ABOVE (\begin{thebibliography}{20}) 
%% AND THE LINE BELOW (\end{thebibliography})

\bibitem{cite2}
{\scshape  Itaï Ben Yaacov, Michal Doucha, André Nies, and Todor Tsankov},
{\itshape Metric Scott analysis},
{\bfseries\itshape Advances in Mathematics},
vol.~318 (2017), %no.~X,
pp.~46--87.

\bibitem{cite1}
{\scshape C. Ward Henson},
{\itshape Nonstandard hulls of banach spaces},
{\bfseries\itshape Israel Journal of Mathematics},
vol.~25 (1976), no.~1--2, pp.~108--144.

\end{thebibliography}


\vspace*{-0.5\baselineskip}
% this space adjustment is usually necessary after a bibliography

\end{document}


%% READ ME
%% READ ME
%% READ ME

INSTRUCTIONS FOR SUPPLYING INFORMATION IN THE CORRECT FORMAT: 

1. Author names are listed as First Last, First Last, and First Last.

\absauth{FirstName1 LastName1, FirstName2 LastName2, and FirstName3 LastName3}


2. Titles of abstracts have ONLY the first letter capitalized,
except for Proper Nouns.

\meettitle{Title of abstract with initial capital letter only, except for
Proper Nouns} 


3. Affiliations and email addresses for authors of abstracts are
  listed separately.

% First author's affiliation
\affil{Department, University, Street Address, Country}
\meetemail{First author's email}
%%% NOTE: email required for at least one author
\urladdr{OPTIONAL}
%
% Second author's affiliation
\affil{Department, University, Street Address, Country}
\meetemail{Second author's email}
\urladdr{OPTIONAL}
%
% Third author's affiliation
\affil{Department, University, Street Address, Country}
% Second author's email
\meetemail{Third author's email}
\urladdr{OPTIONAL}


4. Bibliographic Entries

%%%% IF references are submitted with abstract,
%%%% please use the following formats

%%% For a Journal article
\bibitem{cite1}
{\scshape Author's Name},
{\itshape Title of article},
{\bfseries\itshape Journal name spelled out, no abbreviations},
vol.~XX (XXXX), no.~X, pp.~XXX--XXX.

%%% For a Journal article by the same authors as above,
%%% i.e., authors in cite1 are the same for cite2
\bibitem{cite2}
\bysame
{\itshape Title of article},
{\bfseries\itshape Journal},
vol.~XX (XXXX), no.~X, pp.~XX--XXX.

%%% For a book
\bibitem{cite3}
{\scshape Author's Name},
{\bfseries\itshape Title of book},
Name of series,
Publisher,
Year.

%%% For an article in proceedings
\bibitem{cite4}
{\scshape Author's Name},
{\itshape Title of article},
{\bfseries\itshape Name of proceedings}
(Address of meeting),
(First Last and First2 Last2, editors),
vol.~X,
Publisher,
Year,
pp.~X--XX.

%%% For an article in a collection
\bibitem{cite5}
{\scshape Author's Name},
{\itshape Title of article},
{\bfseries\itshape Book title}
(First Last and First2 Last2, editors),
Publisher,
Publisher's address,
Year,
pp.~X--XX.

%%% An edited book
\bibitem{cite6}
Author's name, editor. % No special font used here
{\bfseries\itshape Title of book},
Publisher,
Publisher's address,
Year.

