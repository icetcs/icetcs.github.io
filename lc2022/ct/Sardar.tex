%% FIRST RENAME THIS FILE <yoursurname>.tex. 
%% BEFORE COMPLETING THIS TEMPLATE, SEE THE "READ ME" SECTION 
%% BELOW FOR INSTRUCTIONS. 
%% TO PROCESS THIS FILE YOU WILL NEED TO DOWNLOAD asl.cls from 
%% http://aslonline.org/abstractresources.html. 


\documentclass[bsl,meeting]{asl}

\AbstractsOn

\pagestyle{plain}

\def\urladdr#1{\endgraf\noindent{\it URL Address}: {\tt #1}.}


\newcommand{\NP}{}
%\usepackage{verbatim}

\begin{document}
\thispagestyle{empty}

%% BEGIN INSERTING YOUR ABSTRACT DIRECTLY BELOW; 
%% SEE INSTRUCTIONS (1), (2), (3), and (4) FOR PROPER FORMATS

\NP  
\absauth{Muhammad Usama Sardar and Christof Fetzer}
\meettitle{Understanding trust assumptions for remote attestation via formal verification} 
\affil{Faculty of Computer Science, Technical University of Dresden, 01069 Dresden, Germany}
\meetemail{muhammad\_usama.sardar@tu-dresden.de}
%%% NOTE: email required for at least one author
%\urladdr{OPTIONAL}
%
% Second author's affiliation
\affil{Faculty of Computer Science, Technical University of Dresden, 01069 Dresden, Germany}
\meetemail{christof.fetzer@tu-dresden.de}
% \urladdr{OPTIONAL}


%% INSERT TEXT OF ABSTRACT DIRECTLY BELOW
Trust is a very critical and yet one of the least understood processes in the computing paradigm. As opposed to typical case studies based on toy examples, we demonstrate how we leverage formal verification to understand the complicated notion of trust in the real-world settings, with a specific focus on remote attestation in confidential computing. In this talk, we present the challenges and lessons learnt in the formal specification and verification of Intel's next generation architecture named Intel Trust Domain Extensions, and demonstrate how we ended up making Intel update the specification. 

The proposed talk will specifically address the following questions:
\begin{itemize}
    \item What is confidential computing? How does it compare with the existing state-of-the-art technologies, such as Homomorphic Encryption? 
    \item Why remote attestation is critical in confidential computing?
    \item What were the challenges in the formal specification of remote attestation in Intel Software Guard Extensions (SGX) and the upcoming Intel Trust Domain Extensions (TDX)?
    \item How we drive formal methods to practice for the automated verification of security properties of remote attestation protocols in Intel SGX and TDX? 
    \item What are interesting open challenges of relevance for logic community for formal verification of remote attestation in confidential computing? 
\end{itemize}

% \begin{thebibliography}{10}

% %% INSERT YOUR BIBLIOGRAPHIC ENTRIES HERE; 
% %% SEE (4) BELOW FOR PROPER FORMAT.
% %% EACH ENTRY MUST BEGIN WITH \bibitem{citation key}
% %%
% %% IF THERE ARE NO ENTRIES  
% %% DELETE THE LINE ABOVE (\begin{thebibliography}{20}) 
% %% AND THE LINE BELOW (\end{thebibliography})

% \end{thebibliography}


\vspace*{-0.5\baselineskip}
% this space adjustment is usually necessary after a bibliography

\end{document}


%% READ ME
%% READ ME
%% READ ME

INSTRUCTIONS FOR SUPPLYING INFORMATION IN THE CORRECT FORMAT: 

1. Author names are listed as First Last, First Last, and First Last.

\absauth{FirstName1 LastName1, FirstName2 LastName2, and FirstName3 LastName3}


2. Titles of abstracts have ONLY the first letter capitalized,
except for Proper Nouns.

\meettitle{Title of abstract with initial capital letter only, except for
Proper Nouns} 


3. Affiliations and email addresses for authors of abstracts are
  listed separately.

% First author's affiliation
\affil{Department, University, Street Address, Country}
\meetemail{First author's email}
%%% NOTE: email required for at least one author
\urladdr{OPTIONAL}
%
% Second author's affiliation
\affil{Department, University, Street Address, Country}
\meetemail{Second author's email}
\urladdr{OPTIONAL}
%
% Third author's affiliation
\affil{Department, University, Street Address, Country}
% Second author's email
\meetemail{Third author's email}
\urladdr{OPTIONAL}


4. Bibliographic Entries

%%%% IF references are submitted with abstract,
%%%% please use the following formats

%%% For a Journal article
\bibitem{cite1}
{\scshape Author's Name},
{\itshape Title of article},
{\bfseries\itshape Journal name spelled out, no abbreviations},
vol.~XX (XXXX), no.~X, pp.~XXX--XXX.

%%% For a Journal article by the same authors as above,
%%% i.e., authors in cite1 are the same for cite2
\bibitem{cite2}
\bysame
{\itshape Title of article},
{\bfseries\itshape Journal},
vol.~XX (XXXX), no.~X, pp.~XX--XXX.

%%% For a book
\bibitem{cite3}
{\scshape Author's Name},
{\bfseries\itshape Title of book},
Name of series,
Publisher,
Year.

%%% For an article in proceedings
\bibitem{cite4}
{\scshape Author's Name},
{\itshape Title of article},
{\bfseries\itshape Name of proceedings}
(Address of meeting),
(First Last and First2 Last2, editors),
vol.~X,
Publisher,
Year,
pp.~X--XX.

%%% For an article in a collection
\bibitem{cite5}
{\scshape Author's Name},
{\itshape Title of article},
{\bfseries\itshape Book title}
(First Last and First2 Last2, editors),
Publisher,
Publisher's address,
Year,
pp.~X--XX.

%%% An edited book
\bibitem{cite6}
Author's name, editor. % No special font used here
{\bfseries\itshape Title of book},
Publisher,
Publisher's address,
Year.

