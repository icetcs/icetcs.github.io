%% FIRST RENAME THIS FILE <yoursurname>.tex. 
%% BEFORE COMPLETING THIS TEMPLATE, SEE THE "READ ME" SECTION 
%% BELOW FOR INSTRUCTIONS. 
%% TO PROCESS THIS FILE YOU WILL NEED TO DOWNLOAD asl.cls from 
%% http://aslonline.org/abstractresources.html. 


\documentclass[bsl,meeting]{asl}

\AbstractsOn

\pagestyle{plain}

\def\urladdr#1{\endgraf\noindent{\it URL Address}: {\tt #1}.}


\newcommand{\NP}{}
%\usepackage{verbatim}

\begin{document}
\thispagestyle{empty}

%% BEGIN INSERTING YOUR ABSTRACT DIRECTLY BELOW; 
%% SEE INSTRUCTIONS (1), (2), (3), and (4) FOR PROPER FORMATS

\NP  
\absauth{Will Stafford}
\meettitle{Compositional proof-theoretic semantics for natural language}
\affil{Institute of Philosophy, Czech Academy of Sciences, Czech Republic}
\meetemail{stafford@flu.cas.cz}
\urladdr{willstafford.info}

%% INSERT TEXT OF ABSTRACT DIRECTLY BELOW
Francez and co-authors \cite{Francez2010-fb,Francez2014-md,Francez2015} attempt to offer a proof-theoretic semantics for natural language.  They take the meaning of sentences to be functions from sets of assumptions (or hypothesis or premises) to canonical proofs of the sentence. A proof for Francez is canonical if it ends in an introduction rule. Using this it has been shown for a fragment of English that one gets the expected behaviour for the grammatical type of the words considered. Francez proposal distinguishes between the meaning of words given in terms of proof rules and the meanings of sentences given in terms of canonical proofs. But to do this a third element must be used which is the set of all proofs of a sentences and called the “sentence contribution”.  This makes the semantics not compositional. It will be argued that the lack of compositionality here is a problem, as we want compositionality for learnability or effectiveness, it will be argued that Francez suggestion to use sentence contributions is unacceptable. I propose an alternative sentence meaning given by the set of proofs in normal form. A proof in normal form is one where all possible elimination rules are applied before introduction rules are.  It then follows that we can give compositional meanings without using the sentence contributions.  This is a small tweek but it allows for a compositional presentation.  The presentation will argue that this definition of propositions is preferable to Francez's original presentation.  

\begin{thebibliography}{10}

%% INSERT YOUR BIBLIOGRAPHIC ENTRIES HERE; 
%% SEE (4) BELOW FOR PROPER FORMAT.
%% EACH ENTRY MUST BEGIN WITH \bibitem{citation key}
%%
%% IF THERE ARE NO ENTRIES  
%% DELETE THE LINE ABOVE (\begin{thebibliography}{20}) 
%% AND THE LINE BELOW (\end{thebibliography})
\bibitem{Francez2010-fb}
{\scshape Francez, N and Dyckhoff, R},
{\itshape Proof-theoretic semantics for a natural language fragment},
{\bfseries\itshape Linguistics and Philosophy},
vol.~33 (2010), no.~6, pp.~447--477.

\bibitem{Francez2014-md}
{\scshape Francez, N and Ben-Avi, G},
{\itshape Proof-theoretic reconstruction of generalized quantifiers},
{\bfseries\itshape Journal of Semantics},
vol.~32 (2015), no.~3, pp.~313--371.

\bibitem{Francez2015}
{\scshape Nissim Francez},
{\bfseries\itshape Proof-Theoretic Semantics},
College Publications,
2015.

\end{thebibliography}


\vspace*{-0.5\baselineskip}
% this space adjustment is usually necessary after a bibliography

\end{document}


%% READ ME
%% READ ME
%% READ ME

INSTRUCTIONS FOR SUPPLYING INFORMATION IN THE CORRECT FORMAT: 

1. Author names are listed as First Last, First Last, and First Last.

\absauth{FirstName1 LastName1, FirstName2 LastName2, and FirstName3 LastName3}


2. Titles of abstracts have ONLY the first letter capitalized,
except for Proper Nouns.

\meettitle{Title of abstract with initial capital letter only, except for
Proper Nouns} 


3. Affiliations and email addresses for authors of abstracts are
  listed separately.

% First author's affiliation
\affil{Department, University, Street Address, Country}
\meetemail{First author's email}
%%% NOTE: email required for at least one author
\urladdr{OPTIONAL}
%
% Second author's affiliation
\affil{Department, University, Street Address, Country}
\meetemail{Second author's email}
\urladdr{OPTIONAL}
%
% Third author's affiliation
\affil{Department, University, Street Address, Country}
% Second author's email
\meetemail{Third author's email}
\urladdr{OPTIONAL}


4. Bibliographic Entries

%%%% IF references are submitted with abstract,
%%%% please use the following formats

%%% For a Journal article
\bibitem{cite1}
{\scshape Author's Name},
{\itshape Title of article},
{\bfseries\itshape Journal name spelled out, no abbreviations},
vol.~XX (XXXX), no.~X, pp.~XXX--XXX.



%%% For a Journal article by the same authors as above,
%%% i.e., authors in cite1 are the same for cite2
\bibitem{cite2}
\bysame
{\itshape Title of article},
{\bfseries\itshape Journal},
vol.~XX (XXXX), no.~X, pp.~XX--XXX.

%%% For a book
\bibitem{cite3}
{\scshape Author's Name},
{\bfseries\itshape Title of book},
Name of series,
Publisher,
Year.
\bibitem{Francez2015}
{\scshape Nissim Francez},
{\bfseries\itshape Proof-Theoretic Semantics},
College Publications,
2015.
Proof-Theoretic Semantics, by Nissim Francez. London:  College Publications,  2015. Pp. xx + 415.

%%% For an article in proceedings
\bibitem{cite4}
{\scshape Author's Name},
{\itshape Title of article},
{\bfseries\itshape Name of proceedings}
(Address of meeting),
(First Last and First2 Last2, editors),
vol.~X,
Publisher,
Year,
pp.~X--XX.

%%% For an article in a collection
\bibitem{cite5}
{\scshape Author's Name},
{\itshape Title of article},
{\bfseries\itshape Book title}
(First Last and First2 Last2, editors),
Publisher,
Publisher's address,
Year,
pp.~X--XX.

%%% An edited book
\bibitem{cite6}
Author's name, editor. % No special font used here
{\bfseries\itshape Title of book},
Publisher,
Publisher's address,
Year.

