%% FIRST RENAME THIS FILE <yoursurname>.tex. 
%% BEFORE COMPLETING THIS TEMPLATE, SEE THE "READ ME" SECTION 
%% BELOW FOR INSTRUCTIONS. 
%% TO PROCESS THIS FILE YOU WILL NEED TO DOWNLOAD asl.cls from 
%% http://aslonline.org/abstractresources.html. 


\documentclass[bsl,meeting]{asl}

\AbstractsOn

\pagestyle{plain}

\def\urladdr#1{\endgraf\noindent{\it URL Address}: {\tt #1}.}


\newcommand{\NP}{}
\DeclareMathSymbol{\mhyph}{\mathalpha}{operators}{`-}
%\usepackage{verbatim}

\begin{document}
\thispagestyle{empty}

%% BEGIN INSERTING YOUR ABSTRACT DIRECTLY BELOW; 
%% SEE INSTRUCTIONS (1), (2), (3), and (4) FOR PROPER FORMATS

\NP  
\absauth{Daichi Hayashi}
\meettitle{Reformulating supervaluational theory of truth}
\affil{Department of Philosophy and Ethics, Hokkaido University, Kita 10, Nishi 7, Kita-ku, Sapporo, Japan}
\meetemail{d-hayashi@eis.hokudai.ac.jp}

%% INSERT TEXT OF ABSTRACT DIRECTLY BELOW


To avoid the liar paradox, Kripke~\cite{cite4} used several monotone operators $\Phi : \mathcal{P}(\mathbb{N}) \to \mathcal{P}(\mathbb{N})$ 
that are based on partial evaluation schemata, such as the strong Kleene three-valued semantics.
Using the well-known fact that such a monotone operator has
 fixed points, Kripke argued that sets $X$ satisfying $\Phi(X) = X$ may be candidates for desirable extensions of the truth predicate.

As an instance of $\Phi$, Kripke suggested a monotone operator $\Phi_{SV}$ for a supervaluation schema, to which Cantini~\cite{cite1} gave a corresponding axiomatic truth theory $\mathsf{VF}$. While $\mathsf{VF}$ is satisfied in every model $\langle \mathbb{N}, X \rangle$ such that $\Phi_{SV}(X) = X$, the converse direction ($\mathbb{N} \mhyph categoricity$~\cite{cite2}) does not hold, that is, $\mathsf{VF}$ fails to completely characterise the fixed points of $\Phi_{SV}$.
Moreover, some authors have criticised $\mathsf{VF}$ because the axioms of $\mathsf{VF}$ do not mirror the structure of $\Phi_{SV}$ and thus these axioms seem somewhat unrelated. 

Even worse, Fischer et al.~\cite{cite2} showed that no recursively enumerable first-order theory can satisfy the $\mathbb{N}$-categoricity for $\Phi_{SV}.$
Therefore, to give a supervaluation-style axiomatisation that properly mirrors the intended operator, we also need to change $\Phi_{SV}$ itself without losing its supervaluational character.

In this talk, we give another operator $\Phi_{SV'},$ which results from replacing the set-theoretic notions used in $\Phi_{SV}$ by purely semantic talk (cf.~\cite{cite3} ). With the help of an additional typed truth predicate, an axiomatisation $\mathsf{VF}'$ for $\Phi_{SV'}$ is naturally induced; we show that $\mathsf{VF}'$ contains $\mathsf{VF}$ and satisfies the $\mathbb{N}$-categoricity-like result for $\Phi_{SV'}.$ If time permits, we also prove that $\mathsf{VF}'$ has the same proof-theoretic strength as $\Pi_{1}^{1} \mhyph \mathsf{CA}^{\mhyph}$.







This work was partially supported by JSPS KAKENHI, Grant Number 20J12361.


\begin{thebibliography}{10}

%% INSERT YOUR BIBLIOGRAPHIC ENTRIES HERE; 
%% SEE (4) BELOW FOR PROPER FORMAT.
%% EACH ENTRY MUST BEGIN WITH \bibitem{citation key}
%%
%% IF THERE ARE NO ENTRIES  
%% DELETE THE LINE ABOVE (\begin{thebibliography}{20}) 
%% AND THE LINE BELOW (\end{thebibliography})

\bibitem{cite1}
{\scshape Cantini Andrea},
{\itshape A theory of formal truth arithmetically equivalent to ID1},
{\bfseries\itshape The Journal of Symbolic Logic},
vol.~55 (1990), no.~1, pp.~244--259.


\bibitem{cite2}
{\scshape Fischer Martin, Halbach Volker, Kriener J{\"o}nne, and Stern Johannes},
{\itshape Axiomatizing semantic theories of truth?},
{\bfseries\itshape The Review of Symbolic Logic},
vol.~8 (2015), no.~2, pp.~257--278.




\bibitem{cite3}
{\scshape Halbach Volker},
{\itshape Truth and reduction},
{\bfseries\itshape Erkenntnis},
vol.~53 (2000), no.~1, pp.~97--126.




\bibitem{cite4}
{\scshape Kripke Saul},
{\itshape Outline of a theory of truth},
{\bfseries\itshape The journal of philosophy},
vol.~72 (1976), no.~19, pp.~690--716.












\end{thebibliography}


\vspace*{-0.5\baselineskip}
% this space adjustment is usually necessary after a bibliography

\end{document}


%% READ ME
%% READ ME
%% READ ME

INSTRUCTIONS FOR SUPPLYING INFORMATION IN THE CORRECT FORMAT: 

1. Author names are listed as First Last, First Last, and First Last.

\absauth{Daichi Hayashi}


2. Titles of abstracts have ONLY the first letter capitalized,
except for Proper Nouns.

\meettitle{Title of abstract with initial capital letter only, except for
Proper Nouns} 


3. Affiliations and email addresses for authors of abstracts are
  listed separately.

% First author's affiliation
\affil{Department, University, Street Address, Country}
\meetemail{First author's email}
%%% NOTE: email required for at least one author
\urladdr{OPTIONAL}
%
% Second author's affiliation
\affil{Department, University, Street Address, Country}
\meetemail{Second author's email}
\urladdr{OPTIONAL}
%
% Third author's affiliation
\affil{Department, University, Street Address, Country}
% Second author's email
\meetemail{Third author's email}
\urladdr{OPTIONAL}


4. Bibliographic Entries

%%%% IF references are submitted with abstract,
%%%% please use the following formats

%%% For a Journal article
\bibitem{cite1}
{\scshape Author's Name},
{\itshape Title of article},
{\bfseries\itshape Journal name spelled out, no abbreviations},
vol.~XX (XXXX), no.~X, pp.~XXX--XXX.

%%% For a Journal article by the same authors as above,
%%% i.e., authors in cite1 are the same for cite2
\bibitem{cite2}
\bysame
{\itshape Title of article},
{\bfseries\itshape Journal},
vol.~XX (XXXX), no.~X, pp.~XX--XXX.

%%% For a book
\bibitem{cite3}
{\scshape Author's Name},
{\bfseries\itshape Title of book},
Name of series,
Publisher,
Year.

%%% For an article in proceedings
\bibitem{cite4}
{\scshape Author's Name},
{\itshape Title of article},
{\bfseries\itshape Name of proceedings}
(Address of meeting),
(First Last and First2 Last2, editors),
vol.~X,
Publisher,
Year,
pp.~X--XX.

%%% For an article in a collection
\bibitem{cite5}
{\scshape Author's Name},
{\itshape Title of article},
{\bfseries\itshape Book title}
(First Last and First2 Last2, editors),
Publisher,
Publisher's address,
Year,
pp.~X--XX.

%%% An edited book
\bibitem{cite6}
Author's name, editor. % No special font used here
{\bfseries\itshape Title of book},
Publisher,
Publisher's address,
Year.

