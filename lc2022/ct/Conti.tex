

\documentclass[bsl,meeting]{asl}

\AbstractsOn

\pagestyle{plain}

\def\urladdr#1{\endgraf\noindent{\it URL Address}: {\tt #1}.}


\newcommand{\NP}{}
%\usepackage{verbatim}

\begin{document}
\thispagestyle{empty}


\NP  
\absauth{Ludovica Conti}
\meettitle{Arbitrary Abstraction and Logicality}
\affil{Complutense University of Madrid}
\meetemail{luconti@ucm.es}

In this talk, I will discuss a criterion (weak invariance) that has been recently suggested in order to argue for the logicality of abstraction operators, when they are understood as arbitrary expressions (cf. Boccuni Woods 2020). The issue of  logicality of the abstractionist vocabulary was originally raised within the seminal abstractionist program, Frege’s Logicism, and represents, still today, a crucial topic in the abstractionist debate. My double aim consists in inquiring this topic both from a formal and from a philosophical point of view. 

On the one side, I will argue that, while weak invariance is not satisfied (except for specific exceptions, cf. \cite{Tarski:1956}, \cite{Woods:2014}) by first-order abstraction principles (APs), it characterises a wide range of higher-order ones. More precisely, by comparing respective schemas of first-order and second-order APs, we will note that logicality (in the chosen meaning) mirrors a relevant distinction between same-order and different-order abstraction principles. 
So, after discussing the controversial case of Ordinal Abstraction, I will note that, if we accept an arbitrary interpretation of APs, not only Neologicism (based on HP), but many current abstractionist programs and even the consistent revisions of Frege’s Logicism (based on weakened versions of BLV) are able to achieve the logicality objective. 
 

On the other side, from a philosophical point of view, I will discuss the role of arbitrariness as a condition for the adoption of the abovementioned logicality criterion. Particularly, I will argue that, on the one hand, the arbitrary interpretation could be considered as the most faithful to abstractionist theories, but, on the other hand, it includes semantic insights that are radically alternative to Logicism. In order to argue for this latter consideration, an analogy between the arbitrary interpretation of the APs and the semantics of some eliminative structuralist reconstructions of the scientific theories will be illustrated.



\begin{thebibliography}{10}

\bibitem{Boccuni}
{\scshape Francesca Boccuni, Jack Woods},
{\itshape Structuralist Neologicism} 
{\bfseries\itshape Philosophia Mathematica}, (2018), 28(3), 296-316.

\bibitem{Fine}
{\scshape Kit Fine},
{\itshape The limits of Abstraction}
{\bfseries\itshape Clarendon Press}, (2002).

\bibitem{Woods:2014}
{\scshape Jack Woods},
{\itshape Logical indefinites , 277-307}
{\bfseries\itshape Logique et Analyse}, (2014) pp. 277-307.

\bibitem{Tarski:1956}
{\scshape Gabriel Uzquiano}
{\itshape The concept of truth in formalized languages}
{\bfseries\itshape Logic, semantics, metamathematics}, (1956) 2(152-278), 7.


\end{thebibliography}


\vspace*{-0.5\baselineskip}

\end{document}


