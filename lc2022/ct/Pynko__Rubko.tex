%% FIRST RENAME THIS FILE <yoursurname>.tex.
%% BEFORE COMPLETING THIS TEMPLATE, SEE THE "READ ME" SECTION
%% BELOW FOR INSTRUCTIONS.
%% TO PROCESS THIS FILE YOU WILL NEED TO DOWNLOAD asl.cls from
%% http://aslonline.org/abstractresources.html.


\documentclass[bsl,meeting]{asl}

\usepackage{amsfonts}
\usepackage{amssymb}
%\usepackage[mathscr]{euscript}
\usepackage{verbatim}
%\usepackage{amsthm}

\AbstractsOn

\pagestyle{plain}

\def\urladdr#1{\endgraf\noindent{\it URL Address}: {\tt #1}.}


\newcommand{\NP}{}
%\usepackage{verbatim}

\newcommand{\mf}[1]{\mathfrak{#1}}
\newcommand{\mc}[1]{\mathcal{#1}}
\newcommand{\mbf}[1]{\mathbf{#1}}
%\newcommand{\ms}[1]{\mathscr{#1}}
\newcommand{\msf}[1]{\mathsf{#1}}
\newcommand{\couple}[2]{\langle{#1},{#2}\rangle}
\newcommand{\FA}{\mf{Fm}}
\newcommand{\inverse}[1]{{#1}^{-1}}
\newcommand{\restr}{{\upharpoonright}}

\def\e{{\frac{1}{2}} }
\def\iff{\Leftrightarrow}

\DeclareMathOperator{\Fm}{Fm}
\DeclareMathOperator{\Con}{Con}
\DeclareMathOperator{\img}{img}
\DeclareMathOperator{\Cn}{Cn}

\newtheorem{thm}{Theorem}
\newtheorem{lem}[thm]{Lemma}
\newtheorem{claim}[thm]{Claim}

\begin{document}
\thispagestyle{empty}

%% BEGIN INSERTING YOUR ABSTRACT DIRECTLY BELOW;
%% SEE INSTRUCTIONS (1), (2), (3), and (4) FOR PROPER FORMATS

\NP
\absauth{Alexej Pynko, Gnat Rubko}
\meettitle{Paraconsistent extensions of three-valued %paraconsistent
logics}
%the logic of paradox}
\affil{Cybernetics Institute,
Glushkov p. 40, Kiev, 03680, Ukraine}
\meetemail{pynko@i.ua}

%% INSERT TEXT OF ABSTRACT DIRECTLY BELOW


Given any propositional language $L$
(viz., a set of connectives,
treated as operation symbols, when dealing
with $L$-algebras),
an {\em $L$-logic\/} $C$ %where $L$ is a propositional
%language, %/signature,
%constituted by {\em [propositional] connectives},
(viz., a structural %--- i.e, commuting
closure operator over the set $\Fm_L$
of $L$-formulas
with variables in %a countable set,
%of the absolutely-free $L$-algebra %$\FA_L$
%freely-generated by
%the set
$\{x_i\}_{i\in\omega}$
%of propositional variables
$\langle$natural numbers, including $0$,
are treated as sets of lesser ones,
the set of all them being denoted by $\omega\rangle$;
pairs of the form $\Gamma\vdash\varphi$, where
$\Gamma\subseteq\Fm_L\ni\varphi$,
being called {\em $L$-rules\/})
%[to be identified with $\varphi$ alone])
%that is structural --- i.e., $\img C$
%commutes with inverse
%propositional $L$-substitutions
%$\langle$viz., endomorphisms of $\FA_L\rangle$)
is said to ``{\em satisfy\/} an $L$-rule $\Gamma\vdash\varphi$''/``be
{\em [(\/\{almost\/\} pre-)maximally]\/
$\langle\neg$-para\/$\rangle$consisent\/}
$\langle$whe\-re $\neg\in L$ is unary$\rangle$'',
if ``$\varphi\in C(\Gamma)$''/``$C$ does not satisfy
$(\langle\{x_0,\neg x_0\}\cup\rangle\varnothing)\vdash x_1$ [and has no (more than $1\{+1\}$)
$\langle\neg$-para$\rangle$consistent
{\em proper\/} --- viz., distinct from $C$ --
{\em extension\/} --- viz., an $L$-logic satisfying all $L$-rules
$C$ satisfies],
an $L$-matrix defining such a logic being said to do/be so too''.
%{\em $\neg$-paraconsistent}.
Likewise, $C$ is said to be {\em weakly\/
$\land$-conjunctive/\/$\lor$-disjunctive},
where $\land/\lor$ is a binary connective of
$L$ (possibly, a {\em secondary\/} one; viz.,
an $L$-formula with {\em at most\/} two variables $x_0$ and $x_1$),
if $C(x_{0|1})\subseteq/\supseteq C(x_0(\land/\lor)x_1)$,
an $L$-matrix defining such a logic being said to be so too.
Then, a {\em theorem/model of\/} $C$ is any
``element of $C(\varnothing)$''/``$L$-matrix defining an
extension of $C$''.
Likewise, the least extension $C^R$ of $C$
satisfying an $L$-rule $R$
is said to be {\em
relatively axiomatized by\/} $R$.
Finally, two-valued $L$-matrices
%(with carrier $2$ of their underlying algebras as well as)
with single
distinguished value %($1$)
and operation $\neg$ %of their underlying algebras
permuting their unique distinguished and non-distinguished values
are said to be {\em %(canonically)\/
$\neg$-classical},
[any {\em sublogic of\/} --- viz., an $L$-logic with an extension being --- any of] their logics
being called {\em $\neg$-[sub]classical}.
Let $\mc{A}=\couple{\mf{A}}{D}$ be
a $\neg$-paraconsistent $L$-matrix
%with underlying algebra $\mf{A}$
with carrier $A\triangleq(2\cup\{\e\})$ and
$(\neg^\mf{A}\restr2)=(2^2\setminus\Delta_2)$,
where $\Delta_S\triangleq\{\couple{s}{s}\mid s\in S\}$,
for any set $S$,
as well as %the set of distinguished values
$D\triangleq(A\setminus1)$,
$\mc{A}_\e\triangleq\couple{\mf{A}}{\{\e\}}$
and $C_{[\e]}$ the logic of $\mc{A}_{[\e]}$. %while
%$\mc{A}$ is said to be {\em regular},
%if %operations of $\mf{A}$ are monotonic w.r.t.
%$(\Delta_A\cup(\{\e\}\times2))\in\mbf{S}\mf{A}^2$,
%whereas a {\em binary discriminator of} $\mc{A}$
%is any secondary binary connective $\beta$ of $L$ %with at most two variables $x_0$ and $
%such that $\forall a\in D,\forall b\in(2\cdot a):
%\beta^\mf{A}(a,b)=(1-(a\cdot(1-b)))$.

%\addvspace{-10pt}

\begin{theorem}
$C$ is not maximally\/ $\neg$-paraconsistent iff\/
$(\neg^\mf{A}\cup\{\couple{\e}{\e}\})\in\mbf{S}\mf{A}^2$ iff\/
$\neg^\mf{A}$ is an automorphism of\/ $\mf{A}$ iff\/
$(C/\mc{A})_\e$ is a\/ $\neg$-paraconsistent extension/model of\/
$C$ iff\/ $C$ has a\/ $\neg$-paraconsistent model with single
distinguished value iff\/ $\couple{\mf{A}}{\{0,\e\}}$
is a\/ $\neg$-paraconsistent/defining model of\/ $C$ iff
the extension of\/ $C$ relatively axiomatized by
$R^{[+]}\triangleq(\{[x_0,\neg x_0,]x_1\}\vdash\neg x_1)$
is\/ $\neg$-paraconsistent, in which case
proper\/$|\neg$-paraconsistent extensions of\/ $C$
are exactly extensions\/$|$sublogics of\/
$C^{R^{+|}}\subseteq|=C_\e[(\varnothing)\linebreak=C(\varnothing)]$,
while\/ $C$ is not pre-maximally\/ $\neg$-paraconsistent iff\/
$C_{[\e]}$ has no theorem iff\/ $C$ is not weakly disjunctive iff\/
$2\in\mbf{S}\mf{A}$ iff\/
$C^{[R^+]}$ ``is\/ $\neg$-subclassical''/``has a consistent
non-$\neg$-paraconsistent extension'' iff\/
$C_{\e|}$ is not maximally\/$|$pre-maximally consistent iff\/
$C_\e\neq C^{R^+}$ iff\/ $C_\e$ is not the only proper
[\/$\neg$-para]consistent extension of\/ $C$,
in which case proper\/ $\neg$-paraconsistent both
extensions/sublogics of\/ $C_{/\e}$ are exactly\/
$|$``\/$\neg$-subclassical\/ $\neg$-paraconsistent''
extensions of\/ $C^{R^+}$ ``with models $\mc{A}\restr2$ and
$\mc{A}_\e$''\/$|$, whereas\/ $C$ is almost pre-maximally\/
$\neg$-paraconsistent iff\/
$C^{R^+}\|$there is a unique proper ``both\/
$\neg$-paraconsistent/-subclassical
extension''\/$|$``\/$\neg$-paraconsistent both
extension/sublogic'' of\/ $C_{|/\e}$ iff proper\/
$\neg$-paraconsistent extensions of\/ $C$ are exactly\/
$C^{R^{[+]}}$ iff\/ $C^{R^+}$ is defined
by\/ $\{%[\mc{A}\times]
(\mc{A}\restr2),\mc{A}_\e\}$
%[as well as $\mc{A}$ is regular and has no binary discriminator]
iff\/ $(\Delta_A\cup(\{\e\}\times2))\in\mbf{S}\mf{A}^2$
and there is no secondary binary connective $\beta$ of $L$ %with at most two variables $x_0$ and $
such that\/ $\forall a\in D,\forall b\in(2\cdot a):
\beta^\mf{A}(a,b)=(1-(a\cdot(1-b)))$,
%$\mc{A}$ is regular and has no binary discriminator,
and so\/ $C$ is [/pre-]maximally\/ $\neg$-paraconsistent,
whenever it is weakly con\-jun\-c\-ti\-ve/``dis\-jun\-c\-ti\-ve (i.e., has a
theorem) and [not]\/ $\wr$-subclassical''.
%$2\in\mbf{S}\mf{A}$''.
\end{theorem}


\vspace*{-0.5\baselineskip}
% this space adjustment is usually necessary after a bibliography

\end{document}


%% READ ME
%% READ ME
%% READ ME

INSTRUCTIONS FOR SUPPLYING INFORMATION IN THE CORRECT FORMAT:

1. Author names are listed as First Last, First Last, and First Last.

\absauth{FirstName1 LastName1, FirstName2 LastName2, and FirstName3 LastName3}


2. Titles of abstracts have ONLY the first letter capitalized,
except for Proper Nouns.

\meettitle{Title of abstract with initial capital letter only, except for
Proper Nouns}


3. Affiliations and email addresses for authors of abstracts are
  listed separately.

% First author's affiliation
\affil{Department, University, Street Address, Country}
\meetemail{First author's email}
%%% NOTE: email required for at least one author
\urladdr{OPTIONAL}
%
% Second author's affiliation
\affil{Department, University, Street Address, Country}
\meetemail{Second author's email}
\urladdr{OPTIONAL}
%
% Third author's affiliation
\affil{Department, University, Street Address, Country}
% Second author's email
\meetemail{Third author's email}
\urladdr{OPTIONAL}


4. Bibliographic Entries

%%%% IF references are submitted with abstract,
%%%% please use the following formats

%%% For a Journal article
\bibitem{cite1}
{\scshape Author's Name},
{\itshape Title of article},
{\bfseries\itshape Journal name spelled out, no abbreviations},
vol.~XX (XXXX), no.~X, pp.~XXX--XXX.

%%% For a Journal article by the same authors as above,
%%% i.e., authors in cite1 are the same for cite2
\bibitem{cite2}
\bysame
{\itshape Title of article},
{\bfseries\itshape Journal},
vol.~XX (XXXX), no.~X, pp.~XX--XXX.

%%% For a book
\bibitem{cite3}
{\scshape Author's Name},
{\bfseries\itshape Title of book},
Name of series,
Publisher,
Year.

%%% For an article in proceedings
\bibitem{cite4}
{\scshape Author's Name},
{\itshape Title of article},
{\bfseries\itshape Name of proceedings}
(Address of meeting),
(First Last and First2 Last2, editors),
vol.~X,
Publisher,
Year,
pp.~X--XX.

%%% For an article in a collection
\bibitem{cite5}
{\scshape Author's Name},
{\itshape Title of article},
{\bfseries\itshape Book title}
(First Last and First2 Last2, editors),
Publisher,
Publisher's address,
Year,
pp.~X--XX.

%%% An edited book
\bibitem{cite6}
Author's name, editor. % No special font used here
{\bfseries\itshape Title of book},
Publisher,
Publisher's address,
Year.
