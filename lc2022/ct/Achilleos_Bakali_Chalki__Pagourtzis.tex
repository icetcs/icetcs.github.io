%% FIRST RENAME THIS FILE <yoursurname>.tex.
%% BEFORE COMPLETING THIS TEMPLATE, SEE THE "READ ME" SECTION
%% BELOW FOR INSTRUCTIONS.
%% TO PROCESS THIS FILE YOU WILL NEED TO DOWNLOAD asl.cls from
%% http://aslonline.org/abstractresources.html.


\documentclass[bsl,meeting]{asl}
\usepackage[greek,english]{babel}
\AbstractsOn

\pagestyle{plain}

\def\urladdr#1{\endgraf\noindent{\it URL Address}: {\tt #1}.}


\newcommand{\NP}{}
%\usepackage{verbatim}
\usepackage{teubner}

\begin{document}
\thispagestyle{empty}



%% BEGIN INSERTING YOUR ABSTRACT DIRECTLY BELOW;
%% SEE INSTRUCTIONS (1), (2), (3), and (4) FOR PROPER FORMATS

%\NP
\absauth{Antonis Achilleos, Eleni Bakali, Aggeliki Chalki, Aris Pagourtzis}
\meettitle{Descriptive complexity for hard counting problems with easy decision version}
\affil{Department of Computer Science, Reykjavik University, Menntavegur 1, IS-102, Reykjavík, Iceland}
%\meetemail{antonios@ru.is}
%%% NOTE: email required for at least one author
%%\urladdr{OPTIONAL}
%
\affil{School of Electrical and Computer Engineering, National Technical University of Athens, Iroon Polytechniou 9, 15780, Athens, Greece}
%\meetemail{bakali@corelab.ntua.gr}
%
%\affil{School of Electrical and Computer Engineering, National Technical University of Athens, Iroon Polytechniou 9, 15780, Athens, Greece}
\meetemail{achalki@corelab.ntua.gr}
%
%\affil{School of Electrical and Computer Engineering, National Technical University of Athens, Iroon Polytechniou 9, 15780, Athens, Greece}
%\meetemail{pagour@cs.ntua.gr}

%% INSERT TEXT OF ABSTRACT DIRECTLY BELOW
The class \textsf{\#P} is the class of functions that count the number of solutions to problems in \textsf{NP}.
Since very few counting problems can be exactly computed in polynomial time (e.g.\ counting spanning trees), the interest of the community has  turned to the complexity of approximating them.
The class \textsf{\#PE} of problems in \textsf{\#P} with decision version in \textsf{P} is of great significance.

We focus on a subclass of \textsf{\#PE}, namely \textsf{TotP}, %which is 
the class of functions that count the total number of paths of NPTMs. 
%This class 
\textsf{TotP}
contains all self-reducible \textsf{\#PE} functions and it is \emph{robust}, in the sense that it has natural complete problems and it is closed under addition, multiplication and subtraction by one.

We present logical characterizations of \textsf{TotP} and two other \emph{robust} subclasses of this class, building upon two seminal works about descriptive complexity for classes of counting problems~\cite{Saluja,Arenas}. Specifically, to capture \textsf{TotP}, we use recursion on functions over second-order variables which, we believe,  is of independent interest.

This work has been partially funded by the Basic Research Program PEVE 2020 of the National Technical  University of Athens, and the project ``MoVeMnt: Mode(l)s of Verification and Monitorability'' (grant no~217987) of the Icelandic Research Fund.

\begin{thebibliography}{10}
\bibitem{Saluja}
{\scshape S. Saluja and K.V. Subrahmanyam and M.N. Thakur},
{\itshape Descriptive Complexity of \#{P} Functions},
{\bfseries\itshape Journal of Computer and System Sciences},
vol.~50, no.~3, pp.~493--505.

\bibitem{Arenas}
{\scshape M. Arenas and M. Mu{\~{n}}oz and C. Riveros},
{\itshape Descriptive Complexity for Counting Complexity Classes},
{\bfseries\itshape Logical Methods in Computer Science},
vol.~16, no.~1.
\end{thebibliography}

\vspace*{-0.5\baselineskip}
%% this space adjustment is usually necessary after a bibliography

\end{document}