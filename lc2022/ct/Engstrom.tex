%% FIRST RENAME THIS FILE <yoursurname>.tex. 
%% BEFORE COMPLETING THIS TEMPLATE, SEE THE "READ ME" SECTION 
%% BELOW FOR INSTRUCTIONS. 
%% TO PROCESS THIS FILE YOU WILL NEED TO DOWNLOAD asl.cls from 
%% http://aslonline.org/abstractresources.html. 


\documentclass[bsl,meeting]{asl}
\usepackage{stmaryrd}
\AbstractsOn

\pagestyle{plain}

\def\urladdr#1{\endgraf\noindent{\it URL Address}: {\tt #1}.}


\newcommand{\NP}{}
%\usepackage{verbatim}

\begin{document}
\thispagestyle{empty}

%% BEGIN INSERTING YOUR ABSTRACT DIRECTLY BELOW; 
%% SEE INSTRUCTIONS (1), (2), (3), and (4) FOR PROPER FORMATS

\NP  
\absauth{Fredrik Engstr\"om}
\meettitle{Foundations of team semantics}
\affil{Department of Philosophy, Linguistics and Theory of Science, University of Gothenburg, Sweden}
\meetemail{fredrik.engstrom@gu.se}

%% INSERT TEXT OF ABSTRACT DIRECTLY BELOW

Dependence logic, see \cite{vaananen}, and its relatives are defined using team semantics, in which formulas are satisfied by sets of assignments, teams, rather than single assignments. 
The team semantics construction is widely applicable and can be used to
understand notions from many different areas; model theory, game theory,
database theory, probabilistic reasoning and program verification, to name a
few. 

The denotatiton of a first-order formula in classical Tarskian semantics is the set of assignments satisfying the formula, $\llbracket \varphi \rrbracket_c$. In team semantics it is the set of teams satisfying the formula, i.e., a set of sets of assignments, $\llbracket \varphi \rrbracket_t$. The standard team semantic construction is via the \emph{flatness principle} according to which $\llbracket \varphi \rrbracket_t = \mathcal P (\llbracket \varphi \rrbracket_c)$. 

This construction can, at least partially, be described using the free functor from the category of partially ordered monoids to the category of quantales, i.e., partially ordered monoids equipped with a complete semilattice structure, see \cite{abramsky}. This functor maps the space of Tarskian denotations, $\mathcal P (X^V)$, where $X$ is the domain and $V$ a set of variables, into the space $\mathcal H(\mathcal P(X^V))$, the set of downwards-closed subsets of $\mathcal P (X^V)$. The embedding is based on the flatness principle in that $\llbracket \varphi \rrbracket_t = \mathcal P (\llbracket \varphi \rrbracket_c)$. 

However, the space of downwards-closed sets can not be used as the space of denotations for some logics: One example is the well-studied Independence logic, which isn't downward-closed; another is a logic constructed to handle branching of non-monotone generalized quantifiers, which isn't based on the flatness principle. I will in this talk revisit the description of the team semantic construction as the free functor from a more general perspective that also includes  these logics.

\begin{thebibliography}{10}


\bibitem{abramsky}
{\scshape Samson Abramsky \and Jouko V\"a\"an\"anen},
{\itshape From IF to BI},
{\bfseries\itshape Synthese},
vol.~167 (2009), pp.~207--230.


\bibitem{engstrom}
{\scshape Fredrik Engstr\"om},
{\itshape Generalized quantifiers in Dependence logic},
{\bfseries\itshape Journal of Logic, Language and Information},
vol.~21 (2012), pp.~299--324.

%%% For a Journal article by the same authors as above,
%%% i.e., authors in cite1 are the same for cite2

%%% For a book
\bibitem{vaananen}
{\scshape Jouko V\"a\"an\"anen},
{\bfseries\itshape Dependence logic. A new approach to independence friendly logic},
London Mathematical Society Student Texts, 
Cambridge University Press,
2007.

%% INSERT YOUR BIBLIOGRAPHIC ENTRIES HERE; 
%% SEE (4) BELOW FOR PROPER FORMAT.
%% EACH ENTRY MUST BEGIN WITH \bibitem{citation key}
%%
%% IF THERE ARE NO ENTRIES  
%% DELETE THE LINE ABOVE (\begin{thebibliography}{20}) 
%% AND THE LINE BELOW (\end{thebibliography})

\end{thebibliography}


\vspace*{-0.5\baselineskip}
% this space adjustment is usually necessary after a bibliography

\end{document}


%% READ ME
%% READ ME
%% READ ME

INSTRUCTIONS FOR SUPPLYING INFORMATION IN THE CORRECT FORMAT: 

1. Author names are listed as First Last, First Last, and First Last.

\absauth{FirstName1 LastName1, FirstName2 LastName2, and FirstName3 LastName3}


2. Titles of abstracts have ONLY the first letter capitalized,
except for Proper Nouns.

\meettitle{Title of abstract with initial capital letter only, except for
Proper Nouns} 


3. Affiliations and email addresses for authors of abstracts are
  listed separately.

% First author's affiliation
\affil{Department, University, Street Address, Country}
\meetemail{First author's email}
%%% NOTE: email required for at least one author
\urladdr{OPTIONAL}
%
% Second author's affiliation
\affil{Department, University, Street Address, Country}
\meetemail{Second author's email}
\urladdr{OPTIONAL}
%
% Third author's affiliation
\affil{Department, University, Street Address, Country}
% Second author's email
\meetemail{Third author's email}
\urladdr{OPTIONAL}


4. Bibliographic Entries

%%%% IF references are submitted with abstract,
%%%% please use the following formats

%%% For a Journal article
\bibitem{cite1}
{\scshape Author's Name},
{\itshape Title of article},
{\bfseries\itshape Journal name spelled out, no abbreviations},
vol.~XX (XXXX), no.~X, pp.~XXX--XXX.

%%% For a Journal article by the same authors as above,
%%% i.e., authors in cite1 are the same for cite2
\bibitem{cite2}
\bysame
{\itshape Title of article},
{\bfseries\itshape Journal},
vol.~XX (XXXX), no.~X, pp.~XX--XXX.

%%% For a book
\bibitem{cite3}
{\scshape Author's Name},
{\bfseries\itshape Title of book},
Name of series,
Publisher,
Year.

%%% For an article in proceedings
\bibitem{cite4}
{\scshape Author's Name},
{\itshape Title of article},
{\bfseries\itshape Name of proceedings}
(Address of meeting),
(First Last and First2 Last2, editors),
vol.~X,
Publisher,
Year,
pp.~X--XX.

%%% For an article in a collection
\bibitem{cite5}
{\scshape Author's Name},
{\itshape Title of article},
{\bfseries\itshape Book title}
(First Last and First2 Last2, editors),
Publisher,
Publisher's address,
Year,
pp.~X--XX.

%%% An edited book
\bibitem{cite6}
Author's name, editor. % No special font used here
{\bfseries\itshape Title of book},
Publisher,
Publisher's address,
Year.

