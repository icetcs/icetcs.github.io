%% FIRST RENAME THIS FILE <yoursurname>.tex.
%% BEFORE COMPLETING THIS TEMPLATE, SEE THE "READ ME" SECTION
%% BELOW FOR INSTRUCTIONS.
%% TO PROCESS THIS FILE YOU WILL NEED TO DOWNLOAD asl.cls from
%% http://aslonline.org/abstractresources.html.


\documentclass[bsl,meeting]{asl}

\AbstractsOn

\pagestyle{plain}

\def\urladdr#1{\endgraf\noindent{\it URL Address}: {\tt #1}.}


\newcommand{\NP}{}
%\usepackage{verbatim}

\begin{document}
\thispagestyle{empty}

%% BEGIN INSERTING YOUR ABSTRACT DIRECTLY BELOW;
%% SEE INSTRUCTIONS (1), (2), (3), and (4) FOR PROPER FORMATS

\NP
\absauth{Aibat Yeshkeyev, Olga Ulbrikht, and Nazerke Mussina}
\meettitle{ Syntactic and semantic similarities of hybrids of classes of
the Jonsson spectrum of Jonsson quasivariety of the class $K$}
\affil{Faculty of Mathematics and Information Technologies, Karaganda Buketov University, University str., 28, building 2, Kazakhstan}
\meetemail{aibat.kz@gmail.com}
\meetemail{ulbrikht@mail.ru}
\meetemail{nazerke170493@mail.ru}

%% INSERT TEXT OF ABSTRACT DIRECTLY BELOW

Let $K$ be the class of structures of countable signature $\sigma$. Let's introduce the notation:
\begin{gather*}
\forall\exists(K)=Th(K)\cup\{\varphi \mid \varphi\text{\;is a\;} \forall\exists\text{-sentence of considered language and\;\;} \\ \varphi\cup Th(K) \text{\;\;is a consistent}\}.
\end{gather*}

Definition 1. A variety (quasivariety) of structures $K$ is called a Jonsson variety (quasivariety) if  $\forall\exists(K)$ is a Jonsson theory.


Consider the $JSpV(K)$ be Jonsson spectrum of the Jonsson variety of class $K$, where $K$ is the Jonsson variety:
\begin{gather*}
    JSpV(K)=\{T \mid  T=\forall\exists(N)\text{ is Jonsson theory,}
    \ N \text{ is a subvariety of } K\}.
\end{gather*}

Then $JSpV(K) /_{\bowtie}$ is denoting the factor set of the Jonsson spectrum of Jonsson quasivariety of the class $K$ by the relation $\bowtie$.

Similarly, we define the Jonsson spectrum of $JSpQV(K)$ quasivariety:
\begin{gather*}
 JSpQV(K)=\{T \mid T=\forall\exists(N)\text{ is Jonsson theory,}
    \ N \text{ is a subquasivariety of } K\}.
\end{gather*}
Then $JSpQV(K) /_{\bowtie}$ denotes the factor set of the Jonsson spectrum of Jonsson quasivariety of the class $K$ by the relation $\bowtie$.




Definition 2. Let $K$ be some Jonsson quasivariety of structures of signature $\sigma$, $[T_1], [T_2] \in JSpQV(K) /_{\bowtie}$.
The hybrid (of the first type) $ H ([T_1], [T_2]) $ of the classes $ [T_1] $ and $ [T_2] $ is the theory $ Th _ {\forall \exists} (C_ {1} \diamond C_ {2 }) $ if it is Jonsson theory in  language of the signature $ \sigma $, where $ C_ {i} $ are semantic models of the classes $ [T_i] $, $ i = 1,2 $ respectively and $ \diamond \in \{\times,  +, \oplus, \prod \limits_ {F}, \prod \limits_ {U} \} $, where $ \times $ is cartesian product$,  +  $ is the sum, $ \oplus $ is the direct sum, $ \prod \limits_ {F} $ is reduced product and $ \prod \limits_ {U} $ is the ultraproduct of models.

The following fact will be necessary for the proof of Theorem~1.

Fact 1. ([1], p. 48) For any complete for $\exists$-sentences Jonsson theory $T$ the following conditions are equivalent:

1) $T^{*}$ is model complete;

2) for each $n<\omega$, $E_{n}(T)$ is Boolean algebra, where $E_{n}(T)$  is a lattice of $\exists$-formulas with  $n$ free variables.

And in the frame above mentioned notions we have the following result.

\begin{theorem} Let $K$ be some Jonsson quasivariety of structures of signature $\sigma$,
$[T_{1}], [T_{2}]$, $[T_{3}], [T_{4}]  \in JSpQV(K)/_{\bowtie}$, $H_{1}=H([T_{1}],[T_{2}])$ and $H_{2}=H([T_{3}],[T_{4}])$ are complete for existential sentences perfect hybrids, then following conditions are equivalent:

1. $H_{1}\overset{S}\sim H_{2}$;

2. $H_{1}^*\overset{S}\approx H_{2}^*$.
\end{theorem}


All additional information regarding Jonsson theories can be found in [1].

This work was supported by the Science Committee of the Ministry of Education and Science of the Republic of Kazakhstan (grand 
AP09260237).


\begin{thebibliography}{10}

%% INSERT YOUR BIBLIOGRAPHIC ENTRIES HERE;
%% SEE (4) BELOW FOR PROPER FORMAT.
%% EACH ENTRY MUST BEGIN WITH \bibitem{citation key}
%%
%% IF THERE ARE NO ENTRIES
%% DELETE THE LINE ABOVE (\begin{thebibliography}{20})
%% AND THE LINE BELOW (\end{thebibliography})



\bibitem{1}
{\scshape Yeshkeyev A.R., Kassymetova M.T.},
{\bfseries\itshape Jonsson theories and their classes of models},
Monograph,
KSU,
2016.



\end{thebibliography}

\vspace*{-0.5\baselineskip}
% this space adjustment is usually necessary after a bibliography

\end{document}


%% READ ME
%% READ ME
%% READ ME

INSTRUCTIONS FOR SUPPLYING INFORMATION IN THE CORRECT FORMAT:

1. Author names are listed as First Last, First Last, and First Last.

\absauth{FirstName1 LastName1, FirstName2 LastName2, and FirstName3 LastName3}


2. Titles of abstracts have ONLY the first letter capitalized,
except for Proper Nouns.

\meettitle{Title of abstract with initial capital letter only, except for
Proper Nouns}


3. Affiliations and email addresses for authors of abstracts are
  listed separately.

% First author's affiliation
\affil{Department, University, Street Address, Country}
\meetemail{First author's email}
%%% NOTE: email required for at least one author
\urladdr{OPTIONAL}
%
% Second author's affiliation
\affil{Department, University, Street Address, Country}
\meetemail{Second author's email}
\urladdr{OPTIONAL}
%
% Third author's affiliation
\affil{Department, University, Street Address, Country}
% Second author's email
\meetemail{Third author's email}
\urladdr{OPTIONAL}


4. Bibliographic Entries

%%%% IF references are submitted with abstract,
%%%% please use the following formats

%%% For a Journal article
\bibitem{cite1}
{\scshape Author's Name},
{\itshape Title of article},
{\bfseries\itshape Journal name spelled out, no abbreviations},
vol.~XX (XXXX), no.~X, pp.~XXX--XXX.

%%% For a Journal article by the same authors as above,
%%% i.e., authors in cite1 are the same for cite2
\bibitem{cite2}
\bysame
{\itshape Title of article},
{\bfseries\itshape Journal},
vol.~XX (XXXX), no.~X, pp.~XX--XXX.

%%% For a book
\bibitem{cite3}
{\scshape Author's Name},
{\bfseries\itshape Title of book},
Name of series,
Publisher,
Year.

%%% For an article in proceedings
\bibitem{cite4}
{\scshape Author's Name},
{\itshape Title of article},
{\bfseries\itshape Name of proceedings}
(Address of meeting),
(First Last and First2 Last2, editors),
vol.~X,
Publisher,
Year,
pp.~X--XX.

%%% For an article in a collection
\bibitem{cite5}
{\scshape Author's Name},
{\itshape Title of article},
{\bfseries\itshape Book title}
(First Last and First2 Last2, editors),
Publisher,
Publisher's address,
Year,
pp.~X--XX.

%%% An edited book
\bibitem{cite6}
Author's name, editor. % No special font used here
{\bfseries\itshape Title of book},
Publisher,
Publisher's address,
Year.

