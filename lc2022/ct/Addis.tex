\documentclass[bsl,meeting]{asl}
\AbstractsOn
\pagestyle{plain}
\def\urladdr#1{\endgraf\noindent{\it URL Address}: {\tt #1}.}
\begin{document}
\thispagestyle{empty}

\absauth{Mark Addis}
\meettitle{Categorical representation of discrete dynamical systems computability}
\affil{Open University and London School of Economics and Political Science}
\meetemail{mark.addis@open.ac.uk}

In discrete dynamical systems computability is characterised by a state space of hereditarily finite sets combined with operations on those sets 
\cite{Sieg4}. A class of states and operations transforms a given state into a succeeding one, and isomorphism and invariance relations between 
states define structural classes \cite{Sieg3}. Such systems can be regarded as a generalisation of Gandy machines \cite{Gandy2} thus enabling 
representation of computable processes which extend beyond Turing machines. Since the representation is complex the logical and philosophical 
gains achieved from simplifying it through the use of the abstract model theory approach of the theory of institutions \cite{Diaconescu1} are 
considered. Such analysis contributes to the development of a category theory approach to the foundations of computability theory and 
philosophical reflection on the geometric aspects of certain kinds of computability.

\begin{thebibliography}{10}

\bibitem{Diaconescu1}
{\scshape Diaconescu, R.},
{\itshape Three decades of institutions}
{\bfseries\itshape Universal Logic: an Anthology},
(Beziau, J-Y, editor)
Springer, Basel,
2012,
pp.~309-322.

\bibitem{Gandy2}
{\scshape Gandy, R.},
{\itshape Church's Thesis and principles for mechanisms},
{\bfseries\itshape The Kleene Symposium},
(Amsterdam, North Holland),
(Barwise, J., Keisler, H., and Kunen, K., editors), 
1980,
pp.~123--148.

\bibitem{Sieg3}
{\scshape Sieg, W.},
{\itshape Calculations by man and machine: mathematical presentation}, 
{\bfseries\itshape The Scope of Logic, Methodology and Philosophy of Science},
(G\"{a}rdenfors, P., Wolesnki, J., and Kijania-Placek, K. editors),
Dordrecht, Kluwer,
2003,
pp.~247--262.

\bibitem{Sieg4}
{\scshape Sieg, W.},
{\itshape Church without dogma: axioms for computability},
{\bfseries\itshape New Computational Paradigms},
(Cooper, B., L\"{o}we, B., and Sorbi, A., editors),
Springer, New York,
2008,
pp.~139--152.

\end{thebibliography}

\vspace*{-0.5\baselineskip}
% this space adjustment is usually necessary after a bibliography

\end{document}
