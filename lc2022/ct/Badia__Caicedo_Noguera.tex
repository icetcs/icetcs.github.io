%% FIRST RENAME THIS FILE <yoursurname>.tex. 
%% BEFORE COMPLETING THIS TEMPLATE, SEE THE "READ ME" SECTION 
%% BELOW FOR INSTRUCTIONS. 
%% TO PROCESS THIS FILE YOU WILL NEED TO DOWNLOAD asl.cls from 
%% http://aslonline.org/abstractresources.html. 


\documentclass[bsl,meeting]{asl}

\AbstractsOn

\pagestyle{plain}

\def\urladdr#1{\endgraf\noindent{\it URL Address}: {\tt #1}.}


\newcommand{\NP}{}
%\usepackage{verbatim}

\begin{document}
\thispagestyle{empty}

%% BEGIN INSERTING YOUR ABSTRACT DIRECTLY BELOW; 
%% SEE INSTRUCTIONS (1), (2), (3), and (4) FOR PROPER FORMATS

\NP  
\absauth{Guillermo Badia, Xavier Caicedo, and Carles Noguera}
\meettitle{Frame definability in finitely-valued modal logic}
\affil{School of Historical and Philosophical Inquiry, University of Queensland, Brisbane, Australia}
\meetemail{g.badia@uq.edu.au}
%%% NOTE: email required for at least one author
\urladdr{https://sites.google.com/site/guillermobadialogic/home}
%
% Second author's affiliation
\affil{Departamento de Matem\'aticas, Universidad de los Andes, Bogot\'a, Colombia}
\meetemail{xcaicedo@uniandes.edu.co}
\urladdr{https://math.uniandes.edu.co/webxcaicedo/}
%
% Third author's affiliation
\affil{Department of Information Engineering and Mathematics, University of Siena, Siena, Italy}
% Second author's email
\meetemail{carles.noguera@unisi.it}
\urladdr{https://sites.google.com/view/carlesnoguera/bio-cv}


%% INSERT TEXT OF ABSTRACT DIRECTLY BELOW

In this paper we study frame definability in finitely-valued modal logics and establish two main results via suitable translations: (1) in finitely-valued modal logics one cannot define more classes of frames than are already definable in classical modal logic (cf.~\cite[Thm.~8]{tho}), and (2)  a large family of finitely-valued modal logics define exactly the same classes of frames as classical modal logic (including modal logics based on finite Heyting and MV-algebras). In this way one may observe, for example, that the celebrated Goldblatt--Thomason theorem applies immediately to these logics. In particular, we obtain the central result from~\cite{te} with a much simpler proof  and answer one of the open questions left in that paper. Moreover, the proposed translations allow us to determine the computational complexity of a big class of finitely-valued modal logics. Finally, we show that the first translation we offer  (from finitely-valued modal logic into two-valued modal logic) yields a  0-1 law over models for the former (cf. \cite{kapron1}) as a corollary of W. Oberschelp's generalization \cite{ober} of  Fagin's 0-1 law. In particular, one can show that, over Kripke models for finitely-valued modal logics based on finite frames, for every modal formula  there is a truth-value that it takes almost surely at all worlds.

%
\begin{thebibliography}{10}
%
%%% INSERT YOUR BIBLIOGRAPHIC ENTRIES HERE; 
%%% SEE (4) BELOW FOR PROPER FORMAT.
%%% EACH ENTRY MUST BEGIN WITH \bibitem{citation key}
%%%
%%% IF THERE ARE NO ENTRIES  
%%% DELETE THE LINE ABOVE (\begin{thebibliography}{20}) 
%%% AND THE LINE BELOW (\end{thebibliography})
%

\bibitem{te} 
B.~Teheux. Modal definability for \L ukasiewicz validity relations. \emph{Studia Logica} 104 (2): 343--363 (2016).

\bibitem{tho} 
S.K.~Thomason. Possible worlds and many truth values. \emph{Studia Logica} 37: 195--204 (1978).

\bibitem{kapron1}
J. Y. Halpern and B. M. Kapron, Zero-one laws for modal logic,
\emph{Annals of Pure and Applied Logic}, 69: 157–193 (1994).


\bibitem{ober}
Walter Oberschelp. Asymptotic 0-1 laws in combinatorics. In D.~Jungnickel (ed.), \emph{Combinatorial theory, Lecture Notes in Mathematics} 969:276--292, Springer, 1982.


\end{thebibliography}


\vspace*{-0.5\baselineskip}
% this space adjustment is usually necessary after a bibliography

\end{document}


%% READ ME
%% READ ME
%% READ ME

INSTRUCTIONS FOR SUPPLYING INFORMATION IN THE CORRECT FORMAT: 

1. Author names are listed as First Last, First Last, and First Last.

\absauth{FirstName1 LastName1, FirstName2 LastName2, and FirstName3 LastName3}


2. Titles of abstracts have ONLY the first letter capitalized,
except for Proper Nouns.

\meettitle{Title of abstract with initial capital letter only, except for
Proper Nouns} 


3. Affiliations and email addresses for authors of abstracts are
  listed separately.

% First author's affiliation
\affil{Department, University, Street Address, Country}
\meetemail{First author's email}
%%% NOTE: email required for at least one author
\urladdr{OPTIONAL}
%
% Second author's affiliation
\affil{Department, University, Street Address, Country}
\meetemail{Second author's email}
\urladdr{OPTIONAL}
%
% Third author's affiliation
\affil{Department, University, Street Address, Country}
% Second author's email
\meetemail{Third author's email}
\urladdr{OPTIONAL}


4. Bibliographic Entries

%%%% IF references are submitted with abstract,
%%%% please use the following formats

%%% For a Journal article
\bibitem{cite1}
{\scshape Author's Name},
{\itshape Title of article},
{\bfseries\itshape Journal name spelled out, no abbreviations},
vol.~XX (XXXX), no.~X, pp.~XXX--XXX.

%%% For a Journal article by the same authors as above,
%%% i.e., authors in cite1 are the same for cite2
\bibitem{cite2}
\bysame
{\itshape Title of article},
{\bfseries\itshape Journal},
vol.~XX (XXXX), no.~X, pp.~XX--XXX.

%%% For a book
\bibitem{cite3}
{\scshape Author's Name},
{\bfseries\itshape Title of book},
Name of series,
Publisher,
Year.

%%% For an article in proceedings
\bibitem{cite4}
{\scshape Author's Name},
{\itshape Title of article},
{\bfseries\itshape Name of proceedings}
(Address of meeting),
(First Last and First2 Last2, editors),
vol.~X,
Publisher,
Year,
pp.~X--XX.

%%% For an article in a collection
\bibitem{cite5}
{\scshape Author's Name},
{\itshape Title of article},
{\bfseries\itshape Book title}
(First Last and First2 Last2, editors),
Publisher,
Publisher's address,
Year,
pp.~X--XX.

%%% An edited book
\bibitem{cite6}
Author's name, editor. % No special font used here
{\bfseries\itshape Title of book},
Publisher,
Publisher's address,
Year.

