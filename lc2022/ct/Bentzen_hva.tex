\documentclass{article}
\begin{document}
Dissatisfaction with the philosophical thought of L.~E.~J. Brouwer has led to a growing interest over the last few decades in the
support of his intuitionism from a phenomenological approach, building on ideas from Husserl. The main supporters of this interpretation are Richard Tieszen and Mark van Atten. It is rooted in Heyting's idea that a proposition is an intention which is fulfilled with a proof-object of it. 

In this talk I argue against this propositions-as-intentions interpretation. I must stress at the outset that the interpretation is already a target of harsh criticisms regarding the incompatibility of Brouwer's and Husserl's positions, mainly from Guillermo Rosado Haddock or Claire Hill. But their objection consists in denying the interpretation its major premise. 
%%
This is not the direction I wish to take in this talk. Instead, I object that even if we grant that the incompatibility can be properly dealt with, as van Atten believes it can, one fundamental issues remain: it is far from clear what the object of an intention corresponding to a proposition should be. I argue that Heyting's own suggestion is inadequate and the most plausible candidates for intentional objects are sets of canonical proof-objects of the propositions. But this thesis immediately leads us to a difficult fulfillment dilemma: for Husserl, an intention is fulfilled when the intended object is genuinely presented to us in just the way it is intended; but here only one element of the set, not the set itself, can fulfill the intention. I conclude that the propositions-as-intentions leads to undesirable consequences. 
\end{document}