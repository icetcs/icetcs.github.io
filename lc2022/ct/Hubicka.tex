%% FIRST RENAME THIS FILE <yoursurname>.tex. 
%% BEFORE COMPLETING THIS TEMPLATE, SEE THE "READ ME" SECTION 
%% BELOW FOR INSTRUCTIONS. 
%% TO PROCESS THIS FILE YOU WILL NEED TO DOWNLOAD asl.cls from 
%% http://aslonline.org/abstractresources.html. 


\documentclass[bsl,meeting]{asl}

\AbstractsOn

\pagestyle{plain}

\def\urladdr#1{\endgraf\noindent{\it URL Address}: {\tt #1}.}


\newcommand{\NP}{}
%\usepackage{verbatim}

\begin{document}
\thispagestyle{empty}

%% BEGIN INSERTING YOUR ABSTRACT DIRECTLY BELOW; 
%% SEE INSTRUCTIONS (1), (2), (3), and (4) FOR PROPER FORMATS

\NP  
\absauth{Jan Hubi\v cka}
\meettitle{Big Ramsey degrees and trees with sucessor operation}
\affil{Department of Applied Mathematics (KAM), Charles University, Malostransk\'e n\'am.~25, Prague, Czech Republic}
\meetemail{hubicka@kam.mff.cuni.cz}

	Recent results on big Ramsey degrees (by Dobrinen on triangle-free and
	Henson graphs, by Zucker on free amalgamation classes) involve
	formulation of special purpose tree Ramsey theorems for trees with
	coding nodes. The proofs of such theorems are quite involved and
	follow the basic scheme of the Harrington's proof of the Milliken tree
	theorem via the method of forcing. The main technical difficulties come from the asymmetric placement
	of coding nodes and complicated definitions of subtrees which need to
	preserve structural properties.

	A recent link to the Carlson--Simpson theorem offers a new direct
	approach to obtaining these results.  We will discuss an abstract tree theorem
	for trees with a sucessor operation which can be used to show all known
	big Ramsey degrees on binary structures and generalises to some cases
	of structures of higher arity.  It can be seen as a joint
	strengthening of the Milliken tree theorem for regular trees and
	the Carlson--Simpson theorem for trees with unbounded branching.

	This is joint work with Balko, Chodounsk\' y, Dobrinen, Kone\v cn\'y, Ne\v set\v ril, Vena and Zucker.
\end{document}
