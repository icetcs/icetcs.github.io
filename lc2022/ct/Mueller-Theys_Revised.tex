% Abstract for LC 2022 
\documentclass[bsl,meeting]{asl}
\AbstractsOn
\pagestyle{plain}
\def\urladdr#1{\endgraf\noindent{\it URL Address}: {\tt #1}.}
\newcommand{\NP}{}
%\usepackage{verbatim}
\begin{document}
\thispagestyle{empty}
\NP  
\absauth{Joachim Mueller-Theys}
\meettitle{The inhomogeneity of concepts}
\affil{Independent researcher, Heidelberg, Germany}
\meetemail{mueller-theys@gmx.de}
\par 
We may think of 
$ P , Q , \ldots \subseteq M $
as properties or concepts. 
For 
$ a , b , \ldots \in M $, 
$ P ( a ) $ 
:iff 
$ a \in P $. 
$ P $ 
is 
{\it total} 
:iff 
$ P = M $, 
$ P $ 
{\it vacuous} 
:iff 
$ - P $
total,
$ P $ 
{\it real} 
:iff 
$ P $ 
not vacuous. 
We naturally call 
$ P $ 
{\it extreme}
:iff
$ P $ 
is 
vacuous 
or
total
iff. 
$ P $ 
real 
implies
$ P $ 
total.  
$ P $ 
{\it singular} 
:iff 
$ | P | = 1 $.
We 
naturally call 
$ P $ 
{\it genuine} 
iff 
$ P $ 
is neither 
vacuous 
nor
singular. 
\par 
We have defined
$ P ${\em -similarity} 
$ a \sim _P b $
by
$ P ( a ) \  \& \  P ( b ) $
and 
$ P ${\em -equality} 
$ a \equiv_P b $
by 
$ P ( a ) \Leftrightarrow P ( b ) $.
The basic connection is 
$ \sim _P \  \subseteq \  \equiv _P $; 
the converse is not true in general. 
\par
We say that 
$ Q $ 
{\it differentiates} 
$ P $ 
:iff 
$ P ( a ) $, 
$ P ( b ) $, 
but
$ a \not\equiv _Q b $
for some 
$ a $, 
$ b $.  
For example, 
evil differentiates human, 
transuranic differentiates element. 
We call 
$ P $ 
{\it inhomogeneous}
iff 
there exists
$ Q $ 
such that 
$ Q $ 
differentiates 
$ P $. 
Accordingly, 
human
and 
element 
are inhomogeneous. 
We have found that, 
in general, 
{\it all genuine concepts are inhomogeneous}:
Let 
$ P $ 
be genuine,
whence 
$ | P | \geq 2 $, 
whereby 
$ P ( a ) $, 
$ P ( b ) $
for some
$ a \neq b $.
Now let 
$ Q : = P \, \backslash \, \{ b \} $, 
whence
$ Q ( a ) $, 
but 
non 
$ Q ( b ) $, 
whereby 
$ a \not\equiv _Q b $. 
Thus
$ {\rm Diff} _Q (P) $, 
whence 
$ {\rm Inhom} ( P ) $. 
\par 
$ {\rm Inhom} ( P ) $
may be seen as formalisation of sayings of the form
``$ P $ {\it is not} $ P $'',
like human is not human, 
element is not element. 
Phrases of the form 
``$ P $ {\it is} $ P $'',
like 
``human is human'',
may be precisefied by 
$ a \equiv _P b $
for all 
$ a , b \in P $,
which is a special case 
of 
{\it ``all $ P $ are $ Q $-equal''}: 
$ {\rm AllEq} _Q (P) $
iff
$ {\rm non} $
$ {\rm Diff} _Q ( P ) $. 
$ {\it Dicho} _Q ( P ) $
$ : =  $
$ P \subseteq Q $
$ {\rm or} $
$ P , Q $
disjoint. 
We had found and proven 
the 
{\it Dichotomy Theorem}: 
$ {\rm AllEq} _Q (P) $ 
{\it iff}
$ {\rm Dicho} _Q ( P ) $. 
\par
Since 
$ P \subseteq - Q $ 
iff
$ P \, \cap \, Q = \emptyset $, 
$ {\it Hom} _Q ( P ) $
$ : = $ 
$ P \subseteq Q $
$ \rm or $ 
$ P \subseteq - Q $
iff
$ {\rm Dicho} _Q ( P ) $,
whence
$ {\rm AllEq} _Q (P) $ 
{\it iff}
$ {\rm Hom} _Q ( P ) $. 
Now as {\it corollaries}, 
$ {\rm AllEq} _P ( M ) $ 
{\it iff} 
Extr($ P $),
whereby 
human beings are equal only with respect to human, 
and 
$ {\rm Inhom} ( P ) $
iff
there is 
$ Q $
with
$ {\rm Inhom} _Q ( P ) $.
Moreover, 
$ {\rm Inhom} _Q ( P ) $
coincides 
with 
{\it heterogeneity}
$ {\it Het} _Q ( P ) $
$ : = $
$ P \cap Q \neq \emptyset $ 
$ \& $ 
$ P \cap - Q \neq \emptyset $. 
\par
A sophisticated formal interpretation of 
``$ P $ {\it is} $ P $''
may now be 
$ \forall Q $: 
$ {\rm Hom} _Q ( P ) $ 
$ \Rightarrow $
$ {\rm AllEq} _Q ( P ) $
(``all $ P $ are equal with respect to all homogeneous $ Q $''). 
It is curious that 
``$ P $ {\it is} $ P $''
and 
``$ P $ {\it is not} $ P $ if $ P $ is genuine''
are tautologies both. 
\par 
Joint work with Wilfried Buchholz.
For related achievements and acknowledgments, 
see 
``Similarity and equality'' 
(abstract \& talk, 
2021 North American Annual Meeting of the Association for Symbolic Logic 
(cf. Long Program),
The Bulletin of Symbolic Logic 27 (2021), p. 329),
``Mathematical theorems on equality and unequality''
(abstract, 
Logic Colloquium 2021, 
by title), 
``Equivalence''
(2022 ASL Annual Meeting, 
Long Program). 
\end{document}