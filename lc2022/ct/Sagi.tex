%% FIRST RENAME THIS FILE <yoursurname>.tex.
%% BEFORE COMPLETING THIS TEMPLATE, SEE THE "READ ME" SECTION
%% BELOW FOR INSTRUCTIONS.
%% TO PROCESS THIS FILE YOU WILL NEED TO DOWNLOAD asl.cls from
%% http://aslonline.org/abstractresources.html.


\documentclass[bsl,meeting]{asl}

\AbstractsOn

\pagestyle{plain}

\def\urladdr#1{\endgraf\noindent{\it URL Address}: {\tt #1}.}


\newcommand{\NP}{}
%\usepackage{verbatim}

\begin{document}
\thispagestyle{empty}

%% BEGIN INSERTING YOUR ABSTRACT DIRECTLY BELOW;
%% SEE INSTRUCTIONS (1), (2), (3), and (4) FOR PROPER FORMATS

\NP \absauth{G\'abor S\'agi} \meettitle{Automorphism invariant
measures on some structures and on their automorphism groups}
\affil{Alfr\'ed R\'enyi Institute of Mathematics, Hungarian
Academy of Sciences, H-1053 Budapest, Re\'altanoda u. 13-15,
Hungary and \\ Department of Algebra, Budapest University of
Technology and Economics, Budapest H-1117 Egry J. u. 1, Hungary}
\meetemail{sagi@renyi.hu}

%% INSERT TEXT OF ABSTRACT DIRECTLY BELOW
\ \\
\indent Let ${\mathcal A}$ be a countable $\aleph_{0}$-homogeneous
structure. Our primary motivation is to study different
amenability properties of (subgroups of) the automorphism group
$\mathit{Aut}({\mathcal A})$ of ${\mathcal A}$. The secondary
motivation is to study the existence of weakly generic
tuples of automorphisms of ${\mathcal A}$.\\
\indent Among others, we present sufficient conditions implying
the existence of automorphism invariant probability measures on
certain subsets of $A$ and $\mathit{Aut}({\mathcal A})$. We also
present sufficient conditions implying that the theory of
${\mathcal A}$ is
amenable. More concretely, our main results are as follows. \\
\\
{\bf Theorem 1.} If the set of locally finite automorphisms of
${\mathcal A}$ is dense (in particular, if ${\mathcal A}$ has
weakly generic tuples of automorphisms of arbitrary finite
length), then there exists a finitely additive probability
measure $\mu$ on the subsets of ${\mathcal A}$ definable with
parameters such that $\mu$ is invariant
under $\mathit{Aut}({\mathcal A})$.  \\
\\
{\bf Theorem 2.} if ${\mathcal A}$ is saturated and the set of its
locally finite automorphisms is dense (in particular, if
${\mathcal A}$ is saturated and has weak generics), then the
theory of ${\mathcal A}$ is amenable.





% \begin{thebibliography}{10}

%% INSERT YOUR BIBLIOGRAPHIC ENTRIES HERE;
%% SEE (4) BELOW FOR PROPER FORMAT.
%% EACH ENTRY MUST BEGIN WITH \bibitem{citation key}
%%
%% IF THERE ARE NO ENTRIES
%% DELETE THE LINE ABOVE (\begin{thebibliography}{20})
%% AND THE LINE BELOW (\end{thebibliography})

% \end{thebibliography}


\vspace*{-0.5\baselineskip}
% this space adjustment is usually necessary after a bibliography

\end{document}


%% READ ME
%% READ ME
%% READ ME

INSTRUCTIONS FOR SUPPLYING INFORMATION IN THE CORRECT FORMAT:

1. Author names are listed as First Last, First Last, and First Last.

\absauth{FirstName1 LastName1, FirstName2 LastName2, and FirstName3 LastName3}


2. Titles of abstracts have ONLY the first letter capitalized,
except for Proper Nouns.

\meettitle{Title of abstract with initial capital letter only, except for
Proper Nouns}


3. Affiliations and email addresses for authors of abstracts are
  listed separately.

% First author's affiliation
\affil{Department, University, Street Address, Country}
\meetemail{First author's email}
%%% NOTE: email required for at least one author
\urladdr{OPTIONAL}
%
% Second author's affiliation
\affil{Department, University, Street Address, Country}
\meetemail{Second author's email}
\urladdr{OPTIONAL}
%
% Third author's affiliation
\affil{Department, University, Street Address, Country}
% Second author's email
\meetemail{Third author's email}
\urladdr{OPTIONAL}


4. Bibliographic Entries

%%%% IF references are submitted with abstract,
%%%% please use the following formats

%%% For a Journal article
\bibitem{cite1}
{\scshape Author's Name},
{\itshape Title of article},
{\bfseries\itshape Journal name spelled out, no abbreviations},
vol.~XX (XXXX), no.~X, pp.~XXX--XXX.

%%% For a Journal article by the same authors as above,
%%% i.e., authors in cite1 are the same for cite2
\bibitem{cite2}
\bysame
{\itshape Title of article},
{\bfseries\itshape Journal},
vol.~XX (XXXX), no.~X, pp.~XX--XXX.

%%% For a book
\bibitem{cite3}
{\scshape Author's Name},
{\bfseries\itshape Title of book},
Name of series,
Publisher,
Year.

%%% For an article in proceedings
\bibitem{cite4}
{\scshape Author's Name},
{\itshape Title of article},
{\bfseries\itshape Name of proceedings}
(Address of meeting),
(First Last and First2 Last2, editors),
vol.~X,
Publisher,
Year,
pp.~X--XX.

%%% For an article in a collection
\bibitem{cite5}
{\scshape Author's Name},
{\itshape Title of article},
{\bfseries\itshape Book title}
(First Last and First2 Last2, editors),
Publisher,
Publisher's address,
Year,
pp.~X--XX.

%%% An edited book
\bibitem{cite6}
Author's name, editor. % No special font used here
{\bfseries\itshape Title of book},
Publisher,
Publisher's address,
Year.
