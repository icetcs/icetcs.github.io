%% FIRST RENAME THIS FILE <yoursurname>.tex. 
%% BEFORE COMPLETING THIS TEMPLATE, SEE THE "READ ME" SECTION 
%% BELOW FOR INSTRUCTIONS. 
%% TO PROCESS THIS FILE YOU WILL NEED TO DOWNLOAD asl.cls from 
%% http://aslonline.org/abstractresources.html. 


\documentclass[bsl,meeting]{asl}

\AbstractsOn

\pagestyle{plain}

\def\urladdr#1{\endgraf\noindent{\it URL Address}: {\tt #1}.}


\newcommand{\NP}{}
%\usepackage{verbatim}

\begin{document}
\thispagestyle{empty}

%% BEGIN INSERTING YOUR ABSTRACT DIRECTLY BELOW; 
%% SEE INSTRUCTIONS (1), (2), (3), and (4) FOR PROPER FORMATS



\NP  
\absauth{Anupamn Das, Avgerinos Delkos}
\meettitle{Proof complexity of monotone branching
programs}
\affil{School of Computer Science, University of Birmingham, B15 2TT, United Kingdom}
\meetemail{A.Das@bham.ac.uk}

\affil{School of Computer Science, University of Birmingham, B15 2TT, United Kingdom}
\meetemail{AxD1010@bham.StuDeNt.ac.uk}


%% INSERT TEXT OF ABSTRACT DIRECTLY BELOW

    We investigate the proof complexity of systems based on positive branching programs, 
i.e.\ non-deterministic branching programs (NBPs) where, for any 0-transition between two nodes, there is also a  1-transition. 
Positive NBPs compute monotone Boolean functions, like negation-free circuits or formulas, but constitute a positive version of (non-uniform) \textbf{NL}, rather than \textbf{P} or $\textbf{NC}^1$, respectively.

The proof complexity of NBPs was investigated in previous work by Buss, Das and Knop, using extension variables to represent the dag-structure, over a language of (non-deterministic) decision trees, yielding the system $\mathsf{eLNDT}$.
Our system $\mathsf{eLNDT}^+$ is obtained by restricting their systems to a positive syntax, similarly to how the `monotone sequent calculus' $\mathsf{MLK}$ is obtained from the usual sequent calculus $\mathsf{LK}$ by restricting to negation-free formulas.

Our main result is that $\mathsf{eLNDT}^+$ polynomially simulates  $\mathsf{eLNDT}$ over positive sequents. Our proof method is inspired by a similar result for $\mathsf{MLK}$ by Atserias, Galesi and Pudl\'ak, that was recently improved to a bona fide polynomial simulation via works 
of Je\v r\'abek and Buss, Kabanets, Kolokolova and Kouck\'y.
Along the way we formalise several properties of counting functions within $\mathsf{eLNDT}^+$ by polynomial-size proofs and, as a case study, give explicit polynomial-size poofs of the propositional pigeonhole principle.
\keywords{Proof Complexity  \and  Branching Programs \and Monotone Complexity}




%% INSERT YOUR BIBLIOGRAPHIC ENTRIES HERE; 
%% SEE (4) BELOW FOR PROPER FORMAT.
%% EACH ENTRY MUST BEGIN WITH \bibitem{citation key}
%%
%% IF THERE ARE NO ENTRIES  
%% DELETE THE LINE ABOVE (\begin{thebibliography}{20}) 
%% AND THE LINE BELOW (\end{thebibliography})




\vspace*{-0.5\baselineskip}
% this space adjustment is usually necessary after a bibliography

\end{document}


%% READ ME
%% READ ME
%% READ ME

INSTRUCTIONS FOR SUPPLYING INFORMATION IN THE CORRECT FORMAT: 

1. Author names are listed as First Last, First Last, and First Last.

\absauth{FirstName1 LastName1, FirstName2 LastName2, and FirstName3 LastName3}


2. Titles of abstracts have ONLY the first letter capitalized,
except for Proper Nouns.

\meettitle{Title of abstract with initial capital letter only, except for
Proper Nouns} 


3. Affiliations and email addresses for authors of abstracts are
  listed separately.

% First author's affiliation
\affil{Department, University, Street Address, Country}
\meetemail{First author's email}
%%% NOTE: email required for at least one author
\urladdr{OPTIONAL}
%
% Second author's affiliation
\affil{Department, University, Street Address, Country}
\meetemail{Second author's email}
\urladdr{OPTIONAL}
%
% Third author's affiliation
\affil{Department, University, Street Address, Country}
% Second author's email
\meetemail{Third author's email}
\urladdr{OPTIONAL}


4. Bibliographic Entries

%%%% IF references are submitted with abstract,
%%%% please use the following formats

%%% For a Journal article
\bibitem{cite1}
{\scshape Author's Name},
{\itshape Title of article},
{\bfseries\itshape Journal name spelled out, no abbreviations},
vol.~XX (XXXX), no.~X, pp.~XXX--XXX.

%%% For a Journal article by the same authors as above,
%%% i.e., authors in cite1 are the same for cite2
\bibitem{cite2}
\bysame
{\itshape Title of article},
{\bfseries\itshape Journal},
vol.~XX (XXXX), no.~X, pp.~XX--XXX.

%%% For a book
\bibitem{cite3}
{\scshape Author's Name},
{\bfseries\itshape Title of book},
Name of series,
Publisher,
Year.

%%% For an article in proceedings
\bibitem{cite4}
{\scshape Author's Name},
{\itshape Title of article},
{\bfseries\itshape Name of proceedings}
(Address of meeting),
(First Last and First2 Last2, editors),
vol.~X,
Publisher,
Year,
pp.~X--XX.

%%% For an article in a collection
\bibitem{cite5}
{\scshape Author's Name},
{\itshape Title of article},
{\bfseries\itshape Book title}
(First Last and First2 Last2, editors),
Publisher,
Publisher's address,
Year,
pp.~X--XX.

%%% An edited book
\bibitem{cite6}
Author's name, editor. % No special font used here
{\bfseries\itshape Title of book},
Publisher,
Publisher's address,
Year.

