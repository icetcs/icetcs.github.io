%% FIRST RENAME THIS FILE <yoursurname>.tex. 
%% BEFORE COMPLETING THIS TEMPLATE, SEE THE "READ ME" SECTION 
%% BELOW FOR INSTRUCTIONS. 
%% TO PROCESS THIS FILE YOU WILL NEED TO DOWNLOAD asl.cls from 
%% http://aslonline.org/abstractresources.html. 


\documentclass[bsl,meeting]{asl}

\AbstractsOn

\pagestyle{plain}

\def\urladdr#1{\endgraf\noindent{\it URL Address}: {\tt #1}.}


\usepackage{stmaryrd}

\newcommand{\NP}{}
\newcommand{\inqB }{\mathtt{inqB}}
\newcommand{\dna }{\mathtt{DNA}}
%\usepackage{verbatim}

\begin{document}
\thispagestyle{empty}

%% BEGIN INSERTING YOUR ABSTRACT DIRECTLY BELOW; 
%% SEE INSTRUCTIONS (1), (2), (3), and (4) FOR PROPER FORMATS

\NP  
\absauth{Joni Puljujärvi, Davide Emilio Quadrellaro}
\meettitle{Compactness and Types in Logics of Dependence}
% First author's affiliation
\affil{Department of Mathematics and Statistics, University of Helsinki, Finland}
\meetemail{joni.puljujarvi@helsinki.fi}
%
% Second author's affiliation
\affil{Department of Mathematics and Statistics, University of Helsinki, Finland}
\meetemail{davide.quadrellaro@gmail.com}

%% INSERT TEXT OF ABSTRACT DIRECTLY BELOW

In first-order logic, the following formulations of the compactness theorem can be easily proved from one another:
\begin{itemize}
    \item[(i)] \textit{Every set of sentences that is finitely satisfiable is satisfiable};
    \item[(ii)] \textit{Every set of formulas that is finitely satisfiable is satisfiable}.
\end{itemize}

\noindent  For dependence logic, the first version of compactness is a well-known result and was proved by V\"a\"an\"anen in \cite{cite1} using the translation between dependence logic and $\Sigma_1^1$. However, in the context of dependence logic, one cannot derive (ii) from (i) by replacing variables with constants, as it is the case for first order logic. 

The second version of compactness (ii) has been recently considered by Kontinen and Yang in \cite{cite3}, who used the translation from dependence logic to $\Sigma_1^1$ to show that ``\textit{every set of formulas with countably many free variables that is finitely satisfiable is satisfiable}". In our talk, we provide a proof of the second version of compactness (ii) for arbitrary sets of formulas by adapting ultraproducts to the context of team semantics, analogously to \cite{cite4}. 

Finally, we briefly touch upon the issue of types in dependence logic, and we see how to obtain a compact space of suitable type.

%
\begin{thebibliography}{10}


\bibitem{cite1}
{\scshape Jouko V\"a\"an\"anen},
{\itshape Dependence Logic: A New Approach to Independence Friendly Logic},
Cambridge University Press,
2007.



\bibitem{cite3}
{\scshape Juha Kontinen and Fan Yang},
{\itshape Complete logics for elementary team properties},
https://arxiv.org/abs/1904.08695.


\bibitem{cite4}
{\scshape Martin Lück},
{\itshape Team Logic
Axioms, Expressiveness, Complexity},
PhD thesis,
University of Hannover,
2020.


%%% INSERT YOUR BIBLIOGRAPHIC ENTRIES HERE; 
%%% SEE (4) BELOW FOR PROPER FORMAT.
%%% EACH ENTRY MUST BEGIN WITH \bibitem{citation key}
%%%
%%% IF THERE ARE NO ENTRIES  
%%% DELETE THE LINE ABOVE (\begin{thebibliography}{20}) 
%%% AND THE LINE BELOW (\end{thebibliography})
%
\end{thebibliography}


\vspace*{-0.5\baselineskip}
% this space adjustment is usually necessary after a bibliography

\end{document}


%% READ ME
%% READ ME
%% READ ME

INSTRUCTIONS FOR SUPPLYING INFORMATION IN THE CORRECT FORMAT: 

1. Author names are listed as First Last, First Last, and First Last.

\absauth{FirstName1 LastName1, FirstName2 LastName2, and FirstName3 LastName3}


2. Titles of abstracts have ONLY the first letter capitalized,
except for Proper Nouns.

\meettitle{Title of abstract with initial capital letter only, except for
Proper Nouns} 


3. Affiliations and email addresses for authors of abstracts are
  listed separately.

% First author's affiliation
\affil{Department, University, Street Address, Country}
\meetemail{First author's email}
%%% NOTE: email required for at least one author
\urladdr{OPTIONAL}
%
% Second author's affiliation
\affil{Department, University, Street Address, Country}
\meetemail{Second author's email}
\urladdr{OPTIONAL}
%
% Third author's affiliation
\affil{Department, University, Street Address, Country}
% Second author's email
\meetemail{Third author's email}
\urladdr{OPTIONAL}


4. Bibliographic Entries

%%%% IF references are submitted with abstract,
%%%% please use the following formats

%%% For a Journal article
\bibitem{cite1}
{\scshape Author's Name},
{\itshape Title of article},
{\bfseries\itshape Journal name spelled out, no abbreviations},
vol.~XX (XXXX), no.~X, pp.~XXX--XXX.

%%% For a Journal article by the same authors as above,
%%% i.e., authors in cite1 are the same for cite2
\bibitem{cite2}
\bysame
{\itshape Title of article},
{\bfseries\itshape Journal},
vol.~XX (XXXX), no.~X, pp.~XX--XXX.

%%% For a book
\bibitem{cite3}
{\scshape Author's Name},
{\bfseries\itshape Title of book},
Name of series,
Publisher,
Year.

%%% For an article in proceedings
\bibitem{cite4}
{\scshape Author's Name},
{\itshape Title of article},
{\bfseries\itshape Name of proceedings}
(Address of meeting),
(First Last and First2 Last2, editors),
vol.~X,
Publisher,
Year,
pp.~X--XX.

%%% For an article in a collection
\bibitem{cite5}
{\scshape Author's Name},
{\itshape Title of article},
{\bfseries\itshape Book title}
(First Last and First2 Last2, editors),
Publisher,
Publisher's address,
Year,
pp.~X--XX.

%%% An edited book
\bibitem{cite6}
Author's name, editor. % No special font used here
{\bfseries\itshape Title of book},
Publisher,
Publisher's address,
Year.
