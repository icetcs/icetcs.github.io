\documentclass[bsl,meeting]{asl}

\AbstractsOn

\pagestyle{plain}

\def\urladdr#1{\endgraf\noindent{\it URL Address}: {\tt #1}.}


\newcommand{\NP}{}
%\usepackage{verbatim}

\begin{document}
\thispagestyle{empty}


\NP  
\absauth{Gemma Robles}
\meettitle{The logic E-Mingle and its Routley-Meyer semantics}
\affil{Dpto. de Psicolog\'{i}a, Sociolog\'{i}a y Filosof\'{i}a, Universidad de Le\'{o}n, Campus Vegazana, s/n, 24071, Le\'{o}n, Spain}
\meetemail{gemma.robles@unileon.es}
\urladdr{http://grobv.unileon.es}

The logic R-Mingle (RM) is axiomatized when adding the ``mingle axiom'' (M: $A\rightarrow (A\rightarrow A)$) to Anderson and Belnap's logic of the relevant implication R. The logic E-Mingle (EM) is the result of  adding the ``restricted mingle axiom'' (Mr: $(A\rightarrow B)\rightarrow [(A\rightarrow B)\rightarrow (A\rightarrow B)]$) to Anderson and Belnap's logic of entailment E (cf. \cite{ABI}).

Contrary to what is the case with RM and its extensions, thoroughly investigated logics since the beginning of the ``relevance enterprise'' (cf. \cite{ABI}), practically everything is ignored about EM. In particular, this logic lacks a semantics whatsoever. The aim of this paper is to remedy this deficiency by providing a Routley-Meyer semantics for EM, despite the fact that the creators of this semantics think that it is no possible to interpret Mr in it (cf. \cite[\S 4.9]{RMPB}). EM is endowed with a Routley-Meyer semantics by giving it a Hilbert-style formulation in which Mr does not appear.


\begin{thebibliography}{10}

\bibitem{ABI}
{\scshape A. R. Anderson, N. D. Belnap},
{\bfseries\itshape Entailment. The Logic of Relevance and Necessity}, 
vol. I, 
Princeton University Press,
1975.

\bibitem{RMPB}
{\scshape R. Routley, R. K. Meyer, V. Plumwood, R. T. Brady},
{\bfseries\itshape Relevant Logics and their Rivals},
vol. 1,
Ridgeview Publishing Co., Atascadero, CA.
1982.

\end{thebibliography}


\vspace*{-0.5\baselineskip}
% this space adjustment is usually necessary after a bibliography

\medskip
{\itshape Acknowledgements}. Work supported by research project PID2020-116502GB-I00, financed by the Spanish Ministry of Science and Innovation (MCIN/AEI/ \\ 10.13039/501100011033).

\end{document}



