%% FIRST RENAME THIS FILE <yoursurname>.tex.
%% BEFORE COMPLETING THIS TEMPLATE, SEE THE "READ ME" SECTION
%% BELOW FOR INSTRUCTIONS.
%% TO PROCESS THIS FILE YOU WILL NEED TO DOWNLOAD asl.cls from
%% http://aslonline.org/abstractresources.html.


\documentclass[bsl,meeting]{asl}

\AbstractsOn

\pagestyle{plain}

\def\urladdr#1{\endgraf\noindent{\it URL Address}: {\tt #1}.}


\newcommand{\NP}{}
%\usepackage{verbatim}

\begin{document}
\thispagestyle{empty}

%% BEGIN INSERTING YOUR ABSTRACT DIRECTLY BELOW;
%% SEE INSTRUCTIONS (1), (2), (3), and (4) FOR PROPER FORMATS

\NP%
\absauth{Revantha Ramanayake}
\meettitle{Sequent calculi with restricted cuts for non-classical logics}
\affil{Bernoulli Institute, University of Groningen, The Netherlands}
\meetemail{d.r.s.ramanayake@rug.nl}
\urladdr{https://www.rug.nl/staff/d.r.s.ramanayake/}


%% INSERT TEXT OF ABSTRACT DIRECTLY BELOW

The primary motivation for cut-elimination is that it leads to a proof calculus with the subformula property. Such a proof calculus has a restricted proof search space and this is a powerful aid for investigating the properties of the logic. Unfortunately, many substructural and modal logics of interest lack a sequent calculus that supports cut-elimination. The overwhelming response since the 1960s has been to generalise the sequent calculus in a bid to regain cut-elimination. The price is that these generalised formalisms are more complicated to reason about and implement.

There is an alternative: remain with the sequent calculus by accepting weaker (but still meaningful) versions of the subformula property. We will discuss how cut-free hypersequent proofs can be transformed into sequent calculus proofs in a controlled way~\cite{CiaLanRam21}. Combined with the quite general methodology~\cite{CiaGalTer08} for transforming Hilbert axiomatic extensions into cut-free hypersequent calculi, this leads to an algorithm taking a Hilbert axiomatic extension to a sequent calculus with a weak subformula property. 

Can we avoid this detour through the hypersequent calculus? This goes to the heart of a new programme called \textit{cut-restriction} that aims to adapt Gentzen’s celebrated cut-elimination argument systematically so that cut-formulas are restricted (when elimination is not possible). We will present the early results in this programme: from arbitrary cuts  to analytic cuts in the sequent calculi for bi-intuitionistic logic and $S5$ via a uniform cut-restriction argument (the results themselves are well-known).

Based on joint work with Agata Ciabattoni (TU Wien) and Timo Lang (UCL).

\begin{thebibliography}{10}

%% INSERT YOUR BIBLIOGRAPHIC ENTRIES HERE; 
%% SEE (4) BELOW FOR PROPER FORMAT.
%% EACH ENTRY MUST BEGIN WITH \bibitem{citation key}
%%
%% IF THERE ARE NO ENTRIES  
%% DELETE THE LINE ABOVE (\begin{thebibliography}{20}) 
%% AND THE LINE BELOW (\end{thebibliography})

\bibitem{CiaLanRam21}
{\scshape Agata Ciabattoni and
               Timo Lang and
               Revantha Ramanayake},
{\itshape Bounded-analytic Sequent Calculi and Embeddings for Hypersequent Logics},
{\bfseries\itshape Journal of Symbolic Logic},
vol.~86 (2021), no.~2, pp.~635--668.

\bibitem{CiaGalTer08}
{\scshape A. Ciabattoni and N. Galatos and K. Terui},
{\itshape From axioms to analytic rules in nonclassical logics},
{\bfseries\itshape Proceedings of the Twenty-Third Annual {IEEE} Symposium on Logic in
               Computer Science (LICS)}
Pittsburgh, PA, {USA},
%(First Last and First2 Last2, editors),
{IEEE} Computer Society,
2008,
pp.~229--240.

\end{thebibliography}


\end{document}
