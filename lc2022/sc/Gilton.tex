
\documentclass[bsl,meeting]{asl}

\AbstractsOn

\pagestyle{plain}

\def\urladdr#1{\endgraf\noindent{\it URL Address}: {\tt #1}.}


\newcommand{\NP}{}
%\usepackage{verbatim}

\begin{document}
\thispagestyle{empty}

%% BEGIN INSERTING YOUR ABSTRACT DIRECTLY BELOW; 
%% SEE INSTRUCTIONS (1), (2), (3), and (4) FOR PROPER FORMATS

\NP  
\absauth{Thomas Gilton, Maxwell Levine, and {\v S}{\'a}rka Stejskalov{\'a}}
\meettitle{Club stationary reflection and consequences of square principles}
\affil{Department of Mathematics. The Dietrich School of  Arts and Sciences, 301 Thackeray Hall, Pittsburgh, PA 15260\\
Albert-Ludwigs-Universit{\"a}t Freiburg, Freiburg, Germany\\
Charles University, Prague, Czech Republic}
\meetemail{tdg25@pitt.edu}
\meetemail{maxwell.levine@mathematik.uni-freiburg.de}
\meetemail{sarka.stejskalova@ff.cuni.cz}

The square principle $\Box_\mu$, for a cardinal $\mu$, exerts a tremendous influence on the combinatorics of $\mu^+$ implying, for example, that on $\mu^+$ stationary reflection and the tree property fail, but that the approachability property holds. In \cite{8fold}, the authors showed that these three consequences of $\Box_\mu$ are mutually independent, in the sense that any of their eight Boolean combinations are consistent, from large cardinals, at $\kappa^{++}$, where $\kappa$ is either singular or regular.

Recently Levine, Stejskalov{\'a}, and I (\cite{GLS:8fold}) have continued this line of research, showing how to obtain \emph{Club Stationary Reflection} together with a variety of other combinatorics at a double successor of a regular. Moreover, Stejskalov{\'a} and I have recently shown (\cite{GS:8foldSingular}) how to fold Prikry-type forcings into these arguments to obtain similar results at the double successor of a cofinality $\omega$ singular.

In this talk, we will briefly review the impact that $\Box_\mu$ has on the combinatorics at $\mu^+$, and then sketch the main ideas for a number of our theorems, both in the regular and singular cases. In particular, we will discuss how we use weakly compact Laver diamonds to build our focings, and we will discuss new preservation theorems for club stationary reflection. If time permits, we will also discuss current work which involves Magidor forcing and uncountable cofinality singulars.


%A fruitful line of research in set theory investigates the tension between compactness and incompactness principles. Given this tension, it is of interest when principles in these categories are in fact jointly consistent. In a recent result with Omer Ben-Neria, we have established such a joint consistency result, showing that Club Stationary Reflection  (\cite{cite2}) is consistent with the Special Aronszajn Tree property (\cite{cite1}) on the cardinal $\omega_2$. 

%The tension between these two principles shows up in the very different properties of our posets (specializing, and club adding) which we must maintain throughout the course of our construction. To build the desired posets, we first introduce the idea of an $\mathcal{F}$\emph{-Strongly Proper} poset ($\mathcal{F}$ is the filter on $\kappa$ dual to the ineffability ideal). These posets use systems of continuous residue functions to witness strong genericity. We then show how to specialize trees on $\omega_2$ following a $\mathcal{F}$-strongly proper forcing, generalizing the classic result of Laver and Shelah.  We also show that the composition of Levy collapsing an ineffable cardinal followed by our club adding is $\mathcal{F}$-strongly proper. 



%We then show how to obtain $\mathcal{F}_{\operatorname{WC}}$-strongly proper posets by introducing the class of $\mathcal{F}_{\operatorname{WC}}$-\emph{Completely Proper} forcings. We show that the composition of the Levy collapse of a weakly compact with an $\mathcal{F}_{\operatorname{WC}}$-completely proper is $\mathcal{F}_{\operatorname{WC}}$-strongly proper. This resembles Abraham's use of guiding reals.

%Additionally, we develop new ideas for preserving Aronszajn trees and for stationary sets which do not make use of the usual closure assumptions. For instance, we show that our club adding posets don't add branches to Aronszajn trees of interest and that quotients of the specializing forcing preserve stationary sets of countable cofinality. 

%In this talk we will survey these two classes of posets and sketch our proof of specializing, as well as our preservation theorems. 


\begin{thebibliography}{10}

\bibitem{GLS:8fold}
Thomas Gilton, Maxwell Levine, and {\v S}{\'a}rka Stejskalov{\'a}. Trees and Stationary Reflection at Double Successors of Regular Cardinals. Accepted to {\it The Journal of Symbolic Logic} 

\bibitem{GS:8foldSingular}
Thomas Gilton and {\v S}{\'a}rka Stejskalov{\'a}. Compactness Principles at $\aleph_{\omega+2}$. In preparation.

\bibitem{8fold}
James Cummings, Sy-David Friedman, Menachem Magidor, Assaf Rinot,  and Dima Sinapova. The Eightfold Way. {\it The Journal of Symbolic Logic}. {\bf 83} (2018) no. 1, 349-371.

%vol.~XX (XXXX), no.~X, pp.~XXX--XXX.


%% INSERT YOUR BIBLIOGRAPHIC ENTRIES HERE; 
%% SEE (4) BELOW FOR PROPER FORMAT.
%% EACH ENTRY MUST BEGIN WITH \bibitem{citation key}
%%
%% IF THERE ARE NO ENTRIES  
%% DELETE THE LINE ABOVE (\begin{thebibliography}{20}) 
%% AND THE LINE BELOW (\end{thebibliography})

\end{thebibliography}


\vspace*{-0.5\baselineskip}

\end{document}

