\documentclass[bsl,meeting]{asl}

\AbstractsOn

\pagestyle{plain}

\def\urladdr#1{\endgraf\noindent{\it URL Address}: {\tt #1}.}

\newcommand{\NP}{}

\begin{document}
\thispagestyle{empty}

\NP  
\absauth{Ben {De Bondt}}
\meettitle{Some remarks on Namba-type forcings}
\affil{Institut de Math\'ematiques de Jussieu (IMJ-PRG), Universit\'e Paris Cit\'e, B\^atiment Sophie Germain, 8 Place Aur\'elie Nemours, 75013 Paris, France}
\meetemail{ben.de-bondt@imj-prg.fr}

For the purpose of this abstract, let ``a Namba-type forcing'' be any forcing that forces $\omega_2$ to get cofinality $\omega$ and doesn't collapse $\omega_1.$
It is well known that the existence of a semiproper Namba-type forcing is equivalent to a Strong Chang's Conjecture, but that instead the existence of a stationary set preserving Namba-type forcing is provable in plain $\mathsf{ZFC}.$ However, in the context of questions on iterated-forcing-using-side-conditions, it is natural to ask whether one can demand more than mere stationary set preservation and get provably in $\mathsf{ZFC}$ a Namba-type forcing that allows many (but not necessarily club many) models for which there exist sufficient semi-generic conditions.
In this talk I will discuss a ``side-condition version'' $\mathbb{P}$ of Namba forcing and explain that there exists a very natural projective stationary family of countable elementary submodels of $H_\theta$ such that  $\mathbb{P}$ is semiproper \emph{with respect to these models}. In fact, we can consider a notion of strong semiproperness, in analogy to the notion of strong properness and verify that $\mathbb{P}$ satisfies it, again with respect to these distinguished models.

As an application of this approach towards Namba forcing, we discuss a particularly natural presentation of an ``ersatz iterated Namba forcing'' which, given an increasing sequence $(\kappa_\alpha : \alpha < \gamma)$ of regular cardinals $\geq \omega_2,$ adds for every $\alpha < \gamma$ a countable cofinal subset of $\kappa_\alpha$, while at the same time preserving stationarity of stationary subsets of~$\omega_1.$ In the proof, we will make strong use of a technique involving labelled trees and games played on such trees that appears in \cite{cite1}.

Finally, we will mention closely related ongoing work and remaining questions.\\
This talk is based on joint work with my thesis supervisor Boban Veli\v{c}kovi\'c.


\begin{thebibliography}{10}

\bibitem{cite1} 
{\scshape Matthew Foreman and Menachem Magidor},
{\itshape Mutually stationary sequences of sets and the non-saturation of the non-stationary ideal on $P_\kappa(\lambda)$},
{\bfseries\itshape Acta Mathematica},
vol.~186 (2001), no.~2, pp.~271--300.

\bibitem{cite2}
{\scshape Saharon Shelah},
{\bfseries\itshape Proper and Improper Forcing},
Perspectives in Logic, vol.~5,
Cambridge University Press,
1998.

\end{thebibliography}

\vspace*{-0.5\baselineskip}

\end{document}

