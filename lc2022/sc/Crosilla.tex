%% FIRST RENAME THIS FILE <yoursurname>.tex. 
%% BEFORE COMPLETING THIS TEMPLATE, SEE THE "READ ME" SECTION 
%% BELOW FOR INSTRUCTIONS. 
%% TO PROCESS THIS FILE YOU WILL NEED TO DOWNLOAD asl.cls from 
%% http://aslonline.org/abstractresources.html. 


\documentclass[bsl,meeting]{asl}

\AbstractsOn

\pagestyle{plain}

\def\urladdr#1{\endgraf\noindent{\it URL Address}: {\tt #1}.}


\newcommand{\NP}{}
%\usepackage{verbatim}

\begin{document}
\thispagestyle{empty}

%% BEGIN INSERTING YOUR ABSTRACT DIRECTLY BELOW; 
%% SEE INSTRUCTIONS (1), (2), (3), and (4) FOR PROPER FORMATS

\NP  
\absauth{Laura Crosilla}
\meettitle{Hermann Weyl and the roots of mathematical logic}
\affil{Department of Philosophy, IFIKK, University of Oslo, Blindern, Norway}
\meetemail{Laura.Crosilla@ifikk.uio.no
}
Hermann Weyl’s book \textit{Das Kontinuum} \cite{Weyl18} presents a coherent and sophisticated approach to analysis from a predicativist perspective. In the first chapter of \cite{Weyl18}, Weyl introduces a system of predicative sets, built ``from the bottom up'' starting  from the natural numbers. He then goes on to show that large portions of 19th century analysis can be developed on that predicative basis. 
\textit{Das Kontinuum} anticipated and inspired fundamental ideas in mathematical logic, ideas that we find in the logical analysis of predicativity of the 1950-60's, in Solomon Feferman’s work on predicativity and in Errett Bishop’s constructive mathematics. The seeds of \textit{Das Kontinuum} are already visible in the early \cite{Weyl10}, where Weyl, among other things, offers a clarification of Zermelo’s axiom schema of Separation. In this talk, I examine key intriguing ideas in \cite{Weyl10}, ideas that witness important debates among mathematicians at the beginning of the 20th century. I then argue that aspects of \cite{Weyl10} foreshadow fundamental features of \textit{Das Kontinuum}. This allows us to consider \cite{Weyl18} under the new light offered by \cite{Weyl10}.
%% INSERT TEXT OF ABSTRACT DIRECTLY BELOW


\begin{thebibliography}{10}


\bibitem{Weyl10} 
Weyl, H., 1910,
{\em \"Uber die Definitionen der mathematischen Grundbegriffe}, Mathematisch-naturwissenschaftliche Bl\"atter, 
	7, pp. 93--95 and pp. 109--113.

\bibitem{Weyl18}
Weyl, H., 1918, {\em Das {K}ontinuum.
  {K}ritische {U}ntersuchungen \"uber die {G}rundlagen der {A}nalysis}, Veit,
  Leipzig.
\newblock Translated in English, Dover Books on Mathematics, 2003. (Page
  references are to the translation).
  
  

%% INSERT YOUR BIBLIOGRAPHIC ENTRIES HERE; 
%% SEE (4) BELOW FOR PROPER FORMAT.
%% EACH ENTRY MUST BEGIN WITH \bibitem{citation key}
%%
%% IF THERE ARE NO ENTRIES  
%% DELETE THE LINE ABOVE (\begin{thebibliography}{20}) 
%% AND THE LINE BELOW (\end{thebibliography})

\end{thebibliography}


\vspace*{-0.5\baselineskip}
% this space adjustment is usually necessary after a bibliography

\end{document}


%% READ ME
%% READ ME
%% READ ME

INSTRUCTIONS FOR SUPPLYING INFORMATION IN THE CORRECT FORMAT: 

1. Author names are listed as First Last, First Last, and First Last.

\absauth{FirstName1 LastName1, FirstName2 LastName2, and FirstName3 LastName3}


2. Titles of abstracts have ONLY the first letter capitalized,
except for Proper Nouns.

\meettitle{Title of abstract with initial capital letter only, except for
Proper Nouns} 


3. Affiliations and email addresses for authors of abstracts are
  listed separately.

% First author's affiliation
\affil{Department, University, Street Address, Country}
\meetemail{First author's email}
%%% NOTE: email required for at least one author
\urladdr{OPTIONAL}
%
% Second author's affiliation
\affil{Department, University, Street Address, Country}
\meetemail{Second author's email}
\urladdr{OPTIONAL}
%
% Third author's affiliation
\affil{Department, University, Street Address, Country}
% Second author's email
\meetemail{Third author's email}
\urladdr{OPTIONAL}


4. Bibliographic Entries

%%%% IF references are submitted with abstract,
%%%% please use the following formats

%%% For a Journal article
\bibitem{cite1}
{\scshape Author's Name},
{\itshape Title of article},
{\bfseries\itshape Journal name spelled out, no abbreviations},
vol.~XX (XXXX), no.~X, pp.~XXX--XXX.

%%% For a Journal article by the same authors as above,
%%% i.e., authors in cite1 are the same for cite2
\bibitem{cite2}
\bysame
{\itshape Title of article},
{\bfseries\itshape Journal},
vol.~XX (XXXX), no.~X, pp.~XX--XXX.

%%% For a book
\bibitem{cite3}
{\scshape Author's Name},
{\bfseries\itshape Title of book},
Name of series,
Publisher,
Year.

%%% For an article in proceedings
\bibitem{cite4}
{\scshape Author's Name},
{\itshape Title of article},
{\bfseries\itshape Name of proceedings}
(Address of meeting),
(First Last and First2 Last2, editors),
vol.~X,
Publisher,
Year,
pp.~X--XX.

%%% For an article in a collection
\bibitem{cite5}
{\scshape Author's Name},
{\itshape Title of article},
{\bfseries\itshape Book title}
(First Last and First2 Last2, editors),
Publisher,
Publisher's address,
Year,
pp.~X--XX.

%%% An edited book
\bibitem{cite6}
Author's name, editor. % No special font used here
{\bfseries\itshape Title of book},
Publisher,
Publisher's address,
Year.

