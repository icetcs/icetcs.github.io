%% FIRST RENAME THIS FILE <yoursurname>.tex. 
%% BEFORE COMPLETING THIS TEMPLATE, SEE THE "READ ME" SECTION 
%% BELOW FOR INSTRUCTIONS. 
%% TO PROCESS THIS FILE YOU WILL NEED TO DOWNLOAD asl.cls from 
%% http://aslonline.org/abstractresources.html. 


\documentclass[bsl,meeting]{asl}

\AbstractsOn

\pagestyle{plain}

\def\urladdr#1{\endgraf\noindent{\it URL Address}: {\tt #1}.}


\newcommand{\NP}{}
%\usepackage{verbatim}

\begin{document}
\thispagestyle{empty}

%% BEGIN INSERTING YOUR ABSTRACT DIRECTLY BELOW; 
%% SEE INSTRUCTIONS (1), (2), (3), and (4) FOR PROPER FORMATS

\NP  
\absauth{Alexis Saurin}
\meettitle{On the dynamics of cut-elimination for
  circular and non-wellfounded proofs}
\affil{IRIF, CNRS, Universit\'e Paris Cit\'e \& INRIA, Paris, France}
\meetemail{alexis.saurin@irif.fr}
%\urladdr{http://www.irif.fr/~saurin}

%% INSERT TEXT OF ABSTRACT DIRECTLY BELOW
%% On the dynamics of circular proofs
%% Computing with circular proofs
%% Good foundations for non-wellfounded proof theory
%% Nonwellfounded proof theory and computations

In this talk, I will consider the structural proof theory of
fixed-point logics and their cut-elimination
theorems, focusing on  their computational content. 

More specifically, I will consider logics with least and
greatest fixed-points, expressing inductive and coinductive
properties, and proof systems for those logics 
admitting ``circular'' and non-wellfounded
proofs~\cite{bouncing22,mumall16,fortier13,santo02}.
Those derivations are finitely branching but admit infinitely
deep branches, possibly subject to some regularity conditions.
Circular derivations are closely related with
proofs by infinite descent~\cite{broth11} and shall be equipped with a
global condition preventing vicious circles in proofs.

In order to unveil the computational content of those logical systems,
%from the cut-elimination procedure,
I will concentrate on linear logic
extended with least and greatest fixed points ($\mu\mathsf{LL}$),
that is, on the $\mu$-calculus considered
in a linear setting, where the structural rules of contraction
and weakening are prohibited (or carefully controlled at least).
%
In particular, following the spirit of structural proof-theory
and of the Curry-Howard correspondence, we will be interested
not only in the structure of provability but also in the
structure of proofs themselves, corresponding to programs (while
formulas correspond to data and codata types).
%% I will survey several topics,
%% notably the global validity condition, the productivity of
%% cut-elimination, the impact of a careful treatment of structural
%% rules, the question of proof invariants and the sequentiality of inferences.

I will first introduce the non-wellfounded proof systems
for $\mu\mathsf{LL}$ and for its exponential-free fragment,
$\mu\mathsf{MALL}$ (that is, multiplicative and additive linear
logic with least and greatest fixed points).
%
After establishing cut-elimination for $\mu\mathsf{MALL}$~\cite{mumall16}, I
will show how to generalize the cut-elimination result to 
$\mu\mathsf{LL}$ (as well as to the intuitionistic and
classical non-wellfounded sequent calculi).
%
After that, I will discuss limitations of the validity condition
considered above, from a computational perspective, and introduce
a more flexible validity condition, called bouncing-validity~\cite{bouncing22},
and establish a cut-elimination theorem for this richer system which,
while proving the same theorems, admits more valid proofs that is,
through the bridge of the Curry-Howard correspondence, more programs.
%% Time allowing, I shall survey some other perspectives on circular and
%% non-wellfounded proofs (focusing results, proof invariants as well as the impact of a careful
%% treatment of structural rules.

%In the spirit of structural proof-theory and of the Curry-Howard correspondence, we will not only be interested in the structure of provability but also in the structure of proofs themselves. I will survey several topics, notably the global validity condition, the productivity of cut-elimination, the impact of a careful treatment of structural rules, the question of proof invariants and the sequentiality of inferences.



%In this talk, I'll explain how by relaxing wellfoundedness assumption on formal proofs, one can design well-behaved proof systems expressing  inductive and coinductive properties. One shall also
%In this seminar, I will consider proof systems for logics expressing inductive and coinductive properties, focusing on "circular" and non-wellfounded proofs. Those derivations are finitely branching but admit infinitely deep branches, possibly subject to some regularity conditions. Circular derivations are closely related with "proofs by infinite descent" and shall be equipped with a global condition preventing vicious circles in proofs.
%After surveying several logics expressing (co)inductive properties and circular proofs,
%I will concentrate on the mu-calculus in a linear setting, that is, on linear logic extended with least and greatest fixed points.


\begin{thebibliography}{10}

%% INSERT YOUR BIBLIOGRAPHIC ENTRIES HERE; 
%% SEE (4) BELOW FOR PROPER FORMAT.
%% EACH ENTRY MUST BEGIN WITH \bibitem{citation key}
%%
%% IF THERE ARE NO ENTRIES  
%% DELETE THE LINE ABOVE (\begin{thebibliography}{20}) 
%% AND THE LINE BELOW (\end{thebibliography})


%%% For a Journal article


%% %David Baelde. 2012. Least and greatest fixed points in linear logic.
%% ACM Transactions on Computational Logic (TOCL) 13, 1 (2012), 2.
%% \bibitem{cite1}
%% {\scshape David Baelde},
%% {\itshape Least and greatest fixed points in linear logic},
%% {\bfseries\itshape ACM Transactions on Computational Logic},
%% vol.~13 (2012), no.~1, pp.~XXX--XXX.

\bibitem{bouncing22}
{\scshape David Baelde, {Amina
  Doumane}, {Denis Kuperberg}, {and} {Alexis Saurin}},
{\itshape Bouncing Threads for Circular and Non-wellfounded
  Proofs -- Towards Compositionality with Circular Proofs},
{\bfseries\itshape To appear in 37th Annual ACM/IEEE Symposium on Logic in Computer Science, LICS 2022}
(Haifa, Israel), {2022}.
%David Baelde, Amina Doumane, Denis Kuperberg, and Alexis Saurin.
%2022. Bouncing Threads for Circular and Non-wellfounded Proofs
  %(extended version). (June 2022).
  long version of the present paper,
available at {https://hal.archives-ouvertes.fr/hal-03682126}.

\bibitem{mumall16}
{\scshape David Baelde, Amina Doumane, and Alexis Saurin},
{\itshape Infinitary
Proof Theory: the Multiplicative Additive Case},
{\bfseries\itshape In 25th EACSL Annual Conference on Computer Science Logic, CSL 2016}
(Marseille, France),
(LIPIcs), Vol. 62. Schloss Dagstuhl -
Leibniz-Zentrum fuer Informatik, 42:1–42:17.
{http://www.dagstuhl.de/dagpub/978-3-95977-022-4}


\bibitem{broth11}
{\scshape James Brotherston and Alex Simpson},
{\itshape Sequent Calculi for Induction and Infinite Descent},
{\bfseries\itshape  Journal of Logic and Computation},
vol.~21 (2011), no.~6, pp.~1177--1216.


%%% For an article in proceedings
\bibitem{fortier13}
{\scshape J\'er\^{o}me Fortier and Luigi Santocanale},
{\itshape Cuts for Circular Proofs: Semantics and Cut-elimination},
{\bfseries\itshape Computer Science Logic 2013 (CSL 2013), CSL 2013}
(Torino, Italy),
(Simona Ronchi Della Rocca, editor),
 (LIPIcs), , Vol. 23. Schloss Dagstuhl - Leibniz-Zentrum
fuer Informatik, 2013, 248–262.

\bibitem{santo02}
{\scshape Luigi Santocanale},
{\itshape A Calculus of Circular Proofs and Its Categorical Semantics},
{\bfseries\itshape Foundations of Software Science and Computation
Structures},
(Mogens Nielsen and Uffe Engberg, editors),
vol.~2303,
Lecture Notes in Computer Science, Springer,
2002,
pp.~357--371


  
  
\end{thebibliography}


\vspace*{-0.5\baselineskip}
% this space adjustment is usually necessary after a bibliography

\end{document}


%% READ ME
%% READ ME
%% READ ME

INSTRUCTIONS FOR SUPPLYING INFORMATION IN THE CORRECT FORMAT: 

1. Author names are listed as First Last, First Last, and First Last.

\absauth{FirstName1 LastName1, FirstName2 LastName2, and FirstName3 LastName3}


2. Titles of abstracts have ONLY the first letter capitalized,
except for Proper Nouns.

\meettitle{Title of abstract with initial capital letter only, except for
Proper Nouns} 


3. Affiliations and email addresses for authors of abstracts are
  listed separately.

% First author's affiliation
\affil{Department, University, Street Address, Country}
\meetemail{First author's email}
%%% NOTE: email required for at least one author
\urladdr{OPTIONAL}
%
% Second author's affiliation
\affil{Department, University, Street Address, Country}
\meetemail{Second author's email}
\urladdr{OPTIONAL}
%
% Third author's affiliation
\affil{Department, University, Street Address, Country}
% Second author's email
\meetemail{Third author's email}
\urladdr{OPTIONAL}


4. Bibliographic Entries

%%%% IF references are submitted with abstract,
%%%% please use the following formats

%%% For a Journal article
\bibitem{cite1}
{\scshape Author's Name},
{\itshape Title of article},
{\bfseries\itshape Journal name spelled out, no abbreviations},
vol.~XX (XXXX), no.~X, pp.~XXX--XXX.

%%% For a Journal article by the same authors as above,
%%% i.e., authors in cite1 are the same for cite2
\bibitem{cite2}
\bysame
{\itshape Title of article},
{\bfseries\itshape Journal},
vol.~XX (XXXX), no.~X, pp.~XX--XXX.

%%% For a book
\bibitem{cite3}
{\scshape Author's Name},
{\bfseries\itshape Title of book},
Name of series,
Publisher,
Year.

%%% For an article in proceedings
\bibitem{cite4}
{\scshape Author's Name},
{\itshape Title of article},
{\bfseries\itshape Name of proceedings}
(Address of meeting),
(First Last and First2 Last2, editors),
vol.~X,
Publisher,
Year,
pp.~X--XX.

%%% For an article in a collection
\bibitem{cite5}
{\scshape Author's Name},
{\itshape Title of article},
{\bfseries\itshape Book title}
(First Last and First2 Last2, editors),
Publisher,
Publisher's address,
Year,
pp.~X--XX.

%%% An edited book
\bibitem{cite6}
Author's name, editor. % No special font used here
{\bfseries\itshape Title of book},
Publisher,
Publisher's address,
Year.

