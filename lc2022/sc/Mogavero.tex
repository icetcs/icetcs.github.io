
%%****************************************************************************%%
%%                                                                            %%
%% Article Title                                                              %%
%%                                                                            %%
%% Article.tex                                                                %%
%%                                                                            %%
%% Copyright (C) 20xx, Fabio Mogavero.                                        %%
%% All rights reserved.                                                       %%
%%                                                                            %%
%%****************************************************************************%%

% Begin of file Article.tex

\let\negmedspace\undefined
\let\negthickspace\undefined

\documentclass[bsl, meeting, bibother]{asl}

\input{Format}

% \input{Macros}

% \input{Figures}

% \input{Tables}

% \input{Algorithms}


\hyphenation{}


\hypersetup
  {
  pdftitle  = {Alternating (In)Dependence-Friendly Logic},
  pdfauthor = {F. Mogavero}
  }


\AbstractsOn
\pagestyle{plain}


\begin{document}

  \thispagestyle{empty}
  \absauth{Fabio Mogavero}
  \meettitle{Alternating (In)Dependence-Friendly Logic~\ddag}
  \affil{Universit\`a degli Studi di Napoli Federico II}
  \meetemail{fabio.mogavero@unina.it}

  \vspace{0.5em}

  \emph{Informational independence} is a phenomenon that emerges quite naturally
  in \emph{game theory}, as players in a game make moves based on what they know
  about the state of the current play~\cite{NM44}.
  %
  In games such as Chess or Go, both players have \emph{perfect information}
  about the current state of the play and the moves they and their adversary
  have previously made.
  For other games, like Poker and Bridge, the players have to make decisions
  based only on \emph{imperfect information} on the state of the play.
  %
  Given the tight connection between games and logics, think for instance at
  \emph{game-theoretic semantics}~\cite{Lor61,Lor68,Hin73}, a number of
  proposals have been put forward to reason with or about informational
  independence, most notably, \emph{Independence-Friendly Logic}~\cite{HS89},
  \emph{Dependence Logic}~\cite{Vaa07}, and logics derived thereof.

  Independence-Friendly Logic (\IF) was originally introduced by Hintikka and
  Sandu~\cite{HS89}, and later extensively studied, \eg, in~\cite{MSS11}, as an
  extension of \emph{First-Order Logic} (\FOL) with informational independence
  as first-class notion.
  Unlike in \FOL, where quantified variables always functionally depend on all
  the previously quantified ones, the values for quantified variables in \IF can
  be chosen independently of the values of specific variables quantified before
  in the formula.
  %
  From a general game-theoretic viewpoint, however, the \IF semantics exhibits
  some limitations.
  It treats the players asymmetrically, truly allowing only one of the two
  players to have imperfect information. In addition, sentences of the logic can
  only encode the existence of a uniform winning strategy for one of the two
  players and, as a consequence, \IF does admit undetermined sentences, which
  are neither true nor false.

  In this talk I will present an extension of \IF, called \emph{Alternating
  (In)Dependence Friendly Logic} (\ADIF), tailored to overcome these limitations
  and that appears more adequate when reasoning about games with full imperfect
  information is the main concern.
  To this end, we introduce a novel compositional semantics, generalising
  Hodges' semantics for \IF based on trumps/teams~\cite{Hod97a,Vaa07,MSS11},
  which
  \begin{inparaenum}[(i)]
    \item
      allows for restricting the two players, aiming at describing both
      symmetric and asymmetric imperfect information games,
    \item
      recovers the law of excluded middle for sentences, and
    \item
      grants \ADIF the full descriptive power of \emph{Second Order Logic}.
  \end{inparaenum}
  We also provide both an equivalent Herbrand-Skolem semantics and a
  game-theoretic semantics for the prenex fragment of \ADIF, the latter being
  defined in terms of a determined infinite-duration game that precisely
  captures the compositional semantics on finite structures.

  \vspace{0.5em}

  \footnotesize
  \bibliographystyle{plain}

  \begin{thebibliography}{1}

  \bibitem{Hin73}
  J.~Hintikka.
  \newblock {\em {Logic, Language-Games and Information: Kantian Themes in the
    Philosophy of Logic.}}
  \newblock Oxford University Press, 1973.

  \bibitem{HS89}
  J.~Hintikka and G.~Sandu.
  \newblock {Informational Independence as a Semantical Phenomenon.}
  \newblock In {\em ICLMPS'89}, pages 571--589. Elsevier, 1989.

  \bibitem{Hod97a}
  W.~Hodges.
  \newblock {Compositional Semantics for a Language of Imperfect Information.}
  \newblock {\em LJIGPL}, 5(4):539--563, 1997.

  \bibitem{Lor68}
  K.~Lorenz.
  \newblock {Dialogspiele als Semantische Grundlage von Logikkalk\"ulen.}
  \newblock {\em AMLG}, 11:32--55, 1968.

  \bibitem{Lor61}
  P.~Lorenzen.
  \newblock {Ein Dialogisches Konstruktivit\"atskriterium.}
  \newblock In {\em SFM'59}, pages 193--200. PWN, 1961.

  \bibitem{MSS11}
  A.L. Mann, G.~Sandu, and M.~Sevenster.
  \newblock {\em {Independence-Friendly Logic - A Game-Theoretic Approach.}}
  \newblock CUP, 2011.

  \bibitem{Vaa07}
  J.A. V{\"a}{\"a}n{\"a}nen.
  \newblock {\em {Dependence Logic: A New Approach to Independence Friendly
    Logic.}}, volume~70 of {\em {London Mathematical Society Student Texts.}}
  \newblock CUP, 2007.

  \bibitem{NM44}
  J.~von Neumann and O.~Morgenstern.
  \newblock {\em {Theory of Games and Economic Behavior.}}
  \newblock Princeton University Press, 1944.

  \end{thebibliography}

  \normalsize
  \noindent
  \ddag~Joint work with Dylan Bellier, Massimo Benerecetti, and Dario Della
  Monica.

\end{document}

% End of file Article.tex
