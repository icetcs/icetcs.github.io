%% FIRST RENAME THIS FILE <yoursurname>.tex.
%% BEFORE COMPLETING THIS TEMPLATE, SEE THE "READ ME" SECTION
%% BELOW FOR INSTRUCTIONS.
%% TO PROCESS THIS FILE YOU WILL NEED TO DOWNLOAD asl.cls from
%% http://aslonline.org/abstractresources.html.

\documentclass[bsl,meeting,11pt]{asl}

\AbstractsOn

\pagestyle{plain}

\def\urladdr#1{\endgraf\noindent{\it URL Address}: {\tt #1}.}

\newcommand{\NP}{}
%\usepackage{verbatim}

\begin{document}
\thispagestyle{empty}

%% BEGIN INSERTING YOUR ABSTRACT DIRECTLY BELOW;
%% SEE INSTRUCTIONS (1), (2), (3), and (4) FOR PROPER FORMATS

\NP
\absauth{Wei Wang}
\meettitle{Ackermann Function and Reverse Mathematics}
\affil{Department of Philosophy and Institute of Logic and Cognition, Sun Yat-sen University, Guangzhou 510275, P. R. China}
\meetemail{wangw68@mail.sysu.edu.cn, wwang.cn@gmail.com}

%% INSERT TEXT OF ABSTRACT DIRECTLY BELOW

In 1928, Ackermann \cite{Ackermann} defined one of the first examples of recursive but not primitive recursive functions.
Later in 1935, R\'{o}zsa P\'{e}ter \cite{Peter} provided a simplification, which is now known as Ackermann or Ackermann-P\'{e}ter function.
The totality of Ackermann-P\'{e}ter function is an interesting subject in the study of fragments of first order arithmetic.
Kreuzer and Yokoyama \cite{KY} prove that the totality of Ackermann-P\'{e}ter function is equivalent to a $\Sigma_3$-proposition called $P\Sigma_1$.
And $P\Sigma_1$ has played important roles in reverse mathematics in recent years.
We will see some examples in this talk, including some joint works \cite{CLWY, CWY} of the speaker and logicians in Singapore.

\begin{thebibliography}{1}

\bibitem{Ackermann}
{\scshape Ackermann, Wilhelm},
{\itshape Zum {H}ilbertschen {A}ufbau der reellen {Z}ahlen},
{\bfseries\itshape Mathematische Annalen}, 99(1):118--133, 1928.

\bibitem{CLWY}
{\scshape Chong, Chitat and Li, Wei and Wang, Wei and Yang, Yue},
{\itshape On the strength of Ramsey's theorem for trees},
{\bfseries\itshape Advances in Mathematics}, 369:107180, 39 pp, 2020.

\bibitem{CWY}
{\scshape Chong, Chitat and Wang, Wei and Yang, Yue},
{\itshape Conservation Strength of The Infinite Pigeonhole Principle for Trees},
{\bfseries\itshape Israel Journal of Mathematics}, to appear,
{\itshape https://arxiv.org/abs/2110.06026.}

\bibitem{KY}
{\scshape Kreuzer, Alexander P. and Yokoyama, Keita},
{\itshape On principles between $\Sigma_1$- and $\Sigma_2$-induction, and monotone enumerations},
{\bfseries\itshape Journal of Mathematical Logic}, 16(1):1650004, 21 pp, 2016.

\bibitem{Peter}
{\scshape P\'{e}ter, R\'{o}zsa},
{\itshape Konstruktion nichtrekursiver {F}unktionen},
{\bfseries\itshape Mathematische Annalen}, 111(1):42--60, 1935.

\end{thebibliography}

\vspace*{-0.5\baselineskip}
% this space adjustment is usually necessary after a bibliography
\end{document}


%% READ ME
%% READ ME
%% READ ME

INSTRUCTIONS FOR SUPPLYING INFORMATION IN THE CORRECT FORMAT:

1. Author names are listed as First Last, First Last, and First Last.

\absauth{FirstName1 LastName1, FirstName2 LastName2, and FirstName3 LastName3}


2. Titles of abstracts have ONLY the first letter capitalized,
except for Proper Nouns.

\meettitle{Title of abstract with initial capital letter only, except for
Proper Nouns}


3. Affiliations and email addresses for authors of abstracts are
  listed separately.

% First author's affiliation
\affil{Department, University, Street Address, Country}
\meetemail{First author's email}
%%% NOTE: email required for at least one author
\urladdr{OPTIONAL}
%
% Second author's affiliation
\affil{Department, University, Street Address, Country}
\meetemail{Second author's email}
\urladdr{OPTIONAL}
%
% Third author's affiliation
\affil{Department, University, Street Address, Country}
% Second author's email
\meetemail{Third author's email}
\urladdr{OPTIONAL}


4. Bibliographic Entries

%%%% IF references are submitted with abstract,
%%%% please use the following formats

%%% For a Journal article
\bibitem{cite1}
{\scshape Author's Name},
{\itshape Title of article},
{\bfseries\itshape Journal name spelled out, no abbreviations},
vol.~XX (XXXX), no.~X, pp.~XXX--XXX.

%%% For a Journal article by the same authors as above,
%%% i.e., authors in cite1 are the same for cite2
\bibitem{cite2}
\bysame
{\itshape Title of article},
{\bfseries\itshape Journal},
vol.~XX (XXXX), no.~X, pp.~XX--XXX.

%%% For a book
\bibitem{cite3}
{\scshape Author's Name},
{\bfseries\itshape Title of book},
Name of series,
Publisher,
Year.

%%% For an article in proceedings
\bibitem{cite4}
{\scshape Author's Name},
{\itshape Title of article},
{\bfseries\itshape Name of proceedings}
(Address of meeting),
(First Last and First2 Last2, editors),
vol.~X,
Publisher,
Year,
pp.~X--XX.

%%% For an article in a collection
\bibitem{cite5}
{\scshape Author's Name},
{\itshape Title of article},
{\bfseries\itshape Book title}
(First Last and First2 Last2, editors),
Publisher,
Publisher's address,
Year,
pp.~X--XX.

%%% An edited book
\bibitem{cite6}
Author's name, editor. % No special font used here
{\bfseries\itshape Title of book},
Publisher,
Publisher's address,
Year.
