
\documentclass[bsl,meeting]{asl}

\AbstractsOn

\pagestyle{plain}

\def\urladdr#1{\endgraf\noindent{\it URL Address}: {\tt #1}.}

\newcommand{\NP}{}

\begin{document}
\thispagestyle{empty}

\NP  
\absauth{Denis R. Hirschfeldt}
\meettitle{The strength of versions of Mycielski's Theorem}
\affil{Department of Mathematics, University of Chicago, 5734
S. University Ave., Chicago, IL 60637, USA}
\meetemail{drh@uchicago.edu}

Mycielski's Theorem is a Ramsey-theoretic result on the reals with
versions for measure and for category. These imply respectively that
there is a perfect tree whose paths are all relatively $1$-random, and
that there is a perfect tree whose paths are all relatively
$1$-generic. In fact, in relativized form, the latter two statements
are equivalent to the two versions of Mycielski's Theorem. I will
discuss joint work with Carl G. Jockusch, Jr. and Paul E. Schupp on
the computability-theoretic and reverse-mathematical strength of these
statements.

\end{document}
