%% FIRST RENAME THIS FILE <yoursurname>.tex. 
%% BEFORE COMPLETING THIS TEMPLATE, SEE THE "READ ME" SECTION 
%% BELOW FOR INSTRUCTIONS. 
%% TO PROCESS THIS FILE YOU WILL NEED TO DOWNLOAD asl.cls from 
%% http://aslonline.org/abstractresources.html. 


\documentclass[bsl,meeting]{asl}

\AbstractsOn

\pagestyle{plain}

\def\urladdr#1{\endgraf\noindent{\it URL Address}: {\tt #1}.}


\newcommand{\NP}{}
%\usepackage{verbatim}

\begin{document}
\thispagestyle{empty}

%% BEGIN INSERTING YOUR ABSTRACT DIRECTLY BELOW; 
%% SEE INSTRUCTIONS (1), (2), (3), and (4) FOR PROPER FORMATS

\NP  
\absauth{Katarzyna W. Kowalik}
\meettitle{A non speed-up result for the chain-antichain principle over a weak base theory}
\affil{Faculty of Mathematics Informatics and Mechanics, University of Warsaw,\\ Banacha 2, 02-097 Warszawa, Poland}
\meetemail{katarzyna.kowalik@mimuw.edu.pl}


%% INSERT TEXT OF ABSTRACT DIRECTLY BELOW
The chain-antichain principle (CAC), a well-known consequence of Ramsey's Theorem for pairs and two colours, says that for every partial order on $\mathbb{N}$ there exists an infinite chain or antichain with respect to this order. We study the strength of this principle over the weak base theory $\mathrm{RCA}_0^*$, which is obtained from $\mathrm{RCA}_0$ by replacing the $\Sigma^0_1$-induction scheme with $\Delta^0_1$-induction. %and totality of exponential function.

It was shown by Patey and Yokoyama in \cite{py} that $\mathrm{RT}_2^2$ is $\Pi^0_3$-conservative over $\mathrm{RCA}_0$ and from \cite{yokoyama} it follows that $\mathrm{RT}_2^2$ is also $\Pi^0_3$-conservative over $\mathrm{RCA}^*_0$ (cf. \cite{kky}). 
The conservativity results lead to the question whether $\mathrm{RT}_2^2$ has significantly shorter proofs for $\Pi^0_3$-sentences. The answer depends on the choice of the base theory: it was proved in \cite{pfsize} that $\mathrm{RT}_2^2$ can be polynomially simulated by $\mathrm{RCA}_0$ for $\Pi^0_3$-sentences but it has non-elementary speed-up over $\mathrm{RCA}^*_0$ for $\Sigma^0_1$-sentences.

The speed-up result was obtained by the use of the exponential lower bound for the finite version of $\mathrm{RT}_2^2$. However, it follows from Dilworth's theorem that the upper bound for the finite version of CAC is polynomial. This suggests that CAC, 
despite being a relatively strong consequence of $\mathrm{RT}_2^2$, might not have an analogous speed-up over $\mathrm{RCA}^*_0$. We confirm this conjecture by constructing a two-step forcing interpretation of $\mathrm{RCA}^*_0+$CAC in $\mathrm{RCA}^*_0$. 



\begin{thebibliography}{10}
 


\bibitem{kky}
{\scshape Leszek A. Ko\l{}odziejczyk, Katarzyna W. Kowalik, Keita Yokoyama},
{\itshape How strong is Ramsey's theorem if infinity can be weak?} Submitted. Available at arXiv:2011.02550.


\bibitem{pfsize}
{\scshape Leszek A. Ko\l{}odziejczyk, Tin Lok Wong, Keita Yokoyama},
{\itshape Ramsey's theorem for pairs, collection, and proof size.} Submitted.
Available at arXiv:2005.06854.

\bibitem{py}
{\scshape Ludovic Patey, Keita Yokoyama},
{\itshape The proof-theoretic strength of Ramsey's theorem
for pairs and two colors},
{\bfseries\itshape Advances in Mathematics},
vol.~330 (2018), pp.~1034--1070.


\bibitem{yokoyama}
{\scshape Keita Yokoyama},
{\itshape On the strength of Ramsey's theorem without {$\Sigma_1$}-induction},
{\bfseries\itshape Mathematical Logic Quarterly},
vol.~59 (2013), no.~1-2, pp.~108--111.


%% INSERT YOUR BIBLIOGRAPHIC ENTRIES HERE; 
%% SEE (4) BELOW FOR PROPER FORMAT.
%% EACH ENTRY MUST BEGIN WITH \bibitem{citation key}
%%
%% IF THERE ARE NO ENTRIES  
%% DELETE THE LINE ABOVE (\begin{thebibliography}{20}) 
%% AND THE LINE BELOW (\end{thebibliography})

\end{thebibliography}


\vspace*{-0.5\baselineskip}
% this space adjustment is usually necessary after a bibliography

\end{document}


%% READ ME
%% READ ME
%% READ ME

INSTRUCTIONS FOR SUPPLYING INFORMATION IN THE CORRECT FORMAT: 

1. Author names are listed as First Last, First Last, and First Last.

\absauth{FirstName1 LastName1, FirstName2 LastName2, and FirstName3 LastName3}


2. Titles of abstracts have ONLY the first letter capitalized,
except for Proper Nouns.

\meettitle{Title of abstract with initial capital letter only, except for
Proper Nouns} 


3. Affiliations and email addresses for authors of abstracts are
  listed separately.

% First author's affiliation
\affil{Department, University, Street Address, Country}
\meetemail{First author's email}
%%% NOTE: email required for at least one author
\urladdr{OPTIONAL}
%
% Second author's affiliation
\affil{Department, University, Street Address, Country}
\meetemail{Second author's email}
\urladdr{OPTIONAL}
%
% Third author's affiliation
\affil{Department, University, Street Address, Country}
% Second author's email
\meetemail{Third author's email}
\urladdr{OPTIONAL}


4. Bibliographic Entries

%%%% IF references are submitted with abstract,
%%%% please use the following formats

%%% For a Journal article
\bibitem{cite1}
{\scshape Author's Name},
{\itshape Title of article},
{\bfseries\itshape Journal name spelled out, no abbreviations},
vol.~XX (XXXX), no.~X, pp.~XXX--XXX.

%%% For a Journal article by the same authors as above,
%%% i.e., authors in cite1 are the same for cite2
\bibitem{cite2}
\bysame
{\itshape Title of article},
{\bfseries\itshape Journal},
vol.~XX (XXXX), no.~X, pp.~XX--XXX.

%%% For a book
\bibitem{cite3}
{\scshape Author's Name},
{\bfseries\itshape Title of book},
Name of series,
Publisher,
Year.

%%% For an article in proceedings
\bibitem{cite4}
{\scshape Author's Name},
{\itshape Title of article},
{\bfseries\itshape Name of proceedings}
(Address of meeting),
(First Last and First2 Last2, editors),
vol.~X,
Publisher,
Year,
pp.~X--XX.

%%% For an article in a collection
\bibitem{cite5}
{\scshape Author's Name},
{\itshape Title of article},
{\bfseries\itshape Book title}
(First Last and First2 Last2, editors),
Publisher,
Publisher's address,
Year,
pp.~X--XX.

%%% An edited book
\bibitem{cite6}
Author's name, editor. % No special font used here
{\bfseries\itshape Title of book},
Publisher,
Publisher's address,
Year.

