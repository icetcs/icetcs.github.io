%% FIRST RENAME THIS FILE <yoursurname>.tex.
%% BEFORE COMPLETING THIS TEMPLATE, SEE THE "READ ME" SECTION
%% BELOW FOR INSTRUCTIONS.
%% TO PROCESS THIS FILE YOU WILL NEED TO DOWNLOAD asl.cls from
%% http://aslonline.org/abstractresources.html.

\documentclass[bsl,meeting]{asl}

\AbstractsOn

\pagestyle{plain}

%\usepackage{amsmath,amsfonts,amssymb,amsthm}
\usepackage[utf8]{inputenc}
%\usepackage[USenglish]{babel}
%\usepackage{natbib}


\newcommand{\rca}{\mathsf{RCA}_0}
\newcommand{\wwklz}{\mathsf{WWKL}_0}
\newcommand{\wklz}{\mathsf{WKL}_0}
\newcommand{\aca}{\mathsf{ACA}_0}

\newcommand{\wkl}{\mathsf{WKL}}
\newcommand{\rt}{\mathsf{RT}}
\newcommand{\srt}{\mathsf{SRT}}
\newcommand{\ads}{\mathsf{ADS}}
\newcommand{\coh}{\mathsf{COH}}
\newcommand{\ocoh}{\mathsf{oCOH}}
\newcommand{\ocohd}{\mathsf{oCOH\mbox{-}in\mbox{-}}\mathbf{\Delta}^0_2}
\newcommand{\infone}{\mathsf{INForONE}}
\newcommand{\infoned}{\mathsf{INForONE\mbox{-}in\mbox{-}}\mathbf{\Delta}^0_2}
\newcommand{\infonesd}{\mathsf{INForONE\mbox{-}in}^*\mathsf{\mbox{-}}\mathbf{\Delta}^0_2}
\newcommand{\tdnr}{\mathrm{DNR}}
\newcommand{\pdnr}{\mathsf{DNR}}
\newcommand{\npdnr}[1]{#1\mbox{-}\mathsf{DNR}}
\newcommand{\rsg}{\mathsf{RSg}}
\newcommand{\rsgr}{\mathsf{RSgr}}
\newcommand{\wrsg}{\mathsf{wRSg}}
\newcommand{\wrsgr}{\mathsf{wRSgr}}
\newcommand{\id}{\mathsf{id}}
\newcommand{\sads}{\mathsf{SADS}}
\newcommand{\cads}{\mathsf{CADS}}
\newcommand{\cac}{\mathsf{CAC}}
\newcommand{\wscac}{\mathsf{WSCAC}}
\newcommand{\dnc}{\mathsf{DNC}}
\newcommand{\nwt}{\mathsf{NWT}}
\newcommand{\mst}{\mathsf{MST}}

\newcommand{\brt}{\mathsf{BRT}}
\newcommand{\urt}{\mathsf{URT}}

\newcommand{\cn}{\mathsf{C}_{\Nb}}
\newcommand{\lpo}{\mathsf{LPO}}
%\newcommand{\pr}{\mathsf{P}}

\newcommand{\bso}{\mathsf{B}\Sigma^0_1}
\newcommand{\iso}{\mathsf{I}\Sigma^0_1}
\newcommand{\bpo}{\mathsf{B}\Pi^0_1}
\newcommand{\bst}{\mathsf{B}\Sigma^0_2}
\newcommand{\ist}{\mathsf{I}\Sigma^0_2}
\newcommand{\isig}[1]{\mathsf{I}\Sigma^0_{#1}}
\newcommand{\bsig}[1]{\mathsf{B}\Sigma^0_{#1}}

\newcommand{\bl}{\bigl(}
\newcommand{\br}{\bigr)}
\newcommand{\andd}{\wedge}
\newcommand{\orr}{\vee}
\newcommand{\la}{\langle}
\newcommand{\ra}{\rangle}
\newcommand{\da}{{\downarrow}}
\newcommand{\ua}{{\uparrow}}
\newcommand{\imp}{\rightarrow}
\newcommand{\Imp}{\Rightarrow}
\newcommand{\biimp}{\leftrightarrow}
\newcommand{\Biimp}{\Leftrightarrow}
\newcommand{\smf}{\smallfrown}
\newcommand{\rst}{{\restriction}}
\newcommand{\rra}{\rightrightarrows}
\newcommand{\wsubseteq}{\!\!\subseteq\!}

\newcommand{\mc}[1]{\mathcal{#1}}
\newcommand{\bs}[1]{{\mathbf #1}}

\newcommand{\leqT}{\leq_\mathrm{T}}
\newcommand{\leT}{<_\mathrm{T}}
\newcommand{\geqT}{\geq_\mathrm{T}}
\newcommand{\geT}{>_\mathrm{T}}
\newcommand{\equivT}{\equiv_\mathrm{T}}

\newcommand{\leqW}{\leq_\mathrm{W}}
\newcommand{\leW}{<_\mathrm{W}}
\newcommand{\geqW}{\geq_\mathrm{W}}
\newcommand{\geW}{>_\mathrm{W}}
\newcommand{\equivW}{\equiv_\mathrm{W}}

\newcommand{\leqsW}{\leq_\mathrm{sW}}
\newcommand{\lesW}{<_\mathrm{sW}}
\newcommand{\geqsW}{\geq_\mathrm{sW}}
\newcommand{\gesW}{>_\mathrm{sW}}
\newcommand{\equivsW}{\equiv_\mathrm{sW}}
\newcommand{\leqc}{\leq_c}

\newcommand{\ol}[1]{\overline{#1}}
\newcommand{\code}[1]{\ulcorner #1 \urcorner}
\newcommand{\mbf}[1]{\mathbf{#1}}
\newcommand{\msf}[1]{\mathsf{#1}}
% \newcommand{\spc}[1]{\mathcal{#1}}

\newcommand{\Pf}{\mathcal{P}_\mathrm{f}}

\newcommand{\fop}[1]{{}^1{#1}}
\newcommand{\dmd}[1]{{#1}^\diamond}
\newcommand{\Ff}{\mathcal{F}}


\def\urladdr#1{\endgraf\noindent{\it URL Address}: {\tt #1}.}


\newcommand{\NP}{}
%\usepackage{verbatim}

\begin{document}
\thispagestyle{empty}

%% BEGIN INSERTING YOUR ABSTRACT DIRECTLY BELOW; 
%% SEE INSTRUCTIONS (1), (2), (3), and (4) FOR PROPER FORMATS

\NP  
\absauth{Giovanni Soldà}
\meettitle{On the strength of some first-order problems corresponding to Ramseyan principles}
%\affil{School of Mathematics, University of Leeds, Leeds, LS2 9JT, United Kingdom}
%\meetemail{mmgs@leeds.ac.uk}

%\absauth{Marta Fiori Carones}
%\meettitle{Reverse Mathematics of some principles related to partial orders}
\affil{Department of Mathematics: Analysis, Logic and Discrete Mathematics, Ghent University, Krijgslaan 281 S8, 9000 Ghent}
\meetemail{giovanni.a.solda@gmail.com}

%\affil{Dipartimento di Scienze Matematiche, Informatiche e Fisiche, Università di Udine, Via delle Scienze 206, Udine, Italy}
%\meetemail{alberto.marcone@uniud.it}

%\affil{School of Mathematics, University of Leeds, Leeds, LS2 9JT, United Kingdom}
%\meetemail{P.E.Shafer@leeds.ac.uk}

%\affil{School of Mathematics, University of Leeds, Leeds, LS2 9JT, United Kingdom}
%\meetemail{mmgs@leeds.ac.uk}
%% INSERT TEXT OF ABSTRACT DIRECTLY BELOW

% Reverse Mathematics is an ongoing program in mathematical logic, the main goal of which is to investigate the role of set existence axioms in the development of mathematics (a standard reference is \cite{sim}). Its standard setting is the theory of second-order arithmetic, $\Zd$. The goal of Reverse Mathematics is achieved by reversing the usual mathematical process, trying to deduce from a theorem the set existence axioms used to prove it.

Given a represented space $X$, we say that a problem $f$ with $\mathrm{dom}(f)\subseteq X$ is \emph{first-order} if its codomain is $\mathbb{N}$. In this talk, we will study, from the point of view of Weihrauch reducibility, some first-order problems corresponding to Ramseyan combinatorial principles.

We will start by analyzing some problems that can be seen naturally as first-order: more specifically, after mentioning some well-established results due to Brattka and Rakotoniaina \cite{brattka-rakotoniaina}, we will proceed to study some principles whose strengths, form a reverse mathematical perspective, lie around $\isig 2$, as proved mainly in \cite{pradic-et-al}.

We will then move to study the \emph{first-order part} $\fop{f}$ of problems $f$ which cannot be presented as first-order ones: intuitively speaking, $\fop{f}$ corresponds the strongest first-order problem Weihrauch reducible to $f$. The first-order part operator was introduced by Dzhafarov, Solomon and Yokoyama in unpublished work, and it has already proved to be a valuable tool to gauge the strengths of various problems according to Weihrauch reducibility. After giving some technical results on this operator, we will focus on $\fop{(\rt^2_2)}$, presenting various results on the position of its degree in the Weihrauch lattice.


% In recent work, Dzhafarov, Solomon and Yokoyama introduced the \emph{first-order part} of a problem $g$, denoted by $\fop g$, as an analog to the first-order part of a theory, which is studied in reverse mathematics. Although the definition of this operator is rather technical, it can be shown that for every problem $g$ the Weihrauch degree of $\fop g$ admits a simple description:
% \[
%   \fop g\equivW\max_{\leqW}\{f: f\text{ is first-order and }f\leqW g\}.
% \]
% Intuitively, we can then say that the first-order part of a problem $g$ represents the projection of $g$ into the space of first-order problems.






The results presented are joint work with Arno Pauly, Pierre Pradic, and Manlio Valenti.

\begin{thebibliography}{1}

\bibitem{brattka-rakotoniaina}
  {\scshape Vasco Brattka and Tahina Rakotoniaina},
  {\itshape On the uniform computational content of Ramsey's theorem},
  {\bfseries\itshape The Journal of Symbolic Logic},
  vol.~82 (2015), no.~4, pp.~1278--1316

\bibitem{pradic-et-al}
  {\scshape Leszek A. Kolodziejczyk, Henryk Michalewski, Pierre Pradic, and Michał
    Skrzypczak},
  {\itshape  The logical strength of Büchi’s decidability theorem},
  {\bfseries\itshape 25th EACSL Annual Conference on Computer Science Logic (CSL 2016)}
  (Marseille, France),
  (Jean-Marc Talbot and Laurent Regnier),
  vol.~62,
  Schloss Dagstuhl--Leibniz-Zentrum für Informatik,
  2016,
  pp.~36:1--36:16.

\end{thebibliography}

\vspace*{-0.5\baselineskip}
% this space adjustment is usually necessary after a bibliography
\end{document}


%% READ ME
%% READ ME
%% READ ME

INSTRUCTIONS FOR SUPPLYING INFORMATION IN THE CORRECT FORMAT:

1. Author names are listed as First Last, First Last, and First Last.

\absauth{FirstName1 LastName1, FirstName2 LastName2, and FirstName3 LastName3}


2. Titles of abstracts have ONLY the first letter capitalized,
except for Proper Nouns.

\meettitle{Title of abstract with initial capital letter only, except for
Proper Nouns}


3. Affiliations and email addresses for authors of abstracts are
  listed separately.

% First author's affiliation
\affil{Department, University, Street Address, Country}
\meetemail{First author's email}
%%% NOTE: email required for at least one author
\urladdr{OPTIONAL}
%
% Second author's affiliation
\affil{Department, University, Street Address, Country}
\meetemail{Second author's email}
\urladdr{OPTIONAL}
%
% Third author's affiliation
\affil{Department, University, Street Address, Country}
% Second author's email
\meetemail{Third author's email}
\urladdr{OPTIONAL}


4. Bibliographic Entries

%%%% IF references are submitted with abstract,
%%%% please use the following formats

%%% For a Journal article
\bibitem{cite1}
{\scshape Author's Name},
{\itshape Title of article},
{\bfseries\itshape Journal name spelled out, no abbreviations},
vol.~XX (XXXX), no.~X, pp.~XXX--XXX.

%%% For a Journal article by the same authors as above,
%%% i.e., authors in cite1 are the same for cite2
\bibitem{cite2}
\bysame
{\itshape Title of article},
{\bfseries\itshape Journal},
vol.~XX (XXXX), no.~X, pp.~XX--XXX.

%%% For a book
\bibitem{cite3}
{\scshape Author's Name},
{\bfseries\itshape Title of book},
Name of series,
Publisher,
Year.

%%% For an article in proceedings
\bibitem{cite4}
{\scshape Author's Name},
{\itshape Title of article},
{\bfseries\itshape Name of proceedings}
(Address of meeting),
(First Last and First2 Last2, editors),
vol.~X,
Publisher,
Year,
pp.~X--XX.

%%% For an article in a collection
\bibitem{cite5}
{\scshape Author's Name},
{\itshape Title of article},
{\bfseries\itshape Book title}
(First Last and First2 Last2, editors),
Publisher,
Publisher's address,
Year,
pp.~X--XX.

%%% An edited book
\bibitem{cite6}
Author's name, editor. % No special font used here
{\bfseries\itshape Title of book},
Publisher,
Publisher's address,
Year.
