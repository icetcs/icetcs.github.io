%% FIRST RENAME THIS FILE <yoursurname>.tex. 
%% BEFORE COMPLETING THIS TEMPLATE, SEE THE "READ ME" SECTION 
%% BELOW FOR INSTRUCTIONS. 
%% TO PROCESS THIS FILE YOU WILL NEED TO DOWNLOAD asl.cls from 
%% http://aslonline.org/abstractresources.html. 


\documentclass[bsl,meeting]{asl}

\AbstractsOn

\pagestyle{plain}

\def\urladdr#1{\endgraf\noindent{\it URL Address}: {\tt #1}.}


\newcommand{\NP}{}
%\usepackage{verbatim}

\begin{document}
\thispagestyle{empty}

%% BEGIN INSERTING YOUR ABSTRACT DIRECTLY BELOW; 
%% SEE INSTRUCTIONS (1), (2), (3), and (4) FOR PROPER FORMATS

\NP  
\absauth{Tim Button}
\meettitle{{MOON} theory: Mathematical Objects with Ontological Neutrality} 
\affil{UCL, Philosophy Department, 19 Gordon Square, London, WC1H 0AG}
\meetemail{tim.button@ucl.ac.uk}
\urladdr{http://www.nottub.com/}

%% INSERT TEXT OF ABSTRACT DIRECTLY BELOW
The iterative notion of set starts with a simple, coherent story, and yields a paradise of mathematical objects, which ``provides a court of final appeal for questions of mathematical existence and proof'' (\cite[p.26]{Maddy:NM}).  But it does not present an attractive mathematical ontology: it seems daft to say that every mathematical object is ``really'' some (pure) set. My goal, in this paper, is to explain how we can inhabit the set-theorist's paradise of \emph{mathematical objects} whilst remaining \emph{ontologically neutral}. 

I start by considering stories with this shape: (1) Gizmos are found in stages; every gizmo is found at some stage. (2) Each gizmo reifies (some fixed number of) relations (or functions) which are defined only over earlier-found gizmos. (3) Every gizmo has (exactly one) colour; same-coloured  gizmos reify relations in the same way; same-coloured gizmos are identical iff they reify the same relations.

Such a story can be told about (iterative) sets: they are monochromatic gizmos which reify one-place properties. But we can also tell such stories about gizmos other than sets. By tidying up the general idea of such stories, I arrive at the notion of a MOON theory (for Mathematical Objects with Ontological Neutrality). 

With weak assumptions, I obtain a metatheorem: \emph{all MOON theories are synonymous}. Consequently, they are (all) synonymous with a theory which articulates the iterative notion of set (LT$_+$; see \cite{Button:LT1}). So: all MOON theories (can) deliver the set-theorist's paradise of mathematical objects. But, since different MOON theories have different (apparent) ontologies, we attain ontological neutrality. 

My metatheorem generalizes some of my work on Level Theory (\cite{Button:LT1}, \cite{Button:LT2}, \cite{Button:LT3}). It also delivers a partial realization of Conway's ``Mathematician's Liberation Movement'' \cite[p.66]{Conway:ONG}. 


\begin{thebibliography}{10}
\bibitem{Button:LT1}Button, T. Level Theory, Part 1: Axiomatizing the bare idea of a cumulative hierarchy of sets. {\em Bulletin Of Symbolic Logic}. \textbf{27}, 436-60 (2021)
\bibitem{Button:LT2}Button, T. Level Theory, Part 2: Axiomatizing the bare idea of a potential hierarchy. {\em Bulletin Of Symbolic Logic}. \textbf{27}, 461-84 (2021)
\bibitem{Button:LT3}Button, T. Level Theory, Part 3: A boolean algebra of sets arranged in well-ordered levels. {\em Bulletin Of Symbolic Logic}. \textbf{28}, 1-26 (2022)
\bibitem{Conway:ONG}Conway, J. On Numbers and Games. (Academic Press, Inc,1976)
\bibitem{Maddy:NM}Maddy, P. Naturalism in Mathematics. (Oxford University Press,1997)

%% INSERT YOUR BIBLIOGRAPHIC ENTRIES HERE; 
%% SEE (4) BELOW FOR PROPER FORMAT.
%% EACH ENTRY MUST BEGIN WITH \bibitem{citation key}
%%
%% IF THERE ARE NO ENTRIES  
%% DELETE THE LINE ABOVE (\begin{thebibliography}{20}) 
%% AND THE LINE BELOW (\end{thebibliography})

\end{thebibliography}


\vspace*{-0.5\baselineskip}
% this space adjustment is usually necessary after a bibliography

\end{document}


%% READ ME
%% READ ME
%% READ ME

INSTRUCTIONS FOR SUPPLYING INFORMATION IN THE CORRECT FORMAT: 

1. Author names are listed as First Last, First Last, and First Last.

\absauth{FirstName1 LastName1, FirstName2 LastName2, and FirstName3 LastName3}


2. Titles of abstracts have ONLY the first letter capitalized,
except for Proper Nouns.




4. Bibliographic Entries

%%%% IF references are submitted with abstract,
%%%% please use the following formats

%%% For a Journal article
\bibitem{cite1}
{\scshape Author's Name},
{\itshape Title of article},
{\bfseries\itshape Journal name spelled out, no abbreviations},
vol.~XX (XXXX), no.~X, pp.~XXX--XXX.

%%% For a Journal article by the same authors as above,
%%% i.e., authors in cite1 are the same for cite2
\bibitem{cite2}
\bysame
{\itshape Title of article},
{\bfseries\itshape Journal},
vol.~XX (XXXX), no.~X, pp.~XX--XXX.

%%% For a book
\bibitem{cite3}
{\scshape Author's Name},
{\bfseries\itshape Title of book},
Name of series,
Publisher,
Year.

%%% For an article in proceedings
\bibitem{cite4}
{\scshape Author's Name},
{\itshape Title of article},
{\bfseries\itshape Name of proceedings}
(Address of meeting),
(First Last and First2 Last2, editors),
vol.~X,
Publisher,
Year,
pp.~X--XX.

%%% For an article in a collection
\bibitem{cite5}
{\scshape Author's Name},
{\itshape Title of article},
{\bfseries\itshape Book title}
(First Last and First2 Last2, editors),
Publisher,
Publisher's address,
Year,
pp.~X--XX.

%%% An edited book
\bibitem{cite6}
Author's name, editor. % No special font used here
{\bfseries\itshape Title of book},
Publisher,
Publisher's address,
Year.

