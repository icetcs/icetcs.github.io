% LC 2022

\documentclass[bsl,meeting]{asl}

\AbstractsOn

\pagestyle{plain}

\def\urladdr#1{\endgraf\noindent{\it URL Address}: {\tt #1}.}


\newcommand{\NP}{}
%\usepackage{verbatim}

\begin{document}
\thispagestyle{empty}



\NP  
\absauth{Bahareh Afshari}
\meettitle{From interpolation to proofs}
\affil{ILLC, University of Amsterdam, The Netherlands. \\ Department of Philosophy, Linguistics and Theory of Science, University of Gothenburg, Sweden}
\meetemail{bahareh.afshari@gu.se}


From a proof-theoretic perspective, the idea that interpolation is tied to provability is a natural one. Thinking about Craig interpolation, if a `nice' proof of a valid implication $\phi\to\psi$ is available, one may succeed in defining an interpolant by induction on the proof-tree, starting from  leaves and proceeding to the implication at the root.
This method has recently been applied even to fixed point logics admitting cyclic proofs \cite{AL22,Sham14}.
In contrast, for uniform interpolation, there is no single proof to work from but a collection of proofs to accommodate:  a witness to each valid implication $\phi\to\psi$ where the vocabulary of $\psi$ is constrained. Working over a set of prospective proofs and relying on the structural properties of sequent calculus is the essence of Pitts' seminal result on uniform interpolation for intuitionistic logic \cite{Pitts92}.

In this talk we explore the opposite direction of the above endeavour, arguing that uniform interpolation can entail completeness of a proof system.
We will demonstrate this in the case of propositional modal $\mu$-calculus by showing that the uniform interpolants obtained from cyclic proofs~\cite{ALM21} play an important role in establishing completeness  for the natural Hilbert axiomatisation of this fixed point logic.

\begin{thebibliography}{10}
%
\bibitem{AL22}
{\scshape Bahareh Afshari and Graham E.~Leigh},
{\itshape Lyndon interpolation for modal mu-calculus},
{\bfseries\itshape Language, Logic, and Computation {TbiLLC }2019}
(Cham),
(Ayb{\"u}ke {\"O}zg{\"u}n
and Yulia Zinova, editors),
vol.~13206,
Lecture Notes in Computer Science,
2022,
pp.~197--213.
%
\bibitem{ALM21}
{\scshape Bahareh Afshari, Graham E.~Leigh and Guillermo Men\'edez Turata},
{\itshape Uniform interpolation from cyclic proofs: The case of modal mu-calculus.},
{\bfseries\itshape Automated Reasoning with Analytic Tableaux and Related Methods - 30th International Conference, {TABLEAUX} 2021}
(Birmingham, UK),
(Anupam Das and Sara Negri, editors),
vol.~12842,
Springer,
2021,
pp.~335--353.
%
\bibitem{Pitts92}
{\scshape Andrew M.~Pitts},
{\itshape On an interpretation of second order quantification in first order intuitionistic propositional logic},
{\bfseries\itshape Journal of Symbolic Logic},
vol.~57 (1992), no.~1, pp.~33--52.
%
\bibitem{Sham14}
{\scshape Daniyar Shamkanov},
{\bfseries\itshape Circular Proofs for {G}\"odel-{L}\"ob Logic},
arXiv preprint arXiv:1401.4002,
2014.
\end{thebibliography}
\vspace*{-0.5\baselineskip}


\end{document}








