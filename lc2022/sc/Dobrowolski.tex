
%% TO PROCESS THIS FILE YOU WILL NEED TO DOWNLOAD asl.cls from
%% http://aslonline.org/abstractresources.html.


\documentclass[bsl,meeting]{asl}

\AbstractsOn

\pagestyle{plain}

\def\urladdr#1{\endgraf\noindent{\it URL Address}: {\tt #1}.}


\newcommand{\NP}{}
%\usepackage{verbatim}
\usepackage{hyperref}


\begin{document}
\thispagestyle{empty}

%% BEGIN INSERTING YOUR ABSTRACT DIRECTLY BELOW;
%% SEE INSTRUCTIONS (1), (2), (3), and (4) FOR PROPER FORMATS

\NP%
\absauth{Jan Dobrowolski}%
\meettitle{Tameness in positive logic}%
\affil{University of Manchester}%
\meetemail{Jan.Dobrowolski@manchester.ac.uk}%
\urladdr{https://www.math.uni.wroc.pl/\string~dobrowol}

%% INSERT TEXT OF ABSTRACT DIRECTLY BELOW

Positive logic is a very flexible framework unifying full first-order logic with several other settings, such as Robinson's logic (which studies existentially closed models of a possibly non-companionable first-order universal theory), hyperimaginary extensions of first-order theories (which are obtained by adding quotients by type-definable equivalence relations), and, in certain aspects, continuous logic.

The study of tameness in those contexts goes back to A. Pillay's work on simple Robinson's theories (\cite{Pi}), and I. Ben Yaacov's work on simple compact abstract theories (\cite{BY}). In the talk, I will present a joint work with M. Kamsma on NSOP$_1$ in positive logic and a joint work in progress with R. Mennuni on NIP in positive logic, discussing in particular the main motivating examples for the two projects: existentially closed exponential fields (studied before by L. Haykazyan and J. Kirby in \cite{HK}) and existentially closed ordered abelian groups with an automorphism.



\begin{thebibliography}{99}

\bibitem{BY} I. Ben Yaacov.
{\em Simplicity in compact abstract theories}, Journal of Mathematical Logic, 03(02):163–191, 2003.

\bibitem{HK} L. Haykazyan, J. Kirby,
{\em Existentially closed exponential fields}, 
Israel
Journal of Mathematics, 241(1):89–117, 2021.

%\bibitem{Hr} E. Hrushovski,
%{\em Definability patterns and their symmetries}, preprint, 2019.	

\bibitem{Pi} A. Pillay.
{\em Forking in the Category of Existentially Closed Structures}, Quaderni di
Matematica, 6:23–42, 2000.

.
\end{thebibliography}

\end{document}
