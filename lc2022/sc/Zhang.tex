%% FIRST RENAME THIS FILE <yoursurname>.tex. 
%% BEFORE COMPLETING THIS TEMPLATE, SEE THE "READ ME" SECTION 
%% BELOW FOR INSTRUCTIONS. 
%% TO PROCESS THIS FILE YOU WILL NEED TO DOWNLOAD asl.cls from 
%% http://aslonline.org/abstractresources.html. 


\documentclass[bsl,meeting]{asl}

\AbstractsOn

\pagestyle{plain}

\def\urladdr#1{\endgraf\noindent{\it URL Address}: {\tt #1}.}


\newcommand{\NP}{}
\usepackage{verbatim}

\begin{document}
\thispagestyle{empty}

%% BEGIN INSERTING YOUR ABSTRACT DIRECTLY BELOW; 
%% SEE INSTRUCTIONS (1), (2), (3), and (4) FOR PROPER FORMATS

\NP
\absauth{Jing Zhang}
\meettitle{Making the diamond principle fail at an inaccessible cardinal}
\affil{Department of Mathematics, Bar-Ilan University, Ramat Gan, Israel}
\meetemail{jingzhan@alumni.cmu.edu}

It is a well-known theorem by Shelah that for any infinite cardinal $\lambda>\aleph_0$, $2^{\lambda}=\lambda^+$ is equivalent to $\diamondsuit(\lambda^+)$. However, the situation at inaccessible cardinals is different. Woodin produced a model where the diamond principle fails at a (greatly) Mahlo cardinal, based on the analysis of the Radin forcing.  We will discuss the advantage and the limitation of such method. Furthermore, we demonstrate a new method giving rise to the failure of the diamond principle at an inaccessible cardinal, fundamentally different from Woodin's method. The differences from the previous method will be highlighted. Joint work with Omer Ben-Neria.


\vspace*{-0.5\baselineskip}
% this space adjustment is usually necessary after a bibliography

\end{document}


